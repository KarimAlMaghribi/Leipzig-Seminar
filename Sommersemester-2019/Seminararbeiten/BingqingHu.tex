\documentclass[11pt,a4paper]{article}
\usepackage{a4wide,url}
\usepackage[utf8]{inputenc}
\usepackage[ngerman]{babel}

\parindent0pt
\parskip3pt

\title{Seminararbeit „TRIZ in China“}
\author{Bingqing Hu}
\date{30. September 2019} 

\begin{document}
\maketitle
\tableofcontents

\section*{Zusammenfassung}
In diesem Aufsatz wird ein Vergleich der Entwicklung von TRIZ in China mit
den Erfinderschulen in der DDR der 1980er Jahre [0] vorgenommen. 

Allgemein wird TRIZ in China durch folgende drei Eigenschaften oder Trends
gekennzeichnet [1]:
\begin{enumerate}
\item Integriert.
  
Vor den 1980er Jahren war TRIZ umständlich und fragmentiert. Boris Zlotin und
Alla Zusman verwendeten als erste Computer, um die Anwendung von TRIZ zu
unterstützen.  Damit haben sich die Methoden der TRIZ vereinheitlicht und
vereinfacht, was einer der Hauptgründe ist, wieso TRIZ heute weltweit bekannt
ist.
\item Ingenieure statt Erfinder als Zielgruppe.
  
Die Förderung und Demonstration von TRIZ in Unternehmen hat einen wichtigen
Anteil an der Verbreitung innovativer Methode in China.
\item Standardisierung von TRIZ
  
Das System der Internationalen TRIZ-Zertifizierung wird auch in China
angewendet. Das System ist ein objektiver Bewertungsstandard für Innovationen
in Firmen und für Ausbildungsstandards für innovative Ingenieure. Aber China
hatte gerade damit angefangen, das System zu implementieren; es ist ein langer
Weg vom Beginn zur Reife.
\end{enumerate}

Die TRIZ-Forschung und -Praxis in China war tief beeinflusst von den
TRIZ-Entwicklungen in USA, besonders vor 2010.

Einer der größten Unterschiede zwischen Erfinderschulen in der DDR und
heutigen TRIZ-Anwendungen in China besteht darin, dass beide TRIZ-Varianten zu
unterschiedlichen Zeiten existierten.  Der entscheidende Unterschied zwischen
diesen zwei Zeiten ist die schnelle Entwicklung der Computertechnologie.

Gleich dagegen ist die Motivation, die wirtschaftlichen Zustände des
jeweiligen Landes zu verbessern.  Im weiteren Vergleich ergeben sich aber mehr
Unterschiede als Ähnlichkeiten. China ist ein Entwicklungsland mit mehr als
1,3 Milliarden Einwohnern und heterogen, die DDR war damals ein homogen
entwickeltes Industrieland.

Um die chinesische Entwicklung detailliert mit der in [0] entwickelten
Perspektive auf die DDR-Erfinderschulen zu vergleichen, habe ich mich am
Inhaltsverzeichnis von [0] orientiert.  In diesem Text geht es vor allem um
Parallelen zu chinesischen Entwicklungen im allgemeinen Zugang, wie er im
ersten Kapitel des Buches entwickelt wird.  Für weitere Studien fehlte mir
in diesem Semester leider die Zeit.

In diesem Aufsatz wird sich auf C-TRIZ als chinesische Version für TRIZ
bezogen, eine TRIZ-Variante, die von Runhua Tan (HeBei Industrie Universität)
entwickelt wurde. TRIZ in China ist allerdings mehr als nur diese Theorie und
umfasst vielerlei weitere Aktivitäten im TRIZ-Umfeld. 

Viele in diesem Artikel referenzierte Dokumente oder Artikel liegen nur in
einer chinesischen Version vor, was das weiterführende eigenständige Studium
des Lesers erschwert. Weitere Unterlagen sind Dokumente der chinesischen
Regierung und in einer entsprechenden politischen Sprache und Logik verfasst.

\section{Hintergrund, Ziel und Weg der chinesischen TRIZ-Version (C-TRIZ)}

\subsection{Staatliche Weisungen}
Nach der Kutur-Revolution in China (1966--1976) hatte die chinesische
kommunistische Patei entschieden, die Wirtschaft zielgerichteter zu entwickeln
und dies als die wichtigste Aufgabe der Regierung deklariert.

Nach 1991 regiert in China eine der größten kommunistischen Pateien der Welt.
Die Regierung möchte ihrem Volk beweisen, dass sie den richtigen politischen
Weg gewählt hat und dies der einzige Weg ist, auf dem eine zielgerichtete
Entwicklung der Wirtschaft möglich ist. Sie hat alles dafür getan, um eine
schnelle wirtschaftliche Entwicklung zu ermöglichen. Sogar die chinesische
Armee darf wirtschaftlich aktiv sein.

In den 1980er Jahren war TRIZ schon einigen chinesichen Wissenschaftlern
bekannt.  In den 1990er Jahren begannen einige chinesische Forscher, an
internationalen TRIZ-Konferenzen teilzunehmen. 2001 führte Iwint [3] eine
spezifisch für China entwickelte TRIZ-Trainingsmethodik ein und versuchte,
diese zu verbreiten. Zwei Jahre später hatte die Firma zwei Software-Programme
für TRIZ-Theorie und TRIZ-Praxis entwickelt.

2007 wird die weitere Verbreitung von TRIZ in China dadurch befördert, dass im
Juni drei Wissenschaftler (Wang, Dong, Yang etc.) einen Brief „Vorschläge zur
Verstäkung von innovativen Methoden unserer Nation” an den damaligen
Premierminister richteten und im Juli darauf eine Antwort erhielten. Danach
organisierten das Ministerium für Wissenschaft, das Bildungsministerium sowie
der Wissenschafts- und Technologieverband Veranstaltungen für ihre Mitarbeiter
in kleinem Kreis, um TRIZ zu popularisieren. Zugleich hatten diese
Organisationen der Regierung die Hauptaufgaben aufgelistet, die zur
Verstärkung und Verbreitung innovativer Methoden in China zu ergreifen sind,
und dazu auch dem Staatsrat einen Bericht vorgelegt.

Im selben Jahr hatten Regierungsstellen zusammen standardisierte
Traingsmaterialien und Trainingssoftware für TRIZ erstellt, von denen
behauptet wurde, dass sie sich am besten für die aktuelle Situation in China
eignen. Weiter wurden die Provinzen Xichuan und Heilongjiang als
Pilot-Provinzen für die Einführung dieser technologischen Innovationsmethodik
ausgewählt. Universitäten wie die Technische Universität Hebei, die Nordost
Forestry Universität, die Sichuan Universität, die Südwest Jiaotong
Universität und weitere waren die ersten Universitäten, die zur TRIZ-Theorie
und TRIZ-Methoden forschten. Jede der oben genannten Universitäten hatte ihr
eigenes Ausbildungssystem erstellt, nach welchem Graduierte und
Master-Studenten in innovativen Methoden ausgebildet wurden, und eine Reihe von
Vorlesungen und Kursen zur Thema „TRIZ-Theorie und -Methodik“ entwickelt.

\subsection{TRIZ und Innovationsmethodiken}

Seit langem gibt es die Debatte, ob es überhaupt einen strukturierten Weg
gibt, das Erfinden zu erlernen, dem die Menschen folgen können.  Einige
Wissenschaftler verneinen dies und begründen ihre Position damit, dass
Innovationen hauptsächlich von der Inspiration und unlogischem Denken der
Erfinder abhängen.  Andere Wissenschaftler gehen davon aus, dass Erfinden ein
systematisch verfolgbarer Prozess ist, zu dem sich methodische Elemente lernen
und vermitteln lassen.

Diese Gruppe von Wissenschaftlern verzweigt sich in zwei Teilgruppen. Eine
Teilgruppe betrachtet die Erfinder als Forschungsobjekt und fokussiert sich
auf die Mechanismen und Eigenschaften des -- im Kern unlogischen --
Erfindungsprozesses.  Ein typisches Beispiel für diesen Zugang ist, dass sich
nach Albert Einsteins Tod Wissenschaftler aus verschiedenen Bereichen daran
gesetzt haben, die Anatomie von Einsteins Gehirn zu untersuchen, um die
Geheimnisse der Innovation zu entdecken. Die andere Teilgruppe betrachtet eher
die Produkte des Erfindungsprozesses als Forschungsgegenstand wie etwa im
Patentamt eingereichte Schriften. TRIZ entstand aus dieser letzteren Strömung.

Trotz seiner theoretischen Unvollkommenheit hat TRIZ eine ausgezeichnete
Wirkung auf Innovationsprozesse. Nach einer allgemeinen Statistik hat TRIZ die
Anzahl der Patent um 80\% erhöht, die Qualität der Patente verbessert und die
Zeit vom Innovationsanfang bis zur Marktreife um mehr als 50\% verkürzt. In
diesem Punkt glaubte die Regierung an die Theorie, da ihr das Ergebnis
wichtiger als die Logik war.

\subsection{TRIZ an der Technischen Universität Hebei}

An der Technischen Universität Hebei haben die Professoren viele Ergebnisse im
Bereich der TRIZ-Theorie, Software für TRIZ, Material für TRIZ, TRIZ Kurse für
Master und Graduierte etc. entwickelt. Darunter ist das C-TRIZ-Modell
besonders erwähnenswert. Beiträge zur TRIZ-Forschung anderer Universitäten
werden hier nicht diskutiert.

C-TRIZ ist keine weitere Verbesserung oder theoretische Variante von TRIZ,
sondern fokussiert mehr auf die TRIZ-Praxis und das Training von Ingenieuren
für den Prozess der Entwicklung neuer Produkte und das Einreichen von
Patenten.

Die Entwicklung von C-TRIZ, das TRIZ in vielen Details und Praxen in China
ergänzt und hauptsächlich der Verbreitung von TRIZ dient, wird von Professor
Runhua Tan koordiniert und verbindet internationale Erfahrungen und
Entwicklungen bei TRIZ mit der konkreten Situation in China. Im Lauf der Zeit
hat C-TRIZ bewiesen, dass es sehr erfolgreich in China eingesetzt wird. Für
C-TRIZ ist MEOTM (mass engineer oriented training model) ein Kernbegriff.
Mehr dazu ist in [4] zu finden.

Das Modell MEOTM ist ein schon ziemlich vollkommenes Modell für die
Verbreitung von TRIZ in China, wie die statistischen Erhebungen in [4] zeigen.
Das Modell wurde relativ systematisch auf der Basis von Fragestellungen und
faktenbasierten Analysen von Herstellungsprozessen bis zu Messung der
Zweckmäßigkeit herausgebildet. Auf jeden Fall ist das Modell ein sehr
umfangreiches und kompliziertes System, welches mehr Gewicht auf
Managementprozesse statt auf die Ingenieurstätigkeit legt. In den Prozessen
7-STEP und 6-GATE entscheiden die Manager, welche Ingenieure geeignet sind für
die Trainings bzw. qualifiziert werden sollen. Das setzt hochqualifizierte
Manager voraus.  In der Gesamtsicht ist das Modell eher praktisch aufgestellt
mit wenigen logisch dargestellten Zusammenhängen und ohne sichtbare
ontologische Modellierungen.  Das erschwert die Konzipierung konkreter
Schritte nach der 7-STEP und 6-GATE Methodik in konkreten Anwendungen. Wenn
ein Begriff wohldefiniert ist, dann ist er leicht zu verstehen und
anzuwenden. Sonst ist das Gegenteil der Fall.

Wenn es um Verbesserungen des Modell geht, dann wären meine Vorschläge,
mehrere Kriterien für die Auswahl der Firma und der Ingenieure zu erstellen,
um weniger Fehler im ersten Schritt zu machen. Und die Trainer in der Phase 1
sollten auch erst standardisiert und dann modifiziert werden, damit das
Training wirklich einen Effekt für die Teilnehmer hat.

Von einer Statistik nach Wahrscheinlichkeitstheorie ist es häufig ein sehr
langer Weg bis zu wirksamen Praxen, und auf dem Weg werden wir verschiedene
schöne und weniger schöne Landschaften sehen.

\subsection{TRIZ in Iwint}
Wenn das Niveau der TRIZ-Forschung an der Technischen Universität Hebei ein
Gipfel ist, dann ist die Firma Iwint ein anderer Gipfel im Forschungsfeld der
TRIZ in China. Iwint steht zu 100\% auf dem ersten Rang aller Firma im Bereich
CAI (Computer Aided Innovation) in China.  Die Firma hatte schon 2002 eine
Software für innovative Methoden auf den Markt gebracht. Diese Software für
innovative Methoden ist eine Plattformen, auf der Ingenieure, Entwickler,
Personal für geistiges Eigentum und Mitarbeiter des Wissensmanagements in
einer gemeinsamen Arbeitsumgebung zusammenarbeiten, welche Pro/Innovator [6]
heißt.  Im Kern geht es um die Analyse des System, in welchem das Produkt
erstellt wird, und um die Auseinandersetzung mit Widersprüchen in den drei
Dimensionen Kausalität, Nutzbarkeit und Betriebsumgebung.  Nach der Eingabe
der entsprechenden Informationen erstellt die Software nach TRIZ automatisch
Lösungsvorschläge. Diese werden dann bewertet, ob es einen Markt für deren
Umsetzung gibt und ob die Firma mit der Lösung ein Patent anmelden kann.
Pro/Innovator basiert auf einer wissenschaftlichen Datenbank. Zugleich ist
Pro/Innovator eine Plattform für Entwurfsmethodiken.

\subsection{Schöpfertum und Methodologie von TRIZ in China}
Schöpfertum bedeutet nichts anderes als zwei Sachen zu kombinieren, welche
früher nicht miteinander kombiniert waren, sodass mit der Kombination ein
Problem gelöst wird. Sind Sie zum Beispiel ein Informatiker und kennen sich
auch in der Philosophie aus, dann sind Sie eine Person mit Schöpfertum, wenn
sie mit diesem Schöpfertum etwas Neues erstellen. Dieses Prozess heißt
Innovation.

Die erste großartige Innovation in China geht zurück auf die Han-Dynastie, wo
Zhai Len als erster Papier hergestellt hat. Am Anfang der industriellen
Revolution hat Benz das erst Auto, Watt die erst Dampfmaschine hergestellt.
Damals blockierte die chinesische Regierung ganz China, mit der äußeren Welt
zu kommunizieren oder zu handeln. Am Anfang der Zeit der Reform und Öffnung
gab es in China keine ausgezeichneten Erfinder. Als Vorbild für das Volk hatte
die Regierung Edison gewählt, mit der Hoffnung, dass sich aus den Erfahrungen
oder Inspirationen von Edisons Biographie etwas Positives gewinnen lässt. Vor
der Einführung von TRIZ verwendeten die meisten Erfinder die Methode von
Versuch und Irrtum, die wie folgt abläuft -- der Erfinder hat eine Idee, die
er versucht umzusetzen. Wenn die Idee nicht funktioniert, sucht der Erfinder
eine neue Idee. Der Erfinder macht so weiter, bis er eine geeignete Lösung
gefunden hat. Die Methode von Versuch und Irrtum ist sehr ineffizient im
Vergleich zu TRIZ.

Obwohl die Antwort auf die Frage, ob Schöpfertum auch anders als allein mit
Übungen oder Training verbessert werden kann, noch unbestimmt ist, scheint
unter den Anhängern von TRIZ in China „ja“ als Antwort zu überwiegen.
Insgesamt lassen sich die Organisation, welche in China eng mit TRIZ verbunden
sind, in drei Gruppen unterteilen: 
\begin{itemize}
\item Die Regierung, welche für die politische Sache zuständig und in der
  entscheidende Position aller Organisationen ist,
\item die Unis, welche hauptsächlich die Forschungen vorantreiben und
\item die Firmen, in denen TRIZ so weit wie möglich angewendet werden soll.
\end{itemize}
Die Logik hinter dieser Anordnungen ist, dass sich neue Theorien und Ansätze
im Bereich von TRIZ schnell theoretisch und praktisch bewerten lassen, welche
Auswirkungen auf den Markt sie haben. Außerdem lassen sich in einer solchen
Anordnung neue Theorien auf der Basis der Rückmeldungen aus den Experimenten
in den Firmen verbessern.

\subsection{Die Sichtbarkeit von TRIZ in China}
%-------------------
Die TRIZ-Organisationen in China setzen nicht so sehr auf theoretisch-formelle
Ableitungen der Theorie, sondern auf deren praktische Anwendung in der
Industrie.  Die Regierung hat nicht nur Interesse an den mit TRIZ
eingefahrenen Gewinnen, sondern auch an der Anzahl der durch TRIZ erworbenen
Patente. Die durch TRIZ gefundenen Lösungen bleiben im Kopf der Ingenieure und
Unternehmen. Zugleich werden die durch TRIZ erworbenen Patenten in Patentamt
gut dokumentiert. Was mehr wichtig ist die Beiden Sache in eine einheitlich
Datenbank einbringen. Damit diese Daten auch mit machine learning method
verarbeiten kann. Dadurch können einige für Mensch schwer abziehbaren
Schlussfolgerung mit Computer gefunden werden. Die billige Anwendung für diese
Datenbank ist ein Bild zu stellen, um anderen, wer kaum Erkenntnis über TRIZ
haben, zu überzeugt, dass TRIZ wirklich wirksam bei Verbesserung der Produkt
oder des Anteils in der Markt ist. Solche Bilder brauchen auch die Politiker,
um ihre Volk zu überzeugen.

In China hat die Regierung zahlreich Uni und Gebäude, in den Veranstaltungen
der TRIZ stattfinden können. Angesicht der Anzahl der absolvierten Studenten
per Jahr(ca. 7 Million in 2018),welche durchschnittlich (weit) nicht so gute
qualifiziert wie die Studenten in Deutschland sind, möchten  sehr größe Menge
der Ingenieur  gerne die TRIZ erkennen.  
  
\subsection{1.7 Wachsende Einfluss}

Seit die Einführung der TRIZ in China in 2001 von iwint hatten TRIZ immer
wachsende Einfluss. Die Gründen dafür sind folgende. 1. die Chinesische
Wirtschaft brauchen TRIZ dringend, um die Struktur der Writschaft zu
verändern,  von ein Wirtschaft mit einer Struktur, mit der bei den
Erstellungen der Produkte viele Ressourcen verschwendet werden,  nach einer
Umweltfreundlichen Wirtschaft. 2. Führung der Regierung spielt bei Wachsende
Einfluss der TRIZ in China auch ein große Rolle, die Regierung hat die
stärkste Nachrichtenagentur und Propaganda. 3. Ein tiefer Grund dafür ist nach
kulturelle Revolution in Sechziger und Siebziger hasst das Volk die politische
Bewegungen, in denen hundert von tausend der Personen getötet. Solche
Bewegungen haben auch die Wirtschaft vollständig ruiniert, Schlecht
wirtschaftlich Zustand führt schlecht Lebensqualität für die Meisten. 

Von Anfang Achtzigerjahre respektiert China die westliche Kultur mehr als die
chinesische alte Kultur, TRIZ ist eine typische Ergebnis von westlichen
Kultur, obwohl TRIZ bis jetzt noch nicht wohl bewiesen werden. Viele bekannte
Uni in China begann TRIZ zu forschen, einige Professor(8) an bestimmten Uni
sogar hoffen, dass alle Studenten und Professoren an der Uni und
Ingenieur,deren Anzahl in 2011 ca 38.5 Millionen, in China in nicht so weit
Zukunft , TRIZ kennenzulernen.  

TRIZ funktioniert in in viele großer staatlich Firma wie China Schiffbau
Schwerindustrie und Chengdu Flugzeugfabrik…, Diese erfolgreichen Beispiel
macht die TRIZ bekannt  und akzeptierbar  für die  Geschäftsinhaber.  Die
Geschäftsinhaber sind die meisten Teilnehmer der Trainingskurs der
TRIZ. Welche normalerweise ziele teür ist im Vergleich zu anderen
Trainingskurs. 

Mit Wachsende Einfluss der TRIZ in China wird die Wirtschaft von einer
niedrigen Niveau auf eine relative höheres Niveau gelandet. Für die
Softwarefirma und Firma wie Bank ist TRIZ fast hilflos. In diese Bereich haben
noch TRIZ keine Plätz. 

\subsection{1.8 Erfinden und Erfinder in der öffentlich Meinung}

Erfinder wird hoch respektiert, zugleich wird der methodologische Fortschritt
bei Denken und erfinderisches Denken wenige beachtet. Weltbekannte Erfinder
wie Zeppelin oder Edison sind jetzt absolute berühmt in weniger als 30 Jahre
junge Menschen. Erfinderisches Denken und methodologische Fortschritte bei
Denken kann nicht von den meisten Personen oder Student erklärt werden. Wenn
es um die Methodologie geht, dann sollten erst die Ontologie und
Erkenntnistheorie kennen, welche den normalen Personen ganz nicht leicht in
kurz Zeit zu verstehen sind. Besondere schwerer ist es in ein asiatisches
Land,wo die alte griechische Kultur der Philosophie verpasst werden. Statt
Mathe und Naturphilosophie hatte es in China soziale Philosophie entwickelt,
so legte die chinesische Philosophie mehr Wert auf die Relationen zwischen
Menschen. Für Konfuzius ist diese Relation Hierarchie und Ordnungen. Solche
Philosophie prägen auf heutigen Chinesen ein. Aus diesen Grund war Edison
früher ganz positives Beispiel, jetzt ist sein Effekt in China nicht so
positiv, wenn mehr und mehr Leute wissen, was passierte zwischen Edison und
Tesla.

In chinesischen Volksmund wird Erfinder mit „gut ausgebildet“,“
leistungsfähig“,“hoch intelligent“ verbunden. Wer in dieser Bereich ein Stelle
haben, zu forschen oder zu trainieren, wird auch respektiert von die meisten
Volk, sodass die Professor aus Fakultät der Kunst die TRIZ forschen möchten,
nachdem die anfänglich Arbeite der Verbreitung der TRIZ in China  meisten von
dem frühen in Ingenieur Studiengang absolvierten Professoren geleistet
haben. Es gibt auch ein klein Kreis von Leute, die TRIZ mit
„Pseudowissenschaft“  bezeichnet.  Solche Meinung haben wenige Unterstützung
der Bevölkerung bekommt. 

Erfinden ist den Chinesen ein Geheimnisvoll Wörter. Fast jede Chinesen möchten
Etwas zu erfinden, damit er die Anerkennung der Gesellschaft erhaltet, um eine
bessere Berufsaussicht zu haben. Wenn Sie ein Chance anbietet werden, die
Kompetenz der Erfinden zu haben, sie werden mit großer Möglichkeit die Chance
erste nehmen, ohne Überlegung ob die Kompetenz wirkliche am Ende schaffbar
ist. Im Gegensatz zum westlichen hoch entwickelte Ländern sind die Gesetz in
vieler Bereich besondere in Hightech-Bereich in China nicht so klar und
vollständig, was macht die Anerkennung der Gesellschaft sehr wichtig. 

\subsection{1.9 Das Nationale Forschungszentrum für technologische Innovation}

Das Nationale Forschungszentrum für technologische Innovation Methode und
Werkzeug(7) als Gründerparkett und Ministerium für Wissenschaft als
Gründerpate

Im Juli 2007 antwortete Premierminister JiabaoWen die Brief von 3
Wissenschaftler DahengWang, DongshengLiu, DuYe und Anweisungen gegeben,welche
lautet “Für die unabhängige Innovation kommt Methode zürst“.  das Ministerium
für Wissenschaft, Ministerium für Wissenschaft,Bildungsministerium und
Wissenschafts- und Technologieverband arbeitete zusammen zur Verbreitungen und
weiteren Forschung der TRIZ. Locale Regierungen in verschiedenen Provinz in
China folgte die oben 4 Organisation nahe. Das Ministerium für Wissenschaft
hatte dafür auch eine Forschungsausschuss gebildet. Alle Unis, die am Anfang
2008 TRIZ forscht, brannten auf die finanzielle Unterstützung von Ministerium
für Wissenschaft. Für die Auswahl der Firmen, welche erst die TRIZ Training
teilnehmen werden, existiert einige Regels . Das Büro für Wissenschaft und
Technologie(8) in HeiLongJiang Provinz, welches ein Ausschluss von Ministerium
für Wissenschaft in HeilongJiang ist, hat uns ein gute Beispiel des Auswahl
der Firma in China gegeben. Einige notwendige Bedingungen dafür sind, erst die
Umsatz der Firma per Jahr sollte mehr als 5 Million RMB, Tax per Jahr sollte
mehr als 0.5 Million. Die Firmen müssen in der Bereich von fortschrittlichen
Herstellung, Information, neün Material, neün Energie, Biologie und Medizin,
Umweltschutz und öffentliche Sicherheit legen. Und mindesten ein Patent der
Erfindung oder andere gleich identifizierten Patent stand auch in der
Bedingungen. Nachdem die Firma als qualifizierte gewählt werden, kommt dann
die kraftvoll Unterstützung von die Organisation nämlich Büro für Wissenschaft
und Technologie in HeiLongJiang Provinz. Die Unterschützen bedeckte
hauptsächlich die Priorität der Firmen bei allen Projekten für die locale
Regierung, Hilfe der Professor der TRIZ, wer von der Regierung organisiert
werden, bei Innovation. Diese Unterstützung ist nicht mit Geld messbar, wenn
die messbar wäre, dann lohnt die mindesten Millionen. Die Anzahl der Firmen
ist 130(9), die Training betrifft 4000 Personen.

Nach 11 Jahren ist jetzt TRIZ fast in aller große Firma in allen Provinz eine
der bekannteste Theorie.

In der kommunistische Partei ist aller Veranstaltungen die Führung von oben
oder die unterstützung von oben in der Partei immer sehr wichtig. Im Jahr 2012
war Likeqiang Preminister, er betont mehr an dem Modelle „Internet+“ der
Wirtschaft. Trotzdem hat TRIZ bis jetzt in China noch eine große Markt.


\section{Rahmenbedingungen der TRIZ in den 21. Jahrhundert}

\subsection{2.1 Das gesellschaftliche Ansehen der Erfinder in China}

Erfinder in China werden gute kümmert von der Regierungen, viele Title wurde
verteilt auf den Erfinder und viele Vorteile,wie Kredit bei Ausleihen der Geld
von staatlichen Banken und so weit, haben die Erfinden im vergleichen zum
anderen normalen Leute. Was nicht gute strikt geschützt werden, ist die
Patenten, welche von dem Erfinder erworben werden. Die gesetzliche Schutz der
Patent und Copyright sind eine kompliziert Aufgabe. Für eine kommunistische
Regierung scheint diese Aufgabe immer eine unmöglich Aufgabe. Die Logik
dahinter ist, dass das Volk in der kommunistische Nation gemein arm und das
Kosten einer Patent ziemlich ist, wenn es eine streng Schutz der Patent gab,
dann ist das Kosten dem meisten Volk bezüglich der Einkommen untragbar, welche
führt zu Schade der Wirtschaft und dem Volk nicht zum Zufriedenheit des
Lebensqualität. Seit 2015 hatte die chinesische Regierung entschieden, außer
die Teufelskreis zu gehen. Eine bedeutsame Änderung ist dass jetzt in
Baidu(die grosseste Search Machine in China, dessen Begründer ist der Nüwa
einem Stellervertretender Gouverneur in einer Provinz in China) fast kein frei
Ressource von Buch gefunden werden. In Alibaba existiert jetzt weniger
Fälschung der Produkt der bekannten Mark.

In China ist Patent in 3 Klasse unterteilt, nämlich Erfindungspatent,
Gebrauchsmuster und Design . Für das Design ist die Beantragung am Patentamt
am einfachsten, entspricht ist die Verteidigung dieser Patent am schwächsten.

Ob mit einer Patent(10) Geld verdienen werden konnte, hängt davon ab, ob die
Patent die Markt haben und ob eine Firma gibt, das diese Patent kaufen oder
benutzen möchten. Die Preis der normalen Patent beim Verkauf ist 2000-3000 RMB
in 2018, für einige wichtige Preis ist die Preis höher von 10 Tausend RMB bis
1000 Tausend RMB. Für die Zulassung der Patent bei Verwendungen bracht ein
ausführlich Vertrag zwischen die Firma und dem Inhaber der Patent. Mit Patent
kann man auch von Bank Geld ausleihen, wie Betrag der Geld entscheidet die
Bewertung der Patent bei der Bank. Patent kann auch mit Stock
austauschen. Hier ist die vier Wegen, auf die Erfinder durch Patent Geld
verdienen, aber realistisch gesagt, die ist schwer , wegen zahlreich ähnliche
Patenten gibt es in chinesische Patentamt. Ohne ausführlich Gesetz ist die
Korruption in Patentamt eine typische Kränkung. Aber für die Firmen,welche von
dem Ministerium der Wissenschaft für das Training der TRIZ gewählt werden,
ist es mehr leicht mit Innovation eine Patent bei Patentamt zu beantragen.

Die in der Patentamt in China von2012-2014 angemeldeten Patent werden nur 2\%
angewendet. Im Falle einer Markenverletzung beläuft sich die
durchschnittlichen Entschädigungsbetrag nach dem Urteil der Gericht auf 62
Tausenden RMB in China, Dieser Betrag steht nicht verhältnismäßig zu den
großen Anstrengungen der Inhaber, die Marke für eine lange Zeit
aufrechtzürhalten. solch Betrag in Europa und USA ist mehr hoch, Es betrug in
USA in selben Jahren 29 Million.

\subsection{2.2 Freiräume, sachbedingten Hürden und Animationspunkte der
  Erfindertätigkeit} 

Die Trainingskurs der TRIZ teilnehmenden Ingenieur zielt neben dem Training
darauf, bei dem Training ihm selbst wertvolle Ingenieur zu kennen und mit der
wertvolle Person dort Freunden zu machen. In der chinesischen Gesellschaft
heißt das Wahl,mit wem zusammen zu arbeiten, Freiräume. Dieses Wahl ist auch
bei innovative Prozess entscheidende Sache. Welche mehr oder wenig wie
MBA(Master Business Adminstration) in China. Wobei die Hauptaufgabe ist
berühmten Person oder Stars kennenzulernen und mit andere Personen eine
Freundschaft zu verstärken. Für eine Person wird die Fähigkeit, mit
verschiedenen Charakter habenden Personen gut zusammenzuarbeiten, wird
wichtiger als andere Kompetenz betrachtet.

Exakte benötige Information zu bekommen, ist die sachbedingten Hürden der
Erfindertätigkeit, wenn ein Erfinder eine Patent erworben möchten. Diese
Hürden vergrößert sich, falls der Gesetz nicht vollständig ist. Dabei spielt
Korruption eine große Rolle. Z.b bei Überprüfung der Mark in dem Patentamt ist
es sehr subjektiv(11), der Sachbearbeiter entscheidet, ob diese Überprüfungen
funktioniert.

Zielgruppen der Training der TRIZ sind Ingenieur, Verwalter und Student. Von
denen auch Erfinderische Aktivität erwartet werden. Nach Organisation hatte
das Training der TRIZ verschiedenen Modell, z.B wenn das Training von Hebei
Industrie Uni ausgeführt werden, dann ist das Modell MEOTM, wenn das Training
von Iwint ausgeführt werden, dann ist das Modell anderes, z.b DAOV(y+2)(12),
DAVO bedeutet Define Analyse Optimise Verify.  Y ist Jahr(year) auf Englische,
y+2 impliziert, was sollten oder müssen eine Firma tun in 2 Jahren. Mit
verschiedenen Trainingsmodell ist die Animationspunkte der Erfindertätigkeit
unterschied. Es gibt keine einheitliche Modell in China, obwohl alle Modell
die gleich Basis haben und alle Organisation versucht hatte, TRIZ zu
standardisieren.

\subsection{2.3 Einfluss des Themencharakters und der Themenvorbereitung auf
  die erfinderische Aktivität} 

In von der kommunistischen Partei geführte Land wie China dient die Wirtschaft
der Politik. Die Staats haben eine stark Kontrolle der Wirtschaft trotz dem
Wille, dass das Wirtschaft mehr Freiheit gegeben werden sollen. Für die
erfinderische Aktivität gibt es auch eine Leitlinie von der Regierung im Namen
des Gesellschaft, welche die erfinderischen Aktivitäten beschränkt und leitet.

In vorbeikommenden vierzig Jahren haben die chinesische Wirtschaft schnell
entwickelt, so haben jetzt die Regierung relativ genüge Ressource ihre
Perspektivplan aus zu führen. Die Firmen bevorzugen, die Themen zu nehmen,
welche in der Perspektivplan der zentralen Regierung liegen, werden. Damit
können sie mehr sogar große Unterstützung von allen Seiten erhalten. In
heutigen China sind alle hoch technologische Erfindung von dem Staat direkt
oder indirekt finanziert. Die 5- Jährige Planung der Regierung dominiert seit
40 Jahren schon die erfinderische Aktivität.

In 2018 hatte ein Professor heißend Jiankui He eine Experimente der
Gene-Editor bei Menschen ohne genug Beweisen der Sicherheit gemacht, was
führte direkt dazu, dass er seine Stelle als Professor an einer der bekannt
Uni in China verloren. Er lebt jetzt unter der Überwachung der Polizei.

\subsection{2.4 Erfinden in China als Basis finanzieller Freiheit}

Abhraham Maslow hatte die Maslowsche Bedürfnishierarchie(12) entwickelt,
welche die menschliche Befürfnisse und Motivationen  in einer hierarchischen
Struktur auf ein vereinfacht Weise. Diese Hierarchie ist aus psychologisch
Sicht. Und ahnlich gibt es auch eine Bedürfnishierarchie aus inanzieleller
Sicht. In der Bedürfnishierarchie gibt insgesamt 9 Schichten.

1.Supermarktsfreiheit

Wenn Sie ins Supermarkt gehen, kaufen die Dings, die sie benotige, egal ob das
Preis der Dings einige hoch ist.

2.Essen-Freiheit

Wenn man ins Restaurants zum Essen gehen, sie gehen ins  Restaurant, die  sie
gehen möchten, was als erstes in Betracht gezogen werden, sind die Gerichte,
die Dekoration, der Service und nicht der Preis des Gerichte.

3.Reisefreiheit

Solange Sie bereit sind zu reisen,  Bei Wählen der Sehenswürdigkeit  kommt
das Preis nicht zum erst, sonder die innere Interesse, wohin Sie reisen
möchten.

4. Auto-Freiheit

wenn das Preis der meisten Autos mit verschiedenen Marken innerhalb ihrer
finanzieller Fähigkeit liegen, die Auto nicht mehr nur ein Transportmittel,
sondern ein wesentlicher Bestandteil ihrer Lebensqualität.

5.Ausbildungsfreiheit

Ihre finanzielle Kraft kann garantieren, dass eine qualitativ hochwertige
öffentliche oder private Ausbildung für die nächste Generation bieten. Ob es
um den Kauf von Hause in Schulbezirken oder die Zahlung hoher Studiengebühren
geht, Es ist der Wunsch jedes Elternteils, um die gute Ausbildung der Kinder
zu sorgen.

6.Arbeitsfreiheit

Solange Sie dazu bereit sind, können Sie diesen Job oder den Job nach ihren
eigenen Interessen wählen, dran arbeiten oder nicht dran arbeiten, es ist
Ihnen egal, ob diese Arbeit viel verdient, aber Sie achten mehr auf den Spaß
und das Erfolgserlebnis, das die Arbeit mit sich bringt.

7.Medizinische Freiheit

Beim Wähl des Krankenhaus kommt die Qualität der medizinische Besorgung erst
sonder nicht das Kosten.

8.Wohnungsfreiheit

Sie kaufen das Haus in der Stadt,wo Sie wohnen, danach, ob das Haus ihnen am
besten gefallt ohne Berücksichtigung des Preis.

9.Staatsangehörigkeitsfreiheit

Sie können ihre Nationalität und ihren Lebensstil über verschiedene Kanäle
frei wählen, z. B. über die Zuwanderung von Investitionsgütern, und Sie sind
nicht mehr an die Landesgrenze gebunden. 

Die meisten Leute in China glauben an Materialismus. Nur die meisten Idealist
betrachten die menschlicher Selbstverwirklichung als die höchsten
Bedürfnis. Die finanzieller Bedürfnishierarchie passen die Chinesen
besser. Die meisten Leute liegen in der ersten Schichten in China und Hundert
von Tausend Chinesen hatte alle diese 9 Sichten beschafft. 

Das Volk in China liegen viele Wert auf das Erfinden, weil der Erfinder mit
ihrer Erfinden große Menge von Geld verdienen kann. Die gute und naive
Qualität der Erfinder mit dem Motto, dass ich will eben erfinden, weil ich das
für besser halte, als nur zu verwalten oder herumzusetzen, war schon langer
seit der Reform und Öffnens -politik in 1978 in China verschwunden.

\section{TRIZ auf dem Weg in die Wirtschaft}

\subsection{3.1   Training der TRIZ: sowohl theoretische als auch praktisch}

Für die internationale Zertifizierung der TRIZ daürt es bei 1 Level(13) 3
Tage, wo nur theoretische Erkenntnisse wie Funktionsanalyse,
Kausalkettenanalyse, Schneiderei, Funktion orientierte Suche, technischer
Widerspruch, physikalischer Widerspruch, Erfindungsprinzip kennengelernt und
diskutiert werden. Theoretische kann die Teilnehmer jede sein, aber praktische
sind die Teilnehmer Mitarbeiter,wie Ingenieur,Manager aus Firmen. Bei Training
für internationale Zertifizierung der Level 2(14),welche daürt insgesamt 8
Tage mit Kosten von 16 Tausend RMB per Teilnehmer in China, sollten
normalerweise praktische Übungen, die hauptsächlich auf die Bereich der
Technische und physikalische Widerspruch und der maßgebende Lösungen basiert,
dazwischen existieren. Die Teilnehmer für das Training der Level 3,welche 14
Tage mit Kosten von 40 Tausend RMB daürt , müssen schon die Zertifizierung
der Level 2 vervollständigen. In Level 3 sollten ARIZ in jeden Tag des
Training theoretische detailliert erklärt werden, Die theoretische Erklärung
werden direkt von einer praktische Übung verflogt. Die Teilnehmer für die
Level 4 brauchen die Empfehlung von TRIZ Master, welche in China erst seit
2018 1 existiert.

Das Training der TRIZ veredelt die Erkenntnisse des Ingenieur aus den Firmen,
mit den das Produkt aus dem Firma verbessert werden kann.
  
\subsection{3.2 Erwartungen von dem Trainer}

Die Teilnehmer sind nicht Jugendliche oder Studenten, Sie sind Ingenieur mit
mehr jährigen Erfahrungen bei der Arbeite.Wegen des hohen Kosten der Training
erwartet der Trainer von den Teilnehmers wesentlich die endlich Prüfung zu
bestehen, damit sie die Zertifizierung bekommen.

Während der Trainings werden allgemein die Teilnehmer mit gute Relationen zum
anderes und scharfem Sinn  hochwertig betrachtet.


\subsection{3.3 Das Training und die Promotion der Stelle}

Viele Leute in dem untere Management einer Firma nehmen die Training der TRIZ
teil. Nachdem Training haben sie einer großer Chance, eine Promotion der
Stelle zu haben. Es ist eine unausgesprochene Regel, nur die Leute, wer
potenzielle Fähigkeit der Leitungen haben, gewählt werden, eine weitere
Ausbildung  teilzunehmen. Wenn sie die Ausbildung erfolgreich fertig gemacht,
dann kommt normalerweise eine schnell Promotion der Stelle in der Firma.  

\section{Das TRIZ-Training in staatlichen Unternehmen}

\subsection{4.1 Innovationspolitik und die Training der TRIZ in den chinesische
staatlichen Unternehmen}

Die Meisten Leute denken, die staatliche Firma in China, haben die Fähigkeit,
etwas innovativ zu machen, weil die staatliche Firma haben die best Talenten
und die staatliche Finanzierung. Aber diese Firma machen wenige Innovation im
Vergleich zum privaten Firma, weil es zu viele Faktoren gibt, die Innovationen
behindern, wie zum Beispiel(16):

1. In der Zielerreichungsindikatoren der CEO der staatlichen Firma in China
sind Innovation nicht enthaltet. Da Innovation nicht direkt im Zusammenhang
mit der Einschätzung der Arbeite der CEO steht, fehlt die CEO Motivation zur
Innovation

2. Das erste versteckte Regel in der staatliche Unternehmen ist  das, dass
neü Geschäft  in dem ersten Jahr rentabel sein muss, sonst bedeutet die neü
Geschäft geeignet nicht zu dem Unternehmen.

3. Der auf Konsens basierende Entscheidungsmechanismus in staatlichen
Unternehmen ist langwierig und ungünstig für die Innovation.

4.  Für die staatlichen Unternehmen ist es leichter durch die politische
Faktor eine bessere Zielerreichungsindikatoren als durch Innovation.

5. Die Zeit, die Innovation braucht ,um Geld zu  verdienen , ist  mit großer
Möglichkeit als die Amtsdaür der CEO.

Durch der Kooperation der ausländischen Unternehmen haben sich die staatliche
Unternehmen technisch in der vergangenen 20 Jahren gerettet.  Die
Kooperationen passierten hauptsächlich in Branch wie Energie, Eisenbahn und
Energieübertragung. In diesen Bereichen haben China große Markt und die
ausländische Firma die best Technik.

Bei der Training der TRIZ steht auf jeden Fall die staatlich Unternehmen auf
der erste Linie. Viele Ingenieur und Manager in der staatlichen Unternehmen
haben schon bis jetzt das Training der TRIZ. Die Teilnehmer der Trainings,
welche von Hebei Industrie Uni ausgeführt werden, sind meisten aus staatlichen
Unternehmen. 

\subsection{4.2  Die Hauptproblem der Verbreitungen der TRIZ in der realen
  Wirtschaft in China}

\subsubsection{4.2.1 Die Probleme von TRIZ selbst (17)}

1. Wegen zu viele parallelen Tools ist es bei Auswähle der Tools, sehr leicht
vermischt zu werden.

In der Problemlösungsprozess der TRIZ gibt es viele Tools, mit denen ein
bestimmte Problem nach einem allgemeinen Problem umgewandelt werden
kann. Diese Werkzeug sind parallel und haben keine spezifische Beschreibung,
unter welche Falle sollten welche Tools verwendet werden. z.B bei der
Umwandelung der Problem von einem spezifischen Problem nach einem
Standardproblem muss das Problem verstanden und beschrieben werden. Dafür sind
verfügbare Werkzeugen die Neun-Bildschirm-Methode, die endgültige ideale
Lösung, die Funktionsanalyse-Methode, das Zuschneiden, die S-Kurve, die
Mind-Setting-Methode usw. Für die Umwandelung der Problem sind die verfügbaren
Werkzeugen Erfindungsprinzipien und WiderspruchsmaTRIZen, Trennungsprinzipien,
Standardlösungen, Ressourcenanalyse und technologische Revolutionstrends. In
der beiden Teilprozesse sind die Tools parallel und verwendbar. Und die bei
der Verstehen und Beschreiben des Problem verwendeten Methoden bestimmen auch
nicht, welche Methode sollen bei Umwandelung des Problem verwendet muss.

2. die allgemeine Löungen sind  zu abstrakt

Bevor wir die spezifische Lösung für das Problem finden können, müssen wir
bei Verwendung der TRIZ erst die allgemeine Lösung finden, aber die mit der
verschiedenen Tools gefundene allgemeine Lösungen sind zu abstrakt, Die Lösung
ist ein Hinweise der Problem, sonder nicht die direkt Antworten zum Problem.

3. Problemlösungsprozess hängt weitgehend von der fachlichen Erkenntnisse der
Anwender ab.

Nicht nur bei Definition und Beschreiben des Problem sondern auch bei Erhalten
einer konkreten Lösungen ist das Problem Lösbar, hängig weitgehend von der
Hintergrundwissen des das Problem lösendes Person  ab, welche die fachlichen
und implizierten Wissen und Erfahrungen enthaltet.

4. Irtum-Versuch werden benötigt bei Definition der Problem und anderem
Teilprozess

Beschreiben und Definition des Problem ist sehr wichtig. Je klar der
Beschreiben und Definition ist, desto leichter ist es, die Lösung zu
finden. In dem praktischen Prozess der TRIZ ist es fast unmöglich das
geeignete Tools zu finden ohne spezifische Anweisungen der Tools, deshalb ist
die Schleife der Irrtum und Versuch unvermeidbar.


\subsubsection{4.2.2  Die spezifischen Probleme bei Verbreitung der TRIZ in
  China.}

1. Die Verbreitungen der TRIZ ist von oben betriebt, sondern nicht von Firma
selbst betribt.

Die Firmen, welche das Training der TRIZ teilgenommen haben, haben nicht eine
schwer Bedürfnisse nach TRIZ, sonder sie von andere Sachen wie finanzielle
Unterstützung der Regierung in das Training anziehen.

2. TRIZ Software ist teür.

Um die Software der TRIZ zu verwenden, wird viel Kosten braucht, sodass das
Unternehmen nicht dazu breit ist, so viel Geld in eine noch nicht unbekannte
Methode zu investieren.

3. Das heutige  Ausbildungssystem in China ist isoliert von der Methode der
TRIZ.

Chinas indoktrinierte und einprägsame Lehrmethoden haben die Entwicklung von
innovativem Denken in gewissem Maße gebremst. Der Materialismus bei Ausbildung
macht die Studenten in China in einige Maße nur weißen, was in der Buch als
Erkenntnisse sind, aber nicht wissen, wie sollen man denken, damit das
Wissenschaft erzeugt werden.

\subsection{4.3 Sozialistisches Sozialbewußstsein und sein Einfluss auf die
  Erfindertätigkeit} 

Die Merkmale der Sozialismus ist heutigen China sehr schwach. Auf der
chinesische Marktwirtschaft gibt es fast keine Sozialbewußstsein und ihre
Einfluss ist null. Das Demokratiekapitalismus haben eine große Einfluß auf die
Erfindertätigkeit. USA nennt das Modelle der chinesische Markt als
Nationalkapitalismus. Das ist ganz ungenau, diese Meinung haben sich nur auf
ein Punkte fokussieren und zugleich die ganz Ebene ignoriert. Die Ziele der
erfinderischen Tätigkeiten ist eine Besserung Leben zu haben. z.B WeChat Pay
erzielt sich auf Vereinfachung des Zahlen. Eine Nebenwirkungen ist, bei dem
Zahlen wird viele persönliche Information automatisch gespeichert. Die
Nebenwirkungen ist veränderbar und die Hauptwirkung, welche das Ziel ist,
sollen bei diesen Sachen nicht änderbar sein. Wenn man die Hauptwirkung und
die Nebenwirkung gemischt, dann wird man sagen, dass diese Erfinder ist
hauptsächlich für die Sammlungen der Daten geboren.

\section{Literatur}
\begin{itemize}
\item 0. Erfinderschuler in DDR
\item 1. Forschung und Verbreitung in Mainland China:Problem und Zustand
\url{https://www.osaka-gu.ac.jp/php/nakagawa/TRIZ/eTRIZ/eforum/e2011Forum/eFromReaders2011/cChen-TRIZinChina2009-111030.pdf}
\item 2. innovative Methoden in modernen Verwaltungssystem (Zukunft und Dynamik)
\url{https://wenku.baidu.com/view/6ff603e1f56527d3240c844769eae009581ba2ff.html}
\item 3 Firma Iwint
\url{http://www.iwintall.com/iplat-cms/portal/aboutUs/expertTeam}
\item 4.C-TRIZ 
\url{https://www.osaka-gu.ac.jp/php/nakagawa/TRIZ/eTRIZ/epapers/e2017Papers/eTan-China-Dissemination/eTan-TRIZCON2017-Paper-171012.pdf}
\item 5 Struktur und Funktionen der Innovator
\url{http://www.iwintall.com/iplat-cms/portal/enterpriseService/productImprovement}
\item 6    Verbreitungen und Beeinflüss der TRIZ in China
\url{https://wenku.baidu.com/view/48e497a0852458fb760b561b.html}
\item 7 Nationales Forschungszentrum für technologische Innovation Methode und
  Werkzeug 
\url{http://TRIZ.hebut.edu.cn/zxgk/zxjg/index.htm}
\item 8 Bekanntmachung über die Organisation von Pilotunternehmen für
  technologische Innovation (TRIZ-Theorie) 
\url{http://219.147.168.134/html/zwgk/TZTG/xztz/show-4536.html}
\item 9 Wörter von Stellvertretender Direktor der Abteilung für Wissenschaft
  und Technologie der Provinz HeiLongJiang  
\url{http://www.TRIZ.gov.cn/index.php?s=/home/index/newsdetail/id/930.html}
\item 10. Gesetz der Patent Chinas
\url{http://m.liuxiaör.com/qz/2755.html}
\item 11. Überprufungen der Markt ist zu subjektiv bei Patentamt
\url{http://www.gov.cn/hudong/2018-04/26/content_5286093.htm}
\item 12 Maslowsche Bedürfnishierarchie
\url{https://de.wikipedia.org/wiki/Maslowsche_Bed%C3%BCrfnishierarchie}
\item 13  beispiel der Training der TRIZ Level 1
\url{https://baijiahao.baidu.com/s?id=1624057263861494837&wfr=spider&for=pc}
\item 14  Skizze der Training der TRIZ Level 2
\url{http://blog.sina.com.cn/s/blog_a651266e01018pcc.html}
\item 15 Skizze der Training der TRIZ Level 3
\url{https://wenku.baidu.com/view/570e78e065ce0508763213cb.html}
\item 16 Innovativ oder nicht in chinesische staatlich Firma
\url{https://www.mckinsey.com.cn/%E4%B8%AD%E5%9B%BD%E5%9B%BD%E6%9C%89%E4%BC%81%E4%B8%9A%E6%98%AF%E5%90%A6%E5%9C%A8%E5%88%9B%E6%96%B0%EF%BC%9F/}
\item 17 Problem und Lösung der Verbreitung innovativer Method TRIZ
\url{http://articles.e-works.net.cn/CAI/Article83271_1.htm}
\end{itemize}

\end{document}
