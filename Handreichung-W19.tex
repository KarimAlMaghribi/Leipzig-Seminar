\documentclass[11pt,a4paper]{article}
\usepackage{a4wide,url}
\usepackage[utf8]{inputenc}
\usepackage[german]{babel}

\parindent0pt
\parskip4pt
\title{Handreichung für das Seminar „Systemwissenschaft“}

\author{Hans-Gert Gr\"abe, Ken Pierre Kleemann, Lydie Laforet, Sabine
  Lautenschläger}

\date{12. Oktober 2019}

\begin{document}
\maketitle

\section{Ziel und Methodik des Seminars}

Der Systembegriff spielt in der Informatik eine herausragende Rolle, wenn es
um Datenbanksysteme, Softwaresysteme\footnote{So die neue Denomination der im
  Besetzungsverfahren befindlichen W3-Professur am Institut für Informatik.},
Hardwaresysteme, Abrechnungssysteme, Zugangssysteme usw. geht.  Überhaupt wird
die Informatik von einer Merhheit als die „Wissenschaft von der
\emph{systematischen} Darstellung, Speicherung, Verarbeitung und Übertragung
von Informationen, besonders der automatischen Verarbeitung mithilfe von
Digitalrechnern“ (Wikipedia) verstanden.  Auch gewisse einschlägige
Professionen wie etwa der \emph{Systemarchitekt} genießen unter IT-Anwendern
hohe Wertschätzung.

Die Bedeutung des Systembegriffs reicht allerdings weit über den Bereich der
Informatik hinaus -- er ist grundlegend für alle Ingenieurwissenschaften und
als \emph{Systems Engineering} mit der ISO/IEC/IEEE-15288 Norm „Systems and
Software Engineering“ auch Gegenstand internationaler Normierungs- und
Standardisierungsprozesse.  Mehr noch spielt der Systembegriff auch bei der
Beschreibung komplexer natürlicher und kultureller Prozesse -- etwa im Begriff
des \emph{Ökosystems} -- eine zentrale Rolle.

Mit dem \emph{Semantic Web} rückt die Bedeutungsanalyse digitaler Artefakte in
den Mittelpunkt, die in letzter Instanz Sprachartefakte sind und damit
ebenfalls in direktem Zusammenhang zu einem sinnvoll zu entfaltenden
\emph{Systembegriff} stehen als Grundlage jeden Verständnisses konkreter
Systeme.

Mit dem Schlagwort \emph{Nachhaltigkeit} werden schließlich komplexe
gesellschaftliche Abstimmungsprozesse angesprochen, mit denen vielfältige
Informations- und Bewertungsprobleme einhergehen. Hierbei ist die Fähigkeit
der beschreibenden Abgrenzung, Entwicklung und Steuerung von sogenannten
Systemen auf bzw. über verschiedene Governance-, Raum- und Zeitebenen hinweg
von großer Bedeutung.

\begin{quote}
  \textbf{Ziel des Seminars} ist es, ein besseres Verständnis für diese
  Vielfalt von Systembegriffen zu gewinnen und dabei die Zugänge
  \emph{verschiedener Systemtheorien} als Gegenstand einer
  \emph{Systemwissenschaft} zu analysieren.
\end{quote}
Das Seminar ist ein Einführungskurs in die Systemwissenschaft auf
Master-Ebene, ihre Entwicklung im Laufe der Zeit, Verzweigung von Ansätzen,
Schlüsselbegriffen und Konzepten.  \emph{Systemwissenschaft} wird hier als
übergeordneter Ausdruck für ein Feld verwendet, zu dem zahlreiche Gelehrte aus
den verschiedensten Disziplinen wie Anthropologie, Biologie, Chemie, Ökologie,
Ökonomie, Mathematik, Physik, Psychologie, Soziologie und andere beigetragen
haben. Entwicklungen wie Kybernetik, Chaostheorie oder Netzwerkanalyse und
-wissenschaft können als Teil von Systemwissenschaft oder zumindest stark
verwandt mit ihr angesehen werden.  Einige Zweige der Systemwissenschaft
gelten in Deutschland sogar als neue Wissenschaftsbereiche mit eigenen Rechten
wie Synergetik oder Komplexitätswissenschaft.

Diese Entwicklungen haben neue Möglichkeiten für eine verbesserte Analyse und
Entscheidungsfindung in wissenschaftlichen, geschäftlichen und politischen
Bereichen eröffnet. Wir stellen jedoch täglich fest, dass in komplizierten
Situationen, insbesondere in der Politik und in der Wirtschaft, einfache und
direkte Entscheidungsfindungsprozesse nach wie vor überwiegen, was zu einer
Zunahme negativer Entwicklungen führt, wenn die ursprünglich beabsichtigten
Wirkungen nicht eintreten. Jede unerwartete Nebenwirkung oder Gegenreaktion,
die die Maßnahmen unbrauchbar machen, sind ein klares Indiz dafür, dass die
grundlegenden mentalen Modelle der Akteure unvollständig waren und breitere
systemische Korrelationen vernachlässigt wurden. Das Systemdenken ist daher in
Deutschland von besonderer Bedeutung für den Übergang zu einer nachhaltigeren
Gesellschaft.

In diesem Seminar soll die historische Entwicklung der Systemwissenschaft (in
Teilen) verfolgt sowie relevante Grundbegriffe studiert werden. Kursteilnehmer
halten sich dabei an kein spezifisches Modell (wie z.B. \emph{Systemdynamik}),
sondern entwickeln ein tieferes Verständnis für die Systemwissenschaft und für
eine spezifische Art des „Systemdenkens“, mit der Nachhaltigkeitsprobleme
erfolgreicher angegangen werden können. Dies erreichen wir durch das Lesen und
Diskutieren von wissenschaftlichen Arbeiten und Buchkapiteln.

Von den Studierenden wird erwartet, dass sie sich aktiv am Seminar beteiligen
durch Seminardiskussionen, Präsentationen, schriftliche Ausarbeitungen und
nicht zuletzt durch Lesen. Die Kursteilnehmer werden angeregt und
aufgefordert, einen eigenen Zugang zum Thema Nachhaltigkeit zu entwickeln.

\section{Kursstruktur}

Der Kurs wird wöchentlich nach dem im Uni-Moodle veröffentlichten Plan in
einem Präsenta"|tions- und Diskussionsformat abgehalten.  Jede Woche müssen
die Seminarteilnehmer die zugewiesene Lektüre vorab studieren und bereit sein,
diese im Seminar zu diskutieren. Nach kurzen Eingaben der Seminarleitung ist
die oder der als \emph{Diskussionsleiter} eingeteilte Studierende
verantwortlich für die Moderation der Seminardiskussion. In Vorbereitung des
Seminars schreiben \emph{Opponenten} kurze Standpunktpapiere.  Von den
Seminarteilnehmern wird also erwartet, dass sie jede Woche entweder das
Seminar leiten oder aktiv an der Diskussion teilnehmen.

Für Wochen, in denen Sie \emph{das Seminar leiten}, müssen Sie
\begin{enumerate}
\item die für die Sitzung angegebene Lektüre komplett studiert haben,
\item eine Präsentation vorbereiten und diese in ca. 20 Minuten als mündliche
  Zusammenfassung der Schwerpunkte der Sitzung präsentieren sowie
\item Diskussionsfragen vorbereiten, um eine Gruppendiskussion für ca. 30
  Minuten zu führen.
\end{enumerate}
Für Wochen, in denen Sie \emph{als Opponent} eingeteilt sind, müssen Sie
\begin{enumerate}
\item die für die Sitzung angegebene Lektüre komplett studiert haben, 
\item ein Standpunktpapier von ca. 800 Wörter schreiben (pdf, 11pt, einzeilig)
  und dieses bis Sonntag Mitternacht vor dem Seminartermin in den
  Materialordner des Seminars im Uni-Moodle hochladen.
\end{enumerate}

Die Präsentationen und die Standpunktpapiere werden (später) ins github
hochgeladen und damit veröffentlicht.


\section{Standpunktpapiere}

Jeder Studierende schreibt drei Standpunktpapiere und leitet eine
Seminardiskussion.  Zur Diskussion, die Sie selbst leiten, können Sie nicht
auch noch ein Standpunktpapier schreiben. Die Standpunktpapiere sollen klaren
Bezug auf das Lesematerial des jeweiligen Seminars nehmen. In den
Standpunktpapieren fassen die Studierenden die wichtigsten Punkte des
Lesematerials kurz zusammen und ergänzen dies um ihren eigenen Input. Diese
Standpunkte können verschiedene Formen annehmen. Sie können
\begin{itemize}
\item den jeweiligen theoretischen Aspekt der Systemwissenschaft mit Fragen
  der nachhaltigen Entwicklung im Großen verbinden,
\item den Ansatz des Autors mit dem eines anderen Autors vergleichen oder
  gegenüberstellen, der im Seminar bereits besprochen wurde oder
\item einen Kommentar zum Anwendungsbereich des Seminarthemas geben.
\end{itemize}
Die Erfüllung dieser Leistungen wird nicht bewertet\footnote{Wir möchten mit
  einem solchen Konzept den akademischen Charakter unseres Vorhabens
  unterstreichen -- es geht nicht darum, Ihre Leistung zu bewerten, sondern um
  \emph{gemeinsamen} Erkenntnisgewinn auf Augenhöhe.}, ist aber als
Prüfungsvorleistung neben dem erfolgreichen Abschluss des Praktikums
Zulassungsvoraussetzung zur Klausur, mit der das Modul abgeschlossen wird.

\section{Literatur}

\begin{itemize}
\item Anderies, John M., Marco A. Janssen, Elinor Ostrom (2004).  Framework to
  Analyze the Robustness of Social-ecological Systems from an Institutional
  Perspective. In: Ecology and Society 9 (1), p. 18.
\item Ashby W.R. (1958).  Requisite variety and its implications for the
  control of complex systems. In: Cybernetica 1:2, 83--99.\\
  \url{http://pcp.vub.ac.be/Books/AshbyReqVar.pdf}
\item Bertalanffy, Ludwig von (1950). An outline of General System Theory,
  The British Journal for the Philosophy of Science, Volume I, Issue 2, 1
  August 1950, 134–165.\\ \url{https://doi.org/10.1093/bjps/I.2.134}
\item Binder, C.R., J. Hinkel, P.W. Bots, C. Pahl-Wostl (2013). Comparison of
  Frameworks for Analyzing Social-ecological Systems. Ecology and Society,
  18(4), 26.
\item Boisot, Max and Bill McKelvey (2011). Complexity and
  Organization-Environment Relations: Revisiting Ashby’s Law of Requisite
  Variety. In: Allen, Peter, Steve Maguire and Bill McKelvey (eds.). The
  Sage Handbook of Complexity and Management, 279--298.
\item Brand, Fridolin Simon and Kurt Jax (2007).  Focusing the Meaning(s) of
  Resilience: Resilience as a Descriptive Concept and a Boundary Object. In:
  Ecology and Society 12 (1), p. 23.
\item Foxon, T.J., M.S. Reed, L.C. Stringer (2009). Governing long‐term
  social–ecological change: what can the adaptive management and transition
  management approaches learn from each other? Environmental Policy and
  Governance, 19 (1), 3--20.
\item Geels, Frank W., Johan Schot (2007). Typology of Sociotechnical
  Transition Pathways. In: Research Policy 36 (2007), 399–417.
\item Holland, John H. (2006). Studying complex adaptive systems. In: Journal
  of systems science and complexity, 19 (1), 1–8.
\item Holling, C.S. (2000). Understanding the Complexity of Economic,
  Ecological, and Social Systems. In: Ecosystems (2001) 4, 390–405.
\item Jantsch, Erich (1992). Die Selbstorganisation des Universums. Vom
  Urknall zum menschlichen Geist.  Hanser, München.
\item Jooß, Christian (2017). Selbstorganisation der Materie.  Verlag Neuer
  Weg, Essen.
\item Mele, C., J. Pels, F. Polese (2010). A brief review of systems theories
  and their managerial applications. Service Science, 2(1--2), 126--135.
\item Mingers, John (1989). An Introduction to Autopoiesis, Implications and
  Applications. In: Systems Practice, Vol. 2, No. 2, 1989. 
\item Ostrom, E. (2007). A diagnostic approach for going beyond panaceas.
  Proceedings of the national Academy of sciences, 104(39), 15181--15187.
\item Prigogine, Ilya, Isabelle Stengers (1993). Das Pardox der Zeit. Piper,
  München, Kap. 3--5.  
\item Ropohl, G. (2009). Allgemeine Technologie: eine Systemtheorie der
  Technik.  KIT Scientific Publishing.
\item Ulanowicz, Robert E. (2009). The dual nature of ecosystem dynamics.
  In: Ecological Modelling 220 (2009), 1886–1892. 
\item Walker, Brian, C. S. Holling, Stephen R. Carpenter, Ann Kinzig (2004).
  Resilience, Adaptability and Transformability in Social-ecological Systems. 
  In: Ecology and Society 9 (2).
\end{itemize}

\end{document}

\section{Zusatzliteratur}

Allgemeine Fachliteratur zum Thema Systemwissenschaft (x markiert leichte
Lesbarkeit).

Ggf. noch zusammenzustreichen (Laforet/Lautenschläger).

\begin{itemize}
\item Arthur, W. Brian (2009). The Nature of Technology. (x)
\item Arthur, Brian (2013). Complexity Economics: A Different Framework for
  Economic Thought. SFI working paper: 2013-04-012. 
\item Ashby, Ross (1956/2015). An Introduction to Cybernetics. 
\item Bateson, Gregory (1972/1987/2000). Ecology and Flexibility in Urban
  civilization. In: Bateson, Gregory. Steps to an Ecology of Mind.
\item Bednar, Jenna (2016). Robust Institutional Design – What Makes Some
  Institutions More Adaptable and Resilient to Changes in Their Environment
  Than Others? In: Wilson, David S., Alan Kirman (Eds.).  Complexity and
  Evolution -- Toward a New Synthesis in Economics, pp. 167--184, Strüngmann
  Forum Reports, MIT Press.
\item Beinhocker, Eric (2006/2007). The Origin of Wealth: Evolution,
  Complexity, And the Radical Remaking of Economics.  (x)
\item Bengtsson, Janne, Per Angelstam, Thomas Elmqvist, Urban Emanuelsson,
  Carl Folke, Margareta Ihse, Fredrik Moberg, Magnus Nyström (2003).
  Reserves, Resilience and Dynamic Landscapes. In: AMBIO: A Journal of the
  Human Environment, 32 (6), pp. 389--396.  
\item Bertalanffy, Ludwig von (1969/2006). General Systems Theory. 
\item Braitenberg, Valentino (1984). Vehicles – Experiments in Synthetic
  Psychology.  (x)
\item Cilliers, Paul (2001). Boundaries, Hierarchies and Networks in Complex
  Systems. In: International Journal of Innovation Management, Vol. 5, No. 2
  (June 2001), pp. 135–147.
\item Colander, David, Roland Kupers (2014). Complexity and the Art of Public
  Policy.  (x)
\item Erdi, Peter (2010). Complexity Explained. 
\item Fath, Brian D. (2017). Systems Ecology, Energy Networks, and a Path to
  Sustainability. In: Int. J. of Design \& Nature and Ecodynamics. Vol. 12,
  No. 1 (2017), pp. 1–15. 
\item Frischmann, Brett M. (2013). Two Enduring Lessons from Elinor Ostrom. In:
  Journal of Institutional Economics, 9, pp. 387--406.
\item Gowdy, John, Mariana Mazzucato, Jeroen C.J.M. van den Bergh, Sander E.
  van der Leeuw, David S. Wilson (2016). In: Wilson, David S., Alan Kirman
  (Eds.). Complexity and Evolution -- Toward a New Synthesis in Economics,
  pp. 327--350, Strüngmann Forum Reports, MIT press.
\item Gunderson, Lance H. (2000). Ecological Resilience -- In Theory and
  Application. In: Annual Review of Ecology and Systematics, Vol. 31 (2000),
  pp. 425--439. 
\item Hartmann, Dominik, Cristian Jara-Figueroa, Miguel Guevara, Alex Simoes,
  César A. Hidalgo (2016). The Structural Constraints of Income Inequality in
  Latin America. In: Integration \& Trade Journal, No. 40, June 2016,
  pp. 70--85. 
\item Hausmann et al. Atlas of Economic Complexity. (New version on sale at
  MIT press, free download of older versions) (x)
\item Heylighen, Francis (2008). Complexity and Self-organization. In: Marcia
  J. Bates and Mary Niles Maack (eds.). Encyclopedia of Library and
  Information Sciences.
\item Kauffman, Stuart A (1993). The Origins of Order. 
\item Kauffman, Stuart A (1996). At Home in the Universe. (x)
\item Kauffman, Stuart A (2008/2010). Reinventing the Sacred. 
\item Kharrazi, Ali, Elena Rovenskaya, Brian D. Fath, Masaru Yarime, Steven
  Kraines (2013). Quantifying the sustainability of economic resource
  networks: An ecological information-based approach. In: Ecological Economics
  90 (2013), pp. 177–186.
\item Latour, Bruno (1996). On actor-network theory: A few clarifications. In:
  Soziale Welt, 47. Jahrg., H. 4 (1996), pp. 369--381.\\
  \url{http://www.jstor.org/stable/40878163}
\item Lichtenstein, B., Bill McKelvey (2011). Four types of emergence: a
  typology of complexity and its implications for a science of management. In:
  Int. J. Complexity in Leadership and Management, Vol. 1, No. 4, 2011.
\item Maturana, Humberto (1975). The Organization of the Living: A Theory of
  the Living Organization. In: Int. J. Man-Machine Studies (1975) 7, 313--332.
\item McKelvey, Bill (2001). Energising Order-Creating Networks of Distributed
  Intelligence Improving the Corporate Brain. In: International Journal of
  Innovation Management, Vol. 5, No. 2 (June 2001), pp. 181–212.
\item Meadows, Donella H. (2008). Thinking in Systems. (x)
\item Mitchell, Melanie (2009). Complexity – A Guided Tour. (x)
\item Morin, Edgar (2008). On Complexity. 
\item Noe, Egon, Hugo Fjelsted Alrøe (2003). Combining Luhmann and
  Actor-Network Theory to see Farm Enterprises as Self-organizing Systems.
  Paper presented at \emph{The Opening of Systems Theory} in Copenhagen, May
  23--25, 2003.
\item Ostrom, Elinor (2009). A General Framework for Analyzing Sustainability
  of Social-Ecological Systems. In: Science 325, 419 (2009). 
\item Page, Scott (2011). Diversity and Complexity. (x)
\item Seidl, David (2004). Luhmann’s theory of autopoietic social systems.
\item Senge, Peter M. (1990/2006). The Fifth Discipline. (x)
\item Sterman, John (2000). Business Dynamics: Systems Thinking and Modeling
  for a Complex World.  (x)
\item Ulanowicz, Robert E. (2007).  Ecosystems becoming. In: Int. Journal of
  Ecodynamics. Vol 2, No. 3 (2007), pp. 153--164 
\item Ulanowicz, Robert E. (2009). A Third Window: Natural Life beyond Newton
  and Darwin.  (x)
\item van der Leeuw, Sander E. (2016). Adaptation and Maladaptation in the
  Past. In: Wilson, David S., Alan Kirman (Eds.) Complexity and Evolution --
  Toward a New Synthesis in Economics, pp. 239--269, Strüngmann Forum Reports,
  MIT Press.
\item Wilson, David S. (2016). Two Meanings of Complex Adaptive Systems. In:
  Wilson, David S., Alan Kirman (Eds.).  Complexity and Evolution -- Toward a
  New Synthesis in Economics, pp. 31–46, Strüngmann Forum Reports, MIT Press.
\item Wilson, David S., Alan Kirman (Editors, 2016). Complexity and Evolution:
  Toward a New Synthesis for Economics. (Strüngmann Forum Reports).  (x)\\
  \url{https://www.esforum.de/publications/sfr19/Complexity and Evolution.html}
\item Woermann, Minka (n.y.). What is Complexity Theory? Features and
  Implications. 
\end{itemize}

\end{document}
\section{Seminarplan}

\paragraph{16. Okt.}
Einführung: Organisation, Methodik, Ziel und Thema
\begin{itemize}
\item Bateson, Gregory (1972/1987/2000). Ecology and Flexibility in Urban
  civilization. In: Bateson, Gregory. Steps to an Ecology of Mind.
  pp. 499--511 or 502–513. Im Internet verfügbar.
\end{itemize}

\paragraph{23. Okt.}
Allgemeine Systemtheorie
\begin{itemize}
\item Bertalanffy, Ludwig von (1950). An outline of General System Theory,
  The British Journal for the Philosophy of Science, Volume I, Issue 2, 1
  August 1950, Pages 134–165. \url{https://doi.org/10.1093/bjps/I.2.134}
\end{itemize}

\paragraph{30. Okt.}
Autopoiesis
\begin{itemize}
\item Maturana, Humberto (1975). The Organization of the Living: A Theory of
  the Living Organization. In: Int. J. Man-Machine Studies (1975) 7, 313-332.
\item Mingers, John (1989). An Introduction to Autopoiesis, Implications and
  Applications. In: Systems Practice, Vol. 2, No. 2, 1989. 
\item Seidl, David (2004). Luhmann’s theory of autopoietic social systems.
\end{itemize}

\paragraph{06. Nov.}
Autokatalyse, Kooperation und Wettbewerb
\begin{itemize}
\item Ulanowicz, Robert E. (2009). The dual nature of ecosystem dynamics.
  In: Ecological Modelling 220 (2009), pp. 1886–1892. 
\item Ulanowicz, Robert E. (2007).  Ecosystems becoming. In: Int. Journal of
  Ecodynamics. Vol 2, No. 3 (2007), pp. 153-164 
\item Latour, Bruno (1996). On actor-network theory: A few clarifications. In:
  Soziale Welt, 47. Jahrg., H. 4 (1996), pp. 369--381.
  \url{http://www.jstor.org/stable/40878163}
\end{itemize}

\paragraph{13. Nov.}
Resilienz
\begin{itemize}
\item Holling, C.S. (2000). Understanding the Complexity of Economic,
  Ecological, and Social Systems. In: Ecosystems (2001) 4, pp. 390–405. 
\item Walker, Brian, C. S. Holling, Stephen R. Carpenter, Ann Kinzig (2004).
  Resilience, Adaptability and Transformability in Social-ecological Systems. 
  In: Ecology and Society 9 (2).
\item Gunderson, Lance H. (2000). Ecological Resilience -- In Theory and
  Application. In: Annual Review of Ecology and Systematics, Vol. 31 (2000),
  pp. 425--439. 
\item Bengtsson, Janne, Per Angelstam, Thomas Elmqvist, Urban Emanuelsson,
  Carl Folke, Margareta Ihse, Fredrik Moberg, Magnus Nyström (2003).
  Reserves, Resilience and Dynamic Landscapes. In: AMBIO: A Journal of the
  Human Environment, 32 (6), pp. 389--396.  
\item Brand, Fridolin Simon and Kurt Jax (2007).  Focusing the Meaning(s) of
  Resilience: Resilience as a Descriptive Concept and a Boundary Object. In:
  Ecology and Society 12 (1), p. 23.
\end{itemize}
  
\paragraph{20. Nov.}
Vielfalt und Diversität
\begin{itemize}
\item Boisot, Max and Bill McKelvey (2011). Complexity and
  Organization-Environment Relations: Revisiting Ashby’s Law of Requisite
  Variety. In: Allen, Peter, Steve Maguire and Bill McKelvey (eds.). The
  Sage Handbook of Complexity and Management, pp. 279--298.
\item Ashby W.R. (1958).  Requisite variety and its implications for the
  control of complex systems. In: Cybernetica 1:2, p. 83--99.
  \url{http://pcp.vub.ac.be/Books/AshbyReqVar.pdf}
\end{itemize}

\paragraph{27. Nov.}
Komplexität
\begin{itemize}
\item Cilliers, Paul (2001). Boundaries, Hierarchies and Networks in Complex
  Systems. In: International Journal of Innovation Management, Vol. 5, No. 2
  (June 2001), pp. 135–147.
\item Woermann, Minka (n.y.). What is Complexity Theory? Features and
  Implications. 
\item Holland, John H. (2006). Studying Complex Adaptive Systems. In: Jrl Syst
  Sci \& Complexity (2006) 19, pp. 1–8.
\item Heylighen, Francis (2008). Complexity and Self-organization. In: Marcia
  J. Bates and Mary Niles Maack (eds.). Encyclopedia of Library and
  Information Sciences.
\end{itemize}

  
\paragraph{04. Dez.}
Anwendungen im Businessbereich
\begin{itemize}
\item McKelvey, Bill (2001). Energising Order-Creating Networks of Distributed
  Intelligence Improving the Corporate Brain. In: International Journal of
  Innovation Management, Vol. 5, No. 2 (June 2001), pp. 181–212.
\item Lichtenstein, B., Bill McKelvey (2011). Four types of emergence: a
  typology of complexity and its implications for a science of management. In:
  Int. J. Complexity in Leadership and Management, Vol. 1, No. 4, 2011.
\item Noe, Egon, Hugo Fjelsted Alrøe (2003). Combining Luhmann and
  Actor-Network Theory to see Farm Enterprises as Self-organizing Systems.
  Paper presented at \emph{The Opening of Systems Theory} in Copenhagen, May
  23--25, 2003.
\end{itemize}

\paragraph{11. Dez.}
Anwendungen in der Wirtschaft
\begin{itemize}
\item Arthur, Brian (2013). Complexity Economics: A Different Framework for
  Economic Thought. SFI working paper: 2013-04-012. 
\item Kharrazi, Ali, Elena Rovenskaya, Brian D. Fath, Masaru Yarime, Steven
  Kraines (2013). Quantifying the sustainability of economic resource
  networks: An ecological information-based approach. In: Ecological Economics
  90 (2013), pp- 177–186.
\item Hartmann, Dominik, Cristian Jara-Figueroa, Miguel Guevara, Alex Simoes,
  César A. Hidalgo (2016). The Structural Constraints of Income Inequality in
  Latin America. In: Integration \& Trade Journal, No. 40, June 2016,
  pp. 70--85. 
\item Wilson, David S. (2016). Two Meanings of Complex Adaptive Systems. In:
  Wilson, David S., Alan Kirman (Eds.).  Complexity and Evolution -- Toward a
  New Synthesis in Economics, pp. 31–46, Strüngmann Forum Reports, MIT Press.
  \url{https://www.esforum.de/publications/sfr19/Complexity and Evolution.html}
\end{itemize}
  
\paragraph{18. Dez.}
Anwendungen auf gesellschaftliche Transformationsprozesse
\begin{itemize}
\item Geels, Frank W., Johan Schot (2007). Typology of Sociotechnical
  Transition Pathways. In: Research Policy 36 (2007), pp. 399–417.
\item Ostrom, Elinor (2009). A General Framework for Analyzing Sustainability
  of Social-Ecological Systems. In: Science 325, 419 (2009). 
\item Frischmann, Brett M. (2013). Two Enduring Lessons from Elinor Ostrom. In:
  Journal of Institutional Economics, 9, pp. 387--406.
\end{itemize}

\paragraph{08. Jan.}
Anwendungen auf gesellschaftliche Transformationsprozesse
\begin{itemize}
\item Anderies, John M., Marco A. Janssen, and Elinor Ostrom (2004).
  Framework to Analyze the Robustness of Social-ecological Systems from an
  Institutional Perspective. In: Ecology and Society 9 (1), p. 18.
\item van der Leeuw, Sander E. (2016). Adaptation and Maladaptation in the
  Past. In: Wilson, David S., Alan Kirman (Eds.) Complexity and Evolution --
  Toward a New Synthesis in Economics, pp. 239--269, Strüngmann Forum Reports,
  MIT Press.
\item Bednar, Jenna (2016). Robust Institutional Design – What Makes Some
  Institutions More Adaptable and Resilient to Changes in Their Environment
  Than Others? In: Wilson, David S., Alan Kirman (Eds.).  Complexity and
  Evolution -- Toward a New Synthesis in Economics, pp. 167--184, Strüngmann
  Forum Reports, MIT Press.
\end{itemize}
  
\paragraph{15. Jan.}
Anwendungen auf gesellschaftliche Transformationsprozesse
\begin{itemize}
\item Fath, Brian D. (2017). Systems Ecology, Energy Networks, and a Path to
  Sustainability. In: Int. J. of Design \& Nature and Ecodynamics. Vol. 12,
  No. 1 (2017), pp. 1–15. 
\item Gowdy, John, Mariana Mazzucato, Jeroen C.J.M. van den Bergh, Sander E,
  van der Leeuw, David S. Wilson (2016). In: Wilson, David S., Alan Kirman
  (Eds.). Complexity and Evolution -- Toward a New Synthesis in Economics,
  pp. 327--350, Strüngmann Forum Reports, MIT press.
\end{itemize}
