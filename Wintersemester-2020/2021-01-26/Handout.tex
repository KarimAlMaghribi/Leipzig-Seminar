\documentclass[a4paper,11pt]{article}
\usepackage{ls}
\usepackage[ngerman]{babel}
\usepackage[utf8]{inputenc}

\title{Evolution allgemeiner Systeme}

\author{Daniel Schubert}

\date{26.01.21 / Seminar nachhaltige Systeme und Semantic Web}

\begin{document}

\maketitle

\section{Grundlagen}
Die Entwicklung von technischen Systemen wurde bereits 1981 in der Konferenz
\emph{Probleme der Entwicklung wissenschaftlich-technischer Kreativität
  ingenieur-technischer Arbeit (ITR)} analysiert.

Die Ergebnisse wurden 1983 überarbeitet und von \textbf{Boris Goldovsky}
veröffentlicht. 

Die darin ausgearbeiteten Gesetze legten auch die Grundlage für
\textbf{Vladimir Petrovs Werk} \emph{Gesetze und Gesetzmäßigkeiten der
  Systementwicklung}.

Beide Werke sind Grundlage dieses Handouts.

\section{Petrovs Zusammenfassung}

Vladimir Petrov ist Autor des Buches \cite{Petrov2020} \emph{Gesetze der
  Systementwicklung}. Die überarbeitete Version ist eine Monographie, welche
ganze vier Bücher umspannt. Diese stellen eine sehr umfangreiche Beschreibung
der Gesetze der Systementwicklung dar. Dabei wird besonderer Wert auf die
ausführliche Erklärung, auch anhand von Beispielen gelegt. Der erste Band
stellt eine Einführung mit grundlegenden Definitionen z.B. bezüglich des
Gesetzes- und Systembegriffes dar, hier wird das System der Gesetze als Ganzes
betrachtet. Der zweite Band befasst sich mit den universellen Gesetzen der
Systementwicklung: Dialektik, S-Kurven und Veränderung von Bedürfnissen sowie
Funktionsänderungen. Erst der dritte Teil beschäftigt sich mit der
Konstruktion von Systemen. Die Inhalte werden im vierten Band weitergeführt,
dabei wird ein Augenmerk auf die Steuerbarkeit, Dynamik und Vorhersage von
Systementwicklung gelegt.

Die Definitionen Petrovs zum Gesetzesbegriff, den Systembegriffen, Prinzipien,
Funktionen und viele weitere werden im Weiteren der Beschreibung der
Systementwicklung zu Grunde gelegt.

\section{Goldovskys Muster der Systementwicklung}

Die in der Konferenz \emph{Probleme der Entwicklung
  wissenschaftlich-technischer Kreativität inge\-nieur-technischer Arbeit
  (ITR)} erarbeiteten Trends sind als Muster aufgefasst, aus denen sich die
weiteren Gesetze ableiten lassen. Die Muster wurden als grundlegend und
methodologisch kategorisiert. Daraus wurden Gesetzmäßigkeiten der Kategorien
zum Bau technischer Systeme, dem Ändern ihrer Funktionen, dem Ändern ihrer
Strukturen sowie ihrer Zusammensetzungen abgeleitet.

Dabei wird bei den Gesetzmäßigkeiten angegeben, aus welchen Mustern oder
anderen Gesetzen sich diese ableiten ließen.

\subsection{Grundlegende Muster}

Zu den grundlegenden Mustern zählt Goldovsky Gesetze der Dialektik,
systemweite Gesetze, Naturgesetze sowie soziale (und ökonomische) Gesetze.
Diese Kategorie ist weit gefasst und setzt ein Fundament aus Logik und
logischen Schlussfolgerungen.  So zählen zu den dialektischen Gesetzen die
Negation der Negation oder die Dualität von Qualität und Quantität. Das
physische Fundament besteht aus Regeln, die als Naturgesetze ausgewiesen sind.
Hierzu zählt die Existenz von Redundanzen, von Stoffaustausch und
Energiedurchsatz oder notwendiger Kompatibilität mit der Umwelt. Dem entgegen
stehen die sozialen Gesetze, in denen der Konflikt zwischen endlichen
Ressourcen und bis zur Unendlichkeit wachsenden Bedürfnissen der Menschen
thematisiert wird.

\subsection{Methodologische Muster}

Die methodologischen Muster beschreiben die Rahmenbedingungen, an denen sich
die Entwicklung technischer Systeme messen lassen muss. So müssen z.B. die
Realisierung eines Systems auch mit den verfügbaren technischen Mitteln
möglich sein, die Funktion des Systems muss seinen Anforderungen genügen und
vertretbare Kosten verursachen. Es wird festgehalten, dass Widersprüche
verschmolzen werden können, um Universalität in einem Element zu erhalten, und
dass Elemente aufgespalten werden können, um sie zu spezialisieren. Dabei
haben Systeme und ihre Funktionen ein theoretisches Optimum, das einer
internen Hierarchie folgt. So ist z.B. das Funktionieren wichtiger als der
Durchsatz und der Durchsatz wichtiger als die Qualität. Im Entwicklungsprozess
dominieren vor allem existierende Systeme bei der Ressourcenallokation. Es
entstehen minimale Änderungen durch schrittweise und kontinuierliche
Verbesserung eines Systems. In der Entwicklung von bestehenden Systemen wird
der Mensch zusehends verdrängt, in neuen Systemen ist entsprechend mehr
menschliche Arbeit involviert.

\subsection{Gesetzmäßigkeiten des Baus arbeitsfähiger technischer Systeme}

Eines der kürzeren Kapitel der Gesetzmäßigkeiten der Systementwicklung ist dem
Bau neuer Systeme gewidmet. Die einzigen zu erfüllenden Bedingungen sind die
funktionale Vollständig\-keit, ein gewährleisteter Energiedurchsatz, eine
gewährleistete Intensität im Energieaustausch und eine minimale Steuerbarkeit.

\subsection{Gesetzmäßigkeiten von Änderungen im Funktionieren des Systems}

Wenn ein System seine Funktion ändert, muss es entweder eine andere Nische
finden, sich spezialisieren, eine höhere Vielseitigkeit oder Rentabilität
garantieren. Wenn eine Funktionsänderung nicht eine dieser Bedingungen
erfüllt, gibt es keinen Grund für eine Änderung.

\subsection{Gesetzmäßigkeiten der Änderung der Struktur technischer Systeme}

Während sich die Funktionen eines Systems nur selten ändern, ist der Großteil
der Verände\-rungen struktureller Natur. In den Bestandteilen des Systems
werden Probleme gelöst oder Eigenschaften verbessert, daher kommt es zu
emergenten Phänomenen, welche in diesem Abschnitt beschrieben werden. Hierzu
zählen u.a. die ungleichmäßige oder inharmonische Entwicklung von
Systemteilen, das Wachstum in Anzahl der Beziehungen zwischen den einzelnen
Systemelementen und dem Obersystem sowie der Variabilität dieser
Beziehungen. Durch strukturelle Veränderung strebt ein System in seiner
Entwicklung wachsende Integrität an, wodurch aber die Komplexität des Systems
steigt. Eine Möglichkeit dazu ist die wachsende Dynamisierung der Elemente des
Systems. Auch kann Komplexität oft nicht endlos zunehmen. Es gibt zu jedem
physikalischen Prinzip eine maximale Schwelle an Kompliziertheit. Ein
wichtiger Ansatz für die Umstrukturierung eines Systems ist eine gesteigerte
Effektivität oder eine Reduzierung der benötigten Eingaben oder schädlichen
Auswirkungen aus dem Betrieb des Systems.  Eine Möglichkeit dies zu erreichen,
ist durch Reduzierung oder Eliminierung von Zwischenketten innerhalb des
Systems sowie das Beseitigen von Funktionsstörungen.

\subsection{Änderungen in der Zusammensetzung des Systems}

Das letzte Kapitel beschreibt Trends oder Muster der Zusammensetzung von
Systemen wäh\-rend ihrer Evolution. Gemeint ist auch hier eine wachsende
Kompliziertheit, diesmal in den Bewegungsformen von Materie, dem wachsenden
künstlichen Anteil der Systemelemente und ihrer Hetero- oder Homogenität. Wenn
Systeme an Nischengrenzen stoßen, entstehen Übergangsformen, und diese
Hybridbildung zeichnet sich vor allem in der Systemzusammensetzung ab.
Abschließend gehört der Einfluss des Menschen als Teil vieler Systeme in diese
Kategorie: in neue Systeme wird der Mensch mit einbezogen, in der Entwicklung
des Systems wird er zunehmend verdrängt.

%\begin{itemize}
%\item Wenn Gesetzmäßigkeiten von Mustern abgeleitet werden, worauf basieren die Muster? Sind sie mit Gesetzen gleichzusetzen?
%\item Sind die vorgestellten Gesetzmäßigkeiten vollständig? Sind einige Gesetzte redundant?
%\item Verändert sich der Entwicklungsprozess von Systemen, wenn ja wie?
%\end{itemize}

\begin{thebibliography}{xx}
\bibitem{Goldovsky1983} B.I. Goldovsky (1983). System der Gesetzmäßigkeiten
  des Aufbaus und der Entwicklung technischer Systeme.
\bibitem{Petrov2020} V.M. Petrov (2020). Gesetze und Gesetzmäßigkeiten der
  Entwicklung von Systemen.
\end{thebibliography}

\end{document}



















