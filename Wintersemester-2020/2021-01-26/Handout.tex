\documentclass[a4paper,12pt]{article}
\usepackage{a4wide,url}

\usepackage[ngerman]{babel}
\usepackage[utf8]{inputenc}

\parindent0cm
\parskip4pt

\title{Evolution allgemeiner Systeme}

\author{Daniel Schubert}

\date{26.01.21 / Seminar nachhaltige Systeme und Semantic Web}

\begin{document}

\maketitle

\section{Grundlagen}
Die Entwicklung von technischen Systemen wurde bereits 1981 in der Konferenz ,,\emph{Probleme der Entwicklung wissenschaftlich- technischer Kreativität ingenieur-technischer Arbeit (ITR)}'' analysiert. \medskip

Die Ergebnisse wurden 1983 überarbeitet und von \textbf{Boris Goldvsky} veröffentlicht. \medskip

Die darin ausgearbeiteten Gesetze legten auch die Grundlage für \textbf{Vladimir Petrov's Werk}, \emph{Gesetze und Gesetzmäßigkeiten der Systementwicklung}''. \medskip

Beide Werke sind Grundlage dieses Handouts.

\section{Petrov's Zusammenfassung}

Vladimir Petrov ist Author des Buches ,,Gesetze der Systementwicklung'' . Die überarbeitete Version ist eine Monographie, welche ganze 4 Bücher umspannt. Diese stellen eine sehr umfangreiche Beschreibung der Gesetze der Systementwicklung dar. Dabei wird besonderer Wert auf die ausführliche Erklärung, auch anhand von Beispielen gelegt. Der erste Band stellt eine Einführung mit grundlegenden Definitionen z.B. bezüglich des Gesetzes- und Systembegriffes dar, hier wird das System der Gesetze als ganzes betrachtet. Der zweite Band befasst sich mit den universellen Gesetzen der Systementwicklung: Dialektik, S-Kurven und Veränderung von Bedürfnissen sowie Funktionsänderungen. Erst der dritte Teil beschäftigt sich mit der Konstruktion von Systemen. Die Inhalte werden im 4ten Band weitergeführt, dabei wird ein Augenmerk auf die Steuerbarkeit und die Dynamik und Vorhersage von Systementwicklung gelegt. \medskip

Die Definitionen Petrov's zum Gesetzesbegriff, den Systembegriffen, Prinzipien, Funktionen und viele weitere werden im folgenden zur Beschreibung der Systementwicklung angewandt.

\clearpage
\section{Goldovsky's Muster der Systementwicklung}

Die in der Konferenz ,,\emph{Probleme der Entwicklung wissenschaftlich- technischer Kreativität ingenieur-technischer Arbeit (ITR)}'' erarbeiteten Trends sind als Muster aufgefasst, aus denen sich die Weiteren Gesetze ableiten lassen. Die Muster wurden als grundlegend und methodologisch kategorisiert. Daraus wurden Gesetzmäßigkeiten der Kategorien zum Bau technischer Systeme, dem Ändern ihrer Funktionen, dem Ändern ihrer Strukturen sowie ihrer Zusammensetzungen abgeleitet. \medskip

Dabei wird bei dem Gesetzmäßigkeiten angegeben aus welchen Mustern oder anderen Gesetzen sich diese ableiten ließen.


\subsection{Grundlegende Muster}

Zu den grundlegenden Mustern zähl Goldovsky Gesetze der Dialektik, Systemweite Gesetze, Naturgesetze sowie Soziale (und ökonomische) Gesetze. diese Kategorie ist weit gefasst und setzt ein Fundament aus Logik und logischen Schlussfolgerungen. So zählen zu den dialektischen Gesetzen die Negation von Negation oder die Dualität von Qualität und Quantität. Das Physische Fundament wird aus als Naturgesetze ausgewiesene Regeln gebaut. Hierzu zählt die Existenz von Redundanzen, von Stoffaustausch und Energiedurchsatz oder notwendiger Kompatibilität mit der Umwelt. Dem entgegen stehen die sozialen Gesetze, in denen der Konflikt zwischen endlichen Ressourcen und bis zur Unendlichkeit wachsenden Bedürfnissen der Menschen thematisiert wird.

\subsection{Methodologische Muster}

Die methodologischen Muster beschreiben die Rahmenbedingungen an denen sich die Entwicklung technischer Systeme messen muss. So müssen z.B. die Realisierung eines Systems auch mit den technischen Mitteln möglich sein, die Funktion des Systems muss seinen Anforderungen genügen und vertretbare Kosten erzeugen. Es wird festgehalten dass Widersprüche verschmolzen werden können um Universalität in einem Element zu erhalten, und das Elemente aufgespalten werden können um sie zu spezialisieren. Dabei haben Systeme und seine Funktionen ein theoretisches Optimum das einer internen Hierarchie folgt, so ist z.B. das Funktionieren wichtiger als der Durchsatz, und der Durchsatz wichtiger als die Qualität. Im Entwicklungsprozess dominieren vor allem existierende Systeme bei der Ressourcenallokation. Es entstehen minimale Änderungen durch schrittweise und kontinuierliche Verbesserung eines Systems. In der Entwicklung von bestehenden Systemen wird der Mensch zusehends verdrängt, in neuen Systemen ist entsprechend mehr menschliche Arbeit involviert.

\subsection{Gesetzmäßigkeiten des Baus arbeitsfähiger technischer Systeme}

Eines der kürzeren Kapitel der Gesetzmäßigkeiten der Systementwicklung ist der Bau neuer Systeme. Die einzigen zu erfüllenden Bedingungen sind die funktionale Vollständigkeit, ein gewährleisteter Energiedurchsatz, eine gewährleistete Intensität im Energieaustausch und eine minimale Steuerbarkeit. 

\subsection{Gesetzmäßigkeiten von Änderungen im Funktionieren des Systems}

Wenn ein System seine Funktion ändert muss es entweder eine andere Nische finden, sich spezialisieren, eine höhere Vielseitigkeit oder Rentabilität garantieren. Wenn eine Funktionsänderung nicht eine dieser Bedingungen erfüllt gäbe es keinen Grund für eine Änderung.

\subsection{Gesetzmäßigkeiten der Änderung der Struktur technischer Systeme}

Während sich die Funktionen eines Systems nur selten ändert ist der Großteil der Veränderungen struktureller Natur. In den Bestandteilen des Systems werden Probleme gelöst, oder Eigenschaften verbessert, daher kommt es zu emergenten Phänomenen, welche in diesem Abschnitt beschrieben werden. Hierzu zählen u.a. die ungleichmäßige oder inharmonische Entwicklung von Systemteilen, das Wachstum in Anzahl der Beziehungen zwischen den einzelnen Systemelementen und dem Obersystem sowie der Variabilität dieser Beziehungen. Durch strukturelle Veränderung strebt ein System in seiner Entwicklung wachsende Integrität an, wodurch aber die Komplexität des Systems steigt. Eine Möglichkeit dazu ist die wachsende Dynamisierung der Elemente des Systems. Auch kann Komplexität oft nicht endlos zunehmen es gibt zu jedem physikalischen Prinzip eine maximale Schwelle an Kompliziertheit. Ein wichtiger Ansatz für die Umstrukturierung eines Systems ist eine gesteigerte Effektivität oder eine Reduzierung der benötigten Eingaben oder schädlichen Auswirkungen aus dem Betrieb des Systems. Eine Möglichkeit dies zu erreichen ist durch Reduzierung oder Eliminierung von Zwischenketten innerhalb des Systems sowie dem beseitigen von Funktionsstörungen.

\subsection{Änderungen in der Zusammensetzung des Systems}

Das letzte Kapitel beschreibt die Trends oder Muster in der Zusammensetzung von Systemen während ihrer Evolution. Gemeint ist auch hier eine wachsende Kompliziertheit, diesmal in den Bewegungsformen von Materie, dem wachsenden künstlichen Anteil der Systemelemente und ihrer Hetero- oder Homogenität. Wenn Systeme an Nischengrenzen stoßen entstehen Übergangsformen, und diese Hybridbildung zeichnet sich vor allem in der Systemzusammensetzung ab. Abschließend gehört der Einfluss des Menschen als Teil vieler Systeme in diese Kategorie: in neue Systeme wird der Mensch mit einbezogen, in der Entwicklung des Systems wird er zunehmend verdrängt.

%\begin{itemize}
%\item Wenn Gesetzmäßigkeiten von Mustern abgeleitet werden, worauf basieren die Muster? Sind sie mit Gesetzen gleichzusetzen?
%\item Sind die vorgestellten Gesetzmäßigkeiten vollständig? Sind einige Gesetzte redundant?
%\item Verändert sich der Entwicklungsprozess von Systemen, wenn ja wie?
%\end{itemize}

\section{Quellen}
\begin{itemize}
\item B.I.Goldovsky 1983 - \emph{System der Gesetzmäßigkeiten des Aufbaus und der Entwicklung technischer Systeme}
\item Vladimir Petrov 2020 - \emph{Gesetze und Gesetzmäßigkeiten der Entwicklung von Systemen}
\end{itemize}

\end{document}



















