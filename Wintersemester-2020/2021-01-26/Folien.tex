\documentclass{beamer}
\mode<presentation>
{
  \usetheme{Copenhagen}
  \setbeamertemplate{headline}[default]
  \setbeamercovered{transparent}
}

\usepackage[ngerman]{babel}
\usepackage[utf8]{inputenc}

\title[Evolution allgemeiner Systeme]{Evolution allgemeiner Systeme}

\author{Daniel Schubert}

\date{26.01.21 / Seminar nachhaltige Systeme und Semantic Web}

\setbeamertemplate{navigation symbols}{}
\setbeamertemplate{blocks}[rounded][shadow=true]

\begin{document}

\begin{frame}
  \titlepage
\end{frame}

\begin{frame}{Inhaltsverzeichnis}
  \tableofcontents
\end{frame}

\section{Grundlagen}
\begin{frame}{Grundlagen}
Die Entwicklung von technischen Systemen wurde bereits 1981 in der Konferenz ,,\emph{Probleme der Entwicklung wissenschaftlich- technischer Kreativität ingenieur-technischer Arbeit (ITR)}'' analysiert. \medskip

Die Ergebnisse wurden 1983 überarbeitet und von \textbf{Boris Goldvsky} veröffentlicht. \medskip

Die darin ausgearbeiteten Gesetze legten auch die Grundlage für \textbf{Vladimir Petrov's Werk}, \emph{Gesetze und Gesetzmäßigkeiten der Systementwicklung}''. \medskip

\end{frame}


\section{Petrov's Zusammenfassung}
\begin{frame}{Gesetze der Systementwicklung}
\begin{itemize}
\item Vladimir Petrov ist Author von ,,Gesetze der Systementwicklung''
\item Monographie ,,Gesetze und Gesetzmäßigkeiten der Systementwicklung'' ist überarbeitete Fassung
\item 4 bändige Ausarbeitung
\end{itemize}

\end{frame}

\begin{frame}{Buch 1}
\begin{itemize}
\item Einführung in Monographie
\item grundlegenden Konzepte und Definitionen
\item Zusammenfassung des Gesamtwerks
\item Vogelperspektive auf Struktur der Gesetze und Gesetzmäßigkeiten
\end{itemize}
\end{frame}

\begin{frame}{Buch 2}
\begin{itemize}
\item universelle Gesetze der Systementwicklung
\item Entwicklung in S-Kurven
\item Entwicklung von Bedürfnissen und Funktionsänderungen
\end{itemize}
\end{frame}

\begin{frame}{Buch 3 und 4}
Gesetzmäßigkeiten in
\begin{itemize}
\item Konstruktion von Systemen
\item Entwicklung von Systemen\vspace{1cm}
\item Veränderung der Steuerbarkeit und Dynamik eines Systems
\item Vorhersage der Entwicklung eines Systems
\end{itemize}
\end{frame}


\section{Goldovsky's Muster der Systementwicklung}
\begin{frame}{Muster der Systementwicklung}

Die in der Konferenz ,,\emph{Probleme der Entwicklung wissenschaftlich- technischer Kreativität ingenieur-technischer Arbeit (ITR)}'' erarbeiteten Trends sind als Muster aufgefasst, aus denen sich die Weiteren Gesetze ableiten lassen. Die Muster wurden als grundlegend und methodologisch kategorisiert. Daraus wurden Gesetzmäßigkeiten der Kategorien zum Bau technischer Systeme, dem Ändern ihrer Funktionen, dem Ändern ihrer Strukturen sowie ihrer Zusammensetzungen abgeleitet. \medskip

Dabei wird bei dem Gesetzmäßigkeiten angegeben aus welchen Mustern oder anderen Gesetzen sich diese ableiten ließen.

\end{frame}

\subsection{Grundlegende Muster}
\begin{frame}{Grundlegende Muster}
\begin{itemize}
\item Gesetze der Dialektik
\item Systemweite Gesetze, Naturgesetze
\item Soziale (und ökonomische) Gesetze
\end{itemize}
\end{frame}

\subsection{Methodologische Muster}
\begin{frame}{Methodologische Muster}
\begin{itemize}
\item Widerspruch zwischen Bedürfnissen und verfügbaren Mitteln
\item Universalisierung und Spezialisierung als Umgangsform mit Widersprüchen
\item Existenz von Widersprüchen
\item primär Nützliche Funktion bestimmt funktionale Nische
\item Hierarchisierung des Optimums eines Systems
\item dominante Ressourcenallokation durch vorhandene Systeme
\end{itemize}
\end{frame}
\begin{frame}{Methodologische Muster}
\begin{itemize}
\item Prinzip minimaler Änderungen während der Entwicklung und schrittweisen Optimierung
\item nützlicher Output muss Anforderung genügen, Input muss angemessen sein und Schaden zumutbar
\item Wachstum in Effizienz und Verdrängung des Menschen
\item Möglichkeit von Realisierung und dessen Kosten
\item Vorrang der Funktion über dem Effekt
\item Zulässigkeit von Verschlechterung
\end{itemize}
\end{frame}

\subsection{Gesetzmäßigkeiten des Baus arbeitsfähiger technischer Systeme}
\begin{frame}{Bau arbeitsfähiger technischer Systeme}
\begin{itemize}
\item funktionale Vollständigkeit
\item Energiedurchlässigkeit
\item Parametereinhaltung
\item minimale Steuerbarkeit
\end{itemize}
\end{frame}

\subsection{Gesetzmäßigkeiten von Änderungen im Funktionieren des Systems}
\begin{frame}{Änderungen im Funktionieren des Systems}
\begin{itemize}
\item Nischenfindung
\item Spezialisierung des Systems
\item Erhöhung der Vielseitigkeit
\item Änderungen im Obersystem
\end{itemize}
\end{frame}

\subsection{Gesetzmäßigkeiten der Änderung der Struktur technischer Systeme}
\begin{frame}{Änderung der Struktur technischer Systeme}
\begin{itemize}
\item Ungleichmäßige und Inharmonische Entwicklung der Systemteile
\item Wechselbeziehungen zu Obersystem
\item Wachstum des Dynamismus im System
\item Variabilität von Elementen und Beziehungen
\item Wachsende Kompliziertheit und dessen Beschränkung
\item Wachstum der Integrität
\end{itemize}
\end{frame}
\begin{frame}{Änderung der Struktur technischer Systeme}
\begin{itemize}
\item Ubergang der Entwicklung ins Obersystem
\item Erhöhung der Nutzung der Umwelt und Reduzierung der verwendeten Ressourcen
\item Elimination von Zwischenketten und Reduktion von Energiekettenlängen
\item Elimination von Funktionsstörungen
\item Eigenständigkeit von Systemteilen
\item Raumfüllung und Reduzierung des Platzbedarfs
\end{itemize}
\end{frame}

\subsection{Änderungen in der Zusammensetzung des Systems}
\begin{frame}{Änderungen in der Zusammensetzung des Systems}
\begin{itemize}
\item wachsende Kompliziertheit von Bewegungsformen
\item Die Verdrängung des Menschen aus alten Systemen und seine Einbeziehung in neue
\item Heterogenität und Homogenität der System-Elemente
\item Hybridisierung von Systemen
\end{itemize}
\end{frame}

\begin{frame}{Diskussion}
\begin{itemize}
\item Wenn Gesetzmäßigkeiten von Mustern abgeleitet werden, worauf basieren die Muster? Sind sie mit Gesetzen gleichzusetzen?
\item Sind die vorgestellten Gesetzmäßigkeiten vollständig? Sind einige Gesetzte redundant?
\item Verändert sich der Entwicklungsprozess von Systemen, wenn ja wie?
\end{itemize}
\end{frame}

\begin{frame}{Quellen}
\begin{itemize}
\item B.I.Goldovsky 1983 - \emph{System der Gesetzmäßigkeiten des Aufbaus und der Entwicklung technischer Systeme}
\item Vladimir Petrov 2020 - \emph{Gesetze und Gesetzmäßigkeiten der Entwicklung von Systemen}
\end{itemize}
\end{frame}

\end{document}



















