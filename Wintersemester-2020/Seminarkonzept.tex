\documentclass[11pt,a4paper]{article}
\usepackage{a4wide,url,enumitem}
\usepackage[utf8]{inputenc}
\usepackage[german]{babel}

\parindent0pt
\parskip4pt

\title{Handreichung für das Forschungsseminar\\ „Entwicklungsmuster
  technischer Systeme“}

\author{Hans-Gert Gr\"abe, Ken Pierre Kleemann, Sabine Lautenschläger}

\date{24. Oktober 2020}

\begin{document}
\maketitle

\section{Ziel und Methodik des Seminars}

Der Systembegriff spielt in der Informatik eine herausragende Rolle, wenn es
um Datenbanksysteme, Softwaresysteme, Hardwaresysteme, Abrechnungssysteme,
Zugangssysteme usw. geht.  Überhaupt wird die Informatik von einer Merhheit
als die „Wissenschaft von der \emph{systematischen} Darstellung, Speicherung,
Verarbeitung und Übertragung von Informationen, besonders der automatischen
Verarbeitung mithilfe von Digitalrechnern“ (Wikipedia) verstanden.  Auch
gewisse einschlägige Professionen wie etwa der \emph{Systemarchitekt} genießen
unter IT-Anwendern hohe Wertschätzung.

Die Bedeutung des Systembegriffs reicht allerdings weit über den Bereich der
Informatik hinaus -- er ist grundlegend für alle Ingenieurwissenschaften und
als \emph{Systems Engineering} mit der ISO/IEC/IEEE-15288 Norm „Systems and
Software Engineering“ auch Gegenstand internationaler Normierungs- und
Standardisierungsprozesse.  Mehr noch spielt der Systembegriff auch bei der
Beschreibung komplexer natürlicher und kultureller Prozesse -- etwa im Begriff
des \emph{Ökosystems} -- eine zentrale Rolle.

Mit dem \emph{Semantic Web} rückt die Bedeutungsanalyse digitaler Artefakte in
den Mittelpunkt, die in letzter Instanz Sprachartefakte sind und damit
ebenfalls in direktem Zusammenhang zu einem sinnvoll zu entfaltenden
\emph{Systembegriff} stehen als Grundlage jeden Verständnisses konkreter
Systeme.

Mit dem Schlagwort \emph{Nachhaltigkeit} werden schließlich komplexe
gesellschaftliche Abstimmungsprozesse angesprochen, mit denen vielfältige
Informations- und Bewertungsprobleme einhergehen. Hierbei ist die Fähigkeit
der beschreibenden Abgrenzung, Entwicklung und Steuerung von sogenannten
Systemen auf bzw. über verschiedene Governance-, Raum- und Zeitebenen hinweg
von großer Bedeutung.

Im Wintersemester 2019/20 hatten wir uns bereits mit diesem Spektrum von
\emph{Systemansätzen} (im Plural) beschäftigt, eine große Spannbreite
entsprechender Konzepte aus verschiedenen Wissenschaftsbereichen identifiziert
und diese im letzten Teil des Seminars mit Entwicklungsansätzen technischer
Systeme im Umfeld der TRIZ verglichen.  Diese Untersuchungen sollen im
aktuellen Forschungsseminar vertieft werden.
\newpage

\begin{quote}
  \textbf{Ziel des Seminars} ist es, ein besseres Verständnis der
  verschiedenen Konzepte zu gewinnen, die für Gesetze, Gesetzmäßigkeiten,
  Trends und Entwicklungsmuster technischer Systeme und allgemeiner Systeme
  im Kontext der TRIZ vorgeschlagen und entwickelt wurden.
\end{quote}

Das Seminar ist ein \textbf{Forschungsseminar}, in dem wir gemeinsam die
Konzepte der Historisierung technischer (und allgemeinerer) Entwicklungen
verschiedener Autoren im TRIZ-Umfeld erschließen und zueinander relatieren
wollen.

Von den Studierenden wird erwartet, dass sie sich aktiv am Seminar beteiligen
durch Seminardiskussionen, Präsentationen und nicht zuletzt durch Lesen der
relevanten Materialien.  Für den erfolgreichen Abschluss des Seminars ist ein
Thema als Diskussionsleiter zu präsentieren und dazu vorab eine 2-3-seitige
Ausarbeitung vorzulegen.

Alle Materialien und Seminarberichte, die öffentlich zur Verfügung gestellt
werden können, werden im github-Repo
\begin{center}
  \url{https://github.com/wumm-project/Leipzig-Seminar}
\end{center}
im Verzeichnis \texttt{Wintersemester-2020} zusammengetragen.

\section{Seminarablauf}

Das Seminar findet dienstags 9-11 Uhr wöchentlich synchron online statt.  Zu
jedem Termin haben die Seminarteilnehmer die zugewiesene Lektüre vorab
studiert und sind so in der Lage, diese im Seminar zu diskutieren.  Das
Seminar wird von einem \emph{Diskussionsleiter} moderiert, der eine kurze
Ausarbeitung zum Thema vorbereitet und diese \emph{vor dem Termin} (bis
Sonntag abend) den Teilnehmern zur Verfügung stellt.

Mehr zum Seminarablauf ist im
OPAL\footnote{\url{https://bildungsportal.sachsen.de/opal/} -- Für den Zugang
  ist im Prinzip ein Account an der Uni Leipzig erforderlich.} (Kurs
W20.BIS.SIM) zu finden.
\begin{itemize}
\item Ein Forum mit den Themen der einzelnen Seminartermine und der
  Diskussionsleiter (als Sekundärquelle) sowie
\item ein Uploadbereich für das Hochladen der Präsentationen und Abstracts im
  pdf-Format sowie seminarinterne Materialien.
\end{itemize}
Die \textbf{Primärquelle für den Seminarplan} ist die Datei
\texttt{Seminarplan.md} im github-Repo \emph{Leipzig-Seminar}.

Für externe Seminarteilnehmer wird ein Zugang zu diesen internen Materialien
organisiert, so weit diese nicht im github-Repo \emph{Leipzig-Seminar}
öffentlich zur Verfügung gestellt werden können.

\section{Prüfungsleistung. Themen für Seminararbeiten}

Grundsätzlich ist für die Zulassung zur Prüfung das Seminar erfolgreich
abzuschließen, wozu in wenigstens einem Seminar die Seminarleitung übernommen
werden muss mit den damit verbundenen Leistungen wie oben beschrieben.

Studierende, die im 10-LP-Modul „Semantic Web“ eingeschrieben sind, müssen
außerdem das TRIZ-Praktikum erfolgreich abschließen und absolvieren danach
eine mündliche Prüfung (30 Minuten), in der die erworbenen Kenntnisse zu
Konzepten Systematischer Innovationsmethodiken sowie zum Semantic Web gefragt
sind.

Studierende, die im 5-LP-Seminarmodul „Angewandte Informatik“ eingeschrieben
sind, erstellen als Prüfungsleistung eine Seminararbeit. Genauere Themen
werden in der zweiten Hälfte des Seminars ab Mitte Dezember vergeben. Die
Seminararbeit muss bis zum Semesterende am 31.03.2021 fertiggestellt und
abgegeben sein. 

\section{Seminararbeiten}

Das Seminar ist als deutschsprachiges Forschungsseminar angelegt, die
Seminararbeiten orientieren sich an den thematischen Rahmenvorgaben aus dem
jeweiligen Semester.

Im Sinne einer weiteren einfachen Arbeit mit den Ansätzen und Texten sollen im
Regelfall die Arbeiten auf der Basis von {\LaTeX} verschriftlicht und auch die
\LaTeX-Quelle selbst zur weiteren Nutzung unter den Bedingungen der
CC-0\footnote{\url{https://creativecommons.org/publicdomain/zero/1.0/deed.de}}
zur Verfügung gestellt werden, um auf diese Weise einen entsprechenden
Textkorpus zu erstellen, der vergleichbare Bemühungen im OpenDiscovery Projekt
begleitet.

Natürlich kann dies nicht „verordnet“ werden. \textbf{Bitte teilen Sie der
  Seminarleitung deshalb mit, wenn Sie Ihre Arbeit für diesen Austausch nicht
  zur Verfügung stellen}.  

\section{Seminarplan}

Dieser ist nun im github Repo des Seminars zu finden und wird dort laufend
aktualisiert.  Dabei wird auf die Literaturübersicht in diesem Konzeptpapier
Bezug genommen. Auch die Literaturliste wird bei Bedarf weiter ergänzt. 

\begin{thebibliography}{xxx}
\bibitem{Altschuller1979} Genrich Altschuller (1979).  Schöpfertum als exakte
  Wissenschaft.  Deutsch 1983 \emph{Erfinden -- (k)ein Problem}.
\bibitem{Goldovsky1983} Boris Goldovsky (1983). System der Gesetzmäßigkeiten
  des Aufbaus und der Entwicklung technischer Systeme.
  \url{https://wumm-project.github.io/TTS.html}
\bibitem{Goldovsky2017} Boris Goldovsky (2017). Über die Gesetze der
  Konstruktion technischer Systeme.  
  \url{https://wumm-project.github.io/TTS.html}
\bibitem{KoltzeSouchkov2017} Karl Koltze, Valeri Souchkov (2017).
  Systematische Innovationsmethoden.  Hanser Verlag, München. ISBN
  9783446451278
\bibitem{TESE2018} Alex Lyubomirsky, Simon Litvin, Sergei Ikovenko et al.
  (2018). Trends of Engineering System Evolution (TESE).  TRIZ Consulting
  Group. ISBN 9783000598463.
\bibitem{Khomenko2007} Nikolai Khomenko, John Cooke (2007?).  Inventive
  problem solving using the OTSM-TRIZ “TONGS” model. \\
  \url{https://otsm-triz.org/sites/default/files/ready/tongs_en.pdf}
\bibitem{Petrov2020a} Vladimir Petrov (2020a). Gesetze und Gesetzmäßigkiten
  der Systemevolution. Buch in 4 Bänden (in Russisch), ISBN
  978-5-0051-5728-7.  Eine deutsche Übersetzung eines frei verfügbaren Auszugs
  aus dem ersten Band siehe \\
  \url{https://wumm-project.github.io/TTS.html}
\bibitem{Petrov2020b} Vladimir Petrov (2020b). Gesetzmäßigkeiten der
  Entwicklung künstlicher Systeme. \\
  \url{https://wumm-project.github.io/TTS.html} 
\bibitem{Rubin2019} Michail Rubin (2019). Zum Zusammenhang der
  Entwicklungsgesetze allgemeiner Systeme und der Entwicklungsgesetze
  technischer Systeme. \\ \url{https://wumm-project.github.io/TTS.html}
\bibitem{Shpakovsky2016} Nikolay Shpakovsky (2016). Tree of Technology
  Evolution. Englische Übersetzung der russischen Ausgabe, die bei Forum,
  Moskau 2010 erschienen ist.\\ \url{https://wumm-project.github.io/TTS.html}
\bibitem{Zobel2007} Dietmar Zobel (2007). Kreatives Arbeiten. Expert Verlag,
  Renningen.\\ ISBN 9783816927136.
\bibitem{Zobel2020} Dietmar Zobel (2020). TRIZ für alle. Expert Verlag,
  Renningen. ISBN 9783816985105.
\end{thebibliography}

\end{document}
