\documentclass[11pt,a4paper]{article}
\usepackage{a4wide,url,enumitem}
\usepackage[utf8]{inputenc}
\usepackage[german]{babel}

\parindent0pt
\parskip4pt
\setcounter{secnumdepth}{-2}

\title{Vorlesungsplan \\[1em] \emph{Modellierung nachhaltiger Systeme und
    Semantic Web} \\[1em] im Wintersemester 2020/21}

\author{Hans-Gert Gr\"abe}

\date{29. Oktober 2020}

\begin{document}
\maketitle
\subsection{Allgemeines}

Die Vorlesung findet synchron online statt (jeweils donnerstags 11-13 Uhr) und
orientert sich am Flipped Classroom Konzept. Die Vorlesung besteht aus drei
Teilen.

Im ersten Teil werden wir das Konzept eines \emph{Technischen Systems}
erschließen und uns mit der TRIZ als wichtigster Systematischer
Innovationsmethodik beschäftigen.  Im Gegensatz zu anderen Kreativitäts- und
Innovationsmethodiken setzt die TRIZ auf die Systematisierung
ingenieur-technischer Erfahrungen. 

Im zweiten Teil befassen wir uns genauer mit Aspekten der Erstellung von
Begriffsnetzen für Datenmodelle auf der Basis des \emph{Resource Description
  Frameworks} (RDF), der \emph{Linked Open Data Cloud}, dem dabei entstehenden
\emph{Giant Global Graph} und der Bedeutung dieser Entwicklungen für die
Organisation kooperativer Handlungszusammenhänge.

Im dritten Teil untersuchen wir schließlich die Rolle von Daten und
Informationen sowie die Erzeugung neuer Sprache für die Entwicklung
technischer Systeme in einer bürgerlichen Gesellschaft.

Neben einer allgemeinen Literaturliste wird dazu zu jeder Vorlesung Literatur
zur Vorbereitung angegeben, die \textbf{vor der Vorlesung zu studieren ist},
um den Ausführungen folgen zu können. In der Vorlesung wird das Thema nur
kursorisch beleuchtet, es besteht aber die Möglichkeit, Fragen zur Literatur
zu stellen und einzelne Aspekte gemeinsam zu diskutieren.

Die meisten Materialien zur Vorlesung sind im öffentlichen Materialordner im
github Projekt \emph{Leipzig-Seminar} (im Weiteren \textbf{LS-Materialordner})
\begin{center}
  \url{https://github.com/wumm-project/Leipzig-Seminar}
\end{center}
im Verzeichnis \texttt{Wintersemester-2020/Material} verfügbar oder sind
anderweitig im Internet leicht aufzufinden. Auf klassische gedruckte Literatur
und Ihre Fähigkeiten, diese zu beschaffen, wird dennoch nicht verzichtet.

Über den Fortgang der Vorlesung wird regelmäßig im o.g. github Repo
berichtet\footnote{\url{https://github.com/wumm-project/Leipzig-Seminar/tree/master/Wintersemester-2020/Vorlesung}}.
Dort finden Sie auch den Vorlesungsplan und es werden die Folien der einzelnen
Termine zur Verfügung gestellt.

\subsection{Datenschutz}

Wir folgen nicht nur theoretisch, sondern auch praktisch einem Open Culture
Ansatz und stellen Kursmaterialien öffentlich zur Verfügung.  Dies gilt auch
für die (kommentierten) Chatverläufe der Vorlesung, in denen auch Ihre Namen
genannt werden.  Wir gehen von Ihrem Einverständnis mit diesem Vorgehen aus,
wenn Sie dem nicht explizit widersprechen.  Die Diskussionen selbst werden
\textbf{nicht} aufgezeichnet.

\subsection{Allgemeine Literaturliste}
\begin{itemize}[noitemsep]
\item Robert Adunka (2020). TRIZ Anwendungsbeispiele. \\
  \url{https://www.triz-consulting.de/ueber-triz/triz-anwendungsbeispiele-2/} 
\item Iouri Belski (2020). Tools of TRIZ. A web repository of TRIZ materials
  on 12 simple TRIZ heuristics.
  \url{https://emedia.rmit.edu.au/triz/content/tools-triz}
\item Karl Koltze, Valeri Souchkov (2017). Systematische Innovationsmethoden.
  Hanser Verlag, München. ISBN 9783446451278
\item Andrei Kuryan, Dmitri Kucharavy (2018). The OTSM-TRIZ Heritage of
  Nikolai N. Khomenko. A General Theory of Powerful Thinking. Folien eines
  Vortrags auf dem TDS 2018 in St. Petersburg. Als \texttt{OTSM-Folien.pdf} im
  LS-Materialordner.
\item Nikolai Khomenko, John Cooke (2007). Inventive problem solving using the
  OTSM-TRIZ “TONGS” model.  Als \texttt{tongs-en.pdf} im LS-Materialordner.
\item Alex Lyubomirskiy, Simon Litvin, Sergei Ikovenko et al. (2018). Trends
  of Engineering System Evolution (TESE).  TRIZ Consulting Group. ISBN
  9783000598463.
\item Dietmar Zobel (2007). Kreatives Arbeiten. Expert Verlag, Renningen.\\
  ISBN 9783816927136.
\item Dietmar Zobel (2020). TRIZ für alle. Expert Verlag, Renningen. ISBN
  9783816985105.
\end{itemize}

\end{document}
