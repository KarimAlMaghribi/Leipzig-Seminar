\documentclass[11pt,a4paper]{article}
\usepackage{ls}
\usepackage[main=english,russian]{babel}

\newcommand{\HGG}[1]{\begin{quote}\textbf{Anmerkung HGG:} #1\end{quote}}

\newenvironment{code}{\tt \begin{tabbing}
\hskip12pt\=\hskip12pt\=\hskip12pt\=\hskip12pt\=\hskip5cm\=\hskip5cm\=\kill}
{\end{tabbing}}
\def\dq{{\char34}}

\title{A Proposal for Modelling the TRIZ Concepts\\ Flow and Flow Analysis}

\author{Immanuel Thoke}

\date{Version of March 9, 2021}

\begin{document}
\maketitle

\section{Aim of the work}

The aim of this paper is to elaborate a proposal for an ontological modelling
of the areas of \emph{TRIZ Flows} and \emph{TRIZ Flow Analysis} based on the
materials of the TRIZ Ontology Project (TOP) of the TRIZ Developer Summit
\cite{TOP} and further own investigations. The work fits into the activities
of the \emph{WUMM Project} \cite{WUMM} to accompany the TOP project and to
transfer it into the language of modern semantic concepts
\cite{WUMM-Ontology}.  The work therefore consists of two parts -- a
\emph{turtle file}, in which the semantic modelling is performed based on the
SKOS framework \cite{SKOS}, and \emph{this elaboration}, in which the
backgrounds and motivations of the concrete modelling decisions are detailed.

\section{Starting point} 

Flows and flow analysis play a rather peripheral role in TRIZ.  In the
standard reference \cite{Petrov2007} on a \emph{TRIZ Body of Knowledge} is
listed as item 2.5.8, but in the 7 references to the literature listed there
are no systematic explanations about the role of flows in TRIZ theory, but
only individual examples in which concrete flows (such as magnetic flow)
played a role in concrete modelling.  

The most comprehensive source on issues of TRIZ modelling of flows and
methodological issues of flow analysis is the thesis \cite{Lebedyev2015} by
Yuri Lebedyev, which he submitted in 2015 under the supervision of S.A. 
Logvinov for graduation as TRIZ Master. Since the thesis is only available in
Russian, only the author's English summary was evaluated.

Such a limitation is justified, since the aim of this seminar paper is not to
completely cover the TRIZ theory of flow analysis, but only to identify its
essential concepts and describe their connection with modern semantic RDF
means based on the SKOS ontology. The basis for this is above all the
presentation of Olga Eckardt in the webinar of the TOP project in October 2020
\cite{Eckardt2020}, in which basic approaches of the Lebedyev's work have
obviously been incorporated.

We start from the definition of both concepts in the TOP Glossary
\cite{TOP-Glossary}: 
\begin{quote}
  \textbf{Flow} is the directional movement in space of particles of mass of
  matter, as well as the directional movement of energy or information. Flow
  has dual properties: the properties of a substance of which the flow
  consists, and the properties of a field, which is formed as a result of the
  directional movement of particles of matter. Flows can be useful, harmful
  and parasitic. The flow model contains static components: source, channel,
  receiver, control system. Flow is a special case of a process in which
  directional movement occurs in physical space.

  \textbf{Flow analysis} is a method of systems analysis to establish
  relationships in a system to flow, find resources, and determine compliance
  with existing system requirements. The result of a flow analysis is a model
  of the flow as is and a model of the flow to be. Flow modification
  techniques are used to develop systems with flows and to solve inventive
  problems in them.
\end{quote}
The use of these concepts is further included there in a single place in the
context of cause-effect analysis (ibid.):
\begin{quote}
  Cause-effect analysis is performed: 
  \begin{itemize}[noitemsep]
  \item When the causes of an undesirable effect are not clear (when we cannot
    go from an administrative contradiction to a technical one $AC\to TC$.
  \item When it is necessary to clarify the causes of an undesirable effect
    (to deepen the understanding of the causes of an undesirable effect, e.g.
    after a functional or a flow analysis).
  \end{itemize}
\end{quote}
Other sources are much more sparing in their statements on these two
individual terms, but list other flow-related concepts, for example in
\cite{Souchkov2018}:
\begin{itemize}[noitemsep]
\item Delay Zone -- A location in a flow in which the integral flow speed is
  significantly lower than local flow speed. A Delay Zone is a typical
  disadvantage identified by Flow Analysis. 
\item Flow -- A sequence of events that have the same common feature.
\item  Flow Analysis-- An analytical method and a tool which identifies
  disadvantages in flows of energy, substances, and information in a technical
  system.
\item  Flow Disadvantage -- A disadvantage of a technical system being
  analyzed identified during Flow Analysis. Examples: Bottlenecks, «Gray
  Zones», «Stagnation Zones», etc.
\item Flow Distribution Analysis-- A part of Flow Analysis that identifies
  distribution of flows and their disadvantages. 
\item Stagnation Zone -- A part of a flow in which the flow stops temporarily
  or permanently. A Stagnation Zone is a typical disadvantage identified by
  Flow Analysis.
\item Transmission -- One of the key components (subsystems) of a Complete
  Technical System which according to the Law of System Completeness of a
  technical system transmits a flow of energy required to operate a working
  unit from an engine to the working unit.
\end{itemize}
Further essential terms and contexts were mentioned by V.M. Petrov in the
discussion on \cite{Eckardt2020}. 

\section{The Conceptualisations}

The conceptualisations to be developed follow the basic assumptions and
positings that are elaborated in more detail in \cite{Graebe2021}. In
particular, the following namespace prefixes are used:
\begin{itemize}[noitemsep]
\item \texttt{ex:} -- the namespace of a special system to be modelled. 
\item \texttt{tc:} -- the namespace of the TRIZ concepts (RDF subjects).
\item \texttt{od:} -- the namespace of WUMM's own concepts (RDF predicates,
  general concepts). 
\end{itemize}

\subsection{\texttt{tc:Flow}}

A flow is a component within a system and hence modeled as subconcept of
\texttt{od:Component}.  It has several properties that are modeled along the
rules for a morphological table, i.e. defining a PropertyDomain and a list of
allowed values as listed below.

\HGG{Alle Subjekt-URIs beginnen mit einem Großbuchstaben.  Alle Subjekte
  sollten Singular sein, da eine Plural suggerierende Auflistung \texttt{A
    od:does B, C, D} in drei Drei-Wort-Sätze mit dreimal Singular aufgelöst
  wird.}

As in Souchkov's Glossary we use URIs as 
\begin{center}\tt
  tc:HarmfulAction, tc:HarmfulFunction, tc:HarmfulInteraction,
  tc:HarmfulMachine
\end{center}
that combine the property and the subject of the property except for
\texttt{tc:FlowSubstance}.

\begin{center}
  \begin{tabular}{|c|c|}\hline
    Property & PropertyDomain \\\hline
    od:hasMode & tc:ModeValue \\
    od:hasFlowType & tc:FlowType  \\
    od:hasFlowSubstance & tc:FlowSubstance  \\
    od:hasFlowFunctionality & tc:FlowFunctionality  \\
    od:hasFlowDefect & tc:FlowDefect  \\
    od:hasFlowSource & tc:FlowSource  \\
    od:hasMethodOfDescription & tc:MethodOfDescription \\\hline 
  \end{tabular}
\end{center}

\begin{center}
  \begin{tabular}{|l|p{10cm}|}\hline
    PropertyDomain & PropertyValues \\\hline
    tc:ModeValue & tc:System, tc:Model, tc:SystemAsIs, tc:SystemAsRequired,
    tc:ModelAsIs, tc:ModelAsRequired \\
    tc:FlowType & tc:ComplexFlow, tc:DiscreteFlow, tc:ContinuousFlow \\
    tc:FlowSubstance & tc:Information, tc:Energy, tc:Substance \\
    tc:FlowFunctionality & tc:UsefulFlow, tc:HarmfulFlow, tc:ParasiticFlow  \\
    tc:FlowDefect & tc:InsufficientFlow, tc:ExcessiveFlow,
    tc:BadControllableFlow, tc:AbsentFlow \\ 
    tc:FlowSource & tc:ExternalSource, tc:InternalSource  \\
    tc:MethodOfDescription & tc:TextDescription, tc:ParametricDescription,
    tc:GraphicDescription \\\hline 
  \end{tabular}
\end{center}

\subsection{Subconcepts of \texttt{tc:Flow}}

A flow consists of several parts (source, channel, receiver, control unit)
that are components by their own and hence modeled with the predicate
\texttt{od:subConceptOf}. For the moment they are attached to the flow with a
single predicate \texttt{od:hasStaticFlowComponent}.

\HGG{Ist es sinnvoll, mit Blick auf das Konzept eines „minimal arbeitsfähigen
  Flusses“, der alle Teile (source, channel, receiver, control unit) enthalten
  muss, hier das Prädikat weiter zu differenzieren?}

\subsection{\texttt{tc:FlowAnalysis}}

A flow analysis is an analytic process concerned with a certain \emph{flow}
(model) in a certain context.  The \emph{context} encapsulates all information
about the exact conditions of the analysis (Goals, Requirements, Description
Method).

The goal of a functional analysis is to create a plan for a flow
transformation based on the \emph{flow analysis rules} that can be applied to
the flow and improves it in the given context. Hence the target of a flow
analysis is a special document in a (yet hypothetic) Flow Transformation
Description Language.

The ontological representation of such a process is extremely difficult and is
not considered here. The ontology of the flow analysis is restricted to the
description of the \emph{formative elements of the analysis} and the creation
of a manually generated \emph{FlowTransformation} document.

\HGG{Ich denke, diese Einschränkung sollte gemacht werden. Es wäre weiter zu
  klären, ob \texttt{od:hasDescriptionMethod} sowohl einen Fluss als auch eine
Flussanalyse als \texttt{rdfs:domain} hat. In Ermangelung einer besseren Idee
habe ich einer \texttt{tc:FlowTransformation} den Typ \texttt{foaf:Document}
zugeordnet. }

\begin{thebibliography}{xxx}
\raggedright
\bibitem{Eckardt2020} Olga Eckardt (2020).  The ontologies “Flow Model” and
  “Flow Analysis”. Presentation (in Russian) at the TRIZ Ontology Webinar at
  Oct. 27, 2020.  An English summary and discussion is available at
  \url{https://wumm-project.github.io/2020-10-27}.
\bibitem{Graebe2021} Hans-Gert Gr\"abe (2021). About the WUMM modelling
  concepts of a TRIZ ontology.  \url{https://github.com/wumm-project/Leipzig-Seminar/blob/master/Wintersemester-2020/Seminararbeiten/Anmerkungen.pdf}.
\bibitem{Lebedyev2015} Yuri Lebedyev (2015). Improving flow analysis tools (in
  Russian).  English summary.
  \url{https://triz-summit.ru/certif/master/matriz/15/302897/}.
\bibitem{Petrov2007} Vladimir Petrov, Michail Rubin, Simon Litvin (2007).
  Fundamentals of TRIZ Theory of Inventive Problem Solving (in Russian).
  Reprinted in 2020. ISBN 978-5-4496-8183-6.
\bibitem{Souchkov2018} Valeri Souchkov (2018).  Glossary of TRIZ and
  TRIZ-related terms. Version 1.2.
  \url{https://matriz.org/wp-content/uploads/2016/11/TRIZGlossaryVersion1_2.pdf}. 
\bibitem{SKOS} SKOS -- The Simple Knowledge Organization System.
  \url{https://www.w3.org/TR/skos-reference/}.  
\bibitem{TOP} The TRIZ Ontology Project (TOP)
  \foreignlanguage{russian}{Онтология ТРИЗ} of the TRIZ Developer Summit.
  \url{https://triz-summit.ru/Onto_TRIZ/}.
\bibitem{TOP-Glossary} TRIZ 100 Glossary. A short glossary of key TRIZ
  concepts and terms (in Russian).
  \url{https://triz-summit.ru/onto_triz/100/}.
\bibitem{WUMM} The WUMM Project. \url{https://wumm-project.github.io/} 
\bibitem{WUMM-Ontology} The WUMM TOP Companion Project.
  \url{https://wumm-project.github.io/Ontology.html} 
\end{thebibliography}

\end{document}

