\documentclass[11pt,a4paper]{article}
\usepackage{ls}
\usepackage[main=english,russian]{babel}

\newenvironment{code}{\tt \begin{tabbing}
\hskip12pt\=\hskip12pt\=\hskip12pt\=\hskip12pt\=\hskip5cm\=\hskip5cm\=\kill}
{\end{tabbing}}
\def\dq{{\char34}}

\title{First remarks on the WUMM modelling of a TRIZ ontology}

\author{Hans-Gert Gr\"abe}

\date{March 5,  2021}

\begin{document}
\maketitle

\section{Background}

As already stated in the \emph{Handreichung zu den Seminararbeiten}, the
modelling to be carried out follows the concepts of the \emph{TRIZ Summit
  Ontology Project} (TOP) \cite{TOP} and should fit in the framework of the
\emph{WUMM Ontology Companion Project} \cite{WUMM-Ontology}.  In contrast to
the «large-scale ontologisations» within TOP, which models the basic
interrelationships of different parts of TRIZ theory, the themes of the
seminar papers are concerned with semantic modelling of different TRIZ
sub-areas in more detail.

\section{Abstraction levels of modelling}

An ontology is about «modelling of models», because the clarification of terms
and concepts aimed at with an ontology is intended to be practically used in
real-world modelling contexts.  This «modelling of models» references a
typical engineering context, in which the \emph{modelling} of systems plays a
central role and serves as basis of further planned action (including project
planning, implementation, operation, maintenance, further development of the
system).

In this process, \emph{several levels of abstraction} are to be distinguished,
which in TOP is not sufficiently worked out. These are
\begin{itemize}[noitemsep]
\item [0.] The level of the \emph{real world system} to which the engineering
  task refers. This level is only \emph{practically} accessible. The model to
  be developed at level 1 must be appropriate to cover all problems arising in
  the process of development and use of the system and express the inherent
  contradictoriness of the system.

  This contradictory nature of the system can be formulated only in language
  form, i.e. on the model level and \emph{applying} the concepts available
  there.  These concepts must therefore not only be able to describe the
  system itself, but also cover a description of the necessary aspects of its
  operation.
\item[1.] The \emph{level of modelling} the system. The ontology provides the
  language means, concepts (RDF subjects) and properties (RDF predicates),
  which are to be \emph{applied} at this level. This level is also the
  \emph{level of methodological practice}.
\item[2.] The \emph{level of the meta-model} as the actual ontology level on
  which the systemic concepts are \emph{defined}. This definition is processed
  \emph{applying} the methodological concepts whose linguistic means are made
  available on meta-level 2.
\item [3.] The \emph{modelling meta-level 2} at which the methodological
  concepts are defined.
\end{itemize}

In the examples below, three namespaces play a role:
\begin{itemize}[noitemsep]
\item \texttt{ex:} as the namespace of a special model (level 1). 
\item \texttt{tc:} as the namespace of TRIZ concepts (level 2 subjects).
\item \texttt{od:} as the namespace of WUMM's own concepts
  (methodological subjects, level 2 predicates). 
\end{itemize}

\section{Basics of the WUMM Ontology Project}

The WUMM ontology project accompanies the TOP activities in order 
\begin{enumerate}[noitemsep]
\item to carry out a remodelling according to semantic standards,
\item to enhance the material multilingually and
\item to build an LOD infrastructure on this basis,
\end{enumerate}
and thus to improve the basis for the necessary social coordination processes.

For this purpose, the SKOS ontology \cite{SKOS} is used with the concepts (K)
\begin{itemize}[noitemsep]
\item \texttt{skos:Concept}, \texttt{skos:prefLabel}, \texttt{skos:altLabel}
  -- concept naming
\item \texttt{skos:definition}, \texttt{skos:example}, \texttt{skos:note} --
  concept properties
\item \texttt{skos:narrower}, \texttt{skos:broader} -- concept relations.
\end{itemize}
It provides an initial descriptive framework for conceptualisations.  For the
meaning of the individual concepts, please refer to \cite{SKOS}.

\subsection{URIs and Namespaces}

One of the central problems of transferring the existing data stocks on TRIZ
concepts is the allocation of meaningful URIs, since the individual glossary
entries in the existing sources are identified solely by their labels.  The
OSA platform is no exception to this since the URIs assigned there (both for
the nodes and the edges of the constructed RDF graph) are not publicly
visible.

During the computer based transformation of the datasets into a valid RDF
format, URIs were automatically generated for all concepts, which are located
in the namespace \texttt{tc:} (like \emph{TRIZ Concepts}). An essential task
still to be done is the unification of these URIs, i.e. merging different URIs
that refer to the same concept. 

\subsection{Provenance of Explanations}

A further problem of this ontological modelling is the representation of the
provenance of the individual explanations. For this purpose the SKOS concepts
listed under (K) were replaced for each individual source by notations from
the namespace \texttt{od:} in order to first identify the «worlds» of the
individual authors and TRIZ schools separately.

\texttt{od:} is the namespace used by the WUMM project to develop its own
concepts.  A more detailed description of this namespace in a separate RDF
file is still pending, the concepts represented by the URIs have so far only
been agreed verbally.

Corresponding notation variations are for example
\begin{itemize}[noitemsep]
\item \texttt{skos:Concept} $\to$ \texttt{od:GSAThesaurusEntry},
  \texttt{od:VDIGlossaryEntry} \ldots
\item \texttt{skos:definition} $\to$ \texttt{od:SouchkovDefinition},
  \texttt{od:VDIGlossaryDefinition} \ldots
\item \texttt{skos:example} $\to$ \texttt{od:VDIGlossaryExample} \ldots
\end{itemize}
etc. See the RDF data itself, which can be accessed via the SPARQL endpoint
\begin{center}
  \url{http://wumm.uni-leipzig.de:8891/sparql}
\end{center}
of the WUMM project.

The \emph{use of different predicate identifiers} allows easily to indicate
the origin -- e.g. of definitions -- to different sources at the level already
of triples:
\begin{center}
  (term X) -- (is defined in source A as) -- (definition of X in
  source A)\\ 
  (term X) -- (is defined in source B as) -- (definition of X in
  source B) 
\end{center}
If \texttt{skos:definition} were used consistently, more complex constructs
\begin{center}
  (term X) -- \texttt{skos:definition} -- \Big[\parbox{8cm}{\centering
      (source A) -- (definition of X in source A)\\
      (source B) -- (definition of X in source B)}\Big]
\end{center}
had to be used. Although this is the final goal of the exercise, in the
current state of the modelling, the simpler form will be used, even though
this inflates the number of predicates. A consolidation towards the «ideal
final result» can be achieved later on by a simple model transformation.  

The same applies to the use of provenance-dependent subclasses of
\texttt{skos:Concept}.  Since the matching of the URIs assigned to concepts
must be carried out across subclasses, the instances of each subclass are also
marked as \texttt{skos:Concept}, although this could also be done via a
general sentence such as
\begin{center}\tt
  od:VDIGlossaryEntry rdfs:subClassOf skos:Concept
\end{center}
and then determined by inference. We proceed this way since inferencing
belongs to the more complex semantic OWL concepts and is not supported by
simple RDF stores. Again, the data can be consolidated later on by a simple
model transformation.  

\section{Typical modelling situations}

\subsection{Morphological box}

A Morphological box describes a concept that can take a finite number of fixed
values (a \text{enum} data type in Java).

\emph{Example:}  Colour (red, green, yellow, blue).

\begin{code}
  ex:MeiersCar od:hasColour tc:green . \\[4pt]
    
  od:hasColour a rdfs:Property;\\
    \>rdfs:domain tc:Flow;\\
    \>rdfs:range tc:Colour .\\[4pt]

  tc:Colour a skos:Concept, od:AdditionalConcept ;\\
  \>skos:prefLabel {\dq}Color{\dq}@en, {\dq}Farbe{\dq}@de ;\\
  \>od:allowedValues tc:red, tc:green, tc:yellow, tc:blue .\\[4pt]

  tc:green a skos:Concept, od:AdditionalConcept ;\\
  \>od:valueOf tc:Colour ; \\
  \>skos:prefLabel {\dq}green{\dq}@en, {\dq}grün{\dq}@de .\\[4pt]
  ...  
\end{code}

Considered as datatypes \texttt{tc:green} is an instance of
\texttt{tc:Colour}.  Since both \texttt{tc:Colour} and \texttt{tc:green} are
modeled as \texttt{skos:Concept} the properties \texttt{od:allowedValues} and
\texttt{od:valueOf} are subproperties of \texttt{skos:narrower} and
\texttt{skos:broader}\footnote{From \cite{SKOS-Primer}: The subject of a
  \texttt{skos:broader} statement is the more specific concept involved in the
  assertion and its object is the more generic one.} and used to indicate the
relation between attribute type and attribute value type

In the TOP model, properties of classes are modelled only sparsely.  For
example, a flow can be a \emph{discrete flow}. This is a value of a property
of this flow. To model such a relationship, the \emph{property itself}, the
\emph{range of values of the property}, and the \emph{possible values} of this
range must be modelled.

\begin{code}
  ex:MeiersFlow od:hasFlowType tc:discreteFlow . \\[4pt]
  
  od:hasFlowType a rdfs:Property;\\
    \>rdfs:domain tc:Flow;\\
    \>rdfs:range tc:FlowType .\\[4pt]

  tc:FlowType a skos:Concept, od:AdditionalConcept ;\\
    \>od:allowedValues tc:complexFlow, tc:discreteFlow, tc:continousFlow ;\\
    \>skos:prefLabel {\dq}Type of flow{\dq}@en, {\dq}Flussart{\dq}@de ;\\
    \>skos:altLabel {\dq}Art der Strömung{\dq}@de .\\[4pt]
  
  tc:discreteFlow a skos:Concept, od:AdditionalConcept;\\
    \>od:valueOf tc:FlowType ; \\
    \>skos:prefLabel {\dq}discrete flow{\dq}@en, {\dq}diskreter Fluss{\dq}@de .\\[4pt] 
  ...  
\end{code}

All this is hard to distinguish from a class hierarchy, which should be
modelled with the predicate refinement \texttt{od:hasSubConcept} of
\texttt{skos:narrower}.  \texttt{od:subConceptOf} as the inverse predicate is
also used for reasons of clarity; however, it can also be added later
automatically.

\begin{code}
  ex:MeiersFlow od:hasStaticFlowComponent tc:Pump . \\[4pt]
  
od:hasStaticFlowComponent a rdfs:Property;\\
    \>rdfs:domain tc:Flow;\\
    \>rdfs:range tc:StaticFlowComponent .\\[4pt]
  
tc:StaticFlowComponent\\
    \>od:hasSubConcept tc:ControlUnit, tc:Receiver, tc:Source, tc:Channel ;\\
    \>a skos:Concept, od:AdditionalConcept ;\\
    \>skos:prefLabel {\dq}static components of the flow{\dq}@en,\\
    \>\>{\dq}statische Flusskomponenten{\dq}@de .\\[4pt]

tc:ControlUnit\\
    \>od:subConceptOf tc:StaticFlowComponent ;\\
    \>od:hasSubConcept tc:Pump, tc:Valve ;\\
    \>skos:prefLabel {\dq}Control Unit{\dq}@en, {\dq}Steuerungssystem{\dq}@de ;\\
    \>skos:altLabel {\dq}Management System{\dq}@en, {\dq}Managementsystem{\dq}@de .\\[4pt]

tc:Pump\\
    \>od:subConceptOf tc:ControlUnit ;\\
    \>skos:prefLabel {\dq}pump{\dq}@en, {\dq}Pumpe{\dq}@de ;\\
    \>a skos:Concept, od:AdditionalConcept .
\end{code}

First of all, it would have to be decided whether \texttt{tc:Pump} should be
included in the set of concepts or whether the term is too specific.  This is
of course a question of taste.

Modelling via class hierarchies has the advantage that one can use as object
of the predicate \texttt{od:hasStaticFlowComponent} instances of the entire
class hierarchy. But its exact type can then only be determined by a second
query to the model.  Alternatively, in this example one could have used
predicates \texttt{od:hasStaticFlowComponent} and \texttt{od:hasControlUnit}
in order to distinguish the object type already from the predicate identifier.


\begin{thebibliography}{xxx}
\raggedright
\bibitem{SKOS} SKOS -- The Simple Knowledge Organization System.
  \url{https://www.w3.org/TR/skos-reference/}.  
\bibitem{SKOS-Primer} SKOS Simple Knowledge Organization System Primer.
  \url{https://www.w3.org/TR/2009/NOTE-skos-primer-20090818/}.  
\bibitem{TOP} The TRIZ Ontology Project (TOP)
  \foreignlanguage{russian}{Онтология ТРИЗ} of the TRIZ Developer Summit.
  \url{https://triz-summit.ru/Onto_TRIZ/}.
\bibitem{WUMM} The WUMM Project. \url{https://wumm-project.github.io/} 
\bibitem{WUMM-Ontology} The WUMM TOP Companion Project.
  \url{https://wumm-project.github.io/Ontology.html} 
\end{thebibliography}


\end{document}
