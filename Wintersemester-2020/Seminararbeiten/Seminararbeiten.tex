\documentclass[11pt,a4paper]{article}
\usepackage{ls}
\usepackage[german]{babel}

\title{Themen für Seminararbeiten\\ im Modul \emph{Angewandte Informatik}\\ im
  Wintersemester 2020/21}

\author{Hans-Gert Gr\"abe}

\date{14. November 2020}

\begin{document}
\maketitle

\section{Hintergrund}

Im Bereich der TRIZ-Forschung wurde vor einem Jahr ein
TRIZ-Ontologie-Projekt\footnote{Siehe \url{https://triz-summit.ru/onto_triz/}
  (in Russisch).}  begonnen, um die verschiedenen Teile der TRIZ-Theorie
ontologisch zu „kartieren“ und die wesentlichen Zusammenhänge zwischen den
einzelnen Teilen mit Mitteln semantischer Technologien zu erfassen. Hierzu
liegen aktuell vor
\begin{itemize}
\item[(1)] eine „Übersichtskarte“ der verschiedenen Theoriefelder und der
  Zusammenhänge zwischen diesen,
\item[(2)] ein „Atlas“ von Ontokarten, die grob einzelne Bereiche markieren,
  die weiter zu detaillieren sind,
\item[(3)] zu einzelnen dieser Ontokarten erste Versuche, Struktur in das
  begriff\-liche Chaos zu bringen,
\item[(4)] ein Glossar (oder auch nur ein Thesaurus) von Begriffen, die
  hierfür wichtig sind.
\end{itemize}

Während zu (1) und (2) weitgehend Konsens besteht, sind die Modellierungen in
(3) stark umstritten, da die entsprechenden Semantiken und Zusammenhänge in
den unterschiedlichen TRIZ-Schulen naturgemäß unterschiedlich verstanden
werden.

Streit gibt es auch zu (4), der aber deutlich einfacher zu klären ist, wenn 
\begin{itemize}
\item[(4a)] zunächst einmal alle Begriffe gesammelt und „URIfiziert“ werden
  (Thesaurus raw),
\item[(4b)] URIs so weit zusammengeführt werden, dass verschiedene URIs auf
  verschiedene Konzepte verweisen, aber Raum bleibt, für gleiche Konzepte
  verschiedene Semantiken zu hinterlegen (Thesaurus final),
\item[(4c)] diese verschiedenen Semantiken auch wirklich zusammengetragen und
  formalisiert werden (Glossar raw) und schließlich
\item[(4d)] die Semantiken in einem komplexen sozialen Abstimmungsprozess so
  weit wie \emph{möglich} abgeglichen und essenzielle Differenzen semantisch
  modelliert werden. 
\end{itemize}
In \cite{Kuryan2019} wird das Projekt genauer beschrieben, in
\cite{Kuryan2020} der aktuelle Stand dargestellt. Andere Vorarbeiten fanden
wenig Berücksichtigung\footnote{Details dazu auf der Webseite
  \url{https://wumm-project.github.io/Ontology.html}.}, insbesondere weder
\cite{IDM2011} noch \cite{VDI}. Im Herbst 2020 läuft eine Webinarreihe (in
Russisch), deren Materialien auch verfügbar\footnote{Siehe
  \url{https://wumm-project.github.io/OntologyWebinar}.} und teilweise ins
Englische übersetzt sind.

Arbeitsgrundlage ist ein Glossar von V. Souchkov \cite{SG}, wobei dessen
Einträge sowohl drei TRIZ-Generationen (TRIZ-1..3) als auch 5 Kategorien
(Basisbegriffe, Modelle, Regeln, Begriffsgruppen, Synonyme) zugeordnet
werden. Ein Webinarteilnehmer wies auf einen weiteren Thesaurus auf den
einschlägigen russischen Altschullerseiten hin, der bereits multilingual ist.

Im Rahmen des WUMM-Projekts wurden und werden Teile dieser Ontologisierungen
nachmodelliert\footnote{Siehe dazu das Verzeichnis \texttt{Ontologies} im
  github-Repo \url{https://github.com/wumm-project/RDFData}}, die im Original
bisher ausschließlich durch grafische Ontogramme sowie die Möglichkeit einer
visuellen Inspektion im verwendeten OSA-Ontologie-Editor\footnote{Siehe
  \url{https://onto.devtas.ru/ts2o1} (in Russisch).} zugänglich sind.  Diese
im Original ausschließlich russischsprachigen Quellen wurden dabei in Teilen
auch ins Englische und Deutsche übertragen.  Weitgehend semantisch erfasst
sind (1) und (2). Weiterhin wurde bereits früher das VDI-Glossar „RDFiziert“
und die dort vorhandenen deutschen und englischen Erläuterungen um eine
russische Übersetzung ergänzt. Dies sowie der einfach zu transformierende
Thesaurus auf den Altschullerseiten bilden die Grundlage für einen eigenen
Thesaurus nach (4a), der mit den Begriff\-lichkeiten des Originalprojekts
weiter abzugleichen ist.  Das Glossar \cite{SG} liegt als pdf, nicht aber als
maschinenlesbare Datei vor. Da mit unserem aktuellen WUMM-Thesaurus bereits
eine gute Abdeckung der Begriff\-lichkeiten erreicht ist und \cite{SG} auch
erst in Phase (4c) komplett relevant wird, wird auf weitere Arbeit an dieser
Front zunächst verzichtet.

\section{Themen für Seminararbeiten}

In den Seminararbeiten soll am Punkt (3) weitergearbeitet werden, indem für
eine konkrete Ontokarte $X$ 
\begin{itemize}
\item[(A)] die Zusammenhänge für das WUMM-Projekt in RDF in einer gemeinsamen
  Rahmensetzung nachmodelliert werden sowie
\item[(B)] differierende Semantiken, Probleme und Widersprüche im Verständnis
  der modellierten TRIZ-Konzepte zusammengetragen und systematisiert werden.  
\end{itemize}
Während (A) primär einen ingenieur-technischen Charakter hat, erfordert (B)
stärker einen akademischen Zugang von Recherche und Vergleich einschlägiger
Publikationen.

Die Seminararbeit soll in {\LaTeX} erstellt werden und möglichst in Englisch
verfasst sein.

Mögliche Themen ($X$, die Liste ist nicht final) sind
\begin{itemize}[noitemsep]
\item Funktionale Analyse
\item Stoff-Feld- und Element-Feld-Modelle
\item Flussmodelle und Flussanalyse
\item Theorie der Entwicklung eines schöpferischen Vorstellungsvermögens
\item Der TRIZ-Modellbegriff
\item Modellierung des IDM-Modells aus Vorgängerarbeiten (insbesondere
  \cite{IDM2011})
\end{itemize}

\begin{thebibliography}{xxx}
\bibitem{IDM2011} D. Cavallucci, F. Rousselot, C. Zanni (2011). An ontology
  for TRIZ. Proc. TRIZ Future Conference 2009. Procedia Engineering 9,
  251–260.  \url{https://doi.org/10.1016/j.proeng.2011.03.116}.
\bibitem{Kuryan2019} A. Kuryan, V. Souchkov, D. Kucharavy (2019). Towards
  ontology of TRIZ. TRIZ Developers Summit, Minsk 2019.
  \\ \url{https://wumm-project.github.io/Texts/Ontology-TDS2019-en.pdf}
\bibitem{Kuryan2020} A. Kuryan, M. Rubin, N. Shchedrin, O. Eckardt, N. Rubina.
  TRIZ Ontology. Current State and Perspectives. TRIZ Developers Summit 2020
  (in Russisch). Englische Übersetzung:
  \\ \url{https://wumm-project.github.io/Texts/Ontology-TDS2020-en.pdf}
\bibitem{SG} V. Souchkov. Glossary of TRIZ and TRIZ-related terms. Mehrere
  Versionen seit 1991. Letzte Version von 2014.
  \\ \url{http://www.xtriz.com/publications/TRIZGlossaryVersion1_0.pdf}.
\bibitem{VDI} VDI Richtlinie 4521. Erfinderisches Problemlösen mit TRIZ.\\
  Blatt 1: Grundlagen und Begriffe (April 2016).\\ Blatt 2: Zielbeschreibung,
  Problemdefinition und Lösungspriorisierung (April 2018).\\ Blatt 3:
  Lösungssuche (Juli 2020). 
\end{thebibliography}

\end{document}
