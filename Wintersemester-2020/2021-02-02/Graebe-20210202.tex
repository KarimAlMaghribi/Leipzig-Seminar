\documentclass[11pt,a4paper]{article}
\usepackage{ls}
\usepackage[ngerman,english]{babel}

\title{OTSM-TRIZ\\[6pt]\Large Diskussionsgrundlage für das Seminar am
  02.02.2021}

\author{Hans-Gert Gr\"abe}

\date{31. Januar 2021}

\begin{document}
\maketitle

\section{Vorbemerkungen}

TRIZ ist ein Theoriekomplex, der ingenieur-technische Erfahrungen
systematisiert und verallgemeinert. Wir hatten uns in mehreren Seminaren damit
auseinandergesetzt, in welchem Sinne diese Verallgemeinerungen und
Systematisierungen den Charakter von \emph{Gesetzen} haben.

Natürlich ist klar, dass es wenig weise ist, diesen Erfahrungsschatz und
dessen Aufarbeitung unberücksichtigt zu lassen, sondern ingenieur-technisches
Handeln sich an diesem Erfahrungs\-schatz orientieren sollte. Wir hatten auch
herausgearbeitet, dass \emph{institutionalisierte Verfahrensweisen} hierfür --
ähnlich wie juristische Gesetze -- verpflichtende gesellschaftliche Rahmen
setzen, die Ingenieure zu \emph{begründetem Handeln} zwingen. Die Frage nach
dem Charakter jener Verallgemeinerungen und Systematisierungen ist damit Teil
der Frage nach der \emph{Struktur jener Begründungszusammenhänge} als
Bindeglied zwischen dem \emph{gesellschaftlich verfügbaren Verfahrenswissen}
und dem \emph{privaten Verfahrenskönnen}. In dieser Frage bündet sich damit
das Wechselverhältnis aller drei Komponenten des in der Vorlesung entwickelten
Wissensbegriffs.

Ähnliche Strukturen finden sich auch in der Wissenschaft, aus deren sozialen
Praxen wir unser Verständnis des Gesetzesbegriffs genommen hatten.  Auch in
den sozialen Praxen der Wissenschaft spielen Begründungszusammenhänge eine
wichtige Rolle, wenn es darum geht, private neue Einsichten in allgemein
anerkanntes gesellschaftliches Wissen zu transformieren.  Allgemein
anerkanntes gesellschaftliches Wissen und private neue Einsichten stehen dabei
in einem koevolutiven Spannungsverhältnis (dazu etwa Thomas S. Kuhn
\cite{Kuhn1962}), das -- im Luhmannschen Sinne -- durch die Codes des
Wissenschaftssystem sozial vermittelt wird.  Nach Berger/Luckmann
\cite{Berger1969} spielt dabei weniger der Begriff \emph{Wahrheit} eine Rolle
als vielmehr die \emph{Legitimität von Sinndeutungen} als
Institutionalisierung von Verkehrsformen im Fortschreiten des
Erkenntnisprozesses der Menschheit als Gattung.

Damit unterscheiden sich aber die sozialen Praxen von Wissenschaft und
ingenieur-techni\-schem Handeln grundlegend und es kann nicht überraschen,
dass der Gesetzesbegriff in der Wissenschaft und in einem
ingenieur-technischen Kontext unterschiedlich zu fassen sind. Beides hängt
zwar mit der Notwendigkeit zusammen, die eigenen Praxen durch valide
begründete \emph{Beschreibungsformen} zu begleiten, die akzeptierten
interpersonalen Standards genügen. Die Rigorosität derartiger Begründungen
unterscheidet sich bereits innerhalb der Wissenschafts\-sphäre -- Herr Thoke
zitierte Scott Aaronson mit der Anmerkung „Was Physiker ein Gesetz nennen,
nennen Mathematiker eine Vermutung.“

Petrovs Unterscheidung von Gesetz (notwendiges, wesentliches, nachhaltiges,
sich wiederholendes Phänomen) und Gesetzmäßigkeit (Existenz und Entwicklung in
Übereinstimmung mit den Gesetzen) legt einen Zusammenhang zwischen
Begründungselementen (Gesetz) und (vernünftiger) Entwicklung durch begründetes
Handeln (Gesetzmäßigkeiten) nahe.  Ingenieur-technisches Handeln in einem
kooperativen Kontext ist mit einem Begründungserfordernis verbunden, das
dieses Handeln auf Grundsätze zurückführt, die in diesem Kontext anerkannt
sind. Nur auf diese Weise kann Handeln von Dritten nachvollzogen und deren
eigenes Handeln anschlussfähig gestaltet werden.

Wir bewegen uns mit einem solchen Ansatz noch stärker im Kontext von
\cite{Berger1969}, allerdings mit der besonderen Betonung einer
Kontextualisierung \emph{innerhalb} kooperativer Zusammenhänge, womit diese
„legitimen Sinndeutungen“ (als \emph{kontextualisierte} „Gesetze“) selbst
stratifiziert sind: Als legitim innerhalb eines Kontexts anerkannte
Sinndeutungen bedürfen einer umfassenderen (oder auch nur anderen)
Legitimation und damit auch weiterer Begründung in anderen Kontexten.

Genau auf diese Weise war aber der große Korpus von Begründungen in
\cite{Goldovsky1983} aufgebaut, wenn man ihn von hinten nach vorn liest. Es
wird ein mehrschichtiges Begründungsuniversum aufgebaut, in dem die
verschiedenen „Muster“ und „Gesetzmäßigkeiten“ (hier im Sinne von Goldovsky
verstanden) technischer Entwicklung (Abschnitt 3-6) in
Begründungszusammenhänge eingebettet werden, die auf „allgemeinere Gesetze“ im
Sinne legitimer Sinndeutungen auf allgemeinerer Ebene verweisen, die
ihrerseits in Begründungszusammenhänge in noch allgemeineren Kontexten
eingebettet sind. 

Im Unterschied zu den „Gesetzen (ingenieur)-technischer Entwicklung“, die wie
moderne Technikpraxen als deren Grundlage erst jüngeren Datums sind, gehören
jene nicht weiter begründeten „allgemeinen Gesetze“ bereits seit Jahrhunderten
zum diskursiven Begründungs\-kanon und sind damit auch scheinbar diskursiv
befestigt, wie eine Vielzahl von Monografien zu jenen Themen nahelegt.
Allerdings steht die Frage, wie sich solche Begründungserfordernisse mit dem
Fortschreiben menschlicher Praxen weiterentwickeln und ob nicht massive
Wechsel in Basistechnologien, wie sie etwa den digitalen Wandel begleiten,
auch zu Umbrüchen in den institutionalisierten Formen jener
Begründungserfordernisse führen.  \cite{Goldovsky1983} ist selbst ein Beleg
für diese Weiterentwicklung von Begründungszusammenhängen.

\section{OTSM-TRIZ und dessen Konzeptualisierungen}

In der Vorlesung am 12.11.2020 \emph{Modellierung widersprüchlicher
  Anforderungen in der TRIZ} bin ich bereits auf OTSM-TRIZ näher eingegangen
und möchte mich hier nur auf den Aspekt beschränken, in welchem Umfang  dort
die Spezifika ingenieur-technischen Handelns berücksich\-tigt sind.

Im Gegensatz zu anderen TRIZ-Schulen ist OTSM-TRIZ auch stärker
theoretisch-philoso\-phisch fundiert. Dabei spielen drei \emph{Axiome} eine
zentrale Rolle, in denen der Gesetzesbegriff wiederum eine zentrale Rolle
spielt.

\newpage
\subsection{Key Problem of OTSM-TRIZ}

\begin{center}
  Quelle: N. Khomenko, R. De Guio. 2010. OTSM System of Axioms
\end{center}

In order to be universal, the rules of problem solving methods should be as
general as possible. But the more general the rules of the problem solving
are, the more general and the less practical the solution will be.

And vice versa: when the rules (and methods) are specific and precise, they
are helpful for solving a specific problem which is of the practical use.
However, the more specific they are the less universal they are as well.

\subsection{World Axioms}

\textbf{Axiom of Unity.} The world is a whole and unique system that evolves
in accordance with objective laws of all the sub-systems.

\textbf{Axiom of Disunity.} The world is a set of different systems, each of
them evolving in accordance with its specific laws.

\textbf{Axiom of Connectedness of Unity and Disunity.} The way the law is
manifested in a specific situation is defined by its resources.

\textbf{Consequences:} 
\begin{itemize}[noitemsep]
\item Unity and diversity of the world are governed by the resources used by
  different systems. Any resource is subject to both general laws and specific
  laws defined by their specific properties.
\item General objective laws are manifested differently in specific
  situations. This difference depends on the nature of the interplay between
  the law and the specificity of the situation.
\end{itemize}

\textbf{Axiom of Description.} There are different ways to describe the world
around us.  There is an infinite number of ways to describe the world.
\begin{quote}
  Nobody is wrong! Everybody describes their perception about something from
  their own standing point.
\end{quote}

\textbf{Axiom of the Process.} Any element should be seen as a process and
vice versa.  This process, which is linked with a human being as soon as we
are in a problem solving context, evolves in accordance with objective laws
and takes into account specific objective and subjective factors.
\begin{quote}
  Modern approaches in system and business engineering related to processes:
  Technology maps; Flow models; Business process models; Product lifecycle
  models; Phylogenies \& ontogenies in the system operator (M. Rubin)
\end{quote}

\subsection{Axioms of Thinking}

\textbf{Axiom of Impossibility.} In order to overcome psychological inertia
during a problem solving process, it is necessary to accept (temporarily) the
assertions, the logical value of which seems „false” at a first glance, and
analyze the consequences of these assertions.

\textbf{Axiom of the Core of any Problem.} Any problem can be stated as a
contradiction between our subjective desires for something appearing in a
specific context on the one hand, and objective laws that cause this specific
situation, one the other hand.

\subsection{Axiom of Independent Observers}

Any perceived problem is a transcription of a situation from the point of view
of the person who is involved in the problem. In order to overcome the
problematic situation; it is necessary to get out of the role of the problem
„owner” and analyze the situation from different points of view.
\begin{enumerate}[noitemsep]
\item The point of view of the problem solver, namely, the person directly
  working on the problem.
\item The point of view of the regulator, namely, a person who checks the
  formal side of the application of rules of OTSM methods and technologies.
\item The point of view of the judge who tries to understand the disagreements
  between the problem solver and the regulator.
\item The point of view of the referee who tries to understand the world
  vision of the problem solver, the regulator and the judge when they
  interact.
\end{enumerate}
\begin{quote}  
  Modern approaches to stakeholder analysis in system engineering and TRIZ:
  Stakeholder analysis; Analysis of Stakeholders’ requirements; Contradiction
  in requirements (M. Rubin).
\end{quote}

\begin{thebibliography}{xxx}
\bibitem{Berger1969} Peter L. Berger, Thomas Luckmann (1969). Die
  gesellschaftliche Konstruktion der Wirklichkeit: eine Theorie der
  Wissenssoziologie. ISBN 978-3-596-26623-4.
\bibitem{Goldovsky1983} Boris I. Goldovsky (1983). System der
  Gesetzmäßigkeiten des Aufbaus und der Entwicklung technischer Systeme.
\bibitem{Kuhn1962} Thomas S. Kuhn (1962). Die Struktur wissenschaftlicher
  Revolutionen. Deutsch: Frankfurt/M. 1967.
\end{thebibliography}
\end{document}

\bibitem{Galanakis2006} Kostas Galanakis (2006).  Innovation process. Make
  sense using systems thinking.  In: Technovation Volume 26, Issue 11,
  p. 1222--1232.
\bibitem{Graebe2020} Hans-Gert Gräbe (2020). Die Menschen und ihre Technischen
  Systeme. LIFIS Online, 19.05.2020.
  \url{http://dx.doi.org/10.14625/graebe_20200519}
\bibitem{TESE2018} Alex Lyubomirsky, Simon Litvin, Sergei Ikovenko u.a.
  (2018).  Trends of Engineering System Evolution (TESE).  TRIZ Consulting
  Group. ISBN 9783000598463.
\bibitem{Preez2006} Niek D Du Preez, Louis Louw, Heinz Essmann (2006). An
  innovation process model for improving innovation capability.  Journal of
  high technology management research, vol 17, 1--24.
\end{thebibliography}
\end{document}
