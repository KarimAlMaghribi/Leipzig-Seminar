\documentclass[aspectratio=169]{beamer}
\usepackage[ngerman]{babel}


\title{Evolution Technischer Systeme bei B.I. Goldovsky}
\author{Tarek Stelzle}
\date{15. Dezember 2020}

\begin{document}
    \begin{frame}
        \maketitle
    \end{frame}

    \begin{frame}{Ansichten von Goldovksy}
        \begin{itemize}
            \item Komplexität des System der Gesetzmäßigkeit \newline
            {\scriptsize Gesetze der Konstruktion und Evolutionsgesetze}
            \item Deduktive Begründung von Gesetzmäßigkeiten
            \item Wichtigkeit der Gesetze der Konstruktion zur Arbeitssicherheit \newline
            {\scriptsize Bedingungslose Folgen bei Nichteinhaltung der Gesetze der Konstruktion}
        \end{itemize}
    \end{frame}

    \begin{frame}{Unterteilung}{Gesetze der Konstruktion}
        \begin{itemize}
            \item Primär nützliche Funktionen \newline
            {\scriptsize Bedürfnisse des Soziums}
            \item Elementar nützliche Funktionen \newline
            {\scriptsize Ausführbarkeit der PNF}
        \end{itemize}
    \end{frame}

    \begin{frame}{Begrifflichkeiten}{Gesetze der Konstruktion}
        \begin{itemize}
            \item Funktionelle Nische \newline
            {\scriptsize Gruppierung der ENF und deren Zusammensetzung in den PNF}
            \item Funktionale Vollständigkeit \newline
            {\scriptsize Realisierung aller Teilsysteme zur Ausführung der PNF}
        \end{itemize}
    \end{frame}

    \begin{frame}{Gesetz der strukturellen Vollständigkeit eines TS}
        \begin{center}
            Die Gesamtheit der Elemente der Struktur und die Wechselwirkungen zwischen ihnen müssen den Durchsatz der natürlichen Flüsse (Stoff,
            Energie und/oder Information) zu den notwendigen Teilen des Systems sichern
            sowie eine solche Umwandlung dieser Ströme, dass alle elementaren Funktionen
            des Systems erfüllt werden.
        \end{center}
        \vfill
        \begin{center}
            Gewährleistung der Ausführbarkeit der elementar nützlichen Funktionen
        \end{center}
    \end{frame}

    \begin{frame}{Begrifflichkeiten}{Wirkprinzipien}
        \begin{itemize}
            \item Wirkprinizip \newline
            {\scriptsize natürliche Prozesse, Effekte und Erscheinungen der Teilsysteme}
        \end{itemize}

        \vfill

        \begin{itemize}
            \item Funktionell-parametrische Nische \newline
            {\scriptsize Verfeinerung der funktionellen Nische}
        \end{itemize}
    \end{frame}

    \begin{frame}{Erweiterung um PNF}{Wirkprinzipien}
        \begin{itemize}
            \item Funktionsänderung des TS aufgrund von Parametern
            \item PNF und grundlegendes (zentrales) Untersystem
        \end{itemize}
        \vfill
        \textbf{Beispiel:} \newline
        Primär nützliche Funktion: \hfill Transport von Gütern auf der Wasseroberfläche \newline
        grundlegendes (zentrales) Untersystem: Fahrzeug auf der Wasseroberfläche halten
    \end{frame}

    \begin{frame}{Vorteile der Wirkprinzipien}{Wirkprinzipien}
        \begin{itemize}
            \item Vollständiges Bild der realen Möglichkeiten der technologischen Effekte
            \item Zwingendes Hervorheben der wirklich wesentlich Parametern
            \item Veränderung eines quantitativen Indikators erfordet wahrscheinlich eine Änderung des Wirkprinizip
        \end{itemize}
    \end{frame}

    \begin{frame}{Anwendungsbedingungen}
        \begin{itemize}
            \item PNF muss qualitativ und qunatitaiv den Anfordungen des Soziums und/oder des technischen Umfelds entsprechen
            \item Gewährleistung der Stabilität des Funktionierens
            \item Gewährleistung des erforderlichen Grad der Steuerbarkeit
            \item Gewährleistung der Benutzerfreundlichkeit des technischen Systems
        \end{itemize}
    \end{frame}

    \begin{frame}{Erst- und Weiterentwicklung}{Entwicklung technischer Systeme}
        \begin{itemize}
            \item Erstentwicklung \newline
            {\scriptsize Beachten der parametrischen Schwellenwerte}
            \item Weiterentwicklung \newline
            {\scriptsize Anpassen der quantitativen Parameter im Arbeitszustand}
        \end{itemize}
        \vfill
        \begin{itemize}
            \item Funktionelle parametrische Schwelle \newline
            {\scriptsize quantitatives Parameter Niveau zur Aberkennung des Status \glqq Prototyp\grqq}
        \end{itemize}
    \end{frame}

    \begin{frame}{Entwicklungsstufen}{Entwicklung technischer Systeme}
        \begin{enumerate}
            \item Anfangsettape \newline
            {\scriptsize Sicherstellung der Arbeitsfähigkeit}
            \item Abstimmung
            \begin{enumerate}
                \item Schwellanabstimmung (Konjugation) \newline
                {\scriptsize Betriebsfähigkeit des technischen Systems}
                \item Optimierungsabstimmung \newline
                {\scriptsize Optimierung des technischen Systems}
            \end{enumerate}
        \end{enumerate}
    \end{frame}

    \begin{frame}{Konformität von Struktur und Funktion}
        \begin{itemize}
            \item Ergebnis der Konjugation
            \item Wichtiges Gesetz der Konstruktion \newline
            {\scriptsize Übereinstimmung der Funktion des TS mit der Struktur}
            \item Nachweis widersprüchlich
            \begin{itemize}
                \item Verfahren der Synthese mit uneindeutigem Ausgang
                \item Analytisches Verfahren mit eindeutigem Ergebnis
            \end{itemize}
        \end{itemize}
    \end{frame}

    \begin{frame}{Folgerung der Konformität}{Konformität von Struktur und Funktion}
        \begin{itemize}
            \item Entsprechung zwischen der Komplexität der Funktion und der Struktur
            \item Manifestation: \glqq Prinzip der notwendigen Vielfalt\grqq{} R.U. Ashby
            \item Zunehmende Komplexität führt zu erhöhrter Komplexität des steuernden Systems
        \end{itemize}
    \end{frame}

    \begin{frame}{Gesetz der Aufrechterhaltung der Komplexität}{Konformität von Struktur und Funktion}
        Keine willkürliche Vereinfachung technischer Systeme
        \vfill
        \begin{itemize}
            \item Vereinfachung der Funktion des Systems
            \item Übertragung der Komplexität in die Teilsysteme (funktional-ideales Trimmen)
            \item Übertragen der Komplexität auf die Mikroebene \newline
            {\scriptsize Änderung des Funktionsprinzipes der Teilsysteme}
        \end{itemize}
    \end{frame}

    \begin{frame}{Welle der Idealität}{Konformität von Struktur und Funktion}
        \begin{enumerate}
            \item Verbesserung der Idealität führt zu erhöhter Komplexität des Systems
            \item Reduktion der Zuverlässigkeit des Funktionierens
            \item Vereinfachung des Systems
        \end{enumerate}
    \end{frame}

    \begin{frame}{Steuerbarkeit des Systems}
        Notwendigkeit einer Steuerung:
        \begin{itemize}
            \item Einige Zustände des Systems sind dynamisch
            \item Notwendigkeit das System in bestimmte Zustände zu versetzen
            \item Durch das Funktionieren des Basisprozesses ist ein Wechsel in diesen Zustand nicht möglich
        \end{itemize}
    \end{frame}

    \begin{frame}{Gesetz der Symmetrie technischer Systeme}
        \begin{itemize}
            \item Einwirkung der Umwelt auf technische Systeme weißen Symmetrie auf
            \item Abweichung von Symmetrie führt manchmal zu Optimierung \newline
            {\scriptsize Gesetzmäßige Tendenz der Symmetrie technischer Systeme}
        \end{itemize}
    \end{frame}

    \begin{frame}
        \begin{center}
            {\LARGE Vielen Dank für Ihre Aufmerksamkeit}
        \end{center}
    \end{frame}

    \begin{frame}{Diskussionsthemen}
        \begin{itemize}
            \item Konformität von Struktur und Funktion
            \begin{itemize}
                \item Zusammensetzung des Systems nicht trivial aber definitiv machbar
            \end{itemize}
            \item Wirkprinizip
            \begin{itemize}
                \item Uneindeutigkeit der Beschreibung des technischen Systems
                \item Definition beruht auf dem Blickwinkel der Person
            \end{itemize}
        \end{itemize}
    \end{frame}
\end{document}