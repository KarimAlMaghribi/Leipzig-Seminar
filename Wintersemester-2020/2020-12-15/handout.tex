\documentclass[DIV=22, 10pt, a4paper]{scrartcl}

\usepackage[ngerman]{babel}
\usepackage{xcolor}
\usepackage{amsmath}

\title{Handout zu Goldovsky (2017)}
\author{Tarek Stelzle}
\date{15. Dezember 2020}

\newcommand{\todo}{{\color{red}\textbf{TODO}}}
\newcommand\descitem[1]{\item{\bfseries #1}}

\begin{document}
    \maketitle
    \tableofcontents
    \newpage

    \section{Einleitung}
    Im folgenden wird auf die übersetzte Ausarbeitung \glqq Über die Gesetze der Konstruktion technischer Systeme\grqq von B.I. Goldovsky aus dem Jahr 2017 eingegangen.
    Hierbei werden die interessantesten und wichtigesten Aspekte aufgezeigt.

    \section{Frührere Arbeit des Autors}
    Der Autor beschäftigte sich mit dem Thema \glqq Entwicklung eines Systems von Gesetzmäßgikeiten der Konstruktion und Entwicklung technischer Systeme\grqq.
    Dabei wollte er das Thema genauer verstehen. 
    Aus dieser Arbeit gingen drei Aussagen hervor. 

    \begin{enumerate}
        \item System der Gesetzmäßgikeit ist komplexer als die Einteilung von G.S. Altshuller (\ref{Altshuller_liste})
        \item Teile der Gesetzmäßgikeit lassen sich deduktiv begründen
        \item Die Gesetze der Konstruktion technischer Systeme, welche die Arbeitssicherheit treffen, sind hervorzuheben
    \end{enumerate}

    \subsection{Anmerkungen zur Klassifizierung der Gesetze}
    In dem Artikel weist der Autor, B.I. Goldovsky, auf die folgenden Klassifizierungen der Entwichlichung technischer Systeme hin.

    Von G.S. Altshuller \label{Altshuller_liste} werden die Gesetze in die drei Bereiche \glqq Statik\grqq, \glqq Kinematik\grqq und \glqq Dynmaik\grqq{} unterteilt.

    Eine natürliche Klassifizierung wurde von V.M. Petrov vorgenommen.
    Dieser fasste die Gesete der \glqq Kinematik\grqq und \glqq Dynamik\grqq zu den \glqq Evolutionsgesetzen\grqq zusammen. 
    Die Gesetze der \glqq Statik\grqq wurden zu den \glqq Organistations-Gesetzen\grqq (Gesetze der Konstruktion) umformuliert.

    Weiterhin bezeichnete N.A. Shpakowski die Gesetze der Konstruktion als zentrale Gesetze.
    Die Evolutionsgesetze sind dabei nur unterstützend.

    \section{Besonderheiten der Gestze der Konstruktion technischer Systeme}
    Die Unterteilung in zentrale und unterstützende Gesetze unterstützt auch Goldovsky. 
    Er erläutert, dass die Gesetze der Konstruktion bei Nichteinhaltung bedingungslose Folgen haben, welche zum Stillstand des technischen Systems führen.
    Andererseits die Entwicklungsgesetze, falls diese missachtet werden, schlägt die Entwicklung und Ausführung der technischen Systeme eine andere Bahn ein. 
    Das technische Systeme würde, wenn auch ineffizienter, dennoch funktionieren. 

    \subsection{Unterteilung der Gesetze der Konstruktion (PNF, ENF)}
    Der wichtigeste Faktor eines technischen System ist das Bedürfnis des Soziums. 
    Dieses wird in den primär nützlichen Funktionen (PNF) formuliert.
    In der zweiten Ebene sind die elementar nützlichen Funktionen (ENF).
    Diese bilden die Ausführbarkeit der PNF.

    Für die Umsetzung der ENF müssen verschieden Teilsysteme vorhanden sein. 
    In anderen Worten muss die \textbf{funktionale Vollständigkeit des technischen Systems} gewährleistet sein.

    \subsection{Gesetz der strukturellen Vollständigkeit eines technischen Systems}
    \begin{center}
        \textbf{Die Gesamtheit der Elemente der Struktur und die Wechselwirkungen zwischen ihnen müssen den Durchsatz der natürlichen Flüsse (Stoff,
        Energie und/oder Information) zu den notwendigen Teilen des Systems sichern
        sowie eine solche Umwandlung dieser Ströme, dass alle elementaren Funktionen
        des Systems erfüllt werden.}
    \end{center}

    Diese Definition bezieht sich auf dynmaische Systeme, welche für den Menschen notwendige Prozesse erfüllen.
    Im Gegensatz dazu existieren auch statische Systeme, auch Anlagen genannt.
    Zur besseren Einteilung werden zur Unterscheidung wesentliche oder unwesentliche dynmaschische Prozesse verwendet.
    Dennoch sind auch in Anlagen Flüsse zu erkenne.
    Demnach kann der Fluss-Ansatz und deswegen auch der Begriff der strukturellen Vollständigkeit auf Anlagen angewendet werden.

    \section{Wirkprinzipien}

    Die Wirkprinzipien werden im Artikel als die natürlichen Prozesse, Effekte und Erscheinungen der Teilsysteme bezeichnet, welche der Ausführung der nützlichen Systemfunktion sichern.
    Da die Parameter für ein System die Funktion deutlich ändern können, kann man funktionale Nische in funktionell-parametrische Nischen unterteilen.
    Als Beispiel werden als System Erngiespeicher genannt. 
    Ein Untersystem davon wäre das Speichern von Energie. 
    Je nachdem wieviel Energie gespeichert wird, ändert sich die Funktion des Systems.

    \subsection{Erweiterung der PNF um Wirkprinzipien}
    Da durch die Parameter, wie oben beschrieben, ein Auswirkung auf die Funktionaliät des Systems haben, wird zusätzlich zu der PNF nun das zentrale Wirkprinizip mit angegeben.

    \section{Verallgemeinerung der Bedinungen für die Anwendung eines technischen Systems}
    Diese vier Punkte beschreiben die Bedinungen die gewährleistet sein müssen, sodass ein technisches System als Anwendung für das Sozim funktioniert.

    \begin{enumerate}
        \item PNF muss qualitativ und qunatitaiv den Anfordungen des Soziums und/oder des technischen Umfelds entsprechen
        \item Gewährleistung der Stabilität des Funktionierens
        \item Gewährleistung des erforderlichen Grad der Steuerbarkeit
        \item Gewährleistung der Benutzerfreundlichkeit des technischen Systems
    \end{enumerate}

    \section{Entwicklung technischer Systeme}

    \subsection{Weiterentwicklung}
        Hierbei werden die quantitativen Parameter im Arbeitszustand festgelegt.

    \subsection{Erstentwicklung}
        Bei der Erstentwicklung eines technischen Systems sollten die \textbf{parametrischen Schwellenwerte} überwunden werden.
        Eine physikalische Schwelle sind Werte, die zum Funktionieren des technischen Systems überwunden werden müssen.
        Funktionelle paramterische Schwellen nennt man das Niveau, welches überschritten werden muss, damit ein technisches System nicht mehr als Prototyp gewertet wird.
        
    \subsection{Entwicklungstsstufen}
    Damit ein technisches System arbeitsfähig ist eine Abstimmung an die Struktur nötig. 
    Die Abstimmmung eines technischen Systems an seine Struktur kann man in drei Etappen einteilen.

    \begin{enumerate}
        \descitem{Anfangsettape}\\
        In dieser Stufe soll die Arbeitsfähigkeit des technischen Systems hergestellt werden.
        \descitem{Abstimmung}
        \begin{enumerate}
            \descitem{Schwellenabstimmung (Konjugation)} Diese Phase ist abgeschlossen, sobald das technische System als betriebsfähig erklärt ist. Hierbei werden Formeln mit \glqq nicht weniger als\grqq{} und \glqq nicht mehr als\grqq{} verwendet.
            \descitem{Optimierungsabstimmung} In dieser Etappe werden die quantitativen Werte des technischen System abgestimmt. Dies kann sich über den kompletten restlichen Lebenszyklus erstrecken. Zur Optimierung werden Gleichung verwendet.
        \end{enumerate}
    \end{enumerate}

    \section{Konformität von Struktur und Funktion}
    Dies ist das Ergebnis der vorher erwähnten Konjugation. 
    In anderen Artikeln ist es unter dem Begriff \glqq Entsprechung von Funktion und Struktur\grqq{} zu finden.
    Das Gesetz legt die Übereinstimmung der Funktion des technischen Systems mit der Struktur fest.
    Der Nachweis dazu führt aber zu widersprüchen, da die Synthese einerseits kein trivaler Vorgang ist und deswegen das Ergebnis nicht eindeutig bestimmt werden kann.
    Andererseits ist diese Zusammensetzung ein analytisches Verfahren was zu einem eindeutigen Ergebnis führen muss.

    \subsection{Entsprechung zwsichen der Kompelexität der Funktion und der Struktur}
    Dies erklärt, dass bei Anstieg der Komplexität des technischen Systems auch die Komplexität der Steuerung ansteigen muss.

    \subsection{Gesetz der Aufrechterhaltung}
    Als Schlussfolgerung, von dem vorherigen Absatz, ist erkenntlich, dass eine Vereinfachung des Systems nicht willkürlich vorgenommen werden kann.
    Eine Vereinfachung kann mittels drei verschiedenen Ansätzen erreicht werden.

    \begin{enumerate}
        \item Vereinfachung der Funktion des Systems
        \item Übertragung der Komplexität in die Teilsysteme (funktional-ideales Trimmen)
        \item Übertragen der Komplexität auf die Mikroebene (Änderung des Funktionsprinzipes der Teilsysteme)
    \end{enumerate}

    \subsection{Welle der Idealität}
    Die Verbesserung der Idealität eines Systems geht einerher mit einem komplizierteren System.
    Ab einem Punkt ist das System so komplex, sodass eine Reduktion der Zuverlässigkeit des Funktionierens voranschreitet.
    Das Phänomen \glqq Welle der Idealität\grqq{}, beschreibt, dass dadurch eine Vereinfachung des Systems folgt.

    \section{Steuerbarkeit des Systems}
    Weiterhin muss bei der Konjugation eines technischen Systems auf die Steuerbarkeit des Systems geachtet werden.
    Nur dynamische Systeme sind steuerbar.
    Dennoch muss eine Steuerung nur gegeben sein, wenn folgende Punkte erfüllt sind:

    \begin{enumerate}
        \item Einige Zustände des Systems sind dynamisch
        \item Notwendigkeit das System in bestimmte Zustände zu versetzen
        \item Durch das Funktionieren des Basisprozesses ist ein Wechsel in diesen Zustand nicht möglich
    \end{enumerate}

    \paragraph{}
    Die Steuerbarkeit ist nur ein Mittel. 
    Dies sollte nur verwendet werden, wenn das System ohne nicht funktionieren würde.

    \paragraph{}
    Zu den wichtigsten Steuereinheit gehört das an- und ausschalten des technischen Systems.
    Weitere werden durch Funktionsweise und -prinzipien festgelegt.

    \paragraph{}
    Bei der Synthese mit eine Subsystem der Steuerung sollte als erstes die Wirkung beachtet werden, mit der das System in den Zielzustand versetzt werden soll.
    Aufgrund dieser wird das Wirkprinzip des Steuersubsytems aufgrund der gegebenen Resourcen festgelegt.

    \paragraph{}
    Die Zustandsänderung durch eine Steuerung muss in dem Fluss als Strukturglied dargestellt werden.
    Dieses sichert die Einwirkung auf den Fluss und ist für Steuereinwirkung empfänglich.

    \section{Gesetz der Symmetrie technischer Objekte}
    Die Einwirkung der Umwelt auf eine System bilden eine Symmetrie dar.
    Diese ist bedingt durch die Kombination dieser.
    Das Gesetzt sollte aber als Tendenz verstanden werden, da in anderen Artikeln ein Abweichen von dieser Symmetrie zu einer Optimierung des technischen Systems geführt hat.
    Demnach ist die gegebene Symmetrie eine Einschränkung der Vielfalt des technischen Systems.

    \section{Diskussionsthemen}
    \begin{itemize}
        \item Konformität von Struktur und Funktion
        \begin{itemize}
            \item Zusammensetzung des Systems nicht trivial aber definitiv machbar
        \end{itemize}
        \item Wirkprinizip
        \begin{itemize}
            \item Uneindeutigkeit der Beschreibung des technischen Systems
            \item Definition beruht auf dem Blickwinkel der Person
        \end{itemize}
    \end{itemize}
\end{document}