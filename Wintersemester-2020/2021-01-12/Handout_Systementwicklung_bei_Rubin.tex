\documentclass[a4paper,11pt]{article}
\usepackage{a4wide,url}
\usepackage[utf8]{inputenc}
\usepackage[main=ngerman,russian]{babel}
\parindent0cm
\parskip4pt

\author{Immanuel Thoke}

\title{Entwicklung technischer und allgemeiner Systeme\\ bei M.S. Rubin}

\date{Version vom 10. Januar 2021}

\begin{document}
\maketitle
\section{Die Zyklen der Systemwissenschaft}
„Die rasante Entwicklung der Welt um uns herum macht es dringend erforderlich,
die allgemeinsten und angewandten Theorien zu entwickeln, mit denen
aufkommende Probleme aus verschiedenen Bereichen effektiv gelöst werden
können.“ \cite{Rubin2002}

Dieses Zitat lesend, fällt es schwer keinen Bezug zu klassischen
Universalgelehrten wie Leonardo da Vinci oder Gottfried Wilhelm Leibniz
herzustellen.  Ist es die vereinfachte Zugänglichkeit zu einem immer breiteren
und tieferen Wissensschatz und die Zugangsmöglichkeit zu Institutionen der
Wissenschaft für eine immer breitere Bevölkerung, die den Typos des
\emph{Universalgelehrten} zu einem verbreiteten Phänomen werden lassen?
Verlangt es neue Methoden, um dieses Wissen ordnen zu können -- eine zyklische
„Überforderung“ (im Sinne einer Obsoleszenz bisheriger Methoden), die die
Reduzierung und Strukturierung notwendig macht, um wenigstens das Gefühl zu
haben, den Überblick nicht zu verlieren -- oder legt die Materie selbst den
Blick zu systematischen Zusammenhängen frei?

Gleichzeitig, bei der Betrachtung insbesondere populärwissenschaftlicher
Medien, scheint es einen Trend zur Fokussierung auf besonders (ökonomisch)
reizvolle Themen zu geben. So gehört es mittlerweile fast zum guten Ton, sich
auch einmal mit Neurologie, Astronomie, Teilchen- oder Metaphysik beschäftigt
zu haben. Wenngleich die ursprüngliche Expertise auf den ersten Blick wenig
mit den bearbeiteten Themen zu tun hat, werden Muster erkannt, die zur
infradiziplinären Untersuchung der Spezifika und möglicher Generika leiten.
Wenngleich in der öffentlichen Diskussion regelmäßig populärwissenschaftliche
Analogien mit Fachtermini verwechselt werden, ist der Aspekt einer
interparadigmatischen strukturellen Kreativität ein Zeichen der „systemischen
Inbezugnahme“ vor allem naturwissenschaftlicher Forschungsfelder -- man denke
nur einmal an die Entwicklung neuronaler Netze.

Rubin unterscheidet zunächst grob zwischen materiellen und immateriellen
Systemen. Die Kategorie der materiellen Systeme reicht dabei von anorganischen
und organischen (physikalisch-chemischen), lebenden
(biophysikalisch-chemischen) bis hin zu sozialen und soziotechnischen
Systemen.  Für ihn haben all diese Systeme „einheitliche Entwicklungsgesetze“
\cite{Rubin2002}. Immaterielle Systeme behandeln „die Entstehung von Mensch,
Vernunft und Zivilisation“ \cite{Rubin2006} und damit „Systeme wie
Wissenschaft, Kunst, Religion, Technologie, Sprache und viele andere“
\cite{Rubin2006}, wobei unklar bleibt, ob dies als eine Art „metaphysisches
System“ zu begreifen ist oder beschreibt, wie sich Menschen in einem
ontologischen System verhalten. Er framt immaterielle Systeme unter dem
Begriff der \emph{Kultur} und weist gleichzeitig auf die Problematik der
normativen Prägung derjenigen Gesetze aufmerksam, die gemeinhin als
Naturgesetze bezeichnet werden. Er bringt dazu Beispiele wie „Weltbilder als
Schlachtfeld von Theorien und Menschen“, die Revidierung der
Phlogiston-Theorie und diverse Beispiele der „Strukturinduktion“ durch Gesetze
als „notwendige, substanzielle, nachhaltige, wiederkehrende Beziehung zwischen
Phänomenen in Natur und Gesellschaft“ -- wobei Natur hier wohl eher im
übertragenen Sinne als Obersystem und allgemeingültiger Handlungsrahmen
verstanden werden muss -- wie bspw. antike Kalender, Traditionen oder
juristische Handlungsrahmen.

Die Verbindung von Transformationswerkzeugen und Gesetzen der Entwicklung von
Systemen ist ein zentraler Aspekt seiner Methodologie. Das erscheint zunächst
trivial, wird aber als wesentliches Charakteristikum von Systemwissenschaften
verstanden \cite{Kleemann2020}: Dass dies aber insbesondere für die
Transformationsschritte der Modellierungsphase und nicht zwangsläufig auf die
Methoden der Implementierung übertragbar ist, wurde im Seminar ausführlich
diskutiert. Gleichzeitig offenbart dies jedoch die Doppeldeutigkeit des
Begriffs \emph{Entwicklung}, und es stellt sich die Frage, inwieweit die
Einheit von System und Methode als Einheit von Theorie und Praxis tatsächlich
relevant sein kann, wenn man die intersubjektiven Begriffssemantiken
betrachtet.

\section{Analogismen infradisziplinärer Systematik}
„Was Physiker ein Gesetz nennen, nennen Mathematiker eine Vermutung.“ (Scott
Aaronson)

Vielleicht ist es schlicht naheliegend, dass sich Rubin eines biologischen
Terminus bedient, um Muster in der Entwicklung von Systemen mittels Ontogenese
und Phylogenese zu charakterisieren und kategorisieren. Schließlich lässt sich
die Biologie historisch als eine frühe interdisziplinäre Systemwissenschaft
begreifen, die sich spätestens seit Mendel und Darwin mit Merkmalsbeschreibung
und Strukturentwicklung beschäftigte.

Jedoch stellt sich die Frage, inwieweit sich Fachtermini grundsätzlich auf
andere Disziplinen übertragen lassen oder ob in deren Anwendung auf fachfremde
Zusammenhänge eine terminologische Differenz zum Ausdruck kommt, da sich der
fachfremden Materie im Abgleich der terminologischen Information lediglich
asymptotisch genähert werden kann. Mit Analogien geht das Spiel schnell
auf. Das Mammut kann mit der Pferdekutsche und der Elefant mit dem
Verbrennerauto gleichgesetzt werden. Die Orthologierelationen, die sich aus
phylogenetischen Informationen herleiten lassen, können auf die
Maschinenelemente des PKWs übertragen werden. Diese Begriffsanalogie führt zu
einer Strukturäquivalenz und damit zu einem Problem. Die Strukturäquivalenz
gilt nur innerhalb der Fächergrenzen. Für den Wissenstransfer ist jedesmal
eine Übersetzungsleistung notwendig. Spätestens wenn es juristisch gesicherte
Definitionen braucht, um die geplanten Systeme alltagstauglich zu machen,
verschwinden objektive Grenzen erster Ordnung und damit die Einheit von
soziotechnischen Systemen und deren Methodik im langfristigen Zeitfenster.

Bekanntlich ist die Justizia jedoch blind, und vielleicht hilft an dieser
Stelle nur das Spannungsfeld der Heuristik, die sich auch in der Luhmannschen
Systemtheorie niederschlägt. Insofern die Grammatiken der Individuen, deren
kommunikative Inferenzen sich nicht als bijektives Mapping darstellen lassen,
irrationale Vorgänge in Form erwartungskonformer Institutionen
rationalisieren, formen sie ihr eigenes „objektives“ Gesetz zweiter Ordnung,
an das sie sich halten müssen, um einander zu verstehen. Indem man sich
gegenseitig auf einen gemeinsamen Wortlaut verständigt und die approximierte
Zuordnung validiert, anstatt Widerspruch in vermeintlich mangelnder Exaktheit
zu suchen, können Nuancen der individuellen Auslegung anhand des Kontextes im
Prozess reinterpretiert werden, sofern die Rückkopplung anhand des dynamischen
gesellschaftlichen Kontexts mit einbezogen werden.

\begin{thebibliography}{yyy}
\bibitem{Kleemann2020} Ken P. Kleemann (2020). Dialektik der kreativen
  Innovation.  In: Erfinderschulen, TRIZ und Dialektik. Rainer Thiel zum
  90. Geburtstag.  Rohrbacher Manuskripte, Heft 20. ISBN 9783751983228.
\bibitem{Rubin2002} M.S. Rubin (2002).  \foreignlanguage{russian}{О теории
  развития материальных систем (ТРМС).} (Über eine Theorie der Entwicklung
  materieller Systeme). \\ \url{http://www.temm.ru/ru/section.php?docId=3878}.
\bibitem{Rubin2006} M.S. Rubin (2006). \foreignlanguage{russian}{Принцип
  захвата и многообразия в развитии систем.} (Das Prinzip des Einfangens und
  Mannigfaltigkeiten in der Entwicklung von Systemen).\\
  \url{http://www.temm.ru/ru/section.php?docId=3433#2}.
\bibitem{Rubin2019} M.S. Rubin (2019).  \foreignlanguage{russian}{ О связи
  комплекса законов развития систем с ЗРТС} (Zum Zusammenhang zwischen dem
  Komplex der Entwicklungsgesetze allgemeiner Systeme und den
  Entwicklungsgesetzen technischer Systeme).  Manuskript, 06.11.2019.\\
  \url{https://wumm-project.github.io/TTS}.
\end{thebibliography}

\end{document}
