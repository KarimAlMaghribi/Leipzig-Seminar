\documentclass[11pt,a4paper]{article}
\usepackage{ls}
\usepackage[main=german,russian]{babel}

\newcommand{\bbox}[2]{\parbox{#1cm}{\small\centering #2}}

\title{Modellierung nachhaltiger Systeme und Semantic Web\\
  Diskussionsgrundlage für das Seminar am 10.11.2020}

\author{Hans-Gert Gr\"abe}

\date{6. November 2020}

\begin{document}
\maketitle

\section{Gesetz, Gesetzmäßigkeit, Trend ... oder was?}

Die in der Überschrift aufgeworfene Frage bleibt genauer auszuloten, der
\emph{Gegenstand} einer solchen epistemischen Kontroverse liegt aber klar
abgrenzbar vor uns. Ich werde im Weiteren dafür das Wort „Gesetz“ verwenden,
so lange wir hier nicht zu tieferen Einsichten gekommen sind.

Es steht außer Zweifel, dass die „Gesetze der Entwicklung technischer Systeme“
eine der zentralen Systematisierungsleistungen innerhalb der TRIZ-Theorie
darstellen. Sie sind das kondensierte Ergebnis selbstreflexiver
\emph{induktiver} Erfahrung aus praktischer erfinderischer Ingenieurstätigkeit
wie auch die TRIZ-Theorie als Ganzes, und Praktiker sind gut beraten, diese
„Best Practices“ in den eigenen Handlungsvollzügen zu berücksichtigen.

Wie steht es aber um die \emph{deduktive} Qualität jener Gesetze?  Diese kann
sich erst innerhalb einer umfassenderen \emph{Theorie} entfalten, in der nicht
nur die Gesetze selbst, sondern auch deren logischer Zusammenhang sowohl
untereinander als auch mit noch genauer zu identifizierenden fundamentaleren
Begriff\-lichkeiten sprachlich gefasst werden kann. Eine genauere Vermessung
dieses \emph{Theoriefundaments} der TRIZ steht im Mittelpunkt unseres
gesamten Seminars.

Die erste Frage, die sich stellt, wenn über die Evolution technischer Systeme
gesprochen werden soll, ist die Frage nach zeitlichen Kontinuitätslinien.  In
welcher Kontinuität stehen das alte, verrostete Fahrrad in meiner Garage zu
dem neuen, aktuell in Gebrauch befindlichen? Evolutionslinien werden sich nur
konstituieren lassen, wenn auch hier vorab eine \emph{Reduktion auf
  Wesentliches} erfolgt. Die Debatte um Evolution setzt also auch hier bereits
eine \emph{Systematik} voraus, deren Begriffsformierung methodisch vor
ähnlichen Herausforderungen steht wie unsere Bemühungen, dem Begriff
\emph{technisches System} mehr Gestalt zu geben.

Hierzu wäre besser zu verstehen, was denn „wesentlich“ bedeutet, wie
Wesentliches von Unwesentlichem geschieden werden kann und was es überhaupt
mit dem Kern des Ganzen, dem Begriff \emph{Wesen} auf sich hat.  Philosophen
unterscheiden „Erscheinung“ und „Wesen“
(russ. \foreignlanguage{russian}{«явление»} und
\foreignlanguage{russian}{«сущность»}), und das werden wir in groben Zügen
auch verstehen müssen. Siehe dazu etwa \cite{Schlemm2004}.
\newpage

\section{Die konkreten Entwicklungsgesetze}

Altschuller \cite[S. 124\,ff.]{Altschuller1984} hat acht solcher Gesetze
identifiziert und genauer besprochen. Ebenda werden diese Gesetze wie folgt
charakterisiert:
\begin{enumerate}[noitemsep]
\item[A1] \textbf{Gesetz der Vollständigkeit der Teile eines Systems:}
  Notwendige Bedingungen für die Lebensfähigkeit eines technischen Systems ist
  das Vorliegen der Hauptteile des Systems und eine minimale
  Funktionsfähigkeit derselben.
\item[A2] \textbf{Gesetz der „energetischen Leitfähigkeit“ eines Systems:}
  Eine notwendige Bedingung für die Lebensfähigkeit eines technischen Systems
  ist der Energiefluss durch alle Teile des Systems.
\item[A3] \textbf{Gesetz der Abstimmung der Rhythmik der Teile eines Systems:}
  Eine notwendige Bedingungen für die Lebensfähigkeit eines technischen
  Systems ist die Abstimmung der Rhythmik (der Schwingungsfrequenzen, der
  Periodizität) aller Teile des Systems.
\item[A4] \textbf{Gesetz der Erhöhung des Grades der Idealität eines Systems:}
  Die Entwicklung aller Systeme verläuft in Richtung auf die Erhöhung des
  Grades der Idealität.
\item[A5] \textbf{Gesetz der Ungleichmäßigkeit der Entwicklung der Teile eines
  Systems:} Die Entwicklung der Teile eines Systems erfolgt ungleichmäßig. Je
  komplizierter das System ist, umso umgleichmäßiger verläuft die Entwicklung
  seiner Teile.
\item[A6] \textbf{Gesetz des Übergangs in ein Obersystem:} Nach Erschöpfung
  seiner Entwicklungsmöglichkeiten wird ein System als Teil in ein Obersystem
  aufgenommen. Dabei erfolgt die weitere Entwicklung auf der Ebene des
  Obersystems.
\item[A7] \textbf{Gesetz des Übergangs von der Makro- zur Mikroebene:} Die
  Entwicklung der Arbeitsorgane eines Systems erfolgt zunächst auf der
  Makroebene und anschließend auf der Mikroebene.
\item[A8] \textbf{Gesetz der Erhöhung des Anteils von Stoff-Feld-Systemen:}
  Die Entwicklung technischer Systeme verläuft in Richtung auf die Erhöhung
  des Anteils und der Rolle von Stoff-Feld-Wechselwirkungen.
\end{enumerate}
Andere Autoren haben diese Liste modifiziert oder ergänzt.  In \cite[Abschnitt
  4.8]{Koltze2017} werden Gesetze (Abschnitt 4.8.5) und Trends (Abschnitt
4.8.6) unterschieden und 5 Gesetze (G1=A1, G3=A4, G4=A5, G5=A8 und ein
\textbf{Gesetz (G2) der Vollständigkeit des Obersystems}) sowie 11 Trends
(S. 151) genannt. Im Inhaltsverzeichnis ist je ein Abschnitt ausgewiesen für
\begin{itemize}[noitemsep]
\item[T1] Dynamisierung
\item[T2] Koordination und Evolution der Rhythmik (A3, aber ergänzt um einen
  Evolutionsgedanken)
\item[T3] Gestalt- und Formkoordination
\item[T4] Evolution der Geometrie
\item[T5] Erhöhung des Energie-Leitvermögens (A2, aber nicht nur die
  \emph{Existenz} entsprechender Flüsse wird thematisiert, sondern auch die
  \emph{Intensivierung} dieser Flüsse)
\item[T6] Übergang auf die Mikroebene (A7)
\item[T8] Erhöhung der Automation 
\item[T9] Übergang zum Obersystem (A6)
\item[T10] Zusammenfall 
\end{itemize}
\newpage
T8 wird bei anderen Autoren auch als „Verdrängung des Menschen aus technischen
Systemen“ thematisiert, T10 in der englischen und deutschen Literatur als
„Trimmen“ bezeichnet\footnote{Interessanterweise spielte in der russischen
  Literatur zunächst auch der gegensätzliche Trend der Ausdifferenzierung als
  \foreignlanguage{russian}{свёртывание} und
  \foreignlanguage{russian}{развёртывание} eine Rolle, davon ist in moderneren
  Texten aber auch dort allein der erste Trend geblieben.}.

Die Darstellung ist etwas inkonsistent, denn Sie werden fragen, wieso 11
Trends, wenn hier nur 10 aufgezählt sind. Aber in \cite[Abb. 4.65]{Koltze2017}
sind in der Tat 11 Trends dargestellt, zusätzlich noch
\begin{itemize}[noitemsep]
\item[T11] Steigerung des Grads der Kontrolle
\item[T12] Funktionale Evolution
\end{itemize}
Einer zuviel? In der Übersicht fehlt T9=A6.  Dafür sind die Trend weiter
untergliedert in insgesamt \textbf{26 Entwicklungslinien}.

Nun also Gesetze, Trends und Entwicklungslinien. Altschuller hatte seine acht
Gesetze in „statische“ („Gesetze, die die Anfangsperiode im Leben technischer
Systeme bestimmen“ \cite[S. 124]{Altschuller1984}), „kinematische“ („Gesetze,
die die Entwicklung technischer Systeme bestimmen, unabhängig von speziellen
technischen und physikalischen Faktoren“ \cite[S. 126]{Altschuller1984}) und
„dynamische“ („Gesetze der Entwicklung moderner technischer Systeme unter der
Wirkung konkreter technischer und physikalischer Faktoren“
\cite[S. 127]{Altschuller1984}) gruppiert, wobei er den statischen und
kinematischen Gesetzen einen universellen, „zeitlosen“ Charakter zuschreibt
(„sie gelten zu allen Zeiten, nicht nur für technische Systeme, sondern für
Systeme überhaupt“, ebenda), den dynamischen dagegen einen eher speziellen
Charakter („sie widerspiegeln die Hauptentwicklungstendenzen der technischen
Systeme speziell in unserer Zeit“, ebenda).

Das bleibt weiter auszuloten und mit den begriff\-lichen Fundamenten
abzugleichen, die etwa V. Petrov \cite{Petrov2020a} gelegt hat:
\begin{quote}
  Ein \textbf{Gesetz} ist ein notwendiges, wesentliches, nachhaltiges, sich
  wiederholendes Phänomen. Ein Gesetz drückt eine Beziehung zwischen
  Gegenständen, den Bestandteilen dieses Gegenstands, zwischen den
  Eigenschaften von Dingen als auch zwischen den Eigenschaften innerhalb
  dieser Dinge aus.

  Aber nicht alle Beziehungen sind Gesetze. Beziehungen können notwendig und
  zufällig sein.  Ein Gesetz ist eine \textbf{notwendige Beziehung}. Es bringt
  die wesentliche Beziehung zwischen im Raum koexistierenden Dingen zum
  Ausdruck. Es ist ein Gesetz des Funktionierens.  Gesetze existieren
  \emph{objektiv}, unabhängig vom Bewusstsein der Menschen.

  Eine \textbf{Gesetzmäßigkeit} ist eine durch objektive Gesetze induzierte
  Bedingtheit; eine Form, in der sich die Existenz und Entwicklung in
  Übereinstimmung mit den Gesetzen entfaltet\footnote{Dies der Versuch einer
    sinngemäßen Präzisierung einer im Original schwer verständlichen
    Formulierung.}.
\end{quote}

\section{Ein Modell der Evolution technischer Systeme}

Wir werden uns mit einzelnen Gesetzen aus der Liste in jedem der Seminare
befassen, deshalb möchte ich die Diskussion in diesem Seminar auf die
Begründung ihrer Systematik konzentrieren, die Koltze/Souchkov in den
Abschnitten 4.8.1.--4. von \cite{Koltze2017} geben. „Modelle der Evolution
technischer Systeme“ ist der Abschnitt 4.8.1 überschrieben, aber von Modellen
im Plural kann keine Rede sein, denn es geht allein um ein Modell der
Entwicklung technischer Systeme längs S-Kurven -- visualisiert durch eine
monoton wachsende Funktion $f(t)$, auf der Abszisse ist die Zeit aufgetragen,
auf der Ordinate „Leistung, Nutzen“, annotiert mit „Grad der Systemleistung
(Hauptparameter der Wertschöpfung) basierend auf einem gegebenen Prinzip“. Im
Bild 4.68 sind drei weitere Kurvenverläufe aufgetragen,
\begin{itemize}[noitemsep]
\item eine „Glockenkurve“ (annotiert mit „Aufwand (Material, Energie,
  Personal, FuE) zur Wertschöpfung und und Lieferung der Funktionalität), die
  in erster Näherung als Ableitung $f'(t)$ durchgeht\footnote{Man beachte
    aber, wenn die S-Kurve eine \emph{logistische Funktion} ist, dann ist
    deren Ableitung \emph{keine} Glockenkurve.}, 
\item eine „Rentabilitäts-Kurve“ und 
\item eine Kurve „Niveau der Ideen“. 
\end{itemize}
Diese Kurven bilden die Grundlage, die Evolution in \textbf{drei Phasen} zu
unterteilen, die Phasen
\begin{itemize}[noitemsep]
\item \textbf{Geburt des Systems} -- Erschaffung der Hauptfunktion,
\item \textbf{Wachstum} -- Wachstum der Hauptfunktion,
\item \textbf{Reife und Zusammenfall} -- Bewahrung der Hauptfunktion. 
\end{itemize}
Später (Bild 4.71) werden die Phasen mit „Wachstum“, „Reife“ und „Sättigung“
bezeichnet, und (rein spekulativ?) der zeitliche Verlauf der vier Parameter
„technische Leistung“, „Anzahl der Erfindungen“, „Niveau der Erfindungen“ und
„Profitabilität“ gegen diese drei Phasen aufgetragen. 

Es bleibt genauer zu besprechen, auf welchen Grundlagen und impliziten
Annahmen ein solches Modell aufbaut.

\begin{thebibliography}{xxx}
\bibitem{Altschuller1984} Genrich S. Altschuller (1984).  Erfinden. Wege zur
  Lösung technischer Probleme. Verlag Technik, Berlin. 
\bibitem{Graebe2020} Hans-Gert Gräbe (2020). Die Menschen und ihre Technischen
  Systeme. LIFIS Online, 19.05.2020.
  \url{http://dx.doi.org/10.14625/graebe_20200519}
\bibitem{Koltze2017} Karl Koltze, Valeri Souchkov (2017). Systematische
  Innovationsmethoden.  Hanser Verlag, München. ISBN 9783446451278
\bibitem{Petrov2020a} Vladimir Petrov (2020). Gesetze und Gesetzmäßigkeiten
  der Systementwicklung. Monografie in 4 Bänden (in Russisch). ISBN
  978-5-0051-5728-7.
\bibitem{Schlemm2004} Annette Schlemm (2004): Hegels Gesetzesbegriff.\\
  \url{https://www.thur.de/philo/hegel/hegel20.htm}
\end{thebibliography}
\end{document}
