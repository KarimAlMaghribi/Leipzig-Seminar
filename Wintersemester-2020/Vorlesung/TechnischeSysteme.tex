\documentclass{beamer}
\usepackage{lsfolien}
\usepackage[german]{babel}
\usepackage[utf8]{inputenc}

\myfootline{Systemmodellierung und Semantic Web -- WS 20/21}{Hans-Gert Gräbe}

\newcommand{\ueberschrift}[1]{\begin{center}\bf #1\end{center}}

\title{Modellierung nachhaltiger Systeme\\ und Semantic Web\\[6pt] \Large
  Zum Begriff „Technisches System“
  \vskip1em}

\subtitle{Vorlesung im Modul 10-202-2330\\ im Master und Lehramt Informatik\\
  sowie im Modul 10-202-2309 im Master Informatik}

\author{Prof. Dr. Hans-Gert Gräbe\\
\url{http://www.informatik.uni-leipzig.de/~graebe}}

\date{Wintersemester 2020/21}
\begin{document}

{\setbeamertemplate{footline}{}
\begin{frame}
  \titlepage
\end{frame}}

\section{Grundlagen}
\begin{frame}{Technische Systeme in der TRIZ}

„... a \emph{number of components} combined to a system by establishing
  \emph{specific interactions} between the components ... assigned to
  \emph{perform a controllable main useful function} ... within a particular
  context.“ (Glossar V. Souchkov) \vskip1em

Ein System ist eine Menge von Elementen, ... die ein \emph{einheitliches
  Ganzes} bilden, das \emph{Eigenschaften} besitzt, die sich erst aus dem
\emph{Zusammenspiel} der Teile ergeben (\emph{Emergenz} der primär nützlichen
Funktion).  (V. Petrov, 2020) \vskip1em

Bedeutung des Systemoperators in der TRIZ, Nichttrivialität des Antisystems
(N. Feygenson, 2020)
\end{frame}

\begin{frame}{Grundlagen}
Trotz seiner Bedeutsamkeit bleibt die Begriffsbestimmung vage.\vskip1em
  
Wie kann der Begriff eines \emph{Technischen Systems} genauer umrissen
werden? \vskip1em
  
  \begin{itemize}
  \item [1.] Welche Aspekte sollten berücksichtigt werden?
  \item [2.] Vier Dimensionen des Begriffs \emph{technisches System} (TS).
  \item [3.] Technische Systeme als Black Box.
  \end{itemize}
\end{frame}
\begin{frame}{Aspekte}
  \begin{block}{Unterscheidung zwischen Designzeit und Laufzeit}
    \begin{itemize}
    \item Während der Entwurfsphase wird die grundlegende kooperative
      Zusammenarbeit \emph{geplant}.
    \item Während der Ausführung wird \emph{dieser Plan ausgeführt}.
    \end{itemize}
  \end{block}
  \begin{block}{Unterscheidung der interpersonellen} 
    \begin{itemize}
    \item Beschreibungsformen, die als \emph{begründete Erwartungen}
      kommuniziert werden, und
    \item Ausführungsformen, deren \emph{erfahrenen Resultate} in einem
      widersprüchlichen Verhältnis zu den begründeten Erwartungen stehen.
    \end{itemize}
  \end{block}
\end{frame}
\begin{frame}{Aspekte}
  \begin{block}{Aspekt der Nachnutzung}
    \begin{itemize}
    \item Dies gilt nicht für die meisten großen TS -- diese sind
      \emph{Unikate}, auch wenn sie aus Standardkomponenten zusammengebaut
      werden.
    \item Die meisten Computerspezialisten erstellen auch solche
      \emph{Unikate}, denn die IT-Systeme, die solche großen technischen
      Systeme operativ steuern, sind ebenfalls einzigartig.
    \item Das gleiche gilt für Ämter, Behörden, Regierungsstellen usw.
    \end{itemize}
  \end{block}
\end{frame}
\begin{frame}{Aspekte}
  \begin{block}{Klare Unterscheidung zwischen den Berufen}
    \begin{itemize}
    \item des Maschinen- und des Industrieanlagenbauers sowie  
    \item des Ausrüsters (Spezialist) und „Baumeisters“
      entsprechender Unikate (Generalist).
    \end{itemize}
  \end{block}\vskip2em
  \begin{alertblock}{These 1:}
    Die Besonderheiten technischer Systemen liegen hauptsächlich im
    \emph{Zusammenwirken von Komponenten} in einer \emph{Welt technischer
      Systeme}.

    \emph{Zwecke} betten diese Relationalität in menschliche Praxen ein. 
\end{alertblock}
\end{frame}
\begin{frame}{Erste Näherung}
  \begin{block}{Die vier Dimensionen des Begriffs \emph{Technisches System}}.
    \begin{itemize}
    \item [1.] Das realweltliche Unikat.
    \item [2.] Die Beschreibung dieses realweltlichen Unikats.
    \end{itemize}
    Für Komponenten, die in größerer Stückzahl hergestellt werden, auch
    \begin{itemize}
    \item [3.] Die Beschreibung des Designs der Systemvorlage.  
    \item [4.] Die Beschreibung und das Funktionieren von Auslieferung,
      Montage und Betrieb der realweltlichen Unikate, die nach dieser Vorlage
      hergestellt wurden (z.B. Produktionsplan, Qualitätssicherungsplan,
      Auslieferungsplanung, Pläne für Betrieb, Wartung und Instandhaltung).
    \end{itemize}
  \end{block}
\end{frame}
\begin{frame}{Erste Näherung}
  \ueberschrift{TS als Black Box}

  Die Grundlage des Konzepts ist der \emph{Begriff des offenes Systems} aus
  der allgemeineren Theorie Dynamischer Systeme.\vskip1em

  Bestehende TS sind normativ charakterisiert 
  \begin{itemize}
  \item auf der Ebene der Beschreibungsform durch die \emph{Spezifikation
    ihrer Schnittstellen} und
  \item auf der Ebene der Ausführungsform durch das \emph{garantierte
    Funktionieren nach dieser Spezifikation}.
  \end{itemize}\vskip1em

  Ein TS besteht aus Komponenten, die wiederum TS sind, deren
  spezifikationskonformes Funktionieren vorausgesetzt wird.\vskip1em

\end{frame}
\begin{frame}{Funktion des Begriffs eines TS}.
  
  Das TS-Konzept hat eine epistemische Funktion der (funktionalen) „Reduktion
  auf das Wesentliche“. \vskip1em

  Die menschliche Praxis ist untrennbar in das Konzept des TS eingebaut,
  denn die Begriffe „wesentlich“, „garantiert“ und „funktioniert“ können nur
  aus diesen Praxen heraus mit Bedeutung gefüllt werden. \vskip1em

  Damit wird die in der TRIZ verbreitete Unterscheidung zwischen technischen
  und sozio-technischen Systemen problematisch.
\end{frame}
  
\section{Das Konzept eines TS}
\begin{frame}{TS als White Box}
  \begin{itemize}
  \item [1.] Definition des Begriffs eines Technischen Systems.
  \item [2.] TS und die Welt der technischen Systeme.
  \end{itemize}\vskip2em
  \begin{block}{Kern eines Technischen Systems ist ... }
  ... die Beschreibung konkreter Prozesse durch Reduktion auf das Wesentliche
    mit dem Ziel ihrer praktischen Anwendung.
  \end{block}
\end{frame}
\begin{frame}{TS als White Box}
  \begin{block}{Die Reduktion auf das Wesentliche ... }
    ... fokussiert auf die folgenden drei Dimensionen:
    \begin{itemize}
    \item[(1)] Abgrenzung des TS nach außen gegen eine \emph{Umwelt},
      Reduktion dieser Beziehungen auf Input/Output-Beziehungen und
      garantierten Durchsatz (Zweckbestimmtheit und Arbeitsfähigkeit).
    \item[(2)] Abgrenzung des TS nach innen durch Gruppierung von Teilen als
      \emph{Komponenten}, deren Funktionieren auf eine „Verhaltenssteuerung“
      über deren Schnittstellen reduziert wird.
    \item[(3)] Reduktion der Beziehungen im TS selbst auf \emph{kausal
      wesentliche}.
    \end{itemize}
  \end{block}
\end{frame}
\begin{frame}{TS als White Box}
  \begin{block}{Das TS in der Welt der technischen Systeme}
    Die Beschreibung des TS selbst ist nur auf der Grundlage von
    Beschreibungen anderer (explizit oder implizit gegebener) TS möglich.  Der
    Beschreibung vorgängig sind:
    \begin{itemize}
    \item[(1)] Eine vage Vorstellung der (funktionierenden)
      Input/Output-Charakteristika der Umwelt.
    \item [(2)] Ein klares Bild von der Funktionsweise der Komponenten über
      die reine Spezifikation hinaus.
    \item [(3)] Eine vage Vorstellung von Ursache-Wirkungs-Beziehungen im
      System selbst, das der detaillierten Modellierung vorausgeht.
    \end{itemize}
  \end{block}
\end{frame}
\begin{frame}{TS als White Box}

  Das Konzept basiert auf der Verfügbarkeit bereits bestehender TS, die in (2)
  als Komponenten und in (3) als Nachbarsysteme eingehen. \vskip1em

  Ingenieur-technische Praktiken vollziehen sich damit in einer \emph{Welt
    Technischer Systeme}.\vskip1em

  In die konkrete Beschreibung eines TS gehen andere Systeme --- Komponenten
  oder Nachbarsysteme -- nur durch ihre Spezifikationen ein. \vskip1em

  Voraussetzung für den reibungslosen Betrieb eines TS ist damit das
  garantierte spezifikationskonforme Funktionieren der entsprechenden
  Infrastruktur.
\end{frame}
\section{Komponenten}
\begin{frame}{Komponenten}
  \begin{itemize}
  \item [1.] Der Begriff der Komponente nach Szyperski.
  \item [2.] core concern, cross cutting concern.
  \item [3.] Komponenten als funktionale Verbindungen.
  \item [4.] Komponenten als Funktions-Objekt-Beziehungen zwischen
    unabhängigen Dritten.
  \item [5.] Komponenten und Infrastruktur.
  \item [6.] Normen und Standards.
  \end{itemize}
\end{frame}

\begin{frame}{Die Welt der Komponenten}
  \begin{block}{Der Begriff der Komponente nach Szyperski}
    Was ist eine Komponente?

    Szyperski gibt eine einfache Antwort: „Components are for composition“.
  \end{block}\vskip2em

  TS werden aus bereits vorhandenen Komponenten zusammengesetzt. Komponenten
  können von Dritten erworben oder selbst entwickelt werden.
\end{frame}

\begin{frame}{Die Welt der Komponenten}

  \begin{block}{core concern, cross cutting concerns}
    Szyperski teilt die Welt der Komponentenherstellung (d.h. der TS) in zwei
    Teilwelten -- „design for component“  und „design from component“.
  \end{block}\vskip1em

  Die erste Welt ist die Welt der Komponentenentwickler, die spezielle
  Komponentenfunktionen für Geschäftsanwendungen -- „core concern“, dies
  entspricht der PNF -- als \emph{Kernsystemfunktion} entwickeln. \vskip1em
\end{frame}

\begin{frame}{Die Welt der Komponenten}

Zusätzlich zu dieser Kernfunktion benötigt die Komponente eine große Anzahl
von \emph{unterstützenden Funktionen} (Protokollierung, Datensicherheit,
Zugriffskontrolle, Druckersteuerung usw. -- „cross cutting concerns“), die auf
\emph{etablierte Konzepte} (Beschreibungsdimension) zurückgreifen und dafür
Dienste anderer, bereits \emph{vorgefertigter Komponenten} integrieren
(Anwendungsdimension), die \emph{andere technische Prinzipien} in anderen
Systemen umsetzen. \vskip1em

\begin{alertblock}{These 2:} 
  Realweltliche Komponenten sind in diesem Sinne immer \emph{Funktionsbündel},
  die prozedurales Wissen aus \emph{mehreren} Bereichen bündeln.
\end{alertblock}
\end{frame}

\begin{frame}{Die Welt der Komponenten}

Der \emph{Komponentenentwickler} muss all diese Beschreibungsformen der
Funktionen von Hilfskomponenten beherrschen, zumindest auf der
Abstraktionsebene ihrer Spezifikationen, um nützliche Komponenten zu bauen.
\vskip1em

Die zweite Welt ist die Welt der \emph{Komponentenassembler}. Diese bauen
(nach vorher entworfenem Plan) das System aus vorhandenen Komponenten
zusammen, entwickeln oder modifizieren zusätzliche Unterstützungsfunktionen
(„glue code“), integrieren und testen das komplette System, bevor Sie es für
die Nutzung durch den Kunden freigeben.
\end{frame}

\section{Standardi- sierung}
\begin{frame}{Modularisierung und Standardisierung}

Dieser Ansatz der Arbeitsteilung in Entwickler und Assembler von Komponenten
im Bereich des Software Engineering wird auch in ingenieur-technischen
Anwendungen umfassend genutzt.\vskip1em

„Modulare Systeme“ sind weit verbreitet und ermöglichen die Standardisierung
des Entwurfs der einzigartigen realweltlichen technischen Systeme. \vskip1em
\end{frame}

\begin{frame}{Komponenten und Frameworks}

Dabei muss die \emph{Anwendungslogik} der Komponente als „core concern“ mit
der \emph{Logik der infrastrukturellen Vernetzung} als „cross cutting
concerns“ verbunden werden.\vskip1em

\begin{alertblock}{These 3:} 
  Die Infrastrukturlogik ist normalerweise Teil des
  \emph{Komponenten-Frameworks}, das nur dann effektiv genutzt werden kann,
  wenn es \emph{gemeinschaftliches „Eigentum“ eines ganzen
    Technologiebereichs} ist.
\end{alertblock}
\end{frame}

\begin{frame}{Standardisierung und Trends der Evolution TS}

Anwendungslogik und Infrastrukturlogik sind orthogonal zueinander, womit die
Trends \emph{4.2 zunehmende Vollständigkeit des Systems} und \emph{4.4
  Migration zum Supersystem} in der Entwicklung TS einander praktisch
entgegenwirken.  \vskip1em

\begin{alertblock}{These 4:} 
  Eine Verbesserung des Verständnis der \emph{Infrastrukturanforderungen}
  miteinander interagierender Komponenten (Übergang zum Supersystem) als
  Beschreibungsform führt zu einer \emph{Verringerung des Niveaus der
    Anforderungen an die Vollständigkeit} einzelner Komponenten.
\end{alertblock}
\end{frame}

\begin{frame}{Standardisierung und Skaleneffekte}

Standardisierung eröffnet die Aussicht auf Skaleneffekte für
Standardkomponenten. Skaleneffekte führen zu geringeren Kosten pro Einheit und
verschieben damit die führende Rolle des Wettbewerb von der Konkurrenz
\emph{um die bessere technische Lösung} zur Konkurrenz um die \emph{billigere
  wirtschaftliche Herstellung}.\vskip1em

Damit wechselt die S-Kurve auf dem Höhepunkt ausgereifter technischer Lösungen
(Standardisierung eingeschlossen) in der Phase der allgemeinen Verfügbarkeit
in einen \emph{anderen Modus}, in dem die Reduzierung der wirtschaftlichen
Kosten für die Verfügbarkeit dieses „Stands der Technik“ die Leitfunktion für
die weitere Entwicklung übernimmt.
\end{frame}

\begin{frame}{Standardisierung und Skaleneffekte}

\begin{alertblock}{These 5:} 
  Der technische „Trend 4.1 des steigenden (technischen) Werts" geht auf der
  dritten Stufe der Entwicklung der S-Kurve über in einen ökonomischen
  „Trend der Verringerung des (ökonomischen) Werts“.
\end{alertblock}\vskip1em

Oder, in ökonomischen Termini: ein nachfragegetriebener Markt geht über in
einen angebotsgesteuerten Markt. Derselbe (reife) Gebrauchswert hat einen
immer geringeren Tauschwert.

\end{frame}

\section{Schluss- folgerungen}
\begin{frame}{Schlussfolgerungen}
  \begin{alertblock}{These 6:} 
    In der TRIZ-Theorie der Evolution TS muss klarer zwischen jungen und
    ausgereiften Technologien unterschieden werden.
  \end{alertblock}\vskip1em
  \begin{block}{In ausgereiften Technologien ... } 
    \begin{itemize}
    \item sind TS \emph{Bündel technischer Prinzipien},
    \item die in der Beschreibungsform \emph{Einheit in der Vielfalt} (global
      denken) zum Ziel haben und
    \item in der Praxisform \emph{Vielfalt in realweltlichen lokalen
      Anwendungskontexten} (lokal handeln) aus dieser Einheit zurückgewinnen.
  \end{itemize}
  \end{block}
\end{frame}
\begin{frame}{Schlussfolgerungen}
  \begin{alertblock}{These 7:} 
    Der gerichtete Graph der \emph{Zweckbestimmungen} ist der Kern der
    Relationalität in der Welt der Technischen Systeme.\vskip1em

    Dieser Graph ist ein globales sozio-technisches Artefakt und entwickelt
    sich seinerseits in der Widersprüchlichkeit von Beschreibungs- und
    Vollzugsform.
  \end{alertblock}
\end{frame}
\end{document}
