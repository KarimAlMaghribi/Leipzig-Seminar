\documentclass[11pt,a4paper]{article}
\usepackage{a4wide,url}
\usepackage[utf8]{inputenc}
\usepackage[german]{babel}

\parindent0pt
\parskip4pt
\title{Handreichung zum Einsatz des TRIZ-Trainers}

\author{Hans-Gert Gr\"abe}

\date{2. März 2020}

\begin{document}
\maketitle
\tableofcontents

\section{Allgemeines}

Im Rahmen einer internationalen Kooperation nutzen wir im Praktikum unseres
Kurses \emph{Semantic Web} den von \emph{Target Invention} in Minsk (Belarus)
entwickelten TRIZ-Trainer \url{https://triztrainer.ru}.  Der TRIZ-Trainer in
der aktuellen Beta-Version ist eine leichtgewichtige Version zur Unterstützung
der Online-Phase von Blended
Learning\footnote{\url{https://de.wikipedia.org/wiki/Integriertes_Lernen}} als
methodischem Praktikumskonzept.

Der TRIZ-Trainer konzentriert sich auf die Basiskonzepte des Einsatzes von
TRIZ an ausgewählten praktischen Beispielen -- die Analyse der jeweiligen
Problemsituation, die Identifizierung entsprechender Wirkfaktoren und
Widersprüche sowie die strukturierte Verwendung entsprechender
Lösungsschemata.  Weitergehende TRIZ-Werkzeuge (strukturierte Analysen von
Stoff-Feld-Interaktionen, Funktionsanalyse, Prozessanalyse,
Root-Conflict-Analysis usw.), die in (Koltze/Souchkov 2017)\footnote{Karl
  Koltze, Valeri Souchkov (2017). Systematische Innovation.  2. Auf"|lage,
  Hanser, München.} ebenfalls besprochen werden, können eingesetzt werden,
sind aber nicht Teil des im TRIZ-Trainer eingebauten strukturierten Vorgehens.

Der TRIZ-Trainer ist selbst noch in Entwicklung, insbesondere der Ausbau
verschiedener Sprachversionen.  Im Rahmen unserer Kooperation habe ich die
Minsker Kollegen bei der Erstellung einer deutschsprachigen Version
unterstützt.  Diese Arbeit ist aktuell zu 80\% umgesetzt, einige Teile des
TRIZ-Trainers im Bereich \emph{Ergänzendes} sind noch nicht in einer deutschen
Version verfügbar. Dies sowie die weitere Konsolidierung der Übersetzungen
erfolgt im Zuge des weiteren Einsatzes des TRIZ-Trainers über das dort
eingebaute Redaktionssystem.  Fortschritte in diesem Bereich werden also
unmittelbar wirksam.

\section{Der Workflow im TRIZ-Trainer}

\subsection{Registrierung und Aktivierung des Accounts}

Für die sechs Studierenden des Kurses wurden Accounts angelegt und der Rolle
\emph{Student} zugewiesen.  Die Probleme mit der Aktivierung der Accounts sind
inzwischen gelöst. Der Account ist bis 30. März 2020 freigeschaltet.  Es
besteht die Möglichkeit, die Arbeit mit einem Zertifikat abzuschließen. Dies
muss im Einzelfall mit Prof. Gräbe abgesprochen werden.

Die weiteren Ausführungen gehen davon aus, dass Sie sich am System
authentifiziert haben (Menü ganz rechts) und in der Rolle \emph{Student}
agieren.  Die beiden Felder daneben (mit den Tooltips \emph{Notifications} und
\emph{Settings}) dienen der Steuerung Ihrer Aktivitäten. Über das Feld
\emph{Notifications} haben Sie Zugriff auf Ihre bisherigen Lösungsversuche.

Die deutsche Version sollte sich automatisch an Hand der Spracheinstellung
Ihres Browsers aktivieren, gegebenenfalls kann dies auch im Auswahlfeld im
Seitenfuß umgeschaltet werden.  

Ich bin Ihnen als Trainer zugewiesen und kann somit Ihre Aktivitäten
verfolgen, kommentieren und bewerten.

\subsection{Einreichen, Kommentieren und Bewerten Ihrer Lösung}

Wenn die Lösungsvorlage vollständig ausgefüllt ist, können Sie die Lösung zur
Überprüfung absenden. In diesem Fall erhält der Trainer einen Hinweis auf
seiner Übersichtsseite und wird die Lösung überprüfen.  Weiter kann der
Trainer Kommentare abgeben. Da in den bisherigen Auswertungen die Sichtbarkeit
des „Teacher's Review“ bemängelt wurde, werde ich solche Kommentare
grundsätzlich nur noch in die Felder „Teacher Comment“ eintragen.  Basierend
auf den Ergebnissen der Durchsicht wird die Aufgabe bewertet und abgeschlossen
(Status \emph{credited}) oder zur Überarbeitung zurückgeben (Status \emph{to
  fix}).

Es gibt also kein „richtig“ oder „falsch“, sondern die Qualität der Lösung
wird begutachtet. Mehr dazu auch weiter unten in den „Anmerkungen von
Anton“\footnote{Anton Ivanov ist mein administrativer Ansprechpartner auf
  Minsker Seite.}.

Primäres Ziel des Einsatzes des TRIZ-Trainers ist die Erprobung dieses
Instruments im praktischen Einsatz eines „Flipped Classroom“ im Kontext
unserer Ausbildung.  Für das \textbf{erfolgreiche Absolvieren des Kurses} sind
10 Aufgaben so weit zu bearbeiten, dass die Lösungen vom Trainer akzeptiert
werden. 

\subsection{Der Workflow im TRIZ-Trainer}

Alles beginnt damit, dass der Bearbeiter mindestens ein Zeichen in die Vorlage
einer Aufgabe einfügt, um das Problem zu lösen. Beim automatischen Speichern
wechselt die Aufgabe dann in den Status \emph{wird gelöst} und wird auf der
Seite \emph{Ergebnisse} dem zugeordneten Trainer angezeigt. Der Trainer kann
so bereits sehen, welche Aufgaben die ihm zugeordneten Bearbeiter zu lösen
begonnen (aber noch nicht beendet) haben und kann den Fortschritt ihrer Arbeit
verfolgen. Der Trainer kann in diesem Stadium bereits Kommentare zu den
einzelnen Schritten hinterlassen.

Wenn der Bearbeiter das Problem gelöst hat, kann er auf die Schaltfläche
\emph{Zur Überprüfung einreichen} klicken und damit die Aufgabe einreichen.
Der Trainer analysiert die Lösung des Bearbeiters, kommentiert einzelne
Schritte und trifft die Entscheidung, die Lösung anzuerkennen (Status
\emph{credited}) oder zur Überarbeitung zurückzugeben (Status \emph{to fix}).
Die Möglichkeit einer \emph{Bewertung} ist aktuell nicht aktiviert, sondern
nur als potenzielle Variante für eine spätere Implementierung angelegt. Dem
Bearbeiter wird der Status der Lösung und die Kommentare des Trainers (falls
vorhanden) über seine Benachrichtigungen und auch per E-Mail mitgeteilt.  Wenn
die Aufgabe zur Überarbeitung zurückverwiesen wird, ist sie vom Bearbeiter
erneut zu bearbeiten. Wenn die Lösung akzeptiert wurde, ist die Bearbeitung
der Aufgabe beendet.

\section{Was ist zu tun?}

\subsection{Den Bearbeitungsprozess starten}

Nach dem Einloggen gehen Sie auf die Seite \emph{Aufgaben} und beginnen, die
Aufgaben zu lösen, die Sie mögen.  Es empfiehlt sich natürlich, vorher die
Hinweise unter \emph{Lösungsprozess} und diese Handreichung genauer zu
studieren.  Im Hilfesystem des TRIZ-Trainers sind zu jedem Schritt im
Lösungsprozess ausführliche Hinweise gegeben, was im jeweiligen Schritt zu tun
ist, und wie an die Teilaufgabe herangegangen werden sollte.

Es werden Ihnen mehrere Aufgabenserien angeboten, was aber eine eher
technische Einteilung ist.  Das Lösen der Aufgaben setzt eine gewisse
Vertrautheit mit der TRIZ-Methodik voraus, die Sie in der Hilfeanleitung
\emph{Lösungsprozess} oder -- in kurzer Form -- in den Tooltipps erwerben
können.  Beide sind ins Deutsche übertragen, inzwischen auch die Mehrzahl der
Grafiken. Probleme können in den Konsultationen besprochen werden.  Schauen
Sie sich auch die \emph{Beispielaufgaben} an.

\subsection{Wie eine Aufgabe lösen?}

Die Aufgaben sind so weit heruntergebrochen, dass sie (meist) nur \emph{eine}
widersprüchliche Grundsituation enthalten bzw. nur auf eine solche fokussiert
wird.  In der \emph{ersten Phase der Lösung} ist dieser Konflikt (zwischen
nützlichen und schädlichen Wirkungen) in einer (genaueren) Modellierung zu
lokalisieren, danach die \emph{Operative Zone} raum-zeitlich zu bestimmen und
eine Hypothese (als „Mini-Aufgabe“) aufzustellen, \emph{wie} das Problem
gelöst werden könnte. Es geht zunächst darum,
\begin{itemize}
\item das zu betrachtende \emph{System} zu fixieren und seine funktionale
  Stellung im Obersystem (zu welchem „Zweck“ gibt es das System --
  Spezifikation, Abschnitt „Präzisierung der Umstände“) zu ermitteln,
\item dessen Komponenten und deren Funktionalitäten zu bestimmen
  (Aufbauorganisation, Abschnitt „Maschine“; beachten Sie den Abschnitt
  „Aufbau und Funktionsweise der Maschine erforschen“ im Hilfeteil
  „Lösungsprozess“),
\item das Zusammenwirken der Komponenten zur Erfüllung des System-Zwecks zu
  beschreiben (Abschnitt „Wie die Maschine arbeitet“) und 
\item den Ort des Konflikts genauer zu lokalisieren.  
\end{itemize}
\textbf{Bitte beachten Sie auch die Hinweise weiter unten zum Modellieren der
  Anforderungssituation.}

Am Ende dieser Analyse (in der Informatik auch als \emph{Anforderungsanalyse}
bezeichnet) steht ein genaues Modell des Systems.  Weiter ist (Abschnitt
„Aufstellen einer Hypothese“) eine \emph{Aufgabe} zu formulieren, deren
Umsetzung das Problem lösen würde. In anderen Ansätzen wird diese Aufgabe auch
bereits als \emph{partielle Lösung} bezeichnet, da mit der Formulierung der
Hypothese im Gegensatz zum Brainstorming der im Weiteren zu analysierende
Lösungsraum \emph{begründet} bereits massiv fokussiert wird, die
\emph{Richtung} der Lösung an dieser Stelle bereits klar ist, auch wenn die
Details im weiteren Prozess noch ausgearbeitet werden müssen.

Es kann sein, dass sich während dieser Systemmodellierung herausstellt, dass
ein anderer Detailgrad als System angemessener ist (siehe \emph{Ergänzendes
  $\to$ Hierarchisches Prozessmodell}).  Dann muss die Modellierung auf jenem
Level wiederholt werden. Mehr dazu finden Sie im Abschnitt \emph{Ergänzendes
  $\to$ Algorithmus zur Korrektur von Problemsituationen (AIPS-2015)}.
Hilfreich ist es hierbei auch, die Ausführungen in (Koltze/Souchkov 2017,
Kapitel 4.3) zum Zusammenhang zwischen technischen (TW) und physikalischen
(PW) Widersprüchen zu beachten und zu einem (gelegentlich offensichtlichen) PW
die TW zu rekonstruieren, um zu verstehen, wie der PW im Gesamtsystem der
Modellierung einzubetten ist.  Vorausgesetzt werden natürlich auch elementare
Kenntnisse zu physikalischen Begriffen und physikalisch-technischen
Zusammenhängen, die Ihnen aus der Schule geläufig sein sollten\footnote{Etwa
  Zusammenhänge zwischen verschiedenen Energieformen, zu Kräften, Momenten,
  Bewegungsgrößen usw.}.

Zur spezifizierten Hypothese werden in der \emph{zweiten Phase der Lösung}
durch genaue Analyse der verfügbaren Ressourcen eine oder mehrere
\emph{Lösungsideen} gefunden, von denen in der \emph{dritten Phase} eine zur
\emph{finalen Lösung} genauer auszuarbeiten ist.  Im Vergleich zur
Anforderungsanalyse stehen dabei andere TRIZ-Werkzeuge im Vordergrund.

\subsection{Modellieren der Anforderungssituation}

Als wesentliches Defizit der bisherigen Bearbeitungen der Aufgaben hat sich
die oberflächliche Modellierung der jeweiligen Anforderungssituationen
herausgestellt, die durch die (scheinbar) relativ kleinteilige Struktur der
Fragestellungen Im Einstiegsbereich „Präzisierung der Umstände“ noch befördert
wird und zu reinen Brainstorming-Lösungen verleitet -- also solchen, wo für
den Bearbeiter von vornherein klar ist, wo die Reise hingehen soll und alle
Begründungen so hingebogen werden, dass am Ende dieses eine Resultat
herauskommen \emph{muss}.  Dabei wird der Prozess der Konkretisierung der
Aufgabenstellung mit ungenügender Präzision durchlaufen und damit das
methodische Potenzial von TRIZ nicht genutzt.

Im Seminar hatten wir herausgefunden, dass der \emph{Systembegriff} am
sinnvollsten im Wechselspiel mit dem \emph{Begriff Komponente} entwickelt
wird.  Komponenten sind zwar auch Systeme, sie werden aber allein durch ihre
\emph{Spezifikation} und damit die nach außen zu erfüllende kontraktuelle
Leistung charakterisiert.  Im Betrieb des Systems wird vorausgesetzt, dass
seine Komponenten spezifikationskonform \emph{funktionieren} -- wobei das
Anzeigen von Problemen nach außen durchaus Teil der Schnittstellendefinition
sein kann.

Das System selbst ist auch Komponente in übergeordneten Strukturen und hat
deshalb selbst eine \emph{Spezifikation}, mit der die Leistung des Systems
nach außen dargestellt wird. Für ein System ist aber weiter die
\emph{Implementierung} wesentlich, d.\,h.\ die genaue Beschreibung, \emph{wie}
die spezifizierte Leistung im Zusammenspiel der Komponenten und weiterer
Ressourcen erbracht wird.

Neben der \emph{Aufbauorganisation} (statisches Modell) ist für ein System
auch dessen \emph{Ablauforganisation} (dynamisches Modell) wichtig. Zur
Darstellung von Abläufen sind entsprechende Diagramme wie Sequenzdiagramme,
Zustandsdiagramme, Zustandsübergangsdiagramme, Prozessketten
usw.\ hilfreich. Die Abläufe im System verbinden Abläufe in den einzelnen
Komponenten, die auf Systemebene als \emph{elementar} betrachtet (und in
Diagrammdarstellungen in Unterdiagramme ausgelagert) werden, können neben dem
Aufruf von Funktionalität aber auch Zustandsänderungen an bearbeiteten
gemeinsamen \emph{Objekten} bewirken.  Komponenten sind in diesem Zusammenhang
Systemteile mit eigener aktiver Funktionalität, Ressourcen und Objekte passive
Targets funktionaler Transformationen\footnote{Der Objektbegriff unterscheidet
  sich damit von dem der OO-Programmierung und folgt den Ansätzen etwa in
  \emph{C. Szyperski.  Component Software. Addison Wesley, Boston 2002.}}.

Für die Modellierung der Gegebenheiten der Aufgabenstellung sind also zu
identifizieren 
\begin{enumerate}
\item das \emph{System}, in dem die abzustellenden problematischen Wirkungen
  auftreten,
\item das \emph{Obersystem}, aus dem heraus klar werden muss,
\begin{enumerate}
\item [2\,a.] warum -- zu welchem \emph{Zweck} -- das System überhaupt für
  menschliche Praxen relevant ist,
\item [2\,b.] wie das System in die \emph{Abläufe} des Obersystems eingebettet
  ist und
\item [2\,c.] welche \emph{Funktionen} das System im Obersystem zu erfüllen
  hat,
\end{enumerate}
(Beides im Abschnitt „Präzisierung der Umstände“)
\item die Aufbauorganisation des Systems -- welche Komponenten und welche
  Ressourcen werden genutzt -- (Abschnitt „Maschine“) und
\item die Ablauforganisation des Systems (Abschnitt „Wie die Maschine
  arbeitet“).
\end{enumerate}
In der Regel kann im Lösungsprozess das System so modifiziert werden, dass es
seine Funktion im Obersystem weiter erfüllt wie bisher spezifiziert oder nur
unwesentliche Modifikationen vorgenommen werden müssen, sich die
Transformation der Ablaufstrukturen also auf den Kontext des Systems selbst
beschränken lässt.

Die Ablauforganisation im Obersystem führt weiterhin oft auf eine klare
Unterscheidung verschiedener \emph{Zustände}, die für optimale Lösungen in der
Modellierung über verschiedene Betriebsmodi im System zu berücksichtigen sind.

\subsection*{Typische Identifizierungen:}
\begin{itemize}
\item Schiffsmast: Obersystem Wasserstraßenverkehr, System Boot.
\item Güterzug anfahren: Obersystem Gütertransport auf der Schiene, System
  Güterzug. 
\item Kipper im Bergbau: Obersystem Erztransport aus der Grube, System
  Kipper. 
\end{itemize}

\section{Weitere Hinweise}

Im TRIZ-Trainer wird in der Analysephase des Systems das Metakonzept der
\emph{Maschine} als grundlegende Struktur der Modellierung eingesetzt, womit
die Wirkung eines \emph{Werkzeugs} (Instrument, Arbeitsorgan) auf ein \emph{zu
  bearbeitendes Objekt} erfasst wird, um ein \emph{nützliches Produkt}
herzustellen.  Dieses \emph{Arbeitsorgan} ist Teil eines Produktionsprozesses,
der durch einen \emph{Antrieb} angetrieben wird, dessen Leistung über eine
\emph{Transmission} auf das Arbeitsorgan übertragen wird.  Das Ganze wird von
einer \emph{Steuereinheit} gesteuert und der Antrieb von einer
\emph{Energiequelle} gespeist.  Dieses stark an ingenieur-technischen Termini
orientierte Konzept wird auch auf allgemeinere Situationen übertragen und ist
dann sinngemäß anzuwenden.  Siehe auch \emph{Lösungsprozess $\to$ Aufbau und
  Bedienung der Maschine}.

Es empfiehlt sich, das zu lösende Problem erst einmal mit Bleistift und Papier
genauer zu analysieren. Dabei sollte man für sich genau klären
\begin{itemize}
\item[1.] Was ist das \emph{System} -- also jener Teil, der für die von den
  widersprüchlichen Anforderungen betroffene \emph{Hauptfunktion} relevant ist
  -- und wie ist dieses ins \emph{Obersystem} eingebettet? Also etwa im
  Beispiel Schiffsmast „das Boot“.

\item[2.] Was ist das nützliche Produkt? Ein Produkt ist etwas (zu einem
  späteren Zeitpunkt) bereits Fertiges, etwa „Das Boot ist auf der anderen
  Seite der Brücke“. Damit wird der \emph{Zweck} im betrachteten Kontext grob
  fixiert. 

\item[3.] Was ist das Nützlichkeitsprinzip? Wie entsteht das nützliche
  Produkt? Etwa, „das Boot fährt unter der Brücke hindurch“. Damit wird die
  Ablauforganisation im System grob fixiert. 

\item[4.] Genaue Analyse der Maschine (mit den oben angegebenen Teilen), dazu
  den Begriff „Maschine“ im Sinne des TRIZ-Trainers verstehen, siehe
  \emph{Ergänzendes $\to$ nützliches technisches System} und wie die Maschine
  arbeitet (siehe \emph{Funktionales Modell} und \emph{Prozessmodell}).  Damit
  werden Aufbau- und Ablauforganisation des Systems so weit detailliert, dass
  auf der Basis dieser Beschreibungen die problematischen Wirkungen genauer
  untersucht werden können. 
\end{itemize}
Müssen Komponenten genauer analysiert werden, so wird dasselbe Analyseschema
rekursiv auf die interessierende Komponenten angewendet, siehe
\emph{Ergänzendes $\to$ Hierarchisches Prozessmodell}.

\section{Anmerkungen von Anton}

TRIZ-Beratungsunternehmen konzentrieren sich nicht darauf, eine einzige starke
Lösung zu finden, die unter bestimmten Bedingungen wahrscheinlich schwierig zu
implementieren ist, sondern bieten eine Reihe von Lösungen. Auf diese Weise
kann der Kunde anhand seiner Ressourcen und Einschränkungen die am besten
geeignete, lokal ideale Lösung auswählen.
 
Dieser Ansatz wird auch im TRIZ-Trainer verfolgt, auch wenn wir hier keinen
\emph{Kunden} als Quelle der Probleme haben. Dementsprechend können wir auch
nur bedingt aus mehreren die beste Lösung auswählen -- diejenige, die uns
aufgrund allgemeiner Erfahrung und Logik als die beste erscheint. Wenn der
Student zu mehreren Lösungsvorschlägen gelangt, kann er bei den letzten
Schritten der Vorlage (Schlussfolgerung, endgültige Entscheidung) die seiner
Meinung nach am besten funktionierenden Lösungen hervorheben und eine (mit
Begründung) als die effektivste herausstellen. Diese Entscheidung wird
(wahlweise) meistens kommentarlos gegeben, da der Kurs keine detaillierten
Informationen sowie Anweisungen zur Bewertung und Auswahl von Lösungen
enthält. Der Trainer kann in solchen Fällen seine Meinung zu dieser Wahl als
Kommentar äußern, wenn die aus seiner Sicht idealste Lösung abgewählt wurde.
 
Zur Kontextualisierung der Lösung: Die Wahl der anfänglichen Systemebene ist
bei diesem Ansatz zweitrangig, weil in der weiteren Bearbeitung auf dem Weg
zum Kern -- dem Konflikt und seinen Ursachen -- die Systembetrachtungsebene
sowieso anzupassen ist. Ein nicht sehr klar definierter Orientierungspunkt für
die Kontextualisierung der Aufgabe ergibt sich (a posteriori) aus der Lösung
der Aufgabe: Für die Lösung sollten nur minimale Änderungen am vorhandenen
System erforderlich sein -- eine Änderung am Ort des Konflikts oder in dessen
Nähe.

Niemand ist jedoch gehindert, die Suche nach einer Lösung auf anderen
Systemebenen zu versuchen. Wenn wir durch andere, weiter entfernte
Systemebenen gehen, finden wir andere Lösungen (für die Aufgabe „Schiffsmast“
zum Beispiel globale Modifikationen der Brücke, des Flussbetts, Installation
von Mitteln zum Umtragen von Booten auf die andere Seite usw.).

Generell ist es ratsam, im Kurs eine Einstellung herauszuarbeiten, dass die
Lösung auf ein akzeptables Niveau zu bringen ist. Das heißt, auch sekundäre
Probleme sind zu lösen.  Wenn der Bearbeiter eine Lösung gefunden hat, die im
Prinzip funktioniert, die aber, wenn man versuchen würde, sie zu
implementieren, auf weitere Hindernisse stößt oder teuer oder kompliziert oder
zu 99\% unrealistisch bzgl. der Ressourcenanforderungen ist, kann man damit
nicht zufrieden sein.

In diesem Fall sollte die Lösung zur Überarbeitung zurückgegeben werden mit
folgenden zwei Optionen:
\begin{itemize}
\item [1)] Erneutes Durchlaufen des Lösungszyklus mit den erweiterten
  Kenntnissen und derselben Hypothese (Iteration).  Modifizieren Sie das
  Modell oder erstellen ein neues, führen daran noch einmal die
  Lösungsschritte aus und reichen die neue Lösung zur Bewertung ein. Es ist
  möglich, weitere Iterationen durchzuführen, es gibt im Prinzip keine
  Einschränkungen.
\enlargethispage{-1em}
\item [2)] Gehen Sie anders an die Lösung heran, formulieren Sie eine neue
  Hypothese (oder wählen eine andere aus den vorher schon aufgestellten
  mehreren Hypothesen aus), modellieren Sie das Problem neu und führen es als
  neue Lösung aus.
\end{itemize}
\end{document}
