\documentclass[11pt,a4paper]{article}
\usepackage{a4wide,url,enumitem}
\usepackage[utf8]{inputenc}
\usepackage[russian,german]{babel}

\parindent0pt
\parskip4pt
\title{Handreichung zum Einsatz des TRIZ-Trainers}

\author{Hans-Gert Gr\"abe}

\date{11. Oktober 2020}

\begin{document}
\maketitle
\tableofcontents

\section{Allgemeines}

Im Rahmen einer internationalen Kooperation nutzen wir im
TRIZ-Online-Praktikum den von \emph{Target Invention} in Minsk (Belarus)
entwickelten TRIZ-Trainer \url{https://triztrainer.ru} als Lernmittel.  Der
TRIZ-Trainer ist eine leichtgewichtige Version zur Unterstützung von Blended
Learning\footnote{\url{https://de.wikipedia.org/wiki/Integriertes_Lernen}} als
methodischem Praktikumskonzept im Sinne eines angeleiteten Selbststudiums. 

Der TRIZ-Trainer konzentriert sich auf die Basiskonzepte des Einsatzes von
TRIZ an ausgewählten praktischen Beispielen -- die Analyse der jeweiligen
Problemsituation, die Identifizierung entsprechender Wirkfaktoren und
Widersprüche sowie die strukturierte Verwendung entsprechender
Lösungsschemata.  Weitergehende TRIZ-Werkzeuge, die in \cite{KS2017} ebenfalls
besprochen werden, können eingesetzt werden, sind aber nicht Teil des im
TRIZ-Trainer eingebauten strukturierten Vorgehens.

Der TRIZ-Trainer ist selbst noch in Entwicklung.  Im Rahmen unserer
Kooperation unterstützen wir die Minsker Kollegen bei der Aktualisierung der
deutschsprachigen Version.  Neue oder noch nicht übersetzte Teile werden zügig
über das redaktionelle System der Anwendung konsolidiert.  Bitte informieren
Sie uns zeitnah über entsprechende Probleme.

\section{Der Workflow im TRIZ-Trainer}

\subsection{Registrierung und Aktivierung des Accounts}

Das Anlegen und Aktivieren der Accounts und die Zuweisung der Rolle
\emph{Student} erfolgt zentral durch die Praktikumsleitung, wenn die
Accountgebühr eingezahlt wurde.  Der Account ist zeitlich befristet.

Die weiteren Ausführungen gehen davon aus, dass Sie sich am System
authentifiziert haben (Menü ganz rechts) und in der Rolle \emph{Student}
agieren.  Die beiden Felder daneben (mit den Tooltips \emph{Notifications} und
\emph{Settings}) dienen der Steuerung Ihrer Aktivitäten. Über das Feld
\emph{Notifications} haben Sie Zugriff auf Ihre bisherigen Lösungsversuche.

Die deutsche Version aktiviert sich automatisch an Hand der Spracheinstellung
Ihres Browsers, gegebenenfalls kann dies auch im Auswahlfeld im Seitenfuß
umgeschaltet werden.

Ich bin Ihnen als Trainer zugewiesen und kann somit Ihre Aktivitäten
verfolgen, kommentieren und bewerten.

\subsection{Den Bearbeitungsprozess starten}

Nach dem Einloggen gehen Sie auf die Seite \emph{Aufgaben} und beginnen, die
Aufgaben zu lösen, die Sie mögen.  Es empfiehlt sich natürlich, vorher die
Hinweise unter \emph{Lösungsprozess} und diese Handreichung genauer zu
studieren.  Im Hilfesystem des TRIZ-Trainers werden zu jedem Schritt im
Lösungsprozess ausführliche Hinweise gegeben, was im jeweiligen Schritt zu tun
ist und wie an die Teilaufgabe herangegangen werden sollte.

Es werden Ihnen mehrere Aufgabenserien angeboten, was aber eine eher
technische Einteilung ist.  Das Lösen der Aufgaben setzt eine gewisse
Vertrautheit mit der TRIZ-Methodik voraus. Hinweise dazu finden Sie über die
Links ins Hilfesystem oder -- in kurzer Form -- in den Tooltipps zu jedem
einzelnen Schritt. Schauen Sie sich auch die \emph{Beispielaufgaben} an.

\subsection{Bearbeiten, Einreichen, Bewerten} 

Alles beginnt damit, dass Sie mindestens ein Zeichen in die Vorlage einer
Aufgabe einfügen, um das Problem zu lösen. Beim automatischen Speichern
wechselt die Aufgabe in den Status \emph{wird gelöst} und wird auf der Seite
\emph{Ergebnisse} dem zugeordneten Trainer angezeigt. Der Trainer kann so
bereits sehen, welche Aufgaben Sie zu lösen begonnen (aber noch nicht beendet)
haben und kann den Fortschritt Ihrer Arbeit verfolgen. Der Trainer kann in
diesem Stadium bereits Kommentare zu den einzelnen Schritten geben.

Ist Ihre Lösung komplett, klicken Sie auf die Schaltfläche \emph{Zur
  Überprüfung einreichen} und reichen damit die Aufgabe zur Bewertung ein.
Der Trainer analysiert Ihre Lösung, kommentiert einzelne Schritte und trifft
die Entscheidung, die Lösung anzuerkennen (Status \emph{angerechnet}) oder zur
Überarbeitung zurückzugeben (Status \emph{zu überarbeiten}).  Kommentare trage
ich grundsätzlich in die offen sichtbaren Felder „Kommentare des Trainers“
ein.

Ihnen werden der Status der Lösung und die Kommentare des Trainers sowohl über
die internen \emph{Notifications} als auch per E-Mail mitgeteilt.  Wenn die
Aufgabe zur Überarbeitung zurückverwiesen wird, ist sie von Ihnen erneut zu
bearbeiten.  Wenn die Lösung akzeptiert wurde, ist die Bearbeitung der Aufgabe
beendet.

Es gibt kein „richtig“ oder „falsch“, sondern die Qualität der Lösung
entsprechend der Methodik wird begutachtet.

\section{Zur Methodik}

Primäres Ziel des Einsatzes des TRIZ-Trainers ist dessen Einbettung in den
Kontext eines Flipped-Classroom-Konzepts, in dem -- einem Spiralmodell des
Kompetenzzuwachses folgend -- praktische Problemstellungen zur Beschäftigung
mit Theorie anregen und umgekehrt die studierte Theorie Ihre Fertigkeiten zum
Lösen praktischer Problemstellungen verbessert. Dazu wird mit
AIPS-2015\footnote{Die Abkürzung steht für \foreignlanguage{russian}{Алгоритм
    Исправления Проблемных Ситуаций} (Algorithmus zur Verbesserung
  problematischer Situationen).}  eine spezielle in Minsk entwickelte
algorithmische Version der TRIZ-Methodik eingesetzt, die den Lösungsprozess in
drei Phasen unterteilt.

Die Aufgaben sind so weit heruntergebrochen, dass sie (meist) nur \emph{eine}
widersprüchliche Grundsituation enthalten bzw. nur auf \emph{eine} solche
fokussiert wird.  Der gelegentlich vorhandene Interpretationsspielraum ist
stets so zu verstehen ist, dass
\begin{itemize}[noitemsep]
\item in der \emph{konkret betrachteten} Situation
\item eine der Situation angemessene \emph{konkrete} Lösung
\item mit möglichst wenig zusätzlichen Hilfsmitteln und 
\item möglichst geringer Modifikation des vorhandenen Systems
\end{itemize}
zu finden ist. 

In der \textbf{ersten Phase} \emph{Analyse der Problemsituation} ist eine
genaue Modellierung der gegebenen Situation auszuführen.  Dieser Teil endet
mit der Formulierung verschiedener Hypothesen\footnote{In anderen
  TRIZ-Methodiken werden diese auch als \emph{partielle Lösungen} bezeichnet,
  da mit einer genauen Modellierung bereits ein wesentlicher Schritt auf dem
  Weg zu einer kompletten Lösung bewältigt ist.}, wie die analysierte
widersprüchliche Situation aufgelöst werden kann, aus denen eine für die
zweite Phase als \emph{Aufgabe} formuliert und genauer analysiert wird.  Bitte
beachten Sie die präzisierenden Ausführungen zu dieser Phase im Abschnitt 4
dieser Handreichung. 

In der \textbf{zweiten Phase} \emph{Lösung der Aufgabe} ist diese präzisierte
Aufgabe nach einem (oder mehreren) von vier Aufgabenmodellen
\begin{itemize}[noitemsep]
\item Bedingungen in der operativen Zone,
\item Aktionen in der operativen Zone, 
\item Technischer Widerspruch,
\item Naturgesetzlicher (früher: physikalischer) Widerspruch
\end{itemize}
entsprechend der für das jeweilige Aufgabenmodell vorgesehenen Methodik
genauer zu analysieren.  Während in der ersten Phase die genaue
\emph{Modellierung der Situation} im Vordergrund stand und damit das
\emph{Problem kontextualisiert} wurde, geht es nun darum, den Kontext
\emph{systematisch} nach Ressourcen (Stoffe, Felder, Werkzeuge, Beziehungen,
Prinzipien, Standardlösungen) zur Lösung des Problems abzusuchen und so die
identifizierte Aufgabe in ein \emph{Bündel vorläufiger Lösungsvorschläge} zu
transformieren.

Es kann sein, dass sich in der zweiten Phase bei der Analyse der Aufgabe
herausstellt, dass der Ansatz nicht zielführend ist. Dann muss in die erste
Phase zurückgekehrt und mit einer anderen Hypothese als Aufgabe derselbe
Prozess noch einmal durchlaufen werden.  Es kann auch sein, dass die
Hypothesen in der ersten Phase sämtlich untauglich sind.  Dies kann geschehen,
wenn die Teile \emph{System} und \emph{Obersystem} in der ersten Phase
fehlerhaft bestimmt wurden. Dann muss die Analyse komplett von vorn auf einer
anderen Ebene der Systemabstraktion begonnen werden.

In der \textbf{dritten Phase} ist aus der Gesamtheit der Ergebnisse die
zufriedenstellendste Lösung auszuwählen bzw.  aus mehreren Lösungen zu
kombinieren.

In jedem Schritt gibt es Tipps und Links zu den entsprechenden Abschnitten der
Theorie im Hilfesystem des TRIZ-Trainers.  Darauf werde ich besonders aktiv
hinweisen, wenn die ersten 5--7 Aufgaben gelöst werden, bis Sie „Fuß gefasst“
haben und besser verstehen, was genau von ihnen verlangt wird.

Für das \textbf{erfolgreiche Absolvieren des Kurses} sind 15 Aufgaben so weit
zu bearbeiten, dass die Lösungen vom Trainer akzeptiert werden.

\section{Weitere Hinweise zum Bearbeiten der Aufgaben}

\subsection{Erste Phase: Analyse der Problemsituation}

Die Erfassungsmaske des TRIZ-Trainers für diese erste Phase geht mit den vier
Punkten
\begin{itemize}[noitemsep]
\item Präzisierung der Umstände 
\item Systemkonflikt
\item Aufstellen einer Hypothese 
\item Bedingungen der Aufgabe 
\end{itemize}
von einer klaren Kontextualisierung der Aufgabenstellung aus.  In der
bisherigen Praxis stellte sich heraus, dass dies eine zusätzliche Hürde für
Studierende ist, so dass der erste Teil der ersten Phase dahingehend
präzisiert wurde, diese Kontextualisierung (\emph{Abgrenzung} des zu
untersuchenden Technischen Systems als Black Box, Bestimmung seines
\emph{Zwecks} in einem \emph{Obersystem}) systematisch vorzunehmen,
\emph{bevor} der innere Aufbau (Wie ist die \emph{Maschine} aufgebaut und wie
arbeitet sie?) genauer analysiert wird.  Deshalb hierzu noch einige
grundsätzliche Ausführungen.

\paragraph{Systeme und Komponenten.} 
Der Systembegriff ist eng mit \emph{Praxen planmäßigen Vorgehens} verbunden,
die als Zusammenspiel vorhandener Ressourcen nach einem vorab als Beschreibung
vorliegenden Plan (in welcher Detailliertheit auch immer) gefasst werden
können.  In jedem Fall ist -- wie in der Informatik üblich -- zwischen
\emph{Designzeit} und \emph{Laufzeit} zu unterscheiden. Weiterhin ist die
Differenz zwischen den aus dem Design resultierenden \emph{begründeten
  Erwartungen} und den zur Laufzeit \emph{erfahrenen Ergebnissen} angemessen
zu berücksichtigen.

Mit Blick auf die grundsätzlich beschränkten Möglichkeiten der Beschreibung
von Prozessen ist eine Komplexitätsreduktion erforderlich. Deshalb müssen
System\emph{beschreibungen} in ihrem Konzept \emph{räumlich, zeitlich und
  kausal} beschränkt (kontextualisiert) werden, während die Performanz der auf
dieser Basis geschaffenen \emph{realen} technischen Systeme zur Laufzeit alle
drei Schranken trans"|zendiert. Der \emph{Zuschnitt} von Systemen muss deshalb
so erfolgen, dass diese Transzendenz in möglichst engen Grenzen verbleibt,
dass also das System in weiten Bereichen auch so funktioniert wie geplant.

Ein typischer ingenieur-technischer Zugang zur Realisierung dieser Anforderung
ist der \emph{Aufbau eines Systems aus Komponenten}, dieses also aus
vorgefundenen, bereits bewährt funktionierenden kleineren Systemen (eben den
Komponenten) zusammenzubauen.  Zu beachten ist, dass dies nur funktioniert,
wenn Komponenten replizierbar sind, also einerseits eine gewisse Komplexität
nicht überschreiten und andererseits eine gewisse Allgemeinheit für mögliche
Wiederverwendung nicht unterschreiten, wie etwa in (Szyperski 2002) genauer
besprochen\footnote{Fast alle großen technischen Spezialsysteme sind in diesem
  Sinne \emph{Unikate}. }.

Der \emph{Systembegriff} lässt sich damit am sinnvollsten im Wechselspiel mit
dem Begriff \emph{Komponente} entwickelt.  Komponenten sind zwar auch Systeme,
sie werden aber allein durch ihre \emph{Spezifikation} und damit die nach
außen zu erfüllende kontraktuelle Leistung charakterisiert.  Im Betrieb des
Systems wird vorausgesetzt, dass seine Komponenten spezifikationskonform
\emph{funktionieren} -- wobei das Anzeigen von Problemen nach außen durchaus
Teil der Schnittstellendefinition sein kann.

Das System selbst ist auch Komponente in übergeordneten Strukturen und hat
deshalb selbst eine \emph{Spezifikation}, mit der die Leistung des Systems
nach außen dargestellt wird. Für ein System ist aber weiter die
\emph{Implementierung} wesentlich, d.\,h.\ die genaue Beschreibung, \emph{wie}
die spezifizierte Leistung im Zusammenspiel der Komponenten und weiterer
Ressourcen erbracht wird.

Neben der \emph{Aufbauorganisation} (statisches Modell) ist für ein System
auch dessen \emph{Ablauforganisation} (dynamisches Modell) wichtig. Zur
Darstellung von Abläufen sind entsprechende Diagramme wie Sequenzdiagramme,
Zustandsdiagramme, Zustandsübergangsdiagramme, Prozessketten
usw.\ hilfreich. Die Abläufe im System verbinden Abläufe in den einzelnen
Komponenten, die auf Systemebene als \emph{elementar} betrachtet (und in
Diagrammdarstellungen in Unterdiagramme ausgelagert) werden, können neben dem
Aufruf von Funktionalität aber auch Zustandsänderungen an bearbeiteten
gemeinsamen \emph{Objekten} bewirken.  Komponenten sind in diesem Zusammenhang
Systemteile mit eigener aktiver Funktionalität, Ressourcen und Objekte passive
Targets funktionaler Transformationen. Der Objektbegriff unterscheidet sich
damit von dem der OO-Programmierung und folgt den Ansätzen eines „beyond
object oriented programming“, wie sie etwa in (Szyperski 2002) dargestellt
sind.

Ein einigermaßen vollständiges System umfasst nach (Lyubomirskiy u.a. 2018,
S. 39)
\begin{itemize}[noitemsep]
\item die Funktionalität des operierenden Agenten,
\item Transmissionsfunktionalitäten,
\item Energiebereitstellungsfunktionalitäten und
\item Steuerungsfunktionalitäten.
\end{itemize}
Dem wird im TRIZ-Trainer mit dem Konzept der \emph{Maschine} Rechnung
getragen, das Sie beim Erfassen der wichtigsten Bestandteile eines Systems
unterstützt (siehe den Abschnitt „Aufbau und Bedienung der Maschine“ im
Hilfesystem).

\paragraph{Bearbeitungsrichtlinie für die erste Phase.}
Für die Modellierung der Gegebenheiten der Aufgabenstellung sind also in der
ersten Phase zu identifizieren 
\begin{enumerate}\itemsep0pt
\item das \emph{System}, in dem die abzustellenden problematischen Wirkungen
  auftreten,
\item das \emph{Obersystem}, aus dem heraus klar werden muss,
\begin{enumerate}\itemsep0pt
\item [2\,a.] warum -- zu welchem \emph{Zweck} -- das System (als Komponente
  im Obersystem) überhaupt für menschliche Praxen relevant ist,
\item [2\,b.] wie das System in die \emph{Abläufe} des Obersystems eingebettet
  ist und
\item [2\,c.] welche \emph{primär nützliche Funktion} (PNF) das System im
  Obersystem erfüllt,
\end{enumerate}
(Beides im Abschnitt „Präzisierung der Umstände“)
\item die Aufbauorganisation des Systems -- welche Komponenten und welche
  Ressourcen werden genutzt -- (Abschnitt „Maschine“) und
\item die Ablauforganisation des Systems (Abschnitt „Wie die Maschine
  arbeitet).
\end{enumerate}
In der Regel kann im Lösungsprozess das System so modifiziert werden, dass es
seine PNF im Obersystem weiter erfüllt wie bisher spezifiziert oder nur
unwesentliche Modifikationen vorgenommen werden müssen, sich die
Transformation der Ablaufstrukturen also auf den Kontext des Systems selbst
beschränken lässt.  In anderen Aufgaben geht es um eine temporär zusätzliche
Funktion des Systems, die in einem anderen Systemzustand auszuführen ist. Auch
in diesem Fall ist die Analyse der PNF wichtig, da über diese Funktion die
verfügbaren Systemressourcen identifiziert werden können, die auch für die
zusätzliche Funktion genutzt werden können (und sollten).

Die Ablauforganisation im Obersystem führt weiterhin oft auf eine klare
Unterscheidung verschiedener \emph{Zustände}, die für optimale Lösungen in der
Modellierung über verschiedene Betriebsmodi im System zu berücksichtigen sind.

\paragraph{Typische Identifizierungen in einzelnen Aufgaben:}
\begin{itemize}[noitemsep]
\item Schiffsmast: Obersystem Wasserstraßenverkehr, System Boot.
\item Güterzug anfahren: Obersystem Gütertransport auf der Schiene, System
  Güterzug. 
\item Kipper im Bergbau: Obersystem Erztransport aus der Grube, System
  Kipper. 
\end{itemize}

Am Ende dieser ersten Phase (in der Informatik auch als
\emph{Anforderungsanalyse} bezeichnet) steht ein genaues Modell des Systems.
Weiter ist (Abschnitt „Hypothesen aufstellen“) auf der Basis dieser genauen
Modellkenntnis eine \emph{präzisierte Aufgabe} zu formulieren, deren Umsetzung
das Problem lösen würde. Im Gegensatz zu Analysemethoden wie \emph{Design
  Thinking} oder \emph{Six Sigma}, die stark auf die Wünsche des Kunden, aber
weniger auf die technischen Gegebenheiten ausgerichtet sind, wird durch die
systematische Modellierung ein gutes Verständnis\footnote{In der Praxis wirken
  hier oft Beratungsfirmen (methodisches Wissen) und Praktiker (fachliches
  Wissen) interdisziplinär zusammen.} für die technischen Gegebenheiten
erarbeitet und damit im Gegensatz zum Brainstorming der zu analysierende
Lösungsraum \emph{begründet} bereits massiv fokussiert.  Diese stark
fokussierende Wirkung einer TRIZ-Analyse auf \emph{praktisch} Umsetzbares wird
von Anwendern immer wieder als großer Vorteil dieser Methodik hervorgehoben.
Mit der Formulierung der Aufgabe ist die \emph{Richtung} der Lösung an dieser
Stelle bereits klar, auch wenn die Details im weiteren Prozess noch
ausgearbeitet werden müssen.

Es kann sein, dass sich während dieser Systemmodellierung herausstellt, dass
ein anderer Detailgrad als System angemessener ist, siehe den Abschnitt
„Hierarchisches Prozessmodell“ im Hilfesystem.  Dann muss die Modellierung auf
jenem Level wiederholt werden. Mehr dazu finden Sie im Abschnitt „AIPS-2015“
des Hilfesystems.  Hilfreich ist es hierbei auch, die Ausführungen in
(Koltze/Souchkov 2017, Kapitel 4.3) zum Zusammenhang zwischen technischen (TW)
und naturgesetzlichen\footnote{auch „physikalischer Widerspruch“, allerdings
  gibt es auch Widersprüche mit chemischem oder biologischem Hintergrund.}
(NW) Widersprüchen zu beachten und zu einem (gelegentlich offensichtlichen) NW
die TW zu rekonstruieren, um zu verstehen, wie der NW im Gesamtsystem der
Modellierung einzubetten ist.  Vorausgesetzt werden natürlich auch elementare
Kenntnisse zu naturgesetzlichen Begriffen und Zusammenhängen, die Ihnen aus
der Schule geläufig sein sollten\footnote{Etwa Zusammenhänge zwischen
  verschiedenen Energieformen, zu Kräften, Momenten, Bewegungsgrößen usw.}.

\subsection{Zweite Phase: Lösung der Aufgabe}

Zur spezifizierten Hypothese werden in der \emph{zweiten Phase der Lösung}
durch genaue Analyse der verfügbaren Ressourcen eine oder mehrere
\emph{Lösungsideen} gefunden. Am Ende ist eine zur \emph{finalen Lösung}
genauer auszuarbeiten und zu prüfen, ob die Lösung auch funktioniert.

Im Zuge der Modellierung wurde in der ersten Phase auch ein \emph{Konflikt}
identifiziert, wo ein (aus Sicht des Systemzwecks) nützlicher Effekt nicht
ohne einen schädlichen weiteren Effekt zu haben ist, sowie die \emph{operative
  Zone} (in Raum und Zeit) genauer bestimmt, in welcher der Konflikt auftritt.
Im TRIZ-Ansatz wird versucht, derartige Konflikte nicht durch Kompromisse zu
lösen, sondern zu prinzipiellen innovativen Ansätzen zu kommen.

\paragraph{Beispiel:}
Ein Teeglas wird mit heißem Tee gefüllt (nützlicher Effekt „heißer Tee
schmeckt“, schädlicher Effekt „beim Anfassen verbrenne ich mir die Finger“).
Die Kompromisslösung „lauwarmer Tee“ stellt niemanden zufrieden.  Mit TRIZ
analysieren wir, wo der Konflikt auftritt (an der Glaswand beim Anheben des
Glases zum Trinken). Typischer Lösungsansatz ist hier das Separationsprinzip
-- kann man das Ganze räumlich oder zeitlich trennen?  Funktion des Glases:
Behälter für den Tee, also kann nur was mit der Hand geändert werden. Fasse
das Glas mit einem Handschuh an (der wirkt wärmedämmend), oder mit einer
Grillzange (Abstand). Oder verwende einen Teeglashalter (perfektioniert die
Idee mit der Grillzange).  Oder pappe den Henkel des Teeglashalters gleich an
das Teeglas (schon perfektere räumliche Separation am Teeglas selbst;
„Trimmen“ des Teeglashalters).  Oder analysiere genauer: Glaswand muss
\emph{innen} heiß und \emph{außen} kalt sein.  Stelle das Glas also aus
wäredämmendem Material her -- wir haben den Coffee-To-Go-Becher erfunden.

\paragraph{Systematische Kontextanalyse.}
In der zweiten Phase der Lösung ist dieser Suchprozess für den Konflikt, seine
Bedingtheiten und die dazu spezifizierte Aufgabe \emph{systematisch}
auszuführen.  Hierfür stehen vier verschiedene Aufgabenmodelle zur Verfügung,
die alle probiert werden können. Aus der spezifischen Konstellation von
Modellierung und Konfliktstruktur kann aber oft eines der Aufgabenmodelle
favorisiert werden.  Die Anwendungsgebiete der vier Aufgabenmodelle sind im
Abschnitt „Lösen der herausgearbeiteten Aufgabe“ des Hilfesystems hinreichend
detailliert erläutert.

In jedem Fall kommt das „Hügelmodell“ zum Einsatz, nach dem die
Aufgabenstellung zu"|nächst entsprechend der Zielstellung in ein
\emph{abstraktes Aufgabenmodell} transformiert wird, auf dieses abstrakte
Modell TRIZ-Werkzeuge angewendet werden, um dieses in ein \emph{abstraktes
  Lösungsmodell} zu transformieren -- also ein \emph{Bündel strukturierter
  Lösungsideen} nach vielfach bewährten TRIZ-Prinzipien zu entwickeln.  Diese
zunächst abstrakten Lösungsideen (etwa „mache irgendwas mit akustischen
Feldern“) werden im zweiten Schritt mit den in der realen Situation
verfügbaren Ressourcen abgeglichen und daraus \emph{vorläufige Lösungen}
entwickelt, die im dritten Schritt miteinander verglichen, ggf. kombiniert und
die beste als \emph{finale Lösung} ausgewählt wird.

Da dieser Teil im Hilfesystem sehr genau beschrieben ist, kann hier auf
weitere Ausführungen verzichtet werden. 

\subsection{Dritte Phase: Finale Lösung auswählen}

\paragraph{Eigene Lösungsansätze bewerten.}
TRIZ-Beratungsunternehmen überlassen die Auswahl der Lösung meist dem Kunden,
da in die Entscheidung oft auch weitere Anforderungen einfließen, die sich aus
betriebsinternen Abläufen ergeben.  Deshalb konzentriert sich die Beratung
nicht darauf, eine einzige starke Lösung zu finden, die unter bestimmten
Bedingungen wahrscheinlich schwierig zu implementieren ist, sondern bietet
eine Reihe von (begründeten) Lösungen, aus denen der Kunde die für ihn am
besten geeignete, lokal ideale Lösung kombinieren kann.
 
Dieser Ansatz wird auch im TRIZ-Trainer verfolgt, auch wenn wir hier keinen
\emph{Kunden} als Quelle der Probleme und zur Evaluierung der Lösungen haben.
Dementsprechend kann auch nur bedingt aus mehreren die beste Lösung ausgewählt
werden -- diejenige, die auf Grund allgemeiner Erfahrung und Logik als die
beste erscheint. Wenn Sie zu mehreren Lösungsvorschlägen gelangen, können Sie
im letzten Schritten der Vorlage (Schlussfolgerung, endgültige Entscheidung)
die Ihrer Meinung nach am besten funktionierende Lösung hervorheben und (mit
Begründung) als die effektivste herausstellen.  Als Bewertungskriterium sollte
die Frage untersucht werden, in welchem Umfang für die Lösung Änderungen am
vorhandenen System erforderlich sind -- optimal sind Lösungen, die nur geringe
Änderungen am Ort des Konflikts oder in dessen Nähe erfordern.

\paragraph{Wie genau muss die finale Lösung ausgearbeitet sein?}
Die finale Lösung muss so weit durchgearbeitet sein, dass auch sekundäre
Probleme gelöst sind.  Mit einer Lösung, die nur im Prinzip funktioniert, die
aber, wenn man versuchen würde, sie zu implementieren, auf weitere Hindernisse
stößt oder teuer oder kompliziert oder zu 99\% unrealistisch bzgl. der
Ressourcenanforderungen ist, kann man nicht zufrieden sein.  In diesem Fall
wird die Lösung zur Überarbeitung zurückgegeben mit folgenden zwei Optionen:
\begin{itemize}
\item Sie bleiben in Phase 2 und durchlaufen den Lösungszyklus erneut mit den
  erweiterten Kenntnissen und derselben Hypothese (Iteration).
  
  Modifizieren Sie das Aufgabenmodell oder verwenden Sie ein anderes, führen
  Sie noch einmal die Lösungsschritte aus und reichen die neue Lösung zur
  Bewertung ein.
\item Sie kehren in Phase 1 zurück, gehen anders an die Modellierung heran,
  formulieren eine neue Hypothese oder wählen eine andere aus den vorher
  schon aufgestellten mehreren Hypothesen aus und durchlaufen dann Phase 2
  noch einmal mit dem neuen  Ansatz.
\end{itemize}

\bibliographystyle{plain}
\begin{thebibliography}{xxx}
\bibitem{KS2017} Karl Koltze, Valeri Souchkov (2017). Systematische
  Innovation.  2. Auf"|lage, Hanser, München.  ISBN: 978-3-446-45127-8
\bibitem{TESE2018} Alex Lyubomirskiy, Simon Litvin, Sergei Ikovenko et al.
  (2018).  Trends of Engineering System Evolution (TESE).  TRIZ Consulting
  Group. ISBN 978-3-00-059846-3.
\bibitem{Szyperski2002} Clemens Szyperski (2002). Component Software. Addison
  Wesley, Boston. 
\end{thebibliography}

\end{document}
