\documentclass[11pt,a4paper]{article}
\usepackage{a4wide,url}
\usepackage[utf8]{inputenc}
\usepackage[ngerman]{babel}

\parindent0pt
\parskip3pt
\setcounter{tocdepth}{2}

\title{Seminararbeit „TRIZ in China“}
\author{Bingqing Hu, editorisch bearbeitet von Hans-Gert Gräbe}
\date{30. September 2019} 

\begin{document}
\maketitle
{\small \tableofcontents}

\section*{Zusammenfassung}
In diesem Aufsatz wird die Entwicklung von TRIZ in China beschrieben.  TRIZ in
China wird allgemein durch die folgenden drei Eigenschaften oder Trends
gekennzeichnet [1]:
\begin{enumerate}
\item Integriert.
  
Vor den 1980er Jahren war TRIZ umständlich und fragmentiert. Boris Zlotin und
Alla Zusman verwendeten als erste Computer, um die Anwendung von TRIZ zu
unterstützen.  Damit haben sich die Methoden der TRIZ vereinheitlicht und
vereinfacht, was einer der Hauptgründe ist, wieso TRIZ heute weltweit bekannt
ist.

\item Ingenieure statt Erfinder als Zielgruppe.
  
Die Förderung und Demonstration von TRIZ in Unternehmen hat einen wichtigen
Anteil an der Verbreitung innovativer Methoden in China.

\item Standardisierung von TRIZ
  
Das System der Internationalen TRIZ-Zertifizierung wird auch in China
angewendet. Das System ist ein objektiver Bewertungsstandard für Innovationen
in Firmen und Ausbildungsstandard für innovative Ingenieure. Aber China hat
erst damit begonnen, das System zu implementieren; es ist ein langer Weg vom
Beginn zur Reife.
\end{enumerate}

Die TRIZ-Forschung und -Praxis in China war tief beeinflusst von den
TRIZ-Entwicklungen in den USA, besonders vor 2010.

Einer der größten Unterschiede zwischen Erfinderschulen in der DDR und
heutigen TRIZ-Anwendungen in China besteht darin, dass beide TRIZ-Varianten zu
unterschiedlichen Zeiten existieren.  Der entscheidende Unterschied zwischen
diesen zwei Zeiten ist die schnelle Entwicklung der Computertechnologie.

Gleich dagegen ist die Motivation, die wirtschaftlichen Zustände des
jeweiligen Landes zu verbessern.  Im weiteren Vergleich ergeben sich aber mehr
Unterschiede als Ähnlichkeiten. China ist ein Entwicklungsland mit mehr als
1,3 Milliarden Einwohnern und heterogen, die DDR war damals ein homogen
entwickeltes Industrieland.

Um die chinesische Entwicklung detailliert mit der in [18] entwickelten
Perspektive auf die DDR-Erfinderschulen zu vergleichen, habe ich mich am
Inhaltsverzeichnis von [18] orientiert.  In diesem Text geht es vor allem um
Parallelen zu chinesischen Entwicklungen im allgemeinen Zugang, wie er im
ersten Kapitel des Buches entwickelt wird.  Für weitere Studien fehlte mir
in diesem Semester leider die Zeit.

In diesem Aufsatz wird sich auf C-TRIZ als chinesische Version für TRIZ
bezogen, eine TRIZ-Variante, die von Runhua Tan (Technische Universität Hebei)
entwickelt wurde. TRIZ in China ist allerdings mehr als nur diese Theorie und
umfasst vielerlei weitere Aktivitäten im TRIZ-Umfeld. 

Viele in diesem Artikel referenzierte Dokumente oder Artikel liegen nur in
einer chinesischen Version vor, was das weiterführende eigenständige Studium
des Lesers erschwert. Weitere Unterlagen sind Dokumente der chinesischen
Regierung und in einer entsprechenden politischen Sprache und Logik verfasst.

\section{Hintergrund, Ziel und Weg der chinesischen TRIZ-Version (C-TRIZ)}

\subsection{Staatliche Weisungen}
Nach der Kutur-Revolution in China (1966--1976) hatte die chinesische
kommunistische Partei entschieden, die Wirtschaft zielgerichteter zu
entwickeln und dies als die wichtigste Aufgabe der Regierung deklariert.

Nach 1991 regiert in China eine der größten kommunistischen Pateien der Welt.
Die Regierung möchte ihrem Volk beweisen, dass sie den richtigen politischen
Weg gewählt hat und dies der einzige Weg ist, auf dem eine zielgerichtete
Entwicklung der Wirtschaft möglich ist. Sie hat alles dafür getan, um eine
schnelle wirtschaftliche Entwicklung zu ermöglichen. Sogar die chinesische
Armee darf wirtschaftlich aktiv sein.

In den 1980er Jahren war TRIZ schon einigen chinesichen Wissenschaftlern
bekannt.  In den 1990er Jahren begannen einige chinesische Forscher, an
internationalen TRIZ-Konferenzen teilzunehmen. 2001 führte Iwint [3] eine
spezifisch für China entwickelte TRIZ-Trainings"|methodik ein und versuchte,
diese zu verbreiten. Zwei Jahre später hatte die Firma zwei Software-Programme
für TRIZ-Theorie und TRIZ-Praxis entwickelt.

2007 wird die weitere Verbreitung von TRIZ in China dadurch befördert, dass im
Juni drei Wissenschaftler (Wang, Dong, Yang etc.) einen Brief „Vorschläge zur
Verstärkung von innovativen Methoden unserer Nation” an den damaligen
Premierminister richteten und im Juli darauf eine Antwort erhielten. Danach
organisierten das Wissenschaftsministerium, das Bildungsministerium sowie
der Wissenschafts- und Technologieverband Veranstaltungen für ihre Mitarbeiter
in kleinem Kreis, um TRIZ zu popularisieren. Zugleich hatten diese
Organisationen der Regierung die Hauptaufgaben aufgelistet, die zur
Verstärkung und Verbreitung innovativer Methoden in China zu ergreifen sind,
und dazu auch dem Staatsrat einen Bericht vorgelegt.

Im selben Jahr hatten Regierungsstellen zusammen standardisierte
Traingsmaterialien und Trainingssoftware für TRIZ erstellt, von denen
behauptet wurde, dass sie sich am besten für die aktuelle Situation in China
eignen. Weiter wurden die Provinzen Xichuan und Heilongjiang als
Pilot-Provinzen für die Einführung dieser technologischen Innovationsmethodik
ausgewählt. Universitäten wie die Technische Universität Hebei, die Nordost
Forestry Universität, die Sichuan Universität, die Südwest Jiaotong
Universität und weitere waren die ersten Universitäten, die zu TRIZ-Theorie
und TRIZ-Methoden forschten. Jede der oben genannten Universitäten hatte ihr
eigenes Ausbildungssystem erstellt, nach welchem Graduierte und
Master-Studenten in innovativen Methoden ausgebildet wurden, und eine Reihe
von Vorlesungen und Kursen zur Thema „TRIZ-Theorie und -Methodik“ entwickelt.

\subsection{TRIZ und Innovationsmethodiken}

Seit langem gibt es die Debatte, ob es überhaupt einen strukturierten Weg
gibt, das Erfinden zu erlernen, dem die Menschen folgen können.  Einige
Wissenschaftler verneinen dies und begründen ihre Position damit, dass
Innovationen hauptsächlich von der Inspiration und unlogischem Denken der
Erfinder abhängen.  Andere Wissenschaftler gehen davon aus, dass Erfinden ein
systematisch verfolgbarer Prozess ist, zu dem sich methodische Elemente
erlernen und vermitteln lassen.

Diese Gruppe von Wissenschaftlern verzweigt sich in zwei Teilgruppen. Eine
Teilgruppe betrachtet die Erfinder als Forschungsobjekt und fokussiert sich
auf die Mechanismen und Eigenschaften des -- im Kern unlogischen --
Erfindungsprozesses.  Ein typisches Beispiel für diesen Zugang ist, dass sich
nach Albert Einsteins Tod Wissenschaftler aus verschiedenen Bereichen daran
gesetzt haben, die Anatomie von Einsteins Gehirn zu untersuchen, um die
Geheimnisse der Innovation zu entdecken. Die andere Teilgruppe betrachtet eher
die Produkte des Erfindungsprozesses als Forschungsgegenstand wie etwa im
Patentamt eingereichte Schriften. TRIZ entstand aus dieser letzteren Strömung.

Trotz seiner theoretischen Unvollkommenheit hat TRIZ eine ausgezeichnete
Wirkung auf Innovationsprozesse. Nach einer allgemeinen Statistik hat TRIZ die
Anzahl der Patente um 80\% erhöht, die Qualität der Patente verbessert und die
Zeit vom Innovationsanfang bis zur Marktreife um mehr als 50\% verkürzt. In
diesem Punkt glaubte die Regierung an die Theorie, da ihr das Ergebnis
wichtiger als die Logik war.

\subsection{TRIZ an der Technischen Universität Hebei}

An der Technischen Universität Hebei haben die Professoren viele Ergebnisse im
Bereich der TRIZ-Theorie, Software für TRIZ, Material für TRIZ, TRIZ Kurse für
Master und Graduierte etc. entwickelt. Darunter ist das C-TRIZ-Modell
besonders erwähnenswert. Beiträge zur TRIZ-Forschung anderer Universitäten
werden hier nicht diskutiert.

C-TRIZ ist keine weitere Verbesserung oder theoretische Variante von TRIZ,
sondern fokussiert mehr auf die TRIZ-Praxis und das Training von Ingenieuren
für den Prozess der Entwicklung neuer Produkte und das Einreichen von
Patenten.

Die Entwicklung von C-TRIZ, das TRIZ in vielen Details und Praxen in China
ergänzt und hauptsächlich der Verbreitung von TRIZ dient, wird von Professor
Runhua Tan koordiniert und verbindet internationale Erfahrungen und
Entwicklungen der TRIZ mit der konkreten Situation in China. Im Lauf der Zeit
hat C-TRIZ bewiesen, dass es sehr erfolgreich in China eingesetzt wird. Für
C-TRIZ ist MEOTM (mass engineer oriented training model) ein Kernbegriff.
Mehr dazu ist in [4] zu finden.

Das Modell MEOTM ist ein schon ziemlich vollkommenes Modell für die
Verbreitung von TRIZ in China, wie die statistischen Erhebungen in [4] zeigen.
Das Modell wurde relativ systematisch auf der Basis von Fragestellungen und
faktenbasierten Analysen von Herstellungsprozessen bis zur Messung der
Zweckmäßigkeit herausgebildet. Auf jeden Fall ist das Modell ein sehr
umfangreiches und kompliziertes System, welches mehr Gewicht auf
Managementprozesse statt auf die Ingenieurstätigkeit legt. In den Prozessen
7-STEP und 6-GATE entscheiden die Manager, welche Ingenieure geeignet sind für
die Trainings bzw. qualifiziert werden sollen. Das setzt hochqualifizierte
Manager voraus.  In der Gesamtsicht ist das Modell eher praktisch aufgestellt
mit wenigen logisch dargestellten Zusammenhängen und ohne sichtbare
ontologische Modellierungen.  Das erschwert die Konzipierung konkreter
Schritte nach der 7-STEP und 6-GATE Methodik in konkreten Anwendungen. Wenn
ein Begriff wohldefiniert ist, dann ist er leicht zu verstehen und
anzuwenden. Sonst ist das Gegenteil der Fall.

Wenn es um Verbesserungen des Modells geht, dann wären meine Vorschläge, mehr
Kriterien für die Auswahl der Firma und der Ingenieure zuzulassen, um weniger
Fehler im ersten Schritt zu machen. Und die Programme in der Phase 1 sollten
auch erst standardisiert und dann modifiziert werden, damit das Training
wirklich einen Effekt für die Teilnehmer hat.

Von einer Statistik nach Wahrscheinlichkeitstheorie ist es häufig ein sehr
langer Weg bis zu wirksamen Praxen, und auf dem Weg werden wir verschiedene
schöne und weniger schöne Landschaften sehen.

\subsection{TRIZ in Iwint}
Wenn das Niveau der TRIZ-Forschung an der Technischen Universität Hebei ein
Gipfel ist, dann ist die Firma Iwint ein anderer Gipfel im Forschungsfeld der
TRIZ in China. Iwint steht auf dem ersten Rang aller Firmen im Bereich CAI
(Computer Aided Innovation) in China.  Die Firma hatte schon 2002 eine
Software für innovative Methoden auf den Markt gebracht. Diese Software für
innovative Methoden ist eine Plattformen, auf der Ingenieure, Entwickler,
Personal für geistiges Eigentum und Mitarbeiter des Wissensmanagements in
einer gemeinsamen Arbeitsumgebung zusammenarbeiten, welche Pro/Innovators [6]
heißt.  Im Kern geht es um die Analyse des System, in welchem das Produkt
erstellt wird, und um die Auseinandersetzung mit Widersprüchen in den drei
Dimensionen Kausalität, Nutzbarkeit und Betriebsumgebung.  Nach der Eingabe
der entsprechenden Informationen erstellt die Software nach TRIZ automatisch
Lösungsvorschläge. Diese werden dann bewertet, ob es einen Markt für deren
Umsetzung gibt und ob die Firma mit der Lösung ein Patent anmelden kann.
Pro/Innovator basiert auf einer wissenschaftlichen Datenbank. Zugleich ist
Pro/Innovator eine Plattform für Entwurfsmethodiken.

\subsection{Schöpfertum und Methodologie von TRIZ in China}
Schöpfertum bedeutet nichts anderes als zwei Sachen zu kombinieren, welche
früher nicht miteinander kombiniert waren, um mit der Kombination ein Problem
zu lösen. Sind Sie zum Beispiel ein Informatiker und kennen sich auch in der
Philosophie aus, dann sind Sie eine Person mit Schöpfertum, wenn sie mit
diesem Schöpfertum etwas Neues erstellen. Dieser Prozess heißt Innovation.

Die erste großartige Innovation in China geht zurück auf die Han-Dynastie, wo
Zhai Len als Erster Papier hergestellt hat. Am Anfang der industriellen
Revolution hat Benz das erste Auto, Watt die erste Dampfmaschine gebaut.
Damals blockierte die chinesische Regierung ganz China, mit der äußeren Welt
zu kommunizieren oder zu handeln. Am Anfang der Zeit der Reform und Öffnung
gab es in China keine ausgezeichneten Erfinder. Als Vorbild für das Volk hatte
die Regierung Edison gewählt, in der Hoffnung, dass sich aus den Erfahrungen
oder Inspirationen von Edisons Biographie etwas Positives gewinnen lässt. Vor
der Einführung von TRIZ verwendeten die meisten Erfinder die Methode von
Versuch und Irrtum, die wie folgt abläuft -- der Erfinder hat eine Idee, die
er versucht umzusetzen. Wenn die Idee nicht funktioniert, sucht der Erfinder
eine neue Idee. Der Erfinder macht so weiter, bis er eine geeignete Lösung
gefunden hat. Die Methode von Versuch und Irrtum ist sehr ineffizient im
Vergleich zu TRIZ.

Obwohl die Antwort auf die Frage, ob Schöpfertum auch anders als allein mit
Übungen oder Training verbessert werden kann, noch unbestimmt ist, scheint
unter den Anhängern von TRIZ in China „ja“ als Antwort zu überwiegen.
Insgesamt lassen sich die Organisationen, welche in China eng mit TRIZ
verbunden sind, in drei Gruppen unterteilen:
\begin{itemize}
\item Die Regierung, welche für die politische Sache zuständig und in der
  entscheidenden Position in allen Organisationen ist,
\item die Unis, welche hauptsächlich die Forschungen vorantreiben und
\item die Firmen, in denen TRIZ so weit wie möglich angewendet werden soll.
\end{itemize}
Die Logik hinter dieser Anordnungen ist, dass sich für neue Theorien und
Ansätze im Bereich von TRIZ schnell theoretisch und praktisch bewerten lässt,
welche Auswirkungen sie auf den Markt haben. Außerdem lassen sich in einer
solchen Anordnung neue Theorien auf der Basis der Rückmeldungen aus den
Experimenten in den Firmen verbessern.

\subsection{Die Sichtbarkeit von TRIZ in China}

Die TRIZ-Organisationen in China setzen nicht so sehr auf theoretisch-formelle
Ableitungen der Theorie, sondern auf deren praktische Anwendung in der
Industrie.  Die Regierung hat nicht nur Interesse an den mit TRIZ
eingefahrenen Gewinnen, sondern auch an der Anzahl der durch TRIZ erworbenen
Patente. Die durch TRIZ gefundenen Lösungen bleiben im Kopf der Ingenieure und
Unternehmen. Zugleich werden die durch TRIZ erworbenen Patente im Patentamt
gut dokumentiert und in eine einheitliche Datenbank eingebracht. Damit können
diese Daten auch mit KI-Methoden analysiert und einige für Menschen schwer
herauszuarbeitende Zusammenhänge gefunden werden. Die einfachste Anwendung für
diese Datenbank wäre, eine Visualisierung zu erstellen, um andere, die kaum
Kenntnis über TRIZ haben, zu überzeugen, dass TRIZ wirklich wirksam ist bei
der Verbesserung von Produkten oder bei der Vergrößerung von Marktanteilen.
Solche Bilder brauchen auch die Politiker, um ihr Volk zu überzeugen.

In China hat die Regierung zahlreiche Unis und Gebäude, in denen
Veranstaltungen zu TRIZ stattfinden können. Viele der studentischen
Absolventen pro Jahr (ca. 7 Million in 2018), welche im Durchschnitt nicht so
gut qualifiziert sind wie die Studenten in Deutschland, möchten als zukünftige
Ingenieure gerne mit TRIZ vertraut werden.
  
\subsection{Wachsender Einfluss}

Seit die Einführung der TRIZ in China in 2001 von iwint ist der TRIZ-Einfluss
immer weiter gewachsen. Die Gründe dafür sind folgende.
\begin{itemize}
\item [1.] Die chinesische Wirtschaft brauchen TRIZ dringend, um sich von
  einer Ressourcen verschwendenden zu einer umweltfreundlichen Wirtschaft zu
  entwickeln.
\item [2.] Die Führung der Regierung spielt beim wachsenden Einfluss der TRIZ
  in China auch ein große Rolle, die Regierung hat die stärkste
  Nachrichtenagentur und Propaganda.
\item [3.] Ein tiefer Grund dafür ist die Kulturrevolution in den 1960er und
  1970er Jahren. Das Volk hasst politische Bewegungen, in denen hundert von
  tausend Personen getötet wurden. Solche Bewegungen haben auch die Wirtschaft
  vollständig ruiniert, der schlechte wirtschaftliche Zustand führt zu
  schlechter Lebensqualität für die meisten.
\end{itemize}
Seit Anfang der 1980er Jahre respektiert China die westliche Kultur mehr als
die alte chinesische Kultur. TRIZ ist ein typisches Element der westlichen
Kultur, obwohl TRIZ bis jetzt seine Potenziale noch nicht bewiesen hat. Viele
bekannte Unis in China begannen, zu TRIZ zu forschen, einige Professoren [8]
an bestimmten Unis hoffen sogar, dass alle Studenten und Professoren an der
Uni und Ingenieure, deren Anzahl in 2011 ca 38.5 Millionen betrug, in China in
nicht so ferner Zukunft TRIZ kennen.

TRIZ wird in vielen großen staatlichen Firmen wie in Chinas Schiffbau, in der
Schwerindustrie und in der Chengdu Flugzeugfabrik usw. eingesetzt.  Diese
erfolgreichen Beispiele machen TRIZ bekannt und akzeptierbar für die
Geschäftsinhaber.  Geschäftsinhaber stellen die größte Zahl von Teilnehmern in
den Trainingskursen der TRIZ, die normalerweise deutlich teurer sind im
Vergleich zu anderen Trainingskursen.

Mit dem wachsenden Einfluss der TRIZ in China wird die Wirtschaft von einem
niedrigen Niveau auf ein relativ höheres Niveau gehoben.  Softwarefirmen und
andere Firmen wie Banken wenden TRIZ noch nicht an. In diesen Bereichen hat
TRIZ noch keinen Anklang gefunden.

\subsection{Erfinden und Erfinder in der öffentlich Meinung}

Erfinder sind hoch respektiert, zugleich wird der methodologische Fortschritt
beim Denken und erfinderischen Denken wenig beachtet. Weltbekannte Erfinder
wie Zeppelin oder Edison sind jetzt absolut berühmt bei jungen Menschen unter
30 Jahren. Erfinderisches Denken und methodologische Fortschritte beim Denken
können von den meisten Personen oder Studenten nicht erklärt werden. Wenn es
um die Methodologie geht, dann sollten erst die Ontologie und
Erkenntnistheorie bekannt sein, was von normalen Personen in kurz Zeit nicht
leicht zu verstehen ist. Besonders schwer ist es in einem asiatisches Land, wo
es keine Traditionen in der alten griechischen Kultur der Philosophie gibt.
Statt Mathe und Naturphilosophie hatte sich in China eine soziale Philosophie
entwickelt. So legte die chinesische Philosophie mehr Wert auf die Beziehungen
zwischen Menschen. Konfuzius unterteilt diese Relationen in Hierarchie und
Ordnung.  Eine solche Philosophie prägt auch die heutigen Chinesen. Aus diesem
Grund war Edison früher ein ganz positives Beispiel, jetzt ist sein Effekt in
China nicht mehr so positiv, wenn mehr und mehr Leute erfahren, was zwischen
Edison und Tesla passiert ist.

Im chinesischen Volksmund wird das Bild eines Erfinder mit den Werten „gut
ausgebildet“, „leistungsfähig“, „hoch intelligent“ verbunden. Wer in diesem
Bereich ein Stelle hat, zu forschen oder zu trainieren, wird auch von den
meisten Menschen respektiert, sodass nun auch die Professoren der
Kunst-Fakultät im Bereich TRIZ forschen möchten, nachdem die anfängliche
Arbeit der Verbreitung der TRIZ in China meistens von Professoren aus
Ingenieurbereichen geleistet wurde. Es gibt auch einen kleinen Kreis von
Leuten, die TRIZ als „Pseudowissenschaft“ bezeichnen.  Solche Meinungen haben
wenig Unterstützung in der Bevölkerung.

Erfinden ist für Chinesen ein geheimnisvolles Wort. Fast jeder Chinese möchte
etwas erfinden, damit er Anerkennung in der Gesellschaft findet, um bessere
Berufsaussichten zu haben. Wenn sich eine Chance anbietet, Kompetenz im
Erfinden zu erwerben, werden viele die Möglichkeit ergreifen, ohne zu
überlegen, ob dies am Ende wirklich zu schaffen ist. Im Gegensatz zu
westlichen höher entwickelten Ländern sind die Gesetze in vielen Bereichen --
besonders im Hightech-Bereich -- in China nicht so klar und eindeutig, was die
Anerkennung der Gesellschaft sehr wichtig macht.

\subsection{Das Nationale Forschungszentrum für technologische Innovation}

Das Nationale Forschungszentrum für technologische Innovationsmethoden und
-werkzeuge [7] als Gründerparkett und das Wissenschaftsministerium als
Gründerpate.

Im Juli 2007 antwortete Premierminister Jiabao Wen auf den Brief der drei
Wissenschaftler Daheng Wang, Dongsheng Liu, Du Ye und gab die Anweisung “Für
unabhängige Innovation kommt die Methode zuerst“ des Wissenschaftsministeriums
heraus. Das Wissenschaftsministerium, das Bildungsministerium und der
Wissenschafts- und Technologieverband arbeiteten zusammen zur Verbreitung und
weiteren Erforschung der TRIZ. Lokale Regierungen in verschiedenen Provinzen
Chinas folgten den oben genannten vier Organisation nach.  Das
Wissenschaftsministerium hatte dafür auch einen Forschungsausschuss
gebildet. Alle Unis, die Anfang 2008 zu TRIZ forschten, brannten auf die
finanzielle Unterstützung vom Wissenschaftsministerium. Für die Auswahl der
Firmen, welche an den ersten TRIZ Trainings teilnehmen durften, existierten
einige Regeln. Das Büro für Wissenschaft und Technologie [8] in der
Heilongjiang-Provinz, welches ein Ausschuss des Wissenschaftsministeriums in
Heilongjiang ist, hat uns ein gutes Beispiel der Auswahl von Firmen in China
gegeben. Notwendige Bedingungen dafür waren, dass der Umsatz der Firma pro
Jahr mehr als 5 Million RMB und die Steuerzahlung pro Jahr mehr als 0.5
Million RMB betragen. Die Firmen müssen im Bereich moderner Produktionen,
Information, neuer Materialien, neuer Energien, Biologie und Medizin,
Umweltschutz oder öffentlicher Sicherheit arbeiten. Und mindesten ein Patent
einer Erfindung oder eine gleichwertige Leistung waren auch gefordert. Nachdem
die Firma als qualifiziert ausgewählt war, kommt dann die kraftvolle
Unterstützung von der Organisation, nämlich vom Büro für Wissenschaft und
Technologie der Heilongjiang Provinz. Die Unterstützung erstreckte sich
hauptsächlich auf die Priorisierung der Firmen bei allen Projekten für die
lokale Regierung sowie auf die Hilfe durch Professoren der TRIZ bei
Innovationen, die von der Regierung organisiert werden. Diese Unterstützung
ist nicht mit Geld messbar, wenn die messbar wäre, dann ginge das in die
Millionen. Die Anzahl der Firmen ist 130 [9], die Trainings betreffen 4000
Personen.

Nach 11 Jahren ist jetzt TRIZ fast in allen großen Firmen in allen Provinz
eine der bekanntesten Theorien.

Bei allen Veranstaltungen ist die Führung von oben oder die Unterstützung von
oben durch die kommunistische Partei immer sehr wichtig. Im Jahr 2012 wurde
Likeqiang Premierminister, er setzt mehr auf das Modell „Internet+“ der
Wirtschaft.  Trotzdem hat TRIZ bis jetzt in China noch einen großen Markt.

\section{Rahmenbedingungen der TRIZ im 21. Jahrhundert}

\subsection{Das gesellschaftliche Ansehen der Erfinder in China}

Um Erfinder in China wird sich von der Regierung gut gekümmert, Titel werden
auf die Erfinder verteilt und viele Vorteile gewährt, wie Kredite bei
staatlichen Banken und so weiter, im Vergleich zu anderen Leuten. Was nicht
gut und strikt geschützt wird, sind die Patente, welche dem Erfinder erteilt
werden. Der gesetzliche Schutz von Patenten und Copyright ist eine
komplizierte Aufgabe. Für eine kommunistische Regierung scheint diese Aufgabe
immer eine unlösbare Aufgabe. Die Logik dahinter ist, dass das Volk in der
kommunistischen Nation gemeinhin arm und die Kosten eines Patents ziemlich
hoch sind. Wenn es einen strengen Schutz der Patente gibt, dann sind die
Kosten für die meisten im Volk bezüglich der Einkommen untragbar, was zu
Schäden in der Wirtschaft führt und und im Volk nicht zu Zufriedenheit und
Lebensqualität beiträgt.  2015 hat die chinesische Regierung entschieden, aus
diesem Teufelskreis auszubrechen. Eine bedeutsame Änderung ist, dass jetzt von
Baidu (die größte Suchmaschine in China, deren Gründer ist ein
Stellervertretender Gouverneur in einer Provinz in China) fast keine freien
Ressourcen von Büchern mehr gefunden werden. In Alibaba existieren jetzt
weniger Fälschungen von Produkten bekannter Marken.

In China ist das Patent in 3 Klassen unterteilt, nämlich Erfindungspatent,
Gebrauchsmuster und Design. Für das Design ist die Beantragung im Patentamt am
einfachsten, entsprechend ist der Schutz dieser Designs am schwächsten.

Ob mit einem Patent [10] Geld verdienen werden kann, hängt davon ab, ob das
Patent einen Markt hat und ob es eine Firma gibt, die dieses Patent kaufen
oder benutzen möchte. Der Preis für ein normales Patent beim Verkauf lag 2018
bei 2000-3000 RMB, für einige wichtige Patente ist der Preis höher, von 10
Tausend RMB bis eine Million RMB. Für die Anwendung eines Patents ist ein
ausführlicher Vertrag zwischen der Firma und dem Inhaber des Patents
erforderlich.  Mit dem Patent kann man auch von der Bank Geld ausleihen, wie
hoch der Geldbetrag ist, entscheidet die Bewertung des Patents durch die Bank.
Ein Patent kann auch an der Börse gehandelt werden. Dies sind die vier Wege,
auf denen die Erfinder durch ein Patent Geld verdienen, aber realistisch
gesagt, dies ist schwer, da es zahlreiche ähnliche Patente im chinesischen
Patentamt gibt. Ohne strenge Gesetze ist die Korruption im Patentamt eine
typische Krankheit.  Aber für die Firmen, welche vom Wissenschaftsministerium
für das Training der TRIZ ausgewählt werden, ist es leichter, mit einer
Innovation ein Patent bei Patentamt zu beantragen.

Die im Patentamt in China von 2012-2014 angemeldeten Patente werden nur zu 2\%
angewendet.  Im Falle einer Markenverletzung beläuft sich der
durchschnittliche Entschädigungs"|betrag nach dem Urteil der Gerichte auf 62
Tausend RMB in China. Dieser Betrag steht in keinem Verhältnis zu den großen
Anstrengungen der Inhaber, die Marke für eine lange Zeit aufrecht zu erhalten.
Solch ein Betrag in Europa und den USA ist höher, er betrug in den USA in
denselben Jahren 29 Million.

\subsection{Freiräume, sachbedingte Hürden und Stimuli der Erfindertätigkeit} 

Die Teilnahme von Ingenieuren an den Trainingskursen der TRIZ zielt neben dem
Training darauf, selbst interessante Ingenieure kennenzulernen und in Kontakt
zu bleiben. In der chinesischen Gesellschaft heißt das wählen, mit wem
zusammenzuarbeiten, das ist ein Freiraum. Diese Wahl ist auch bei innovativen
Prozessen die entscheidende Sache. Mehr oder weniger wie bei einem MBA (Master
Business Adminstration) in China. Wobei die Hauptaufgabe ist, berühmte
Personen oder Stars kennenzulernen und mit anderen Personen die Freundschaft
zu verstärken. Für eine Person wird die Fähigkeit, mit Personen gut
zusammenzuarbeiten, die verschiedenen Charakter haben, wichtiger als andere
Kompetenzen.

Exakte Information zu bekommen, ist eine sachbedingte Hürde der
Erfindertätigkeit, wenn ein Erfinder ein Patent erwerben möchte. Diese Hürde
vergrößert sich, falls das Gesetz nicht vollständig ist. Dabei spielt
Korruption eine große Rolle. Z.B. ist eine Überprüfung der Marke im Patentamt
sehr subjektiv [11], der Sachbearbeiter entscheidet, ob diese Überprüfung
erfolgreich ist.

Zielgruppen der TRIZ-Trainings sind Ingenieure, Manager und Studenten, von
denen auch erfinderische Aktivitäten erwartet werden. Organisiert wird das
Training nach verschiedenen Modellen, z.B. wenn das Training von der
Technischen Universität Hebei angeboten wird, dann ist das Modell MEOTM, wenn
das Training Iwint anbietet, dann ist das Modell ein anderes, z.B.  DAOV(y+2)
[12], DAVO bedeutet „Define, Analyse, Optimise, Verify“.  $Y$ ist Jahr (year
auf Englisch), $y+2$ impliziert, was soll oder muss die Firma in 2 Jahren
erreicht haben. Mit verschiedenen Trainingsmodellen werden unterschiedliche
Punkte der Erfindertätigkeit angesprochen. Es gibt kein einheitliches Modell
in China, obwohl alle Modelle die gleiche Basis haben und alle Organisation
versuchen, TRIZ zu standardisieren.

\subsection{Einfluss des Themencharakters und der Themenvorbereitung auf
  die erfinderische Aktivität} 

In einem von der kommunistischen Partei geführten Land wie China dient die
Wirtschaft der Politik. Der Staat hat eine starke Kontrolle über die
Wirtschaft trotz des Willens, der Wirtschaft mehr Freiheit zu geben. Für
erfinderische Aktivitäten gibt es auch eine Leitlinie der Regierung im Namen
des Gesellschaft, welche die erfinderischen Aktivitäten beschränkt und leitet.

In den vergangenen vierzig Jahren hat sich die chinesische Wirtschaft schnell
entwickelt, so hat die Regierung jetzt genügend Ressourcen, ihren
Perspektivplan auch umzusetzen. Die Firmen bevorzugen, Themen zu nehmen, die
im Perspektivplan der zentralen Regierung gelistet sind. Damit können sie mehr
und sogar große Unterstützung von allen Seiten erhalten. Im heutigen China
sind alle hochtechnologischen Erfindungen vom Staat direkt oder indirekt
finanziert. Die 5-Jahres-Planung der Regierung dominiert schon seit 40 Jahren
die erfinderischen Aktivitäten.

2018 hatte Professor Jiankui He eine Experiment mit einem Gen-Editor bei
Menschen durchgeführt, ohne genug auf Sicherheit zu achten, was direkt dazu
führte, dass er seine Stelle als Professor an einer der bekanntesten Unis in
China verlor. Er lebt jetzt unter Überwachung der Polizei.

\subsection{Erfinden in China als Basis finanzieller Freiheit}

Abraham Maslow hat die Maslowsche Bedürfnishierarchie [12] entwickelt, welche
die menschlichen Befürfnisse und Motivationen in einer hierarchischen Struktur
auf eine vereinfachte Weise darstellt. Diese Hierarchie ist aus
psychologischer Sicht aufgestellt. Ähnlich gibt es eine Bedürfnishierarchie
aus finanzieller Sicht. In dieser Bedürfnishierarchie gibt es insgesamt 9
Schichten.
\begin{itemize}
\item[1.] Supermarktsfreiheit

Im Supermarkt kaufen Sie die Dinge, die Sie benötigen, egal ob der Preis
einiger Dinge zu hoch ist.

\item[2.] Essen-Freiheit

Wenn Sie ins Restaurant zum Essen gehen, ziehen Sie als erstes die Gerichte,
die Dekoration, den Service in Betracht und nicht den Preis der Gerichte.

\item[3.] Reisefreiheit

Wenn Sie reisen, spielt bei der Auswahl der Sehenswürdigkeit nicht zuerst der
Preis, sondern das innere Interesse eine Rolle, wohin Sie reisen möchten.

\item[4.] Auto-Freiheit

Wenn der Preis verschiedener Auto-Marken innerhalb ihrer finanziellen
Möglichkeiten liegt, dann ist das Auto nicht mehr nur ein Transportmittel,
sondern ein wesentlicher Bestandteil ihrer Lebensqualität.

\item[5.] Ausbildungsfreiheit

Ihre finanzielle Kraft garantiert Ihren Kindern eine qualitativ hochwertige
öffentliche oder private Ausbildung. Ob es um den Kauf eines Hauses im
richtigen Schulbezirk oder die Zahlung hoher Studiengebühren geht, es ist der
Wunsch jedes Elternteils, den Kindern eine gute Ausbildung zu ermöglichen.

\item[6.] Arbeitsfreiheit

Solange Sie dazu bereit sind, können Sie diesen Job oder jenen Job nach ihren
eigenen Interessen wählen, dran arbeiten oder nicht dran arbeiten, es ist
Ihnen egal, was Sie bei dieser Arbeit verdienen, Sie achten mehr auf den Spaß
und das Erfolgserlebnis, das die Arbeit mit sich bringt.

\item[7.] Medizinische Freiheit

Bei der Auswahl eines Krankenhauses steht die Qualität der medizinischen
Versorgung vor den Kosten.

\item[8.] Wohnungsfreiheit

Sie kaufen ein Haus in der Stadt, wo Sie wohnen, danach, ob das Haus ihnen am
besten gefällt ohne Berücksichtigung des Preises.

\item[9.] Staatsangehörigkeitsfreiheit

Sie können ihre Nationalität und ihren Lebensstil über verschiedene Kanäle
frei wählen, z.B. durch Auswanderung, womit Sie nicht mehr an die
Landesgrenzen gebunden sind.
\end{itemize}

Die meisten Leute in China glauben an Materialismus. Nur Idealisten betrachten
die menschliche Selbstverwirklichung als höchstes Bedürfnis. Die finanzielle
Bedürfnishierarchie passt besser auf die Chinesen. Die meisten Leute in China
gehören zu den ersten Schichten in und einer von zehn Chinesen hat die oberste
Schicht 9 erreicht.

Das Volk in China legt viel Wert auf das Erfinden, weil Erfinder mit ihren
Erfindungen eine große Menge Geld verdienen können. Die gute und naive
Motivation von Erfindern, das will ich eben erfinden, weil ich das für besser
halte als nur zu verwalten oder herumzusetzen, war schon lange vor der Reform
und Öffnungspolitik 1978 in China verschwunden.

\section{TRIZ auf dem Weg in die Wirtschaft}

\subsection{Training der TRIZ: sowohl theoretisch als auch praktisch}

Für die internationale Zertifizierung der TRIZ dauert es bei Level 1 [13] 3
Tage, wo nur theoretische Kenntnisse wie Funktionsanalyse,
Kausalkettenanalyse, Trimmen, funktionsorientierte Suche, technischer
Widerspruch, physikalischer Widerspruch, Erfindungsprinzip kennengelernt und
diskutiert werden. Theoretisch kann jeder Teilnehmer sein, aber praktisch sind
die Teilnehmer Mitarbeiter, wie Ingenieure oder Manager aus Firmen.

Dem Training für die internationale Zertifizierung auf Level 2 [14], welche
insgesamt 8 Tage dauert mit Kosten von 16 Tausend RMB per Teilnehmer in China,
sollten normalerweise praktische Übungen vorausgehen, die hauptsächlich auf
die Bereiche der technischen und physikalischen Widersprüche und der
maßgebenden Lösungen ausgerichtet sind.

Teilnehmer für das Training auf Level 3, welches 14 Tage dauert mit Kosten von
40 Tausend RMB, müssen schon die Zertifizierung auf Level 2 vorweisen. Auf
Level 3 soll ARIZ an jedem Tag des Training theoretisch detailliert erklärt
werden. Die theoretischen Erklärungen werden direkt von einer praktische Übung
begleitet.

Teilnehmer für die Zertifizierung auf Level 4 brauchen die Empfehlung eines
TRIZ Masters, von denen in China 2018 erst einer existierte.

Das Training der TRIZ erhöht die Fähigkeit der Ingenieure, die Produkte in den
Firmen zu verbessern.
  
\subsection{Erwartungen an den Trainer}

Die Teilnehmer sind nicht Jugendliche oder Studenten, sondern Ingenieure mit
mehrjährigen Erfahrungen in der Arbeit. Wegen der hohen Kosten der Trainings
erwartet der Trainer von den Teilnehmern, die Abschlussprüfung zu bestehen,
damit sie die Zertifizierung bekommen.  Während der Trainings werden allgemein
die Teilnehmer mit guten Beziehungen zu anderen und scharfem Sinn höher
geschätzt.

\subsection{Das Training und Karrierechancen}

Viele Leute im unteren Management einer Firma nehmen an den TRIZ-Trainings
teil.  Nach dem Training haben sie eine größere Chance, auf der Karriereleiter
aufzusteigen. Es ist eine unausgesprochene Regel, nur die Leute, die
potenzielle Leitungsfähigkeiten haben, für die weitere Ausbildung auszuwählen.
Wenn sie die Ausbildung erfolgreich absolviert haben, dann kommt normalerweise
ein schneller Aufstieg in der Firma.

\section{Das TRIZ-Training in staatlichen Unternehmen}

\subsection{Innovationspolitik und TRIZ-Training in chinesischen staatlichen
  Unternehmen}  

Die meisten Leute denken, staatliche Firmen in China haben die Fähigkeit,
innovativ zu sein, weil sie die besten Talente und eine staatliche
Finanzierung bekommen. Aber diese Firmen verzeichnen weniger Innovationen im
Vergleich zu privaten Firmen, weil es zu viele Faktoren gibt, die Innovationen
behindern, wie zum Beispiel [16]:
\begin{itemize}
\item[1.] In den Zielerreichungsindikatoren der CEO der staatlichen Firmen in
  China sind Innovation nicht enthalten. Da Innovation nicht direkt im
  Zusammenhang mit der Einschät"|zung der Arbeit der CEO steht, fehlt den CEO
  die Motivation zur Innovation.
\item[2.] Die erste versteckte Regel in staatlichen Unternehmen ist, dass
  neues Geschäft im ersten Jahr rentabel sein muss, sonst ist das Neugeschäft
  für das Unternehmen nicht geeignet.
\item[3.] Der auf Konsens basierende Entscheidungsmechanismus in staatlichen
  Unternehmen ist langwierig und ungünstig für die Innovation.
\item[4.]  Für staatliche Unternehmen ist es leichter, durch politische
  Faktoren die Zielerreichungsindikatoren zu erfüllen als durch Innovation.
\item[5.] Die Zeit, die eine Innovation braucht, um Geld zu erwirtschaften,
  ist mit großer Wahrscheinlichkeit länger als die Amtsdauer des CEO.
\end{itemize}
Durch die Kooperation mit ausländischen Unternehmen haben sich die staatlichen
Unternehmen in der vergangenen 20 Jahren technisch gerettet.  Kooperationen
gab es hauptsächlich in Branchen wie Energie, Eisenbahn und
Energieübertragung. In diesen Bereichen hat China einen großen Markt und die
ausländischen Firmen die beste Technik.

Bei TRIZ-Trainings stehen auf jeden Fall die staatlichen Unternehmen auf der
erste Linie. Viele Ingenieure und Manager staatlicher Unternehmen haben bis
jetzt schon TRIZ-Trainings absolviert. Die Teilnehmer der Trainings, welche
von der Technischen Universität Hebei durchgeführt wurden, kamen meistens aus
staatlichen Unternehmen.

\subsection{Die Hauptprobleme der Verbreitungen der TRIZ in der realen
  Wirtschaft in China} 

\subsubsection{Die Probleme von TRIZ selbst [17]}

\paragraph{1.}
Wegen zu vieler verschiedener Tools ist es bei der Auswahl der Tools sehr
leicht, falsch zu kombinieren.

Im Problemlösungsprozess der TRIZ gibt es viele Tools, mit denen ein
bestimmtes Problem in ein allgemeines Problem umgewandelt werden kann. Diese
Werkzeuge können alternativ eingesetzt werden und haben keine spezifische
Beschreibung, in welchem Fall welche Tools verwendet werden sollten. Z.B.
muss bei der Umwandlung eines spezifischen Problems in ein Standardproblem das
Problem verstanden und beschrieben werden. Dafür sind verschiedene Werkzeuge
verfügbar wie die Neun-Felder-Methode, das ideale Endresultat, die Methode der
Funktionsanalyse, Trimmen, die S-Kurve, die Mind-Setting-Methode usw. Für die
Transformation des Problems sind die verfügbaren Werkzeuge TRIZ-Prinzipien und
Widerspruchsmatrix, Separationsprinzipien, Standardlösungen, Ressourcenanalyse
und Trends der technologischen Evolution. In beiden Teilprozessen sind die
Tools alternativ verwendbar. Und die beim Verstehen und Beschreiben des
Problems verwendeten Methoden bestimmen auch nicht, welche Methode bei der
Transformation des Problem anzuwenden ist.

\paragraph{2.}
Die allgemeine Löung ist zu abstrakt.

Bevor die spezifische Lösung für das Problem gefunden werden kann, müssen wir
bei Verwendung der TRIZ erst die allgemeine Lösung finden, aber die mit
verschiedenen Tools gefundenen allgemeinen Lösungen sind zu abstrakt. Die
abstrakte Lösung ist ein Hinweis auf die Lösung des spezifischen Problems und
nicht die direkt Antwort zum Problem.

\paragraph{3.}
Der Problemlösungsprozess hängt weitgehend von den fachlichen Kenntnissen der
Anwender ab.

Nicht nur bei der Definition und Beschreibung des Problems, sondern auch der
Weg zu einer konkreten Lösung hängt weitgehend vom Hintergrundwissen der das
Problem lösenden Person ab, deren fachliches und impliziertes Wissen
eingeschlossen.

\paragraph{4.}
Ansätze von Versuch-Irrtum sind bei der Definition der Probleme und in anderen
Teilprozessen erforderlich.

Beschreibung und Definition des Problem ist sehr wichtig. Je klarer dies
erfolgt, desto leichter ist es, eine Lösung zu finden. Im praktischen Prozess
der TRIZ ist es fast unmöglich, das geeignete Tools zu finden ohne spezifische
Anweisungen zu den Charakteristika der Tools, deshalb ist die Schleife der
Versuch-Irrtum unvermeidbar.

\subsubsection{Die spezifischen Probleme bei der Verbreitung der TRIZ in
  China} 

\paragraph{1.}
Die Verbreitungen der TRIZ wird von oben betrieben und nicht von der Firma
selbst.

Die Firmen, welche an TRIZ-Trainings teilgenommen haben, haben kein großes
Bedürfnis nach TRIZ, sondern werden von anderen Sachen wie finanzieller
Unterstützung der Regierung angezogen.

\paragraph{2.}
TRIZ Software ist teuer.

Um TRIZ-Software zu verwenden, fallen hohe Investitionskosten an.  Unternehmen
sind nicht bereit, so viel Geld in eine noch nicht bekannte Methode zu
stecken.

\paragraph{3.}
Das Ausbildungssystem in China ist isoliert von der TRIZ-Methode.

Chinas indoktrinierenden und vom Auswendiglernen geprägte Lehrmethoden haben
die Entwicklung von innovativem Denken in gewissem Maße gebremst. Im
Materialismus der Ausbildung der Studenten in China steht Faktenwissen im
Vordergrund, was in Büchern als Kenntnisse präsentiert wird, aber nicht die
Art und Weise, wie man denken soll, um Wissen zu erzeugen.

\subsection[Sozialistisches Sozialbewusstsein und dessen Einfluss auf die
  Erfindertätigkeit]{Sozialistisches Sozialbewusstsein und dessen Einfluss\\ auf
  die Erfindertätigkeit}

Die Merkmale der Sozialismus sind im heutigen China sehr schwach ausgeprägt.
In der chinesischen Marktwirtschaft gibt es fast kein Sozialbewusstsein. Der
demokratische Kapitalismus hat einen großen Einfluss auf die
Erfindertätigkeit.  Die USA bezeichnet das Modell des chinesischen Markts als
Nationalkapitalismus. Das ist allerdings ungenau, diese Charakterisierung
fokussiert nur auf einen Punkt und ignoriert zugleich die ganze Breite der
Entwicklung. Ziel der erfinderischen Tätigkeiten ist die Verbesserung der
Lebensbedingungen. So zielt z.B. WeChat Pay auf die Vereinfachung des
Bezahlens. Eine Nebenwirkungen ist, dass beim Bezahlen viel persönliche
Information automatisch gespeichert wird. Die Nebenwirkungen sind veränderbar,
ohne die Hauptwirkung zu verändern. Wenn man die Hauptwirkung und die
Nebenwirkung vermischt, dann wird man sagen, dass diese Erfindung
hauptsächlich dem Sammeln von Daten dient.

\section{Literatur}
\begin{itemize}
\item[{[1]}] Forschung und Förderung der TRIZ in Festlandchina: Aktueller
  Stand und Probleme (in Chinesisch).
  \url{../2019-05-16/cChen-TRIZinChina2009.pdf} 
\item[{[2]}] Innovative Methoden in modernen Verwaltungssystem. Zukunft und
  Dynamik (in Chinesisch).\\{\small
  \url{https://wenku.baidu.com/view/6ff603e1f56527d3240c844769eae009581ba2ff.html}} 
\item[{[3]}] Die Firma Iwint (in Chinesisch).\\
  \url{http://www.iwintall.com/iplat-cms/portal/aboutUs/expertTeam}
\item[{[4]}] Runhua Tan. TRIZ, The development and dissemination in industries
  in China.  Proceedings TRIZCON 2017.\\
  \url{../2019-05-16/RunhuaTan-TRIZCON2017.pdf}
\item[{[5]}] Struktur und Funktionen der Pro/Innovator
  F\&E-Innovationsplattform (in Chinesisch).\\{\small
  \url{http://www.iwintall.com/iplat-cms/portal/enterpriseService/productImprovement}}
\item[{[6]}] Die Verbreitung der TRIZ-Theorie und ihre Auswirkungen auf unser
  Land (in Chinesisch).
  \url{https://wenku.baidu.com/view/48e497a0852458fb760b561b.html}
\item[{[7]}] Nationales Forschungszentrum für technologische
  Innovationsmethoden und Werkzeuge (in Chinesisch).
  \url{http://TRIZ.hebut.edu.cn/zxgk/zxjg/index.htm}
\item[{[8]}] Bekanntmachung über die Organisation von Pilotunternehmen für
  technologische Innovation (TRIZ-Theorie, in Chinesisch).\\
  \url{http://219.147.168.134/html/zwgk/TZTG/xztz/show-4536.html}
\item[{[9]}] Rede des Stellvertretenden Direktors der Abteilung für
  Wissenschaft und Technologie der Provinz Heilongjiang (in Chinesisch).\\
  \url{http://www.TRIZ.gov.cn/index.php?s=/home/index/newsdetail/id/930.html}
\item[{[10]}] Patentgesetz Chinas (in Chinesisch).
  \url{http://m.liuxiaoer.com/qz/2755.html}
\item[{[11]}] Überprüfungen am Markt sind zu subjektiv beim Patentamt,
  26.04.2018 (in Chinesisch). \\
  \url{http://www.gov.cn/hudong/2018-04/26/content_5286093.htm}
\item[{[12]}] Maslowsche Bedürfnishierarchie.\\
  \url{https://de.wikipedia.org/wiki/Maslowsche_Bed%C3%BCrfnishierarchie}
\item[{[13]}] Materialien der TRIZ-Trainings Level 1 (in Chinesisch).\\
  \url{https://baijiahao.baidu.com/s?id=1624057263861494837}
\item[{[14]}] Materialien der TRIZ-Trainings Level 2 (in Chinesisch).\\
  \url{http://blog.sina.com.cn/s/blog_a651266e01018pcc.html}
\item[{[15]}] Materialien der TRIZ-Trainings Level 3 (in Chinesisch).\\
  \url{https://wenku.baidu.com/view/570e78e065ce0508763213cb.html}
\item[{[16]}] Sind chinesische staatliche Unternehmen innovativ? (in
  Chinesisch).  McKinsey China, 18.12.2018.
\item[{[17]}] Probleme und Lösungen der Verbreitung innovativer
  TRIZ-Methoden.\\ 
  \url{http://articles.e-works.net.cn/CAI/Article83271_1.htm}
\item[{[18]}] Hans-Jochen Rindfleisch, Rainer Thiel. Erfinderschulen in der
  DDR.  Trafo Verlag, Berlin 1994.
\end{itemize}

\end{document}
