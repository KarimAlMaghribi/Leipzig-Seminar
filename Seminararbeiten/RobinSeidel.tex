\documentclass[11pt,a4paper]{article}
\usepackage{a4wide,url,graphicx}
\usepackage[utf8]{inputenc}
\usepackage[ngerman]{babel}

\parindent0pt
\parskip3pt

\newcommand{\HGG}[1]{\begin{quote} \emph{Anmerkung Gräbe:} #1  \end{quote}}

\title{Seminararbeit Patentrecherche zum\\ „Prinzip der Dynamisierung“}

\author{Robin Seidel, mit Anmerkungen von Hans-Gert Gräbe}
\date{30. September 2019} 

\begin{document}
\maketitle

\HGG{Die Arbeit wurde im Rahmen der Begutachtung orthografisch und
stilistisch überarbeitet.}

\section{Allgemeines}
In dieser Seminararbeit soll anhand von fünf Erfindungen dargestellt werden,
wie Altschullers Gedanken zum schöpferischen Prozess in der Ingenieurskunst
angewendet werden können, um eine Herangehensweise zu entwickeln, wie aus
allgemeinen Problemen eine Lösungsstrategie für spezielle Probleme abgeleitet
werden kann.  Speziell soll hier gezeigt werden, wie auf der Basis des
\emph{Prinzips der Dynamisierung} Probleme gelöst werden können. Allgemein ist
dieses Innovative Prinzip aus der Widerspruchsmatrix in folgenden
Charakteristika der spezielle Lösung zu finden:
\begin{itemize}
\item
Die Kennwerte des Objektes (oder des umgebenden Mediums) müssen sich so
verändern, dass sie in jeder Arbeitsetappe optimal sind.
\item
Das Objekt ist in Teile zu zerlegen, die sich zueinander verstellen oder
verschieben lassen.
\item 
Falls das Objekt insgesamt unbeweglich ist, ist es beweglich (verstellbar) zu
gestalten.
\end{itemize}

\section{Beispiel: Der Abgasturbolader}

\subsection{Metadaten}
\begin{itemize}\itemsep0pt
\item Titel: Turbolader
\item Patentnummer: EP2158386B1
\item Quelle: \url{https://patents.google.com/patent/EP2158386B1}
\item Patentinhaber: LINDENMAIER GMBH, SYCOTEC GmbH and Co KG
\item Patentdaten: Veröffentlichung 2017-07-12
\end{itemize}

\subsection{Beschreibung des Stands der Technik}
Das Patent beruht auf dem Prinzip der Gleichdruck- oder Stauaufladung (Patent
von Alfred Büchi -- 1905).  Dieses beschreibt den Turbolader, der den Stand
der Technik darstellt.  Ziel ist es die Leistung eines Motors zu erhöhen,
indem man den Druck in der Brennkammer erhöht.  Die Funktionsweise eines
Turboladers ist folgende: Ein Teil der Luft des Abgasstromes wird in eine
Turbine umgeleitet. Diese Turbine treibt dann einen Verdichter an. Dadurch
wird der Druck im Ansaugsystem des Motors erhöht. Somit kann viel mehr Luft in
die Brennkammern gelangen.  Die Verbrennung des Benzins kann effektiver
stattfinden. Es wird mehr Energie frei.  Dies führt dazu, dass trotz
gleichbleibender Motorgröße eine größere Leistung erzielt werden kann.
Schwachstelle dieses Systems ist das sogenannte „Turboloch“. Es beschreibt die
Verzögerung, bevor die Leistung explosionsartig einsetzt. Durch mechanische
und thermische Trägheit kommt es im niedrigen Drehzahlbereich zu
Verzögerungen, bis die komprimierte Luft in den Brennkammern ankommt.

\HGG{Stand der Technik ist die Verwendung von Abgasenergie, um durch
  Übertragung mechanischer Energie (von der Turbine im Abgasstrom auf den
  Verdichter im Ansaugstrom) der angesaugten Luft Energie durch Kompression
  zuzuführen.  Mit der Druckerhöhung erhöht sich zugleich die Temperatur der
  angesaugten Luft. Damit wird der Verbrennungsprozess optimaler und die
  Abgasenergie steigt. Der Prozess hat also einen speziellen optimalen
  Betriebsbereich (Operative Zone), der erst erreicht werden muss. }

\subsection{Funktionales Modell}
\paragraph{Obersystem:}
Die Energie fließt vom Abgasstrang in eine Turbine. Das Abgas wird
beschleunigt in einen Verdichter geleitet. Schließlich gelangt die Luft wieder
in die Brennkammer. Dort wird sie zusammen mit Benzin zur Entzündung gebracht.

\HGG{Das ist falsch dargestellt. Turbine und Verdichter sind über ein
  mechanisches System gekoppelt. Abgas und Ansaugluft werden nicht
  vermischt. }

\paragraph{Komponenten:}
\begin{itemize}
\item Elektromotor: Steuert die Energiezufuhr in die Brennkammer, indem er den
  Luftstrom regelt in Betriebszuständen, wo der Abgasladedruck nicht ideal
  ist.
\item Turbine: Mechanisches Bauteil, das Luftfluss beschleunigt.
\item Verdichter: Bauteil, das die Luft komprimiert. 
\end{itemize}

\HGG{Das ist falsch dargestellt. Die Turbine koppelt mechanische Energie aus
  dem Abgasstrom aus. Diese wird auf den Verdichter übertragen, der diese
  Energie zum Komprimieren der Ansaugluft verwendet.  Der Elektromotor stellt
  diese mechanische Energie bereit, wenn die Turbine noch nicht „liefert“. }
  
\subsection{Formulierung des Miniproblems}
Ein Problem bei dieser Druckerhöhung in der Brennkammer ist eine höhere
mechanische Belastung, woraus eine kürzere Lebensdauer des Motors resultiert.
Außerdem ist die Konstruktion aufwendiger, da weitere Bauteile hinzukommen.
Das eigentliche Problem ist jedoch, dass durch den Turbolader zwar zu
bestimmten Zeitpunkten des Beschleunigungsvorganges mehr Druck in der
Brennkammer herrscht, dies aber sehr verzögert und nicht konstant ist.

Die durch Alfred Büchi vorgestellte Lösung ist nur halb-dynamisch. Mithilfe
des vorgestellten Patents EP2158386B1 wird dieses Problem beseitigt. Der
Ladedruck des Turboladers ist in allen Betriebszuständen optimal.

\HGG{Das Miniproblem ist das schnelle Erreichen der Operativen Zone des
  basalen Prozesses.}

\subsection{Beschreibung der Lösung des Problems in der Patentschrift}

Mithilfe eines Elektromotors kann exakt gesteuert werden, zu welchem Zeitpunkt
wie viel Luft in die Brennkammer gegeben wird. Dadurch wird das Turboloch so
weit verringert, dass der Ladedruck immer gleichmäßig ist.

\subsection{Einordnung des Problems in die TRIZ-Systematik}
Das Problem lässt sich nur sehr schwer in die TRIZ Matrix einordnen. Wenn man
als sich verbessernden Parameter die Leistung nimmt und als sich
verschlechternden den Energieverlust, bekommt man das Prinzip der
Eigenschaftsänderung (35) vorgeschlagen.

\HGG{Es kommt ganz klar das \emph{Prinzip 10 der vorgezonenen Wirkung} zur
  Anwendung.  Die Verdichterwirkung wird durch den Elektromotor bereits
  \emph{vor} Einsetzen der Hauptwirkung durch den Antrieb mittels Turbine
  erreicht.} 

\subsection{Bezug der speziellen Lösung zu den TRIZ-Strukturen}

Das Prinzip der Eigenschaftsänderung ist in diesem speziellen Zusammenhang ein
vergleichbares Prinzip wie das der Dynamisierung. Die Eigenschaft des Druckes
verändert sich so, dass die Kennwerte des Objektes in jeder Phase optimal
sind (siehe Prinzip der Dynamisierung).

\HGG{Relevant sind das \emph{Prinzip 12 der Äquipotenzialität} (die anfangs
  fehlende Energie wird durch eine zusätzliche Komponente zugeführt, die
  später auch überschüssige Eneregie abführen kann).  Relevant für das
  Grundprinzip des Turboladers ist das \emph{Prinzip 23 der Rückkopplung}.}

\section{Beispiel: Automatische Zylinderabschaltung}
\subsection{Metadaten}
\begin{itemize}\itemsep0pt
\item Titel: Zylinderabschaltung bei Otto-Motoren
\item Patentnummer: DE19606402C2
\item Quelle: \url{https://patents.google.com/patent/DE19606402C2}
\item Patentinhaber:  Rainer Born 
\item Patentdaten: Veröffentlichung 1998-08-13
\end{itemize}

\subsection{Beschreibung des Stands der Technik}
Alle Zylinder des Motors arbeiten zu allen Betriebszuständen gleichzeitig und
produzieren die gleiche Energie. Die Zylinder bekommen alle die gleiche
Treibstoffmenge zugeführt.

\HGG{In der Patentschrift wird der Stand der Technik anders beschrieben. Es
  werden mehrere Verfahren referenziert, in denen bereits
  Zylinderabschaltungen eingesetzt werden, die aber alle ihre Nachteile
  besonders in den im Patent genannten Bereichen haben.  Gegenüber dem Stand
  der Technik wird eine optimale Kombination von Ansätzen behauptet, die zu
  einer deutlichen Verbesserung des Betriebs nachgelagerter Komponenten
  (Abgaswerte und Katalysator) führen.

  Stattdessen wird im Weiteren das Grundverfahren analysiert, überhaupt
  Zylinder im Betrieb abzuschalten, das aber Stand der Technik ist und nicht
  Gegenstand der im aktuellen Patent erteilten Schutzrechte. }

\subsection{Funktionales Modell}

\HGG{Dieser Abschnitt fehlt. Obersystem ist die gesamte Antriebseinheit, also
  die über die Kurbelwelle gegebene Kopplung der Zylinder, das Abgassystem,
  der Katalysator sowie die zugehörigen Steuereinheiten.  Relevante
  Komponenten (laut Patentschrift) sind wenigstens: die Zylinder,
  Drosselklappen, Kraftstoffzufuhr, Ansaugrohre, Ventilsteuerung, Abgas
  (relevant, da hier Emissionsnormen einzuhalten sind), Katalysator. }

\subsection{Formulierung des Miniproblems}
Um ein Fahrzeug schneller zu beschleunigen, beziehungsweise damit es schneller
fahren kann, muss die Anzahl der Zylinder oder der Brennraum vergrößert
werden. Dies hat zur Folge, dass das Kraftfahrzeug mehr Motorleistung
generiert, aber dadurch auch mehr Benzinverbrauch verursacht.  Durch die
erhöhte Anzahl an Zylindern hat das Fahrzeug auch in Zuständen geringer
Motorlast einen verhältnismäßig erhöhten Kraftstoffverbrauch.

\HGG{In der Patentschrift ist das anders formuliert: „Aufgabe der Erfindung
  ist es, eine Zylinderabschaltung zum Zwecke der Kraftstoffeinsparung
  anzugeben, mit der die Abgasnormen eingehalten werden können, ein besserer
  Rundlauf erreicht wird und eine längere Aufrechterhaltung der
  Betriebstemperatur des Katalysators gewährleistet ist.“}

\subsection{Beschreibung der Lösung des Problems in der Patentschrift}
Durch eine Zylinder Ab- oder Zuschaltung, kann jeder Zylinder mit der
optimalen oder gar keiner Treibstoffmenge versorgt werden.  Somit ist eine
optimale Leistungsbeanspruchung zu jedem Zeitpunkt gewährleistet.  Es sind
hohe Geschwindigkeiten und eine Benzineinsparung gleichzeitig möglich.

\HGG{Es geht nicht um die  Ab- oder Zuschaltung von Zylindern an sich, sondern
um die Modi, in denen dies geschieht und den Auswirkungen auf den Betrieb des
Gesamtsystems. }

\subsection{Einordnung des Problems in die TRIZ-Systematik}
Die Lösung lässt sich in die Widerspruchsmatrix einordnen, da der verbesserte
Faktor die Geschwindigkeit oder die Leistung ist (9) und der sich
verschlechternde Faktor die Energiekonsumption eines beweglichen Objektes
(19). Die Matrix schlägt neben dem Prinzip des Gegengewichts (8), dem Prinzip
der Eigenschaftsänderung (35) auch das Prinzip der Dynamisierung vor.

\HGG{In der Patentschrift wird ausgeführt: „Das Leerlaufabschaltmuster ... ist
  der optimale Kompromiß für Vierzylinder-Ottomotoren zwischen den Kriterien
  hoher thermischer Wirkungsgrad durch hohe Kompression bei geringen
  Arbeitstakten einerseits, und geringe Stickoxidemission, der Standzeit des
  Katalysators oberhalb der Anspringtemperatur und Rundlaufeigenschaften
  andererseits.“ Es handelt sich also um eine klassische Kompromisslösung und
  nicht um eine Widerspruchslösung nach TRIZ.}

\subsection{Bezug der speziellen Lösung zu den TRIZ-Strukturen}
 
Die Lösung erfolgt hier durch Dynamisierung, da die Kennwerte des Objektes
sich so verän"|dern, dass sie in jeder Arbeitsetappe optimal sind. Es arbeiten
nur eine bestimmte Anzahl an Zylindern zur gleichen Zeit. Somit ist die in
einer Zeitspanne umgesetzte Energie im System optimal im Bezug auf die
Leistung, die das System oder das Fahrzeug zu einem bestimmten Zeitpunkt
gerade benötigt. Es geht wenig Energie verloren. Dies bedeutet im Speziellen
einen geringeren Kraftstoffverbrauch.

\HGG{Das trifft auf den Stand der Technik zu, überhaupt über
  Zylinderabschaltungen nachzudenken.  Die Kompromisslösung hier hat eine
  vollkommen andere Ausrichtung: Durch spezifische Betriebsbedingungen im
  System sollen spezifische Effekte (Abgaswerte, Lebensdauer des Katalysators)
  in anderen Systemteilen erreicht werden. Es wird hier also die
  \emph{Lösungsstrategie 3 Übergang ins Super- und Subsystem (Makro- und
    Mikrolevel)} aus den „76 Standardlösungen“ angewendet. 

  Als weiteres kommt das \emph{Prinzip 28 des Austauschs mechanischer
    Wirkschemata} zur Anwendung, da die neuen Betriebsstrategien nicht mehr
  auf Drosselklappen und ähnlichen mechanischen Steuerelementen aufsetzen,
  sondern unmittelbar auf der (digitalen) Einspritzsteuerung selbst. }

\section{Beispiel: Schwenkflügel bei Flugzeugen}
\subsection{Metadaten}
\begin{itemize}\itemsep0pt
\item Titel: Schwenkflügel bei Flugzeugen
\item Patentnummer: US2523427A
\item Quelle: \url{https://patents.google.com/patent/US2523427A}
\item Patentinhaber:  William J. Hampshire
\item Patentdaten: Veröffentlichung 1950-09-26
\end{itemize}

\subsection{Beschreibung des Stands der Technik}
Flugzeuge waren bis dahin so konzipiert, dass sie möglichst unter einem
Flugverhaltenstyp optimal funktionieren. Manche Flugzeuge waren so konzipiert,
dass sie möglichst schnell fliegen können, andere so gebaut, dass sie viel
transportieren oder Langsamflugfahrten möglich waren. Es war bisher nicht
möglich, die Flugzeugflügel so zu entwerfen, dass sie mehrere
Flugeigenschaften in einer Flugzeugzelle vereinen konnten.

\HGG{Für verschiedene Einsatzszenarien waren also verschiedene Flugzeuge
  erforderlich.} 

\subsection{Funktionales Modell}

\HGG{Hier fehlt dieser Abschnitt bereits komplett. Damit wird aber unklar,
  worüber überhaupt geredet wird.

  Allein auf die verstellbaren Flügel als \emph{System} zu schauen verkennt
  die Komplexität der Patentschrift, die sich auch mit Aspekten der Steuerung
  des \emph{Systems Flugzeug} befasst.  Als \emph{Obersystem} kann nur das
  System der Einsatzbedingungen am Himmel dienen, die Flügel sind damit nur
  ein \emph{Untersystem} des \emph{Systems Flugzeug}.

  In der Patentschrift wird genauer ausgeführt: „This invention relates to
  improvements in airplanes, and the primary object of the invention is to
  provide an airplane with wings which are adjustable to selectively provide
  the characteristics of a family or plurality of airfoils, and to provide
  mechanical means inside the wings to adjust the latter and thereby shift
  from one airfoil characteristic to another by changing the cross sectional
  contour of the wings.“ Es geht also (1) um die Vereinigung verschiedener
  Flugeigenschaften in einem Flugzeug.

  Weiter wird ausgeführt: „Another object of the invention is to provide
  mechanisms in the wings and the tail plane of an airplane that are operated
  together by electrical means so as to effect automatic trim for directional
  flight.“ Es geht also (2) um eine verbesserte Steuerungsfähigkeit des
  Flugzeugs durch einen „automatic trim for directional flight“.

  Weiter wird ausgeführt: „Another object of the invention is to provide
  improvements in the form of an airplane wing, whereby the same is most
  effectively suited to use of the automatic trim mechanism to combine speed
  and performance or maneuverability.“ Es geht also (3) um eine einheitliche
  Steuerungsfähigkeit in zwei verschiedenen („performance or maneuverability“)
  Betriebsmodi.

  Weiter wird ausgeführt: „Another object of the invention is to provide
  improved means for obtaining rolling action of the airplane without the use
  of ailerons and by means of a stick-operated reversible switch controlling
  motor-operated mechanism for shifting relatively movable hingedly connected
  wing sections.“ Auch dies bezieht sich auf (3).

  Weiter wird ausgeführt: „Other objects of the invention are to provide
  improved mounting and operating means for jack mechanisms controlling the
  movable wing sections, and for correlating parts of the electrical mechanism
  with the wing sections.“ Es geht also (4) um Herstellungs- und
  Wartungsbedingungen für derartige Flugzeuge. }

\subsection{Formulierung des Miniproblems}
Im Militärbereich ist es nötig, dass Flugzeuge möglichst schnell fliegen und
trotzdem in der Lage sind, ihre Flugrichtung schnell zu wechseln oder die
Geschwindigkeit plötzlich zu reduzieren. Mit starren, nicht verstellbaren
Flügeln ist dies nur sehr bedingt möglich.

\HGG{Dies bleibt mit Blick auf die obigen Anmerkungen weit hinter den
  Problemstellungen des Patents zurück. Nicht einmal ein Hauptwiderspruch wird
  formuliert, der in der Patentschrift klar ausgeführt ist: „At the present
  time, either speed or performance (maneuverability) must be sacrificed, one
  for the other. ...  These statements are general, however, as there are many
  other things that enter into actual performance calculations. The point is
  that designers have never been able to combine the best features of speed
  and maneuverability in the same airplane, although they have approached it
  to a reasonable extent in several types of modern combat ships.  However,
  there has always been a tendency, to sacrifice maneuverability for speed,
  and the purpose of this invention is to make possible the combination of
  speed, maneuverability, long range, and more horsepower per unit weight in
  the same airplane by controlling automatically the lift needed for each
  change in velocity during operation.“ }

\subsection{Beschreibung der Lösung des Problems in der Patentschrift}
Gelöst wurde das Problem erstmals durch eine Erfindung im 2. Weltkrieg für
militärische Zwecke. Das erste Flugzeug mit schwenkbaren Flügeln war die
Messerschmitt P1101 im Jahr 1944.  Es wurde allerdings dazu kein Patent
beantragt, da das Patentwesen in Deutschland Ende des Krieges brach lag,
beziehungsweise Erfindungen aus Geheimhaltungsgründen nicht veröffentlicht
wurden. Nach Kriegsende wurden die Schwenkflügel durch William J. Hampshire
weiterentwickelt. Mithilfe einer Lenksäulenvorrichtung wurde es ermöglicht,
die Lenkwelle drehbar zu machen. Somit können die Flügel des Flugzeuges
mehrere Positionen einnehmen.

\HGG{Diese Lösungsbeschreibung bleibt weit hinter der Komplexität des Patent
  Claims zurück.  Die Aussage zur Lenkwelle gehört nicht hierher.}

\subsection{Einordnung des Problems in die TRIZ-Systematik}
Die Lösung ist nur schwer in die Widerspruchsmatrix einzuordnen, da hier eine
hohe Geschwindigkeit und die Beweglichkeit eines Objektes im Widerspruch
zueinander stehen.

\HGG{Zwei weitere Prinzipien sind -- neben dem \emph{Prinzip 15 der
    Dynamisierung} („Falls das Objekt insgesamt unbeweglich ist, ist es
  beweglich zu gestalten“) -- offensichtlich: Das \emph{Prinzip 6 der
    Kopplung} („zur Koordinierung bestimmte Systeme sind zu koppeln“) sowie
  das \emph{Prinzip 7 der Universalität} („Das Objekt erfüllt mehrere
  unterschiedliche Funktionen, wodurch weitere gesonderte Objekte überflüssig
  werden“).  Die Prinzipien werden aber auf unterschiedlichen
  Abstraktionsebenen wirksam -- Prinzip 15 auf der Ebene des Teilsystems
  „Flügel“, die anderen beiden Prinzipien auf der Ebene des Systems
  „Flugzeug“.  }

\subsection{Bezug der speziellen Lösung zu den TRIZ-Strukturen}
Allgemein kann man sagen, dass das Prinzip der Dynamisierung hier Anwendung
findet. Die Kennwerte des Flugzeuges (Leistung im Bezug auf Energieeinsparung)
werden hier so optimiert, dass sie in jedem Zeitpunkt optimal sind.

\HGG{Auch das bleibt weit hinter den TRIZ-Konzepten zurück. Die Lösung des
  Widerspruchs zwischen „speed or maneuverability“ wird klar durch das
  Separationsprinzip \emph{Separation in der Zeit} gelöst (und funktioniert
  damit auch nur, wenn die verschiedenen Operativen Zonen zeitlich separiert
  werden können einschließlich der „Umrüstzeit“, die ein Verstellen der Flügel
  ggf. erfordert).  Die „Kennwerte des Flugzeuges“ orientieren sich hierbei
  allerdings in beiden Fällen \emph{nicht} an der „Leistung im Bezug auf
  Energieeinsparung“. }


\section{Beispiel: Verstellbare Lenksäule}
\subsection{Metadatyen}
\begin{itemize}\itemsep0pt
\item Titel: Verstellbare Lenksäule
\item Patentnummer: DE976562C
\item Quelle: \url{https://patents.google.com/patent/DE976562C}; \\
  \url{https://depatisnet.dpma.de}, \texttt{docid=DE000000976562B} 
\item Patentinhaber: BMW AG 
\item Erfinder: Fritz Fiedler 
\item Patentdaten: Veröffentlichung 1963-11-14
\end{itemize}

\subsection{Beschreibung des Stands der Technik}
Lenksäulen, die mit dem Lenkrad eines beliebigen Fahrzeuges verbunden sind,
sind in einer festen Position angebracht. Sie können verstellt werden, indem
die Postion des Sitzes verstellt wird.

\HGG{In der Patentschrift selbst wird der Stand der Technik deutlich
  fortgeschrittener beschrieben.  Es gibt bereits auf verschiedene Weise
  verstellbare Lenksäulen.}

\subsection{Funktionales Modell}

\HGG{Auch hier fehlt dieser Abschnitt komplett. Das Miniproblem kann aber
  nicht sinnvoll formuliert werden, wenn die Begriffe eines solchen Modells
  nicht vorab eingeführt werden. In der Patentschrift sind dazu Begriffe wie
  „Lenksäule“, „Lenkrad“, „Lenkwelle“, „Lenkgetriebe“, „Führungs- und
  Feststell"|einrichtungen“, „gefederte und ungefederte Massen“,
  „Abstandsänderungen“ usw. zu finden.}

\subsection{Formulierung des Miniproblems}
Das Problem besteht darin, die Lenksäule so zu gestalten, dass sie in jeder
Situation stufenlos verstellbar ist und trotzdem ohne Abzüge voll
funktionsfähig sein muss.

\HGG{Das wird in der Patentschrift deutlich anders ausgeführt, da auch ein
  anderer Stand der Technik vorausgesetzt wird.}

\subsection{Beschreibung der Lösung des Problems in der Patentschrift}
Das Patent beschreibt ein System, das sich unter allen Betriebszuständen auf
optimale Performance einstellt. Es ist eine Lenksäulenvorrichtung, die die
Lenkwelle drehbar und stufenlos verstellbar macht.  Die gesamte Konstruktion
besteht aus Lenkrad, Lenksäule und einer drehbaren Lagervorrichtung.  Somit
wird eine vertikale und axiale Verstellbarkeit des Lenkrades gewährleistet.
Das Objekt hat somit optimale Kennwerte in jeder Position seiner Anwendung.

\HGG{In der Patentschrift wird formuliert: „In Längsachse der Lenk"|säule
  insbesondere stufenlos verstellbares Lenkrad für Kraftfahrzeuge, wobei die
  Lenksäule und das Lenkgetriebe getrennt voneinander angeordnet sind und
  mittels einer ... Gelenkwelle verbunden sind ... und in an sich bekannter
  Weise die teleskopartig ausgebildete Gelenkwelle die auftretenden
  Abstandsänderungen in sich aufnimmt.“}

\subsection{Einordnung des Problems in die TRIZ-Systematik}
Lässt sich nur schwer in die Widerspruchsmatrix einordnen.

\HGG{Offensichtlich kommt mit der Gelenkwelle das \emph{Prinzip 24 des
    Vermittlers} („es ist ein Zwischenobjekt zu benutzen, das die Wirkung
  überträgt, weitergibt oder in sich aufnimmt“) zum Einsatz. }

\subsection{Bezug der speziellen Lösung zu den TRIZ-Strukturen}
Das Prinzip der Dynamisierung findet Anwendung. Aus einem unbeweglichem
Objekt, wird ein verstellbares Objekt gemacht. Eine Lenksäule, die vielen
Sicherheitsansprüchen genügen muss, wurde so verändert das die einzelne
untereinander verstellbar sind. Sodass sie in jeder Lage optimal sind.

\HGG{Auch hier ist die Frage, was sind System, Teilsystem und Obersystem.  Die
  Patentschrift konzentriert sich vollständig auf die Lenkvorrichtung, die
  Frage, \emph{warum} eine Verstellbarkeit überhaupt wünschenswert ist, wird
  nicht thematisiert, womit auch Fragen der „Sicherheit“ oder „Optimalität“,
  die ja nur auf der Ebene des Obersystems „Fahrzeug“ besprochen werden
  können, kein Thema sind.  Das \emph{Prinzip der Dynamisierung} in Form einer
  überhaupt verstellbaren Lenksäule spielt sich auf der Ebene jenes
  Obersystems „Fahrzeug“ ab, ist aber bereits Stand der Technik und nicht
  Gegenstand des betrachteten Patents.}

\section{Ergänzung: Kanban}
Im folgenden Abschnitt möchte ich noch vorstellen, dass die TRIZ-Prinzipien
auch auf nicht-technische Prozesse anwendbar sind. Ein Beispiel für die
Anwendung des Prinzips der Dynamisierung ist das sogenannte Kanban-System.  Es
gibt genau wie bei den bereits beschriebenen Patenten einen Widerspruch, der
mithilfe dieses Prinzips aufgelöst werden kann.

\subsection{Metadaten}
\begin{itemize}\itemsep0pt
\item Titel: Die Kanban Methode
\item Patentnummer: kein Patent vorhanden
\item Erfinder: Taiichi Ohno 1947, Toyota Motor Cooperation
\end{itemize}

\HGG{Die folgenden Ausführungen weichen von der methodisch gegebenen Vorgabe
  ab.  „Stand der Technik“ ist klar ausgeführt, aber das „Funktionale Modell“
  fehlt wieder.  Es geht hier offensichtlich um \emph{Produktionsplanungs- und
    Steuerungssysteme}\footnote{Siehe dazu etwa
    \url{https://de.wikipedia.org/wiki/PPS-System}.} (PPS), in deren Kontext
  Kanban nicht nur als grafisches Werkzeug zu betrachten ist, sondern als
  komplexe \emph{produktionsorganisatorische Innovation}.  Diese Innovation
  bezieht sich aber nicht nur auf die \emph{Planung} entsprechender Prozesse,
  sondern auch auf deren Implementierung und Steuerung. Ein entsprechendes
  Begriffsgebäude als Funktionales Modell wäre deshalb an dieser Stelle
  zunächst zu entwickeln gewesen.

  „Beschreibung des Systems“ und „Beschreibung der Lösung“ sind offensichtlich
  zusammenzulegen, wobei in ersterem Abschnitt nur ein sehr spezifischer Teil
  der Lösung vorgetragen wird (Signalisierung von Handlungsbedarfen durch
  Dritte).  Siehe dazu weitere Anmerkungen unten.}

\subsection{Beschreibung des Systems}
Kanban bedeutet Karte, Tafel, Beleg und wurde von dem Japaner Taiichi Ohno
entwickelt. Diesen frustrierte die sehr unflexible Ressourcenplanung in der
Autoindustrie. Er entwickelte ein System, in dem die Mitarbeiter bestimmter
Abschnitte im Unternehmen mithilfe von Karten anzeigen konnten, wann neue
Ressourcen benötigt wurden. Er dynamisierte das Produktionswesen.

\subsection{Stand der Technik}
In traditionellen, zentralen Planungssystemen der Produktionssteuerung wird
der gesamte Materialbedarf bis ins Detail weit im Voraus geplant. Bei
Schwankungen in der Nachfrage ist der Zufluss von neuem Material schwer zu
beeinflussen. Die Lagerbestände sind nicht zu jedem Zeitpunkt optimal.

\subsection{Formulierung des Miniproblems}
Es herrscht ein Konflikt zwischen der Produktionsgeschwindigkeit und der
Lieferbereitschaft von Waren.  Materialstoffe können nicht sofort nachbestellt
werden, wenn sie aufgrund von Schwankungen im Durchsatz auf einmal mehr
benötigt werden, da der Bedarf von zentraler Stelle geplant wurde. Einzelne
Produktionsstellen haben kaum die Möglichkeit, auf Engstellen zu reagieren.

\subsection{Beschreibung der Lösung}
Kanban ist eine Methode der Produktionsprozesssteuerung. Es wird sich hier nur
am tatsäch"|lichen Verbrauch von Materialien orientiert. Ziel der
Kanban-Methodik ist es, die Wert"|schöpfungs"|kette auf jeder Stufe der
Produktion einer mehrstufigen Kette kosten-optimal zu steuern. Durch
kurzlebige Pufferlager wurde ohne moderne Informationssysteme und mit kurzen
Wegen des Transports eine einfache Lösung erreicht.  Folglich können kürzere
Durchlaufzeiten durch schnelle Reaktionszeiten erlangt werden.

\subsection{Einordnung des Problems in die TRIZ-Systematik}
Trotzdem die Kanban Methode kein technisches Problem beschreibt, kann der
Widerspruch in die Widerspruchsmatrix eingeordnet werden.  Wenn die
Liefergeschwindigkeit zunimmt, und damit der sich verbessernde Parameter 9:
die Geschwindigkeit ist, und der sich verschlechternde Parameter 35: sie
Anpassungsfähigkeit, ist das Problem mit Hilfe der Dynamisierung lösbar.  Mit
Anpassungsfähigkeit ist hier die Lieferbereitschaft gemeint.

\HGG{Das fehlende Funktionale Modell erlaubt keine genaue Einordnung dieser
  Innovation, womit die Argumentation der Anwendbarkeit des Prinzips der
  Dynamisierung reine Spekulation bleibt und letztlich wertlos ist.

  Es geht offensichtlich um die wirtschaftliche Bemessung von „Pufferlagern“
  in PPS in verschiedenen Betriebsmodi (planwirtschaftlich, „just in time“)
  sowie eine dazu adäquate Betriebsführung.  Dazu wäre zunächst eine genauere
  Analyse erforderlich gewesen, was hier Teilsysteme, System und Obersystem
  sind.  Als System ist wohl das Bestellsystem des Unternehmens anzusehen, als
  Obersystem damit das System der Beziehungen des Unternehmens zu seinen
  Lieferanten.  Kanban selbst wird nur in Teilsystemen wirksam, nämlich in den
  Bestellvorgängen der einzelnen Einheiten des Unternehmens.  Die Lösung des
  Problems erfolgt also hier nicht durch eine Änderung am bestehenden System,
  sondern durch einen kompletten Systemwechsel.}

\HGG{ Es geht hier um die vordigitale Organisation von
  Informa"|tions"|verarbeitungsprozessen.  Eine Verbesserung wird vor allem
  durch eine Verbesserung des bisherigen SuField-Modells erreicht, indem eine
  weitere Ressource \emph{Information} adäquat modelliert wird. In moderneren
  TRIZ-Ansätzen werden solche Ressourcen als \emph{Felder} modelliert und dazu
  der klassische 5-Felder-Ansatz (mechanische, akustische, thermische,
  chemische, elektromagnetische Felder) aufgebohrt.  (Petrov 2015) schlägt
  vor, den SuField-Ansatz zu einem EI-Ansatz zu erweitern, in dem Elemente
  (E), Aktionen (A) und Wissen (K) interagieren.  }


\subsection{Nachwort}
Die vorgestellte Methode Kanban in der Produktionssteuerung ist heute auch
eine Methodik in der Softwareentwicklung.

\HGG{Das ist nicht verständlich.  Sicher wird heute das Wort „Kanban“ auch in
  der Softwareentwicklung verwendet, aber wirklich in der ursprüng"|lichen
  Bedeutung des Ansatzes? Wenn nicht, welcher „Feature Transfer“ hat
  stattgefunden und warum?}

\section{Fazit}
In bin in meiner Suche hauptsächlich vorgegangen, indem ich mir Erfindungen
herausgesucht habe, in denen das Prinzip der Dynamisierung eine Rolle gespielt
haben könnte. Ich habe die Beispiele für die Patente beziehungsweise für die
Erfindungen so ausgewählt, dass sie möglichst breitgefächerte Bereiche
abdecken. Neben technischen Problemen habe ich mich auch mit
gesellschaftlichen Problemen auseinandergesetzt.  Durch die Patentanalyse
konnte ich nachvollziehen, wie man vorgefertigte allgemeine
Problemlösungsstrategien auf spezielle Probleme anwenden kann.  Auch erkannt
habe ich, dass man auch in alltäglichen Problemlagen, die TRIZ-Prinzipien
anwenden kann und dass sie sich nicht nur auf konkrete technische Probleme
anwenden lassen.

\HGG{In der Analyse wird deutlich, dass das innovative Potenzial in einzelnen
  Patentschriften eine genaue Analyse des jeweiligen Stands der Technik
  voraussetzt, um die genaue Ausrichtung des entwickelten innovativen
  Potenzials richtig einschätzen zu können.  Nur auf einer solchen Basis
  lassen sich aber allgemeine innovative Prinzipien und Methodiken aus
  konkreten Patenten extrahieren (Altschuller) oder wiederfinden (Ziel der
  Seminararbeiten).  Im Lichte einer solchen Erfahrung erscheint eine
  werkzeuggestützte Auswertung von Patentschriften auf der Basis \emph{allein}
  textanalytischer Methodiken (wie in der Seminararbeit von René Brückner
  genauer untersucht) nur von begrenzter Aussagekraft. }

\section{Quellen}

\begin{itemize}
\item {Widerspruchsmatrix}
  \url{http://www.triz40.com/aff_Tabelle_TRIZ.php}
\item {Patent zur automatischen Zylinderabschaltung}\\
  \url{https://patents.google.com/patent/DE19606402C2}
\item {Patent zum elektrischen gesteuerten Abgasturbolader}\\
	\url{https://patents.google.com/patent/EP2158386B1}
\item {Patent zum stufenlos verstellbares Lenkrad fuer Kraftfahrzeuge}\\
	\url{https://patents.google.com/patent/DE976562C}
\item \url{https://www.michael-patra.de/triz/loesungsverfahren}
\item {Kanban} \url{https://de.wikipedia.org/wiki/Kanban}
\item Vladimir Petrov (2015).  A New Approach to Su-Field: Structural
  Analysis.  TRIZ Journal, 27.07.2015. 
\end{itemize}

\end{document}
