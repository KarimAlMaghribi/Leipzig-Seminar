\documentclass[11pt,a4paper]{article}
\usepackage{a4wide,url}
\usepackage[utf8]{inputenc}
\usepackage[german]{babel}

\parindent0pt
\parskip4pt
\title{Handreichung zum Einsatz des TRIZ-Trainers}

\author{Hans-Gert Gr\"abe}

\date{23. Januar 2020}

\begin{document}
\maketitle

\section{Allgemeines}

Im Rahmen einer internationalen Kooperation nutzen wir in diesem Semester im
Praktikum unseres Kurses „Semantic Web“ wie angekündigt probeweise den von
\emph{Target Invention} in Minsk (Belarus) entwickelten TRIZ-Trainer
\url{https://triztrainer.ru}.  Der TRIZ-Trainer ist eine leichtgewichtige
Version zur Unterstützung der Online-Phase von Blended
Learning\footnote{\url{https://de.wikipedia.org/wiki/Integriertes_Lernen}} als
methodischem Praktikumskonzept.  Ab 7. Januar 2020 werden wir Probleme sowohl
technischer als auch inhaltlicher Art sowie die Lernfortschritte in
wöchentlichen Besprechungen dienstags ab 17 Uhr genauer analysieren.

Der TRIZ-Trainer konzentriert sich auf die Basiskonzepte des Einsatzes von
TRIZ an ausgewählten praktischen Beispielen -- die Analyse der jeweiligen
Problemsituation, die Identifizierung entsprechender Wirkfaktoren und
Widersprüche sowie die strukturierte Verwendung entsprechender
Lösungsschemata.  Weitergehende TRIZ-Werkzeuge (strukturierte Analysen von
Stoff-Feld-Interaktionen, Funktionsanalyse, Prozessanalyse,
Root-Conflict-Analysis usw.), die in (Koltze/Souchkov 2017)\footnote{Karl
  Koltze, Valeri Souchkov (2017). Systematische Innovation.  2. Auf"|lage,
  Hanser, München.} ebenfalls besprochen werden, können eingesetzt werden,
sind aber nicht Teil des im TRIZ-Trainer eingebauten strukturierten Vorgehens,
das Ihnen auf der Hauptseite angezeigt wird und in der Fachwelt auch als
„Weihnachtsbaum“ (christmas tree) bekannt ist.

Der TRIZ-Trainer ist selbst noch in Entwicklung, insbesondere der Ausbau
verschiedener Sprachversionen.  Im Rahmen unserer Kooperation habe ich die
Minsker Kollegen bei der Erstellung einer deutschsprachigen Version
unterstützt.  Diese Arbeit ist aktuell zu 80\% umgesetzt, ein Teil der
deutschen Texte sind noch die Rohversionen, die Google Translate geliefert
hat.  Einige Teile des TRIZ-Trainers (insbesondere im Bereich „Ergänzendes“)
sind noch nicht in einer deutschen Version verfügbar. Dies sowie die weitere
Konsolidierung der Übersetzungen erfolgt im Zuge des weiteren Einsatzes des
TRIZ-Trainers über das dort eingebaute Redaktionssystems. Fortschritte in
diesem Bereich werden also unmittelbar wirksam.
\newpage

\section{Registrierung und Aktivierung des Accounts}

Für die sechs Studierenden des Kurses wurden Accounts angelegt und der Rolle
„Student“ zugewiesen.  Die Probleme mit der Aktivierung der Accounts sind
inzwischen gelöst. Der Account ist derzeit bis 23. Februar 2020
freigeschaltet.  Es besteht die Möglichkeit, die Arbeit mit einem Zertifikat
abzuschließen und dazu den Account auch noch zu verlängern. Dies muss aber im
Einzelfall mit Prof. Gräbe abgesprochen werden. 

Die weiteren Ausführungen gehen davon aus, dass Sie sich am System
authentifiziert haben (Menü ganz rechts) und in der Rolle \emph{Student}
agieren.  Die beiden Felder daneben (mit den Tooltips „Notifications“ und
„Settings“) dienen der Steuerung Ihrer Aktivitäten. Über das Feld
„Notifications“ haben Sie Zugriff auf Ihre bisherigen Lösungsversuche.

Die deutsche Version sollte sich automatisch an Hand der Spracheinstellung
Ihres Browsers aktivieren, gegebenenfalls kann dies auch im Auswahlfeld im
Seitenfuß umgeschaltet werden (dort sind die drei Sprachversionen Russisch,
Ukrainisch und Deutsch verfügbar).  Bitte beachten Sie, dass einzelne Teile
noch nicht ins Deutsche übersetzt sind und dann in der Fallback-Sprache
Russisch angezeigt werden. Wenn Sie dies am weiteren Arbeiten hindert, dann
teilen Sie mir dies bitte mit, damit ich mich um die vorrangige Übersetzung
jener Teile ins Deutsche kümmere.

Ich bin Ihnen als Trainer zugewiesen und kann somit Ihre Aktivitäten
verfolgen, kommentieren und bewerten.

\section{Was ist zu tun?}

Nach dem Einloggen gehen Sie einfach auf die Seite \emph{Aufgaben} und
beginnen, die Aufgaben zu lösen, die Sie mögen.  Es empfiehlt sich natürlich,
vorher die Hinweise unter \emph{Lösungsprozess} genauer zu studieren.  Dort
sind zu jedem Schritt im Lösungsprozess ausführliche Hinweise gegeben, was im
jeweiligen Schritt zu tun ist, und wie an die Teilaufgabe herangegangen werden
sollte. 

Es werden Ihnen insgesamt 8 Aufgabenserien angeboten, was aber eine eher
technische Einteilung ist.  Das Lösen der Aufgaben setzt eine gewisse
Vertrautheit mit der TRIZ-Methodik voraus, die Sie aber in den
Hilfeanleitungen oder -- in kurzer Form -- in den Tooltipps erwerben können.
Beide sind ins Deutsche übertragen, inzwischen auch die Mehrzahl der Grafiken
im Hilfesystem. Probleme können im Chat vorgetragen oder in den Konsultationen
besprochen werden.  Schauen Sie sich auch die \emph{Beispielaufgaben} an.

Die Aufgaben sind so weit heruntergebrochen, dass sie (meist) nur \emph{eine}
widersprüchliche Grundsituation enthalten bzw. nur auf eine solche fokussiert
wird.  In der \emph{ersten Phase der Lösung} ist dieser Konflikt (zwischen
nützlichen und schädlichen Wirkungen) in einer (genaueren) Modellierung zu
lokalisieren, danach die \emph{Operative Zone} raum-zeitlich zu bestimmen und
eine Hypothese (als „Mini-Aufgabe“) aufzustellen, \emph{wie} das Problem
gelöst werden könnte. Es geht an der Stelle zunächst darum, das zu
betrachtende \emph{System} zu fixieren, dessen Komponenten und deren
Funktionalitäten zu bestimmen und den Ort des Konflikts genauer zu
lokalisieren.  Am Ende dieser Analyse (in der Informatik auch als
\emph{Anforderungsanalyse} bezeichnet) steht ein genaues Modell des Systems.
Weiter ist eine \emph{Aufgabe} formuliert, deren Umsetzung das Problem löst.
Es kann sein, dass sich während dieser Systemmodellierung herausstellt, dass
ein anderer Detailgrad als System angemessener ist (siehe „Ergänzendes $\to$
Hierarchisches Prozessmodell“).  Dann muss die Modellierung auf jenem Level
wiederholt werden. Mehr dazu finden Sie im Abschnitt „Algorithmus zur
Korrektur von Problemsituationen (AIPS-2015)“ im Tab „Ergänzendes“.  Hilfreich
ist es hierbei auch, die Ausführungen in (Koltze/Souchkov 2017, Kapitel 4.3)
zum Zusammenhang zwischen technischen (TW) und physikalischen (PW)
Widersprüchen zu beachten und zu einem (gelegentlich offensichtlichen) PW die
TW zu rekonstruieren, um zu verstehen, wie der PW im Gesamtsystem der
Modellierung einzubetten ist.

Zur spezifizierten Aufgabe werden in der \emph{zweiten Phase der Lösung} durch
genaue Analyse der verfügbaren Ressourcen eine oder mehrere
\emph{Lösungsideen} gefunden, von denen in der \emph{dritten Phase} eine zur
\emph{finalen Lösung} genauer auszuarbeiten ist.  Im Gegensatz zur
Anforderungsanalyse stehen dabei andere TRIZ-Werkzeuge im Vordergrund.

Wenn die Lösungsvorlage vollständig ausgefüllt ist, können Sie die Lösung zur
Überprüfung absenden. In diesem Fall erhält der Trainer einen Hinweis auf
seiner Übersichtsseite und wird die Lösung überprüfen.  Weiter kann der
Trainer Kommentare abgeben. Basierend auf den Ergebnissen der Durchsicht wird
der Trainer die Aufgabe bewerten und abschließen oder zur Überarbeitung
zurückgeben.

Es gibt also kein „richtig“ oder „falsch“, sondern die Qualität der Lösung
wird begutachtet. Mehr dazu auch weiter unten in den „Anmerkungen von
Anton“\footnote{Anton Ivanov ist mein administrativer Ansprechpartner auf
  Minsker Seite.}.

Primäres Ziel des Einsatzes des TRIZ-Trainers ist die Erprobung dieses
Instruments im praktischen Einsatz eines „flipped Classroom“ im Kontext
unserer Ausbildung.  Für das \textbf{erfolgreiche Absolvieren des Kurses} sind
10 Aufgaben so weit zu bearbeiten, dass die Lösungen vom Trainer akzeptiert
werden. 

\section{Der Workflow im TRIZ-Trainer}

Alles beginnt damit, dass der Bearbeiter mindestens ein Zeichen in die Vorlage
einer Aufgabe einfügt, um das Problem zu lösen. Beim automatischen Speichern
wechselt die Aufgabe dann in den Status \emph{wird gelöst} und wird auf der
Seite \emph{Ergebnisse} dem zugeordneten Trainer angezeigt. Der Trainer kann
so bereits sehen, welche Aufgaben die ihm zugeordneten Bearbeiter zu lösen
begonnen (aber noch nicht beendet) haben und kann den Fortschritt ihrer Arbeit
verfolgen. Der Trainer kann in diesem Stadium bereits Kommentare zu den
einzelnen Schritten hinterlassen.

Wenn der Bearbeiter das Problem gelöst hat, kann er auf die Schaltfläche
\emph{Zur Überprüfung einreichen} klicken und damit die Aufgabe zur
Überprüfung einreichen.  Der Trainer analysiert die Lösung des Bearbeiters und
kann Kommentare zu den Schritten hinterlassen. Dabei ist die Entscheidung zu
treffen, die Lösung anzuerkennen oder zur Überarbeitung zurückzugeben. Die
Möglichkeit einer \emph{Bewertung} ist aktuell nicht aktiviert, sondern nur
als potenzielle Variante für eine spätere Implementierung angelegt. Dem
Bearbeiter wird der Status der Lösung und die Kommentare des Trainers (falls
vorhanden) über seine Benachrichtigungen und auch per E-Mail mitgeteilt.  Wenn
die Aufgabe zur Überarbeitung zurückverwiesen wird, ist sie vom Bearbeiter
erneut zu bearbeiten. Wenn die Lösung akzeptiert wurde, ist die Bearbeitung
der Aufgabe beendet.

\section{Weitere Hinweise}

Im TRIZ-Trainer wird in der Analysephase das Konzept der \emph{Maschine} als
grundlegende Struktur der Modellierung eingesetzt, womit die Wirkung eines
\emph{Werkzeugs} (Instrument, Arbeitsorgan) auf ein \emph{zu bearbeitendes
  Objekt} erfasst wird, um ein \emph{nützliches Produkt} herzustellen.  Dieses
\emph{Arbeitsorgan} ist Teil eines Produktionsprozesses, der durch einen
\emph{Antrieb} angetrieben wird, dessen Leistung über eine \emph{Transmission}
auf das Arbeitsorgan übertragen wird.  Das Ganze wird von einer
\emph{Steuereinheit} gesteuert und der Antrieb von einer \emph{Energiequelle}
gespeist.  Dieses stark an ingenieur-technischen Termini orientierte Konzept
wird auch auf allgemeinere Situationen übertragen und ist dann sinngemäß
anzuwenden.  Siehe auch „Lösungsprozess $\to$ Aufbau und Bedienung der
Maschine“.

Es empfiehlt sich, das zu lösende Problem erst einmal mit Bleistift und Papier
genauer zu analysieren. Dabei sollte man für sich genau klären
\begin{itemize}
\item[1.] Was ist das „System“ (also jener Teil, der für die Hauptfunktion in
  der Problemstellung in der Folge Energiequelle $\to$ Antrieb $\to$
  Transmission $\to$ Arbeitsorgan $\to$ Objekt $\to$ nützliches Produkt +
  Steuerung relevant ist)? Also etwa im Beispiel Schiffsmast „das Boot“.

\item[2.] Was ist das nützliche Produkt? Ein Produkt ist etwas (zu einem
  späteren Zeitpunkt) bereits Fertiges, etwa „Das Boot ist auf der anderen
  Seite der Brücke“.

\item[3.] Was ist das Nützlichkeitsprinzip? Wie entsteht das nützliche
  Produkt? Etwa, „das Boot fährt unter der Brücke hindurch“.

\item[4.] Genaue Analyse der Maschine (mit den oben angegebenen Teilen), dazu
  den Begriff „Maschine“ im Sinne des TRIZ-Trainers verstehen, siehe
  „Ergänzendes $\to$ nützliches technisches System“ und wie die Maschine
  arbeitet (siehe Funktionales Modell und Prozessmodell).
\end{itemize}
Das System hat einen äußeren Zweck (nach dem Prinzip hergestelltes nützliches
Produkt) -- die Spezifikation, mit der Maschine wird beschrieben, wie das
innen implementiert ist. Ein System und damit die Maschine besteht aus
Komponenten, die ggf. weiter analysiert werden müssen. Dazu wird dasselbe
Analyseschema rekursiv auf die interessierende Komponenten angewendet, siehe
„Ergänzendes $\to$ Hierarchisches Prozessmodell“.

In vielen Fällen wurde diese Analyse sehr ungenau vorgenommen, aber eine
plausible Lösung als Endergebnis vorgeschlagen, die dann aber meist vom Himmel
fällt (in dem Sinne, dass von der ersten Zeile eigentlich klar ist, wo die
Reise hingehen soll). In einem solchen Fall lohnt es, das Ganze von hinten her
aufzurollen und zu überlegen, wie man mit TRIZ auf diese Lösung gekommen
wäre. Insbesondere steht die Frage, ob eines der 40 TRIZ-Prinzipien (oder
mehrere anwendbar wäre und ggf. welche(s). Durch den Übergang auf eine solche
abstraktere Ebene wird oft auch klarer, wie man die Modellierung aufziehen
sollte.

\section{Anmerkungen von Anton}

TRIZ-Beratungsunternehmen konzentrieren sich nicht darauf, eine einzige starke
Lösung zu finden, die unter bestimmten Bedingungen wahrscheinlich schwierig zu
implementieren ist, sondern bieten eine Reihe von Lösungen. Auf diese Weise
kann der Kunde anhand seiner Ressourcen und Einschränkungen die am besten
geeignete, lokal ideale Lösung auswählen.
 
Dieser Ansatz wird auch im TRIZ-Trainer verfolgt, auch wenn wir hier keinen
\emph{Kunden} als Quelle der Probleme haben. Dementsprechend können wir auch
nur bedingt aus mehreren die beste Lösung auswählen -- diejenige, die uns
aufgrund allgemeiner Erfahrung und Logik als die beste erscheint. Wenn der
Student zu mehreren Lösungsvorschlägen gelangt, kann er bei den letzten
Schritten der Vorlage (Schlussfolgerung, endgültige Entscheidung) die seiner
Meinung nach am besten funktionierenden Lösungen hervorheben und eine (mit
Begründung) als die effektivste herausstellen. Diese Entscheidung wird
(wahlweise) meistens kommentarlos gegeben, da der Kurs keine detaillierten
Informationen sowie Anweisungen zur Bewertung und Auswahl von Lösungen
enthält. Der Trainer kann in solchen Fällen seine Meinung zu dieser Wahl als
Kommentar äußern, wenn die aus seiner Sicht idealste Lösung abgewählt wurde.

Ich möchte eine Bemerkung zum Schritt \emph{Nützliches Produkt} machen. Dies
sollte keine \emph{Aktion} sein, sondern ein \emph{Objekt} in einem bestimmten
Zustand: ein empfangenes Funksignal; ein Schiff unter einer Brücke
vorbeifahrend usw.
 
Zur Kontextualisierung der Lösung: Die Wahl der anfänglichen Systemebene ist
bei diesem Ansatz zweitrangig, weil wir in der weiteren Bearbeitung auf dem
Weg zum Kern -- dem Konflikt und seinen Ursachen -- die
Systembetrachtungsebene sowieso anpassen. Ein nicht sehr klar definierter
Orientierungspunkt für die Kontextualisierung der Aufgabe ergibt sich (a
posteriori) aus der Lösung der Aufgabe: Für die Lösung sollten nur minimale
Änderungen am vorhandenen System erforderlich sein -- eine Änderung am Ort des
Konflikts oder in der Nähe.

Niemand ist jedoch gehindert, die Suche nach einer Lösung auf anderen
Systemebenen zu versuchen. Wenn wir durch andere, weiter entfernte
Systemebenen gehen, finden wir andere Lösungen (für die Aufgabe „Schiffsmast“
zum Beispiel globale Modifikationen der Brücke, des Flussbetts, Installation
von Mitteln zum Umtragen von Booten auf die andere Seite usw.).

Generell ist es ratsam, im Kurs eine Einstellung herauszuarbeiten, dass die
Lösung auf ein akzeptables Niveau zu bringen ist. Das heißt, auch sekundäre
Probleme sind zu lösen.  Wenn der Student eine Lösung gefunden hat, die im
Prinzip funktioniert, die aber, wenn man versuchen würde, sie zu
implementieren, auf weitere Hindernisse stößt oder teuer oder kompliziert oder
zu 99\% unrealistisch bgzl. der Ressourcenanforderungen ist, kann man damit
nicht zufrieden sein.

In diesem Fall sollte die Lösung zur Überarbeitung zurückgegeben werden mit
folgenden zwei Optionen:
\begin{itemize}
\item [1)] Erneutes Durchlaufen des Lösungszyklus mit den erweiterten
  Kenntnissen und derselben Hypothese (Iteration).  Modifizieren Sie das
  Modell oder erstellen ein neues, führen daran noch einmal die
  Lösungsschritte aus und reichen die neue Lösung zur Bewertung ein. Es ist
  möglich, weitere Iterationen durchzuführen, es gibt im Prinzip keine
  Einschränkungen.
\enlargethispage{-1em}
\item [2)] Gehen Sie anders an die Lösung heran, formulieren Sie eine neue
  Hypothese (oder wählen eine andere aus den vorher schon aufgestellten
  mehreren Hypothesen aus), modellieren Sie das Problem neu und führen es als
  neue Lösung aus.
\end{itemize}
\end{document}
