\documentclass[11pt,a4paper]{article}
\usepackage{a4wide,url}
\usepackage[utf8]{inputenc}
\usepackage[german]{babel}

\parindent0pt
\parskip4pt
\title{Handreichung zum Einsatz des TRIZ-Trainers}

\author{Hans-Gert Gr\"abe}

\date{25. Dezember 2019}

\begin{document}
\maketitle

\section{Allgemeines}

Im Rahmen einer internationalen Kooperation nutzen wir in diesem Semester im
Praktikum unseres Kurses „Semantic Web“ wie angekündigt probeweise den von
\emph{Target Invention} in Minsk (Belarus) entwickelten TRIZ-Trainer
\url{https://triztrainer.ru}.  Der TRIZ-Trainer ist eine leichtgewichtige
Version zur Unterstützung der Online-Phase von Blended
Learning\footnote{\url{https://de.wikipedia.org/wiki/Integriertes_Lernen}} als
methodischem Praktikumskonzept.  Ab 7. Januar 2020 werden wir Probleme sowohl
technischer als auch inhaltlicher Art sowie die Lernfortschritte in
wöchentlichen Besprechungen dienstags ab 17 Uhr genauer analysieren.

Der TRIZ-Trainer konzentriert sich auf die Basiskonzepte des Einsatzes von
TRIZ an ausgewählten praktischen Beispielen -- die Analyse der jeweiligen
Problemsituation, die Identifizierung entsprechender Wirkfaktoren und
Widersprüche sowie die strukturierte Verwendung entsprechender
Lösungsschemata.  Weitergehende TRIZ-Werkzeuge (strukturierte Analysen von
Stoff-Feld-Interaktionen, Funktionsanalyse, Prozessanalyse,
Root-Conflict-Analysis usw.), die in (Koltze/Souchkov 2017)\footnote{Karl
  Koltze, Valeri Souchkov (2017). Systematische Innovation.  2. Auf"|lage,
  Hanser, München.} ebenfalls besprochen werden, können eingesetzt werden,
sind aber nicht Teil des im TRIZ-Trainer eingebauten strukturierten Vorgehens,
das Ihnen auf der Hauptseite angezeigt wird und in der Fachwelt auch als
„Weihnachtsbaum“ (christmas tree) bekannt ist.

Der TRIZ-Trainer ist selbst noch in Entwicklung, insbesondere der Ausbau
verschiedener Sprachversionen.  Im Rahmen unserer Kooperation habe ich die
Minsker Kollegen bei der Erstellung einer deutschsprachigen Version
unterstützt.  Dazu wurden die von Google Translate gelieferten Ergebnisse
editorisch überarbeitet, um einen einheitlichen Gebrauch der Terminologie zu
gewährleisten. Diese Arbeit ist aktuell zu 50\% umgesetzt, die restlichen
deutschen Texte sind die Rohversionen von Google Translate.  Einige Teile des
TRIZ-Trainers (insbesondere im Bereich „Ergänzendes“) sind noch nicht in einer
deutschen Version verfügbar. Dies sowie die weitere Konsolidierung der
Übersetzungen erfolgt im Zuge des weiteren Einsatzes des TRIZ-Trainers über
das dort eingebaute Redaktionssystems. Fortschritte in diesem Bereich werden
also unmittelbar wirksam.

\section{Registrierung und Aktivierung des Accounts}

Für die sechs Studierenden des Kurses wurden Accounts angelegt und der Rolle
„Student“ zugewiesen. Username ist die im Moodle hinterlegte Email-Adresse.
An diese Adresse sollten Sie einen Aktivierungslink geschickt bekommen haben,
mit dem Sie ein eigenes Passwort einrichten und damit Ihre Authentifizierung
am System (ganz rechts im Login-Feld) ermöglichen.  Der Account ist für zwei
Monate, also bis etwa Ende Februar freigeschaltet. 

Die weiteren Ausführungen gehen davon aus, dass Ihnen dies gelungen ist.  Die
beiden Felder daneben (mit den Tooltips „Notifications“ und „Settings“) dienen
der Steuerung Ihrer Aktivitäten. Über das Feld „Notifications“ haben Sie
Zugriff auf Ihre bisherigen Lösungsversuche.

Die deutsche Version sollte sich automatisch an Hand der Spracheinstellung
Ihres Browsers aktivieren, gegebenenfalls kann dies auch im Auswahlfeld im
Seitenfuß umgeschaltet werden (dort sind die drei Sprachversionen Russisch,
Ukrainisch und Deutsch verfügbar).  Bitte beachten Sie, dass einzelne Teile
noch nicht ins Deutsche übersetzt sind und dann in der Fallback-Sprache
Russisch angezeigt werden. Wenn Sie dies am weiteren Arbeiten hindert, dann
teilen Sie mir dies bitte mit, damit ich mich um die vorrangige Übersetzung
jener Teile ins Deutsche kümmere.

Ich bin Ihnen als Trainer zugewiesen und kann auch die 6 Accounts sehen und
somit Ihre Aktivitäten verfolgen. 

\section{Was ist zu tun?}

Nach dem Einloggen gehen Sie einfach auf die Seite \emph{Aufgaben} und
beginnen die zu lösen, die Sie mögen.  Es werden Ihnen insgesamt 8
Aufgabenserien angeboten, was aber eine eher technische Einteilung ist.  Das
Lösen der Aufgaben setzt eine gewisse Vertrautheit mit der TRIZ-Methodik
voraus, die Sie aber in den Hilfeanleitungen oder -- in kurzer Form -- in den
Tooltipps erwerben können. Beide sind ins Deutsche übertragen, allein die
Grafiken im Hilfesystem sind noch weitgehend in Russisch. Dies kann in den
Konsultationen besprochen werden.  Schauen Sie sich dazu auch die
\emph{Beispielaufgaben} sowie die Anmerkungen zum \emph{Lösungsprozess} an.

Die Aufgaben und das Lösungsschema orientieren sich am „Weihnachtsbaum“ auf
der Einstiegsseite.  Die Aufgaben sind so weit heruntergebrochen, dass sie nur
\emph{eine} widersprüchliche Grundsituation enthalten bzw. nur auf eine solche
fokussiert wird.  Dieser Konflikt (zwischen nützlichen und schädlichen
Wirkungen) ist in einer Modellierung genauer zu lokalisieren, die
\emph{Operative Zone} raum-zeitlich zu bestimmen und entsprechende Hypothesen
aufzustellen, um auf dieser Basis zu entscheiden, welches der vier
Problemmodelle sinnvoll angewendet werden kann, um die Ausgangssituation in
eine Lösung zu transformieren.  Mehr dazu finden Sie im Abschnitt „Algorithmus
zur Korrektur von Problemsituationen (AIPS-2015)“ im Tab „Ergänzendes“, der
inzwischen ins Deutsche übersetzt ist.

Im TRIZ-Trainer wird das Konzept der \emph{Maschine} als grundlegende Struktur
der Modellierung eingesetzt, womit die Wirkung eines \emph{Werkzeugs}
(Instrument, Arbeitsorgan) auf ein \emph{zu bearbeitendes Objekt} erfasst
wird, um ein \emph{nützliches Produkt} herzustellen.  Dieses
\emph{Arbeitsorgan} ist Teil eines Produktionsprozesses, der durch eine
\emph{Motor} angetrieben wird, dessen Leistung über eine \emph{Transmission}
auf das Arbeitsorgan übertragen wird.  Das Ganze wird von einer
\emph{Steuereinheit} gesteuert und der Motor von einer \emph{Energiequelle}
gespeist.  Dieses stark an ingenieur-technischen Termini orientierte Konzept
wird auch auf allgemeinere Situationen übertragen und ist dann sinngemäß
anzuwenden.

Wenn die Lösungsvorlage vollständig ausgefüllt ist, können Sie die Lösung zur
Überprüfung absenden. In diesem Fall erhält der Trainer einen Hinweis auf
seiner Übersichtsseite und kann die Lösung überprüfen.  Weiter kann der
Trainer Kommentare abgeben. Basierend auf den Ergebnissen der Durchsicht kann
der Trainer die Aufgabe bewerten und abschließen oder zur Überarbeitung
zurückgeben.

Es gibt also kein „richtig“ oder „falsch“, sondern die Qualität der Lösung
wird begutachtet. 

\section{Der Workflow im TRIZ-Trainer}

Alles beginnt damit, dass der Bearbeiter mindestens ein Zeichen in die Vorlage
einer Aufgabe einfügt, um das Problem zu lösen. Beim automatischen Speichern
wechselt die Aufgabe dann in den Status \emph{wird gelöst} und wird auf der
Seite \emph{Ergebnisse} dem zugeordneten Trainer angezeigt. Der Trainer kann
so bereits sehen, welche Aufgaben die ihm zugeordneten Bearbeiter zu lösen
begonnen (aber noch nicht beendet) haben und kann den Fortschritt ihrer Arbeit
verfolgen. Der Trainer kann in diesem Stadium bereits Kommentare zu den
einzelnen Schritten oder im Chat zur Aufgabe hinterlassen.

Wenn der Bearbeiter das Problem gelöst hat (formal: alle Felder der Vorlage
sind ausgefüllt), kann er auf die Schaltfläche \emph{Zur Überprüfung
  einreichen} klicken und damit die Aufgabe \emph{zur Überprüfung} einreichen.
Der Trainer analysiert die Lösung des Bearbeiters und kann Kommentare zu den
Schritten und im Chat hinterlassen. Dabei muss er die Entscheidung treffen,
die Lösung anzuerkennen oder zur Überarbeitung zurückzureichen. Dies erfolgt
über die Schaltfläche \emph{Bearbeiten}. Im geöffneten Formular kann der
Trainer den Status ändern und einen Kommentar hinterlassen (Review).  Die
Möglichkeit einer \emph{Bewertung} ist aktuell nicht aktiviert, sondern nur
als potenzielle Variante für eine spätere Implementierung angelegt. Dem
Bearbeiter wird der Status der Lösung und die Kommentare des Trainers (falls
vorhanden) über seine Benachrichtigungen und auch per E-Mail mitgeteilt.  Wenn
die Aufgabe zur Korrektur von Fehlern zurückverwiesen wird, kann sie der
Bearbeiter erneut bearbeiten. Wenn die Lösung akzeptiert wurde, ist die
Bearbeitung der Aufgabe beendet.

Damit kann der Trainer die Aufgabe in allen Stadien der Bearbeitung sehen,
beginnend mit der Zuweisung des Status \emph{wird gelöst} über die Zustände
\emph{Warten auf Bewertung}, \emph{Lösung zur Korrektur zurückgegeben} bis hin
zum finalen Zustand \emph{Lösung akzeptiert}.

\section{Was ist zum Bestehen erforderlich?}

Anton hat mir dazu folgendes geschrieben:
\begin{quote}
  Die Studenten entscheiden, welche Aufgaben sie lösen möchten. Von den
  verfüg\-baren 48 Aufgaben müssen unseres Erachtens mindestens 50\% gelöst
  werden, um das Zertifikat zu erhalten.
  
  Alle Aufgabenlösungen und alle Nachrichten in den Aufgabenchats werden zu
  Ihnen als Trainer weitergeleitet -- Sie erhalten entsprechende
  Benachrichtigungen.
\end{quote}

Wie wir genau damit umgehen, besprechen wir in den Konsultationen. Das hängt
auch davon ab, welche Hürden sich bei der Verwendung des TRIZ-Trainers noch
auftun werden. 

\end{document}
