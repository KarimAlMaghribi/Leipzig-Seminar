\documentclass[11pt,a4paper]{article}
\usepackage{a4wide,url,amsmath}
\usepackage[utf8]{inputenc}
\usepackage[german]{babel}

\parindent0pt
\parskip4pt
\title{Reader zum Interdisziplinären Gespräch}

\author{Hans-Gert Gr\"abe, Ken Pierre Kleemann, Lydie Laforet, Sabine
  Lautenschläger}

\date{1. Januar 2020}

\begin{document}
\maketitle
\tableofcontents

\section{Ziel und Methodik des Seminars}

Der Systembegriff spielt in der Informatik eine herausragende Rolle, wenn es
um Datenbanksysteme, Softwaresysteme\footnote{So die neue Denomination der im
  Besetzungsverfahren befindlichen W3-Professur am Institut für Informatik.},
Hardwaresysteme, Abrechnungssysteme, Zugangssysteme usw. geht.  Überhaupt wird
die Informatik von einer Merhheit als die „Wissenschaft von der
\emph{systematischen} Darstellung, Speicherung, Verarbeitung und Übertragung
von Informationen, besonders der automatischen Verarbeitung mithilfe von
Digitalrechnern“ (Wikipedia) verstanden.  Auch gewisse einschlägige
Professionen wie etwa der \emph{Systemarchitekt} genießen unter IT-Anwendern
hohe Wertschätzung.

Die Bedeutung des Systembegriffs reicht allerdings weit über den Bereich der
Informatik hinaus -- er ist grundlegend für alle Ingenieurwissenschaften und
als \emph{Systems Engineering} mit der ISO/IEC/IEEE-15288 Norm „Systems and
Software Engineering“ auch Gegenstand internationaler Normierungs- und
Standardisierungsprozesse.  Mehr noch spielt der Systembegriff auch bei der
Beschreibung komplexer natürlicher und kultureller Prozesse -- etwa im Begriff
des \emph{Ökosystems} -- eine zentrale Rolle.

Mit dem \emph{Semantic Web} rückt die Bedeutungsanalyse digitaler Artefakte in
den Mittelpunkt, die in letzter Instanz Sprachartefakte sind und damit
ebenfalls in direktem Zusammenhang zu einem sinnvoll zu entfaltenden
\emph{Systembegriff} stehen als Grundlage jeden Verständnisses konkreter
Systeme.

Mit dem Schlagwort \emph{Nachhaltigkeit} werden schließlich komplexe
gesellschaftliche Abstimmungsprozesse angesprochen, mit denen vielfältige
Informations- und Bewertungsprobleme einhergehen. Hierbei ist die Fähigkeit
der beschreibenden Abgrenzung, Entwicklung und Steuerung von sogenannten
Systemen auf bzw. über verschiedene Governance-, Raum- und Zeitebenen hinweg
von großer Bedeutung.

\begin{quote}
  \textbf{Ziel des Seminars} ist es, ein besseres Verständnis für diese
  Vielfalt von Systembegriffen zu gewinnen und dabei die Zugänge
  \emph{verschiedener Systemtheorien} als Gegenstand einer
  \emph{Systemwissenschaft} zu analysieren.
\end{quote}
Das Seminar ist ein Einführungskurs in die Systemwissenschaft auf
Master-Ebene, ihre Entwicklung im Laufe der Zeit, Verzweigung von Ansätzen,
Schlüsselbegriffen und Konzepten.  \emph{Systemwissenschaft} wird hier als
übergeordneter Ausdruck für ein Feld verwendet, zu dem zahlreiche Gelehrte aus
den verschiedensten Disziplinen wie Anthropologie, Biologie, Chemie, Ökologie,
Ökonomie, Mathematik, Physik, Psychologie, Soziologie und andere beigetragen
haben. Entwicklungen wie Kybernetik, Chaostheorie oder Netzwerkanalyse und
-wissenschaft können als Teil von Systemwissenschaft oder zumindest stark
verwandt mit ihr angesehen werden.  Einige Zweige der Systemwissenschaft
gelten in Deutschland sogar als neue Wissenschaftsbereiche mit eigenen Rechten
wie Synergetik oder Komplexitätswissenschaft.

Diese Entwicklungen haben neue Möglichkeiten für eine verbesserte Analyse und
Entscheidungsfindung in wissenschaftlichen, geschäftlichen und politischen
Bereichen eröffnet. Wir stellen jedoch täglich fest, dass in komplizierten
Situationen, insbesondere in der Politik und in der Wirtschaft, einfache und
direkte Entscheidungsfindungsprozesse nach wie vor überwiegen, was zu einer
Zunahme negativer Entwicklungen führt, wenn die ursprünglich beabsichtigten
Wirkungen nicht eintreten. Jede unerwartete Nebenwirkung oder Gegenreaktion,
die die Maßnahmen unbrauchbar machen, sind ein klares Indiz dafür, dass die
grundlegenden mentalen Modelle der Akteure unvollständig waren und breitere
systemische Korrelationen vernachlässigt wurden. Das Systemdenken ist daher in
Deutschland von besonderer Bedeutung für den Übergang zu einer nachhaltigeren
Gesellschaft.

In diesem Seminar soll die historische Entwicklung der Systemwissenschaft (in
Teilen) verfolgt sowie relevante Grundbegriffe studiert werden. Kursteilnehmer
halten sich dabei an kein spezifisches Modell (wie z.B. \emph{Systemdynamik}),
sondern entwickeln ein tieferes Verständnis für die Systemwissenschaft und für
eine spezifische Art des „Systemdenkens“, mit der Nachhaltigkeitsprobleme
erfolgreicher angegangen werden können. Dies erreichen wir durch das Lesen und
Diskutieren von wissenschaftlichen Arbeiten und Buchkapiteln.

\section{Anmerkungen zum Seminar am 22.10.2019}
\begin{itemize}
\item \textbf{Thema:} Systembegriff in der Theorie dynamischer Systeme (Gräbe)
\item \textbf{Literatur:} (Prigogine 1993)
\item \textbf{Zusatzliteratur:} (Jantsch 1992), (Jooß 2017)
\end{itemize}

\subsection{Fragestellungen der Theorie dynamischer Systeme}

Siehe hierzu
\url{https://de.wikipedia.org/wiki/Dynamisches_System_(Systemtheorie)} 

\subsection{Erste Beispiele}

Beispiele im homogenen Gravitationsfeld mit zunehmend „chaotischem“ Verhalten
\begin{itemize}
\item Das Pendel: \url{https://de.wikipedia.org/wiki/Pendel}
\item Gekoppeltes Pendel:
  \url{https://de.wikipedia.org/wiki/Gekoppelte_Pendel}
\item Doppelpendel\footnote{\url{https://de.wikipedia.org/wiki/Doppelpendel}}
  und chaotische Trajektorien, deterministisches Chaos
\item Magnetisches Pendel:
  \url{https://de.wikipedia.org/wiki/Magnetisches_Pendel}
\end{itemize}
Beispiele mit gravitativer Wechselwirkung mit zunehmend „chaotischem“
Verhalten
\begin{itemize}
\item Zweikörperproblem: \url{https://de.wikipedia.org/wiki/Zweikörperproblem}
\item Dreikörperproblem: \url{https://de.wikipedia.org/wiki/Dreikörperproblem}
\item Das
  Kolmogorow-Arnold-Moser-Theorem\footnote{\url{https://de.wikipedia.org/wiki/Kolmogorow-Arnold-Moser-Theorem}}
  über stablie und instabile Bahnen
\end{itemize}

Das sind bereits -- notwendigerweise reduktionistische -- Beschreibungsformen
der Wirklichkeit: Etwa Pendel mit und ohne Dämpfungsglied.  Aber: Sinnvolle
Reduktionen von Beschreibungsformen \textbf{verbessern} unsere Einsicht in die
Zusammenhänge der Welt.  Hätte Galileo Galilei diese Denkmethodik nicht
angewendet, wäre ihm niemals aufgefallen, dass Eisen und Feder gleich schnell
fallen.

Nicht alles, was wie Chaos aussieht, muss auch Chaos sein:\\
\url{https://i.redd.it/zr7tet9mdfl01.gif} 

\subsection{Grenzzyklen und Attraktoren}

\begin{itemize}
\item Grenzzyklen: \url{https://de.wikipedia.org/wiki/Grenzzyklus}
\item Als \textbf{Attraktor} bezeichnet man eine stabile Lösung des
  entsprechenden Differentialglei\-chungs-Systems

  Beispiel: Attraktoren des Magnetpendels waren die drei stabilen Endlagen,
  also drei Punkte im Phasenraum.

\item Hysterese
\begin{itemize}
  \item Beispiel: Temperaturregelung einer Heizungsanlage
  \item \url{https://de.wikipedia.org/wiki/Hysterese}
\end{itemize}
\item Räuber-Beute-Zyklen
\begin{itemize}
  \item \url{https://de.wikipedia.org/wiki/Räuber-Beute-Beziehung}
  \item \url{https://de.wikipedia.org/wiki/Lotka-Volterra-Regeln}
\end{itemize}
\end{itemize}

Zur Bedeutung \emph{stabiler} zyklischer Prozesse in der Natur. 

Wir sind in der Lage, solche sich \emph{näherungsweise} wiederholenden Muster
in natürlichen Prozessen (d.h. Attraktoren) wahrzunehmen, also auch unabhängig
von der Mathematik eine solche Reduktionsleistung zu vollbringen.
\begin{itemize}
\item Frage: Wie kompliziert können solche Attraktoren werden?
\item Der Lorenzattraktor:
  \url{https://de.wikipedia.org/wiki/Lorenz-Attraktor}.
    
  Achtung, bei den dort verwendeten numerischen Verfahren zur Visualisierung
  ist schwer zu unterscheiden, ob sie eine chaotische Trajektorie berechnen
  oder wirklich den Attraktor, der ja ein \emph{globales} Artefakt ist.
\item Seltsame
  Attraktoren\footnote{\url{https://de.wikipedia.org/wiki/Seltsamer_Attraktor}}
  als „Endzustand eines dynamischen Prozesses, dessen fraktale Dimension nicht
  ganzzahlig und dessen Kolmogorov-Entropie echt positiv ist. Es handelt sich
  damit um ein Fraktal, das nicht in geschlossener Form geometrisch
  beschrieben werden kann“. (Wikipedia)
\end{itemize}
\textbf{Damit ist der Trajektorienbegriff der klassischen Physik für derartige
Phänomene nicht mehr anwendbar.} („Schmetterlingseffekt“)

\subsection{Systeme auf multiplen Zeitskalen}

Ein wichtiger Ansatz ergibt sich für Systeme, deren Dynamiken auf
verschiedenen Zeitskalen ablaufen. Man kann dann methodisch als weiteren
Abstraktionsschritt zunächst die Dynamiken auf den einzelnen Zeitskalen
untersuchen und später in einem erweiterten Modell die Wechselwirkungen
zwischen den Zeitskalen hinzunehmen.  Massiv neue Phänomene ergeben sich
bereits bei der Betrachtung von \emph{zwei} Zeitskalen, was als \emph{Mikro-
  und Makroevolution} bezeichnet wird. Hier wird es in der Wikipedia bereits
dünn.
\begin{itemize}
\item Beispiel: Das Doppelpendel kann in gewissen Grenzen als Pendel
  aufgefasst werden, dessen Pendelkörper selbst noch eine innere Dynamik hat.
  Das Obersystem prägt dem Untersystem durch Energieeintrag eine gemeinsame
  Dynamik auf.  Obwohl Doppelpendel, ist das System damit (final) \emph{nicht}
  chaotisch, sondern verhält sich wie ein einfaches Pendel mit Masse im
  Schwerpunkt.
\item In der Literatur als \emph{Versklavungseffekt} bekannt und besonders in
  methodisch schlecht fundierten soziologischen Betrachtungen als
  Verbalargument verbreitet.\\  Siehe aber
  \url{https://de.wikipedia.org/wiki/Netzwerkforschung}.
\end{itemize}

\subsection{Immersiver und submersiver Systembegriff}

Welche Probleme treten beim Zusammensetzen von (verstandenen) Mikroevolutionen
von Teilsystemen zu einem Verständnis der Dynamik auf der Makroebene auf?

Wie lassen sich zwei Systeme $S_1$ und $S_2$ zu einem Obersystem
zusammenfassen? 

\subsubsection*{Immersiver Zugang}

Mathematische Formulierung der Fragestellung: Dies geschieht durch Abbildungen
(im einfachsten Fall Einbettungen, Immersionen) $f_1: S_1 \to S$, $f_2: S_2
\to S$.  Wie lassen sich solche Abbildungen konstruieren?

Gibt es für diese Konstellation ein \emph{universelles kategorielles Objekt},
d.h. ein universelles $U$ und universelle Abbildungen
\begin{gather*}
  p_1: S_1 \to  S, p_2: S_2 \to  S,  
\end{gather*}
so dass sich für jedes Tripel $(f_1 , f_2 , S)$ die obige Konstellation als
\begin{gather*}
  f_1 = f \circ p_1: S_1 \to U \to  S, f_2 = f \circ p_2: S_2 \to U \to  S
\end{gather*}
für ein geeignetes $f = f_1 \oplus f_2: S \to U$ schreiben lässt? $U$ heißt in
dem Fall \emph{direkte Summe} und man schreibt $U = S_1 \coprod S_2$.

Die meisten mathematischen Modelle bewegen sich in konkreten
\emph{Kategorien}, zum Beispiel der Kategorie der Mengen, der Vektorräume, der
Faserbündel, der algebraischen Varietäten usw.  Jede solche Kategorie zeichnet
sich dadurch aus, dass dort die Begriffe \emph{Objekt} und \emph{Morphismus}
eine klare Bedeutung haben.  Morphismen zwischen Vektorräumen sind zum
Beispiel operationstreue Abbildungen, also lineare Abbildungen, die sich für
endlichdimensionale Vektorräume durch Matrizen beschreiben lassen.

Nicht in jeder Kategorie existieren solche universellen Objekte.

\emph{Anmerkung:} Die Konstruktion lässt sich leicht auf endlich viele $S_i$
und sogar auf unendlich viele $S_i, i\in I$, verallgemeinern, und so ist es in
der Mathematik auch gemeint.

In der \emph{Kategorie der Mengen} existieren direkte Summen $U$ sowohl für
endliche als auch unendliche Indexmengen I. Dies ist gerade die
\emph{disjunkte Vereinigung} der Mengen $S_i$ .

Die Abbildungen $p_i$ sind gerade die Einbettungen $p_i: S_i \to S$ der
Teilmengen in deren disjunkte Vereinigung.

Die Abbildung $f: U \to S$ ergibt sich wie folgt: Für jedes $a\in U$
existieren genau ein $i$ und ein $a'\in S_i$ mit $a=p_i(a')$. Setze
$f(a)=f_i(a')$.

Ist $|S_1| = a$, $|S_2| = b$, so ist $|S_1 \coprod S_2| = a+b$.

Das Ganze ist nicht mehr als die Summe seiner Teile.

\subsubsection*{Submersiver Zugang}

Alle Pfeile umdrehen (TRIZ Prinzip 13)

Es ergeben sich Abbildungen (im einfachsten Fall Projektionen) $f_1: S_1
\leftarrow S$, $f_2: S_2 \leftarrow S$.

Gibt es auch für diese Konstellation ein \emph{universelles kategorielles
  Objekt}, d.h. ein universelles $U$ und universelle Abbildungen
\begin{gather*}
  p_1: S_1 \leftarrow  S, p_2: S_2 \leftarrow  S,  
\end{gather*}
so dass sich für jedes Tripel $(f_1 , f_2 , S)$ die obige Konstellation als
\begin{gather*}
  f_1 = p_1 \circ f: S_1 \leftarrow U \leftarrow S, f_2 = p_2 \circ f: S_2
  \leftarrow U \leftarrow S
\end{gather*}
für ein geeignetes $f = f_1 \otimes f_2: S \to U$ schreiben lässt? $U$ heißt
in dem Fall \emph{direktes Produkt} und man schreibt $U = S_1 \prod S_2$.

In der \emph{Kategorie der Mengen} existieren direkte Produkte $U$ sowohl für
endliche als auch unendliche Indexmengen I. Dies ist gerade das
\emph{Karthesische Produkt} der Mengen $S_i$ .

Die Abbildungen $p_i$ sind die Projektionen $p_i: U \to S_i$ des karthesischen
Produkts auf die einzelnen Komponenten.

Die Abbildung $f: S\to U$ ergibt sich wie folgt: Für jedes $a\in S$
ist $f(a)=(f_i(a):i\in I)$.

Ist $|S_1| = a$, $|S_2| = b$, so ist $|S_1 \prod S_2| = a\cdot b$.

\textbf{Das Ganze ist deutlich mehr als die Summe seiner Teile, der größte
  Teil der „Information“ ist relationaler Natur.}

\subsection{Emergente Phänomene}
\begin{itemize}
\item Nichtlineare Systeme und Phasenübergänge:\\
  \url{https://de.wikipedia.org/wiki/Phasenübergang}.
\item Selbstorganisation in dissipativen Strukturen
\begin{itemize}
  \item \url{https://de.wikipedia.org/wiki/Rayleigh-Bénard-Konvektion}
  \item \url{https://de.wikipedia.org/wiki/Belousov-Zhabotinsky-Reaktion}
\end{itemize}
\item Dissipative Strukturen:
  \url{https://de.wikipedia.org/wiki/Dissipative_Struktur} 
\item Temperatur\footnote{\url{https://de.wikipedia.org/wiki/Temperatur}} als
  emergentes Phänomen.
\item Entropie\footnote{\url{https://de.wikipedia.org/wiki/Entropie}} und
  Enthalpie\footnote{\url{https://de.wikipedia.org/wiki/Enthalpie}}.
\item Leben auf der Erde als dissipatives System
\end{itemize}

\section{Anmerkungen zum Seminar am 29.10.2019}
\begin{itemize}
\item \textbf{Thema:} Einführung in Systemwissenschaft und Nachhaltigkeit,
  Allgemeine System\-theorie (IIRM)
\item \textbf{Literatur:} (Bertalanffy 1950), (Mele 2010), (Binder 2013)
\end{itemize}

Im Seminar wurde das Thema in zwei Vorträgen von Sabine Lautenschläger und
Lydie Laforet vorgestellt.

Hans-Gert Gräbe hat im Nachgang den folgenden Text verfasst, in dem die
Verbindungen zu konzeptionellen Fragen der Theorie Dynamischer Systeme sowie
zu den Argumentationen zum Thema Nachhaltigkeit aus der Vorlesung aufgenommen
werden. Dabei werden einige Aspekte der Theorie dynamischer Systeme (TDS) mit
den von Lautenschläger und Laforet vorgetragenen Systemtheorieansätzen
abgeglichen und damit zugleich einige Punkte der TDS genauer ausgeführt.

\subsection{Bertalanffys Allgemeine Systemtheorie}

Bertalanffy entwickelt in seinem Text (Bertalanffy 1950) zunächst genau die
Grundlagen der TDS im Verständnis jener Zeit.  Der Bezugstext steht damit ganz
am Anfang einer stürmischen Entwicklung der TDS in den 1960er und 1970er
Jahren, die zu fundamental neuen Einsichten in die Vielfalt von Formen der
Lösungen gewöhnlicher Differentialgleichungssysteme geführt haben.  Bereits in
diesem Gebiet\footnote{In den Gleichungen werden nur zeitabhängige Ableitungen
  zugelassen, keine partiellen Ableitungen nach auch noch anderen Parametern,
  das Gebiet der \emph{partiellen Differentialgleichungen} wird also noch
  nicht betreten.} finden sich erstaunliche Phänomene wie der Lorenzattraktor,
deterministisches Chaos, das Ende des Trajektorienbegriffs und fraktale
Gebilde. Mit partiellen Differentialgleichungen kommen noch
Solitonen\footnote{Auf dieses Phänomen bin ich in meinem Seminar nicht
  eingegangen, obwohl diese Strukturen, die in vielen Systemen partieller
  Differentialgleichungen als Lösungen auftreten, zu einem vollkommen neuen
  Verständnis des Welle-Teilchen-Dualismus führen. Siehe dazu
  \url{https://de.wikipedia.org/wiki/Soliton}. } hinzu. Bertalanffy hat also
nur eine erste Ahnung möglicher Phänomene. Seine mathematischen Betrachtungen
verwenden allein Taylorreihen und beschränken sich damit auf Phänomene nahe
einer Gleichgewichtslage, können also mathematisch auf steady state
Situationen (ohne wesentlich vereinfachende Annahmen) nicht einmal angewendet
werden.

Seine wissenschaftstheoretischen Überlegungen fußen auf der Analogie
entsprechender mathematischer Beschreibungsformen in verschiedenen
Wissenschaftsgebieten\footnote{Komplexe Systemtheorie stellt die Adäquatheit
  derartiger Beschreibungen heute selbst in Frage.} und stellen damit nach
meinem Verständnis auf \emph{methodologische} Ähnlichkeit von Zugängen und
\emph{nicht} auf Isomorphie von Strukturen (so Lautenschläger) ab. Dass
Bertalanffys Zugang \emph{deduktiv} sei, kann sich damit auch maximal auf den
mathematisch-deduktiven Kern seiner Argumentation beziehen, nicht aber auf die
weitergehenden wissenschaftstheoretischen Beobachtungen, bzw. dies wäre noch
genauer zu belegen.

\subsection{Der Raumbegriff der TDS}

Der Raumbegriff der TDS entwickelt sich aus dem physikalischen Begriff des
\emph{Phasenraums}. So „lebt“ ein klassisches Vier-Teilchen-System in einem
12-dimensionalen Phasenraum, der durch die $4\times 3$ Raumkoordinaten
aufgespannt wird. Derartige Phasenräume dienen zunächst der Koordinatisierung
der Bewegungsgleichungen, allerdings sieht bereits die Physik in solchen
Koordinatenabhängigkeiten einen Mangel, da die Gesetze unter
Koordinatentransformationen invariant sein müssen, also letztlich
koordinatenfreie Zusammenhangsbeschreibungen mehr Einsicht in bestehende
Zusammenhänge vermitteln. Damit steht zugleich die Frage, invariante
geometrische Strukturen in solchen höherdimensionalen Phasenräumen zu
beschreiben.

Derartige Fragen sind Gegenstand zum Beispiel der algebraischen Geometrie oder
der Differentialgeometrie. In diesen Beschreibungen (der invarianten
geometrischen Gebilde) treten ihrerseits Räume auf, die sich etwa im Konzept
der \emph{Vektorbündel} „materialisieren“ als \emph{Sprache}, um geometrische
Eigenschaften der betrachteten invarianten Strukturen zu beschreiben (wie
Fasern, Keime, Schnitte, Obstruktionen zur Fortsetzbarkeit von Schnitten,
Homologieklassen als Strukturen derartiger Obstruktionen usw.).

Im Bereich der Analysis wird der Raumbegriff weiter verallgemeinert zu
unendlich-dimensio"|nalen Banach- und Sobolev-Räumen, in denen sich gewisse
mathematische Konzepte (etwa das Lebesgue-Integral) überhaupt erst entfalten
lassen für Situationen, wo man mit „klassischen“ Lösungen nicht mehr
weiterkommt.  Theorien (wie etwa der Banachsche Fixpunktsatz) lassen sich
überhaupt erst auf der Basis derart verallgemeinerter Raumbegriffe konsistent
entwickeln.

\subsection{Steady State und Fließgleichgewichte}

Diese Begriffe entwickeln sich später zum Begriff des \emph{Attraktors}
weiter.  Zugleich wird erkannt, dass derartige Attraktoren extrem komplexe
Gestalt haben können, womit eine Unterscheidung zu chaotischem Verhalten
allein auf phänomenologischer Ebene schwierig wird.  Zugleich wird die Rolle
auch \emph{negativer Attraktoren} erkannt.  Derartige Strukturen und
Strukturbildungsprozesse sind typisch für dissipative Prozesse fern von
Gleichgewichtszuständen, die durch einen gewissen Durchsatz von Materie und
Energie getrieben werden. Der Durchsatz von Information spielt dabei keine
Rolle\footnote{Siehe dazu etwa noch einmal
  \url{https://de.wikipedia.org/wiki/Dissipative_Struktur}.}. Ich komme unten
auf diese Frage zurück.

\subsection{Komplexe und komplizierte Systeme}

Diese Unterscheidung habe ich überhaupt nicht begriffen. Sicher kann man einen
solchen Unterschied nicht an der Zerlegbarkeit eines technischen Artefakts
(„ein Auto ist kompliziert, nicht aber komplex“) festmachen, da ein
entsprechender Technikbegriff noch deutlich hinter dem des VDI (siehe meine
1. Vorlesung) zurückbliebe, der zum System wenigstens noch „Herstellung“ und
„Verwendung“ des Artefakts (oder -- dort bereits deutlich -- „Sachsysteme“)
rechnet.

Eine solche Unterscheidung lässt sich nach meinem Verständnis ausschließlich
an den Beschreibungsmethodiken festmachen, die etwa im Potsdamer Manifest (VDW
2005) als „mechanisch-materialistisch“ und „geistig-lebendig“ unterschieden
werden.  Damit kommen wir aber sofort auf grundlegende Fragen, welche Technik-
und Wissenschaftsverständnisse überhaupt nur Grundlage für „Nachhaltigkeit“
sein können und welchen Anteil das Wert-Nutzen-Denken des homo oeconomicus
oder auch nur des homo faber an der aktuellen Krise unserer fossil basierten
Produktionsweise hat.

Carlowitz hat vor 250 Jahren wenigstens noch über eine nachhaltige
Bewirtschaftung der nachwachsenden Ressource „Holz“ raisonniert\footnote{Dass
  Carlowitz' Probleme eng mit der aufkommenden kapitalistischen
  Produktionsweise zusammenhängen und vergleichbare Probleme der
  Bewirtschaftung von Infrastrukturen vorher mit den lokalen Allmendegesetzen
  stabil prozessiert werden konnten, hat Elinor Ostrom klar gezeigt, siehe
  etwa (Stollorz 2011). }. Unsere gesamte Technik und Wissenschaft hat sich
seither rasant weiterentwickelt, allerdings auf der Basis \emph{fossiler}
Rohstoffe, die sich definitiv \emph{nicht} in so kurzen Zeiten regenerieren
wie sie verbraucht werden.  Die damit verbundenen grundlegenden Probleme habe
ich bereits in der 2. Vorlesung („Peak Oil? Peak Everything!“) angeschaut.
Siehe dazu auch (Davis 2008), (Gräbe 2012).

\subsection{Informationsbegriff}

„Komplexe Systeme sind lernfähig“ (Laforet). Lernfähigkeit setzt nach meinem
Verständnis 1) Reflexionsfähigkeit und 2) Selbstreflexionsfähigkeit voraus.
Ich denke nicht, dass der Begriff „komplexes System“ derart eingeengt werden
sollte.  Insgesamt sind wir bei diesem Ansatz bei Informationstheorien auf dem
Stand der 1970er Jahre, etwa (Steinbuch 1969)\footnote{„Geschichte ist die uns
  überlieferte Information über frühere Versuche, die Zukunft zu gestalten.“
  (ebenda, S. 5)}, die Klaus Fuchs-Kittowski (Fuchs-Kittowski 2002) in der
Unterscheidung zwischen Kybernetik 1. und 2. Ordnung noch einmal resümierte.
Dieser Ansatz wurde bereits Ende der 1990er Jahre in Debatten zwischen Janich,
Capurro, Fleissner, Hofkirchner u.a. fundamental kritisiert. Dazu etwa (Janich
2006), (Capurro 1996), (Capurro 1998), (Capurro 2002), (Klemm 2003).

\section{Literatur}

\begin{itemize}
\item Anderies, John M., Marco A. Janssen, Elinor Ostrom (2004).  Framework to
  Analyze the Robustness of Social-ecological Systems from an Institutional
  Perspective. In: Ecology and Society 9 (1), 18.\\
  \url{https://www.ecologyandsociety.org/vol9/iss1/art18/}
\item Ashby, William Ross (1958).  Requisite variety and its implications for
  the control of complex systems. In: Cybernetica 1:2, 83--99.\\
  \url{http://pcp.vub.ac.be/Books/AshbyReqVar.pdf}
\item Bertalanffy, Ludwig von (1950). An outline of General System Theory,
  The British Journal for the Philosophy of Science, Volume I, Issue 2, 1
  August 1950, 134–165.\\ \url{https://doi.org/10.1093/bjps/I.2.134}
  (Verlagsintern) 
\item Binder, C.R., J. Hinkel, P.W. Bots, C. Pahl-Wostl (2013). Comparison of
  Frameworks for Analyzing Social-ecological Systems. Ecology and Society,
  18 (4), 26.\\ \url{https://www.ecologyandsociety.org/vol18/iss4/art26/}
\item Boisot, Max and Bill McKelvey (2011). Complexity and
  Organization-Environment Relations: Revisiting Ashby’s Law of Requisite
  Variety. In: Allen, Peter, Steve Maguire and Bill McKelvey (eds.). The Sage
  Handbook of Complexity and Management, 279--298. (Available at
\item Capurro, Rafael, Peter Fleissner, Wolfgang Hofkirchner (1996). Is a
  unified theory of information feasible?
  \url{http://www.capurro.de/trialog.htm}
\item Capurro, Rafael (1998). Das Capurrosche Trilemma.
  \url{http://www.capurro.de/janich.htm}.
\item Capurro, Rafael (2002). Menschengerechte Information oder
  informationsgerechter\linebreak Mensch?
  \url{http://www.capurro.de/gotha.htm}.
\item Davis, Mike (2008). Wer wird die Arche bauen?  Das Gebot zur Utopie im
  Zeitalter der Katastrophen.  Telepolis, 11.12.2008.
\item Dobusch, Leonhard, Volker, Sigrid Quack (2011). Auf dem Weg zu einer
  Wissensallmende? Argumente Politik und Zeitgeschichte 28--30, S. 41--46.
\item Foxon, T.J., M.S. Reed, L.C. Stringer (2009). Governing long‐term
  social–ecological change: what can the adaptive management and transition
  management approaches learn from each other? Environmental Policy and
  Governance, 19 (1), 3--20.\\ \url{https://doi.org/10.1002/eet.496}
  (Verlagsintern)
\item Fuchs-Kittowski, Klaus (2002). Wissens-Ko-Produktion.  Verarbeitung,
  Verteilung und Entstehung von Informationen in kreativ-lernenden
  Organisationen.  Festschrift zum 65. Geburtstag von Klaus Fuchs-Kittowski.
\item Geels, Frank W., Johan Schot (2007). Typology of Sociotechnical
  Transition Pathways. In: Research Policy 36 (2007), 399–417.\\
  \url{https://doi.org/10.1016/j.respol.2007.01.003} (Verlagsintern)
\item Gräbe, Hans-Gert (2012). Wie geht Fortschritt? LIFIS ONLINE [12.11.12]. 
\item Goldovsky, B.I. (1983). System der Gesetzmäßigkeiten des Aufbaus und der
  Entwicklung technischer Systeme.
  \url{https://wumm-project.github.io/Texts.html} 
\item Gräbe, Hans-Gert (2019). Zur Entwicklung Technischer Systeme.
  Manuskript. \\ \url{https://wumm-project.github.io/Texts.html}
\item Holland, John H. (2006). Studying complex adaptive systems. In: Journal
  of systems science and complexity, 19 (1),
  1–8.\\ \url{https://link.springer.com/article/10.1007/s11424-006-0001-z}
  (Verlagsintern)
\item Holling, C.S. (2000). Understanding the Complexity of Economic,
  Ecological, and Social Systems. In: Ecosystems (2001) 4, 390–405.
  \url{https://www.esf.edu/cue/documents/Holling_Complexity-EconEcol-SocialSys_2001.pdf}
\item Jacobasch, Gisela (2019). Bienensterben -- Ursachen und Folgen.  Leibniz
  Online 37 (2019).
  \url{https://leibnizsozietaet.de/bienensterben-ursachen-und-folgen/}
\item Janich, Peter (2006). Was ist Information? Frankfurt/Main.
\item Jantsch, E. (1992). Die Selbstorganisation des Universums. Vom Urknall
  zum menschlichen Geist.  Hanser, München.
\item Jooß, Christian (2017). Selbstorganisation der Materie.  Verlag Neuer
  Weg, Essen.
\item Klemm, Helmut (2003). Ein großes Elend. Informatik Spektrum,
  S. 267--273. 
\item Koltze, Karl, Valeri Souchkov (2017). Systematische Innovation.
  2. Auf"|lage, Hanser, München.
\item Kozhemyako, Anton (2019). Features of TRIZ applications for solving
  organizational and management problems: schematization of an inventive
  situation and working with models of contradictions. (In Russisch, englische
  Übersetzung in Vorbereitung).\\ \url{https://matriz.org/kozhemyako/}
\item Lyubomirskiy, A., S. Litvin, S. Ikovenko, C. M. Thurnes, R. Adunka
  (2018).  Trends of Engineering System Evolution (TESE).
\item Mann, Darrell (2019).  Systematic innovation in complex
  environments. Proceedings of the TRIZ Summit 2019 Minsk.\\
  \url{https://triz-summit.ru/file.php/id/f304797-file-original.pdf} 
\item Mele, C., J. Pels, F. Polese (2010). A brief review of systems theories
  and their managerial applications. Service Science, 2(1--2), 126--135.\\
  \url{https://doi.org/10.1287/serv.2.1_2.126} (Open Access)
\item Mingers, John (1989). An Introduction to Autopoiesis -- Implications and
  Applications. In: Systems Practice, Vol. 2, No. 2, 1989.\\
  \url{https://link.springer.com/article/10.1007/BF01059497} (Verlagsintern) 
\item Ostrom, Elinor (2007). A diagnostic approach for going beyond panaceas.
  Proceedings of the national Academy of sciences, 104(39), 15181--15187.\\
  \url{https://doi.org/10.1073/pnas.0702288104} (Open Access)
\item Prigogine, Ilya, Isabelle Stengers (1993). Das Pardox der Zeit. Piper,
  München, Kap. 3--5.  
\item Ropohl, Günter (2009). Allgemeine Technologie: eine Systemtheorie der
  Technik.  KIT Scientific Publishing.
  \url{https://books.openedition.org/ksp/3007} (Open Access) 
\item Rubin, Michail (2019).  Zum Zusammenhang der Entwicklungsgesetze
  allgemeiner Systeme und der Entwicklungsgesetze technischer Systeme. \\
  \url{https://wumm-project.github.io/Texts.html} 
\item Souchkov, Valeri (2014). Breakthrough Thinking with TRIZ for Business
  and Management: An Overview. \url{https://www.semanticscholar.org}
\item Steinbuch, Karl (1969). Die informierte Gesellschaft.  Stuttgart,
  2. Auflage. 
\item Stollorz, Volker (2011). Elinor Ostrom und die Wiederentdeckung der
  Allmende. Argumente Politik und Zeitgeschichte 28--30, S. 3--15. 
\item Ulanowicz, Robert E. (2009). The dual nature of ecosystem dynamics.
  In: Ecological Modelling 220 (2009), 1886–1892.\\
  \url{https://people.clas.ufl.edu/ulan/files/Dual.pdf} (Green Paper des
  Autors) 
\item VDW -- Verein Deutscher Wissenschaftler (2005). „We have to learn to
  think in a new way“. Potsdamer Denkschrift.
\item Vernadsky, V.I. (1997, Original 1936--38). Scientific Thought as a
  Planetary Phenomenon. \url{https://wumm-project.github.io/Texts.html}
\item Walker, Brian, C. S. Holling, Stephen R. Carpenter, Ann Kinzig (2004).
  Resilience, Adaptability and Transformability in Social-ecological Systems. 
  In: Ecology and Society 9 (2).
  \url{https://www.ecologyandsociety.org/vol9/iss2/art5/}
\end{itemize}

\end{document}
